\documentclass{sigplanconf}

 \usepackage{cmap,mmap}
 \usepackage{accsupp}

 % The following packages are needed because unicode
 % is translated (using the next set of packages) to
 % latex commands. You may need more packages if you
 % use more unicode characters:

 % This handles the translation of unicode to latex:

 \newcommand{\unicodesub}[1]{\ensuremath{{}_{\text{#1}}}}

 \usepackage[autogenerated,mathletters]{ucs}
 \usepackage[utf8x]{inputenc}
 % \usepackage{autofe}

 \usepackage{agda}

 \usepackage{amssymb}
 \usepackage{amsmath}
 \usepackage{minted} % must be before hyperref to not get duplicate table identifiers
 \usepackage{hyperref}
 \usepackage{xcolor}
 \usepackage{upquote}
 \usepackage{stmaryrd}
 \usepackage{slashbox}
 % \usepackage{bbm}
 \usepackage[english]{babel}

 % disable minted red error boxes on syntax error
 \makeatletter
 \expandafter\def\csname PYGdefault@tok@err\endcsname{\def\PYGdefault@bc##1{{\strut ##1}}}
 \makeatother

 % Some characters that are not automatically defined
 % (you figure out by the latex compilation errors you get),
 % and you need to define:

 \newcommand*{\ucopyable}[2]{%
   \BeginAccSupp{%
     method=hex,%
     unicode,%
     ActualText=#1%
   }%
   #2%
   \EndAccSupp{}%
 }

 \DeclareUnicodeCharacter{8988}{\ensuremath{\ulcorner}}
 \DeclareUnicodeCharacter{8989}{\ensuremath{\urcorner}}
 \DeclareUnicodeCharacter{8990}{\ensuremath{\llcorner}}
 \DeclareUnicodeCharacter{8991}{\ensuremath{\lrcorner}}
 %\DeclareUnicodeCharacter{9633}{\ensuremath{\square}}
 \DeclareUnicodeCharacter{931}{\ucopyable{03A3}{\ensuremath{\Sigma}}}
 \DeclareUnicodeCharacter{915}{\ucopyable{0393}{\ensuremath{\Gamma}}}
 \DeclareUnicodeCharacter{928}{\ucopyable{03A0}{\ensuremath{\Pi}}}
 %\DeclareUnicodeCharacter{8803}{\ensuremath{\overline{\equiv}}}
 \DeclareUnicodeCharacter{9659}{\ensuremath{\triangleright}}
 \DeclareUnicodeCharacter{1255}{\"o}
 %\DeclareUnicodeCharacter{7488}{\ensuremath{{}^T}}
 %\DeclareUnicodeCharacter{7511}{\ensuremath{{}^t}}
 %\DeclareUnicodeCharacter{7580}{\ensuremath{{}^c}}
 %\DeclareUnicodeCharacter{8336}{\ensuremath{{}_a}}
 %\DeclareUnicodeCharacter{8348}{\ensuremath{{}_t}}
 \DeclareUnicodeCharacter{0738}{\ensuremath{{}^{\text{s}}}}
 \DeclareUnicodeCharacter{8216}{\text{\textquoteleft}}
 \DeclareUnicodeCharacter{8217}{\text{\textquoteright}}
 \DeclareUnicodeCharacter{8220}{\text{\textquotedblleft}}
 \DeclareUnicodeCharacter{8221}{\text{\textquotedblright}}
 % Add more as you need them (shouldn’t happen often).

 %\undef\bibfont
 %\usepackage[style=abbrvnat,authoryear,square,natbib=true]{biblatex}
 %\bibpunct{(}{)}{;}{a}{}{,}
 %\let \cite = \citep

 %\addbibresource{lob.bib}

\newcommand{\todo}[1]{\textcolor{red}{TODO: #1}}

\makeatletter
\let\@oldauthorinfo=\authorinfo
\newcommand{\@emptyauthorinfo}[3]{}
\newcommand{\anonymousauthorinfo}[3]{\@oldauthorinfo{Anonymous}{}{}}
\newcommand{\@newauthorinfo}[3]{\anonymousauthorinfo{#1}{#2}{#3}\let\authorinfo=\@emptyauthorinfo}
\@ifclasswith{sigplanconf}{preprint}{
  \let\authorinfo=\@newauthorinfo
  \newcommand{\inputacknowledgements}{\acks{} [Redacted for the anonymous submission]}
}{
  \newcommand{\inputacknowledgements}{\InputIfFileExists{./acknowledgments.tex}{}{[Redacted for the anonymous submission]}}
}
\makeatother

%\overfullrule=1mm

%\renewcommand{\AgdaCodeStyle}{\tt}

\begin{document}

\special{papersize=8.5in,11in}
\setlength{\pdfpageheight}{\paperheight}
\setlength{\pdfpagewidth}{\paperwidth}

\conferenceinfo{ICFP '16}{September 19--21, 2016, Nara, Japan}
\copyrightyear{2016}
%\copyrightdata{978-1-nnnn-nnnn-n/yy/mm}
%\copyrightdoi{nnnnnnn.nnnnnnn}

% Uncomment the publication rights you want to use.
%\publicationrights{transferred}
\publicationrights{licensed}     % this is the default
%\publicationrights{author-pays}

%\titlebanner{banner above paper title}        % These are ignored unless
%\preprintfooter{short description of paper}   % 'preprint' option specified.

\IfFileExists{./authorinfo.tex}{%
\authorinfo{Jason Gross}
           {MIT CSAIL}
           {\href{mailto:jgross@mit.edu}{jgross@mit.edu}}
\authorinfo{Jack Gallagher}%Name2\and Name3}
           {MIRI}%Affiliation2/3}
           {\href{mailto:jack@gallabytes.com}{jack@gallabytes.com}}%Email2/3}
%
}{%
\anonymousauthorinfo{}{}{}%
}%


\title{Lӧb's Theorem}
\subtitle{A functional pearl of dependently typed quining}

\maketitle

%\category{CR-number}{subcategory}{third-level}

% general terms are not compulsory anymore,
% you may leave them out
%\terms
%Agda, Lob, quine, self-reference

\keywords
Agda, Lӧb's theorem, quine, self-reference, type theory

\AgdaHide{
  \begin{code}%
\>\AgdaKeyword{module} \AgdaModule{lob} \AgdaKeyword{where}\<%
\end{code}
}

\begin{abstract}
Lӧb's theorem states that to prove that a proposition is provable, it
is sufficient to prove the proposition under the assumption that it is
provable.  The Curry-Howard isomorphism identifies formal proofs with
abstract syntax trees of programs; Lӧb's theorem thus implies, for
total languages which validate it, that self-interpreters are
impossible.  We formalize a few variations of Lӧb's theorem in Agda
using an inductive-inductive encoding of terms indexed over types.  We
verify the consistency of our formalizations relative to Agda by
giving them semantics via interpretation functions.
\end{abstract}

% \todo{Should we unify the various repr-like functions (repr, add-quote, ⌜_⌝ᵀ, ⌜_⌝ᵗ, ⌜_⌝ᶜ)?}

\section{Introduction}

\begin{quotation}
\noindent \textit{If P's answer is `Bad!', Q will suddenly stop. \\
But otherwise, Q will go back to the top, \\
and start off again, looping endlessly back, \\
till the universe dies and turns frozen and black.}
\end{quotation}
\begin{flushright}
Excerpt from \emph{Scooping the Loop Snooper: A proof that the Halting Problem is undecidable} \cite{loopsnoop}
\end{flushright}

 Lӧb's theorem has a variety of applications, from providing an
 induction rule for program semantics involving a ``later''
 operator~\cite{appel2007very}, to proving incompleteness of a logical
 theory as a trivial corollary, from acting as a no-go theorem for a
 large class of self-interpreters, to allowing robust cooperation in
 the Prisoner's Dilemma with Source
 Code~\cite{BaraszChristianoFallensteinEtAl2014}, and even in one case
 curing social anxiety~\cite{Yudkowsky2014}.

 In this paper, after introducing the content of Lӧb's theorem, we
 will present in Agda three formalizations of type-theoretic languages
 and prove Lӧb's theorem in and about these languages: one that shows
 the theorem is admissible as an axiom in a wide range of situations,
 one which proves Lӧb's theorem by assuming as an axiom the existence
 of quines (programs which output their own source code), and one
 which constructs the proof under even weaker assumptions; see
 \autoref{sec:prior-work-and-new} for details.

 ``What is Lӧb's theorem, this versatile tool with wondrous
 applications?'' you may ask.

 Consider the sentence ``if this sentence is true, then you, dear
 reader, are the most awesome person in the world.''  Suppose that
 this sentence is true.  Then you are the most awesome person in the
 world.  Since this is exactly what the sentence asserts, the sentence
 is true, and you are the most awesome person in the world.  For those
 more comfortable with symbolic logic, we can let $X$ be the statement
 ``you, dear reader, are the most awesome person in the world'', and
 we can let $A$ be the statement ``if this sentence is true, then
 $X$''.  Since we have that $A$ and $A → X$ are the same, if we assume
 $A$, we are also assuming $A → X$, and hence we have $X$.  Thus since
 assuming $A$ yields $X$, we have that $A → X$.  What went
 wrong?\footnote{Those unfamiliar with conditionals should note that
 the ``if \ldots\space then \ldots'' we use here is the logical
 ``if'', where ``if false then $X$'' is always true, and not the
 counter-factual ``if''.}

 It can be made quite clear that something is wrong; the more common
 form of this sentence is used to prove the existence of Santa Claus
 to logical children: considering the sentence ``if this sentence is
 true, then Santa Claus exists'', we can prove that Santa Claus
 exists.  By the same logic, though, we can prove that Santa Claus
 does not exist by considering the sentence ``if this sentence is
 true, then Santa Claus does not exist.''  Whether you consider it
 absurd that Santa Claus exist, or absurd that Santa Claus not exist,
 surely you will consider it absurd that Santa Claus both exist and
 not exist.  This is known as Curry's Paradox.

 The problem is that the phrase ``this sentence is true'' is not a
 valid mathematical assertion; no language can encode a truth
 predicate for itself~\cite{tarski1936undefinability}.  However, some
 languages \emph{can} encode assertions about
 \emph{provability}~\cite{godel1931formal}.  In
 \autoref{sec:quine-curry}, we will dig into the difference between
 truth predicates and provability predicates from a computational
 perspective.  We will present an argument for the indefinability of
 truth and for the definability of provability, which we hope will
 prove enlightening when we get to the formalization of Lӧb's theorem
 itself.

 Now consider the sentence ``if this sentence is provable, then Santa
 Claus exists.''  To prove that that sentence is true, we suppose that
 it is provable.  We must now show that Santa Claus exists.  \emph{If
 provability implies truth}, then the sentence is true, and thus Santa
 Claus exists.  Hence, if we can assume that provability implies
 truth, then we can prove that the sentence is true.  This, in a
 nutshell, is Lӧb's theorem: to prove $X$, it suffices to prove that
 $X$ is true whenever $X$ is provable.  If we let $□ X$ denote the
 assertion ``$X$ is provable,'' then, symbolically, Lӧb's theorem
 becomes: $$□ (□ X → X) → □ X.$$ Note that Gӧdel's incompleteness
 theorem follows trivially from Lӧb's theorem: by instantiating $X$
 with a contradiction, we can see that it's impossible for provability
 to imply truth for propositions which are not already true.

 Note that Lӧb's theorem is specific to the formal system and to the
 notion of probability used.  In particular, the formal system must be
 powerful enough to talk about which of its sentences are provable;
 examples of such formal systems include Peano Arithmetic, Martin--Lӧf
 Type Theory, and Gӧdel-Lӧb Modal Logic.  In this paper, we fix formal
 systems by formalizing them as object languages in Agda, and we fix
 formalizations of provability in those systems by treating each
 formalized language as the metalanguage for some formalization of
 itself.

\section{Quines and the Curry--Howard Isomorphism} \label{sec:quine-curry}

 Let us now return to the question we posed above: what went wrong
 with our original sentence?  The answer is that self-reference with
 truth is impossible, and the clearest way we know to argue for this is
 via the Curry--Howard Isomorphism; in a particular technical sense,
 the problem is that self-reference with truth fails to terminate.

 The Curry--Howard Isomorphism establishes an equivalence between
 types and propositions, between (well-typed, terminating, functional)
 programs and proofs.  See \autoref{table:curry-howard} for some
 examples.  Now we ask: what corresponds to a formalization of
 provability?  A proof of $P$ is a terminating functional program
 which is well-typed at the type corresponding to $P$.  To assert that
 $P$ is provable is to assert that the type corresponding to $P$ is
 inhabited.  Thus an encoding of a proof is an encoding of a program.
 Although mathematicians typically use Gӧdel codes to encode
 propositions and proofs, a more natural choice of encoding programs
 is abstract syntax trees (ASTs).  In particular, a valid syntactic
 proof of a given (syntactic) proposition corresponds to a well-typed
 syntax tree for an inhabitant of the corresponding syntactic type.
 Other formalizations of self-representation of programs in programs
 abound~\cite{church1940formulation,Davies:2001:MAS:382780.382785,geuvers2014church,Kiselyov2012,DBLP:conf/ershov/Mogensen01,PFENNING1991137,scott1963system,nott31169,Berarducci1985,brown2016breaking}.

 Note well that the type \mintinline{Agda}|(□ X → X)| is the type of
 functions that take syntax trees and evaluate them; it is the type
 of an interpreter or an unquoter.

  \begin{table}
  \begin{center}
  \begin{tabular}{ccc}
  Logic & Programming & Set Theory \\ \hline
  Proposition & Type & Set of Proofs \\
  Proof & Program & Element \\
  Implication (→) & Function (→) & Function  \\
  Conjunction (∧) & Pairing (,) & Cartesian Product (×)  \\
  Disjunction (∨) & Sum (+) & Disjoint Union (⊔) \\
  Gӧdel codes & ASTs & --- \\
  □ X → X & Interpreters & --- \\
  (In)completeness & Halting problem & ---
  \end{tabular}
  \end{center}
  \caption{The Curry-Howard Isomorphism between mathematical logic and functional programming} \label{table:curry-howard}
  \end{table}

% Unless otherwise specified, we will henceforth consider only
% well-typed, terminating programs; when we say ``program'', the
% adjectives ``well-typed'' and ``terminating'' are implied.

% Before diving into Lӧb's theorem in detail, we'll first visit a
% standard paradigm for formalizing the syntax of dependent type
% theory. (\todo{Move this?})

 What is the computational equivalent of the sentence ``If this
 sentence is provable, then $X$''?  It will be something of the form
 ``??? → $X$''.  As a warm-up, let's look at a Python program that
 outputs its own source code.

 There are three genuinely distinct solutions, the first of which is
 degenerate, and the second of which is cheeky.  These solutions are:
 \label{sec:python-quine}
 \begin{itemize}
   \item The empty program, which outputs nothing.
   \item The code
     \mintinline{Python}|print(open(__file__, 'r').read())|,
     which relies on the Python interpreter to get the
     source code of the program.

   \item A program with a ``template'' which contains a copy of the
     source code of all of the program except for the template itself,
     leaving a hole where the template should be.  The program then
     substitutes a quoted copy of the template into the hole in the
     template itself.  In code, we can use Python's
     \mintinline{Python}|repr| to get a quoted copy of the template,
     and we do substitution using Python's replacement syntax: for
     example, \mintinline{Python}|("foo %s bar" % "baz")| becomes
     \mintinline{Python}|"foo baz bar"|.  Our third solution, in code,
     is thus:
\begin{minted}[mathescape,
%               numbersep=5pt,
               gobble=2,
%               frame=lines,
%               framesep=2mm%
]{Python}
  T = 'T = %s\nprint(T %% repr(T))'
  print(T % repr(T))
\end{minted}

    The functional equivalent, which does not use assignment, and
    which we will be using later on in this paper, is:
\begin{minted}[mathescape,gobble=2]{Python}
  (lambda T: T % repr(T))
   ('(lambda T: T %% repr(T))\n (%s)')
\end{minted}

  \end{itemize}

 We can use this technique, known as
 quining~\cite{hofstadter1980godel,kleene1952introduction}, to
 describe self-referential programs.

 Suppose Python had a function □ that took a quoted representation of
 a Martin--Lӧf type (as a Python string), and returned a Python object
 representing the Martin--Lӧf type of ASTs of
 Martin--Lӧf programs inhabiting that type.  Now consider the program
\begin{minted}[mathescape,gobble=2,]{Python}
  φ = (lambda T: □(T % repr(T)))
       ('(lambda T: □(T %% repr(T)))\n (%s)')
\end{minted}

  The variable \mintinline{Python}|φ| evaluates to the type of ASTs of
  programs inhabiting the type corresponding to
  \mintinline{Python}|T % repr(T)|, where \mintinline{Python}|T| is
  \mintinline{Python}|'(lambda T: □(T %% repr(T)))\n (%s)'|. What
  Martin--Lӧf type does this string, \mintinline{Python}|T % repr(T)|,
  represent? It represents \mintinline{Python}|□(T % repr(T))|, of
  course. Hence \mintinline{Python}|φ| is the type of syntax trees of
  programs that produce proofs of \mintinline{Python}|φ|----in other
  words, \mintinline{Python}|φ| is a Henkin sentence.


  Taking it one step further, assume Python has a function
  \mintinline{Python}|Π(a, b)| which takes two Python representations
  of Martin--Lӧf types and produces the Martin--Lӧf type
  \mintinline{Agda}|(a → b)| of functions from \mintinline{Agda}|a| to
  \mintinline{Agda}|b|.  If we also assume that these functions exist
  in the term language of string representations of Martin--Lӧf types,
  we can consider the function
\begin{minted}[mathescape,gobble=2]{Python}
  def Lob(X):
    T = '(lambda T: Π(□(T %% repr(T)), X))(%s)'
    φ = Π(□(T % repr(T)), X)
    return φ
\end{minted}

  What does \mintinline{Python}|Lob(X)| return?  It returns the type
  \mintinline{Python}|φ| of abstract syntax trees of programs
  producing proofs that ``if \mintinline{Python}|φ| is provable, then
  \mintinline{Python}|X|.''  Concretely, \mintinline{Python}|Lob(⊥)|
  returns the type of programs which prove Martin--Lӧf type theory
  consistent, \mintinline{Python}|Lob(SantaClaus)| returns the variant
  of the Santa Claus sentence that says ``if this sentence is
  provable, then Santa Claus exists.''

  Let us now try producing the true Santa Claus sentence, the one
  that says ``If this sentence is true, Santa Claus exists.'' We need
  a function \mintinline{Python}|Eval| which takes a string
  representing a Martin--Lӧf program, and evaluates it to produce a
  term. Consider the Python program
\begin{minted}[mathescape,gobble=2,]{Python}
  def Tarski(X):
    T = '(lambda T: Π(Eval(T %% repr(T)), X)(%s)'
    φ = Π(Eval(T % repr(T)), X)
    return φ
\end{minted}

  Running \mintinline{Python}|Eval(T % repr(T))| tries to produce a
  term that is the type of functions from
  \mintinline{Python}|Eval(T % repr(T))| to
  \mintinline{Python}|X|. Note that \mintinline{Python}|φ| is itself
  the type of functions from \mintinline{Python}|Eval(T % repr(T))| to
  \mintinline{Python}|X|.  If \mintinline{Python}|Eval(T % repr(T))|
  could produce a term of type \mintinline{Python}|φ|, then
  \mintinline{Python}|φ| would evaluate to the type
  \mintinline{Python}|φ → X|, giving us a bona fide Santa Claus
  sentence. However, \mintinline{Python}|Eval(T % repr(T))| attempts
  to produce the type of functions from \mintinline{Python}|Eval(T %
  repr(T))| to \mintinline{Python}|X| by evaluating
  \mintinline{Python}|Eval(T % repr(T))|.  This throws the function
  \mintinline{Python}|Tarski| into an infinite loop which never
  terminates. (Indeed, choosing \mintinline{Python}|X = ⊥| it's
  trivial to show that there's no way to write
  \mintinline{Python}|Eval| such that \mintinline{Python}|Tarski|
  halts, unless Martin--Lӧf type theory is inconsistent.)

\section{Abstract Syntax Trees for Dependent Type Theory} \label{sec:local-interpretation}

  The idea of formalizing a type of syntax trees which only permits
  well-typed programs is common in the
  literature~\cite{mcbride2010outrageous,Chapman200921,danielsson2006formalisation}.
  For example, here is a very simple (and incomplete) formalization
  with dependent function types ($\Pi$), a unit type (⊤), an empty
  type (⊥), and functions ($\lambda$).

  We will use some standard data type declarations, which are provided
  for completeness in \autoref{sec:common}.
 \AgdaHide{
  \begin{code}%
\>\AgdaKeyword{open} \AgdaKeyword{import} \AgdaModule{common} \AgdaKeyword{public}\<%
\end{code}
}
\AgdaHide{
  \begin{code}%
\>\AgdaKeyword{module} \AgdaModule{dependent-type-theory} \AgdaKeyword{where}\<%
\end{code}
}

\noindent
\begin{code}%
\> \AgdaKeyword{mutual}\<%
\\
\>[0]\AgdaIndent{2}{}\<[2]%
\>[2]\AgdaKeyword{infixl} \AgdaNumber{2} \AgdaFixityOp{\_▻\_}\<%
\\
%
\\
\>[0]\AgdaIndent{2}{}\<[2]%
\>[2]\AgdaKeyword{data} \AgdaDatatype{Context} \AgdaSymbol{:} \AgdaPrimitiveType{Set} \AgdaKeyword{where}\<%
\\
\>[2]\AgdaIndent{3}{}\<[3]%
\>[3]\AgdaInductiveConstructor{ε} \AgdaSymbol{:} \AgdaDatatype{Context}\<%
\\
\>[2]\AgdaIndent{3}{}\<[3]%
\>[3]\AgdaInductiveConstructor{\_▻\_} \AgdaSymbol{:} \AgdaSymbol{(}\AgdaBound{Γ} \AgdaSymbol{:} \AgdaDatatype{Context}\AgdaSymbol{)} \AgdaSymbol{→} \AgdaDatatype{Type} \AgdaBound{Γ} \AgdaSymbol{→} \AgdaDatatype{Context}\<%
\\
%
\\
\>[0]\AgdaIndent{2}{}\<[2]%
\>[2]\AgdaKeyword{data} \AgdaDatatype{Type} \AgdaSymbol{:} \AgdaDatatype{Context} \AgdaSymbol{→} \AgdaPrimitiveType{Set} \AgdaKeyword{where}\<%
\\
\>[2]\AgdaIndent{3}{}\<[3]%
\>[3]\AgdaInductiveConstructor{‘⊤’} \AgdaSymbol{:} \AgdaSymbol{∀} \AgdaSymbol{\{}\AgdaBound{Γ}\AgdaSymbol{\}} \AgdaSymbol{→} \AgdaDatatype{Type} \AgdaBound{Γ}\<%
\\
\>[2]\AgdaIndent{3}{}\<[3]%
\>[3]\AgdaInductiveConstructor{‘⊥’} \AgdaSymbol{:} \AgdaSymbol{∀} \AgdaSymbol{\{}\AgdaBound{Γ}\AgdaSymbol{\}} \AgdaSymbol{→} \AgdaDatatype{Type} \AgdaBound{Γ}\<%
\\
\>[2]\AgdaIndent{3}{}\<[3]%
\>[3]\AgdaInductiveConstructor{‘Π’} \AgdaSymbol{:} \AgdaSymbol{∀} \AgdaSymbol{\{}\AgdaBound{Γ}\AgdaSymbol{\}}\<%
\\
\>[3]\AgdaIndent{5}{}\<[5]%
\>[5]\AgdaSymbol{→} \AgdaSymbol{(}\AgdaBound{A} \AgdaSymbol{:} \AgdaDatatype{Type} \AgdaBound{Γ}\AgdaSymbol{)} \AgdaSymbol{→} \AgdaDatatype{Type} \AgdaSymbol{(}\AgdaBound{Γ} \AgdaInductiveConstructor{▻} \AgdaBound{A}\AgdaSymbol{)} \AgdaSymbol{→} \AgdaDatatype{Type} \AgdaBound{Γ}\<%
\\
%
\\
\>[0]\AgdaIndent{2}{}\<[2]%
\>[2]\AgdaKeyword{data} \AgdaDatatype{Term} \AgdaSymbol{:} \AgdaSymbol{\{}\AgdaBound{Γ} \AgdaSymbol{:} \AgdaDatatype{Context}\AgdaSymbol{\}} \AgdaSymbol{→} \AgdaDatatype{Type} \AgdaBound{Γ} \AgdaSymbol{→} \AgdaPrimitiveType{Set} \AgdaKeyword{where}\<%
\\
\>[2]\AgdaIndent{3}{}\<[3]%
\>[3]\AgdaInductiveConstructor{‘tt’} \AgdaSymbol{:} \AgdaSymbol{∀} \AgdaSymbol{\{}\AgdaBound{Γ}\AgdaSymbol{\}} \AgdaSymbol{→} \AgdaDatatype{Term} \AgdaSymbol{\{}\AgdaBound{Γ}\AgdaSymbol{\}} \AgdaInductiveConstructor{‘⊤’}\<%
\\
\>[2]\AgdaIndent{3}{}\<[3]%
\>[3]\AgdaInductiveConstructor{‘λ’} \AgdaSymbol{:} \AgdaSymbol{∀} \AgdaSymbol{\{}\AgdaBound{Γ} \AgdaBound{A} \AgdaBound{B}\AgdaSymbol{\}} \AgdaSymbol{→} \AgdaDatatype{Term} \AgdaBound{B} \AgdaSymbol{→} \AgdaDatatype{Term} \AgdaSymbol{\{}\AgdaBound{Γ}\AgdaSymbol{\}} \AgdaSymbol{(}\AgdaInductiveConstructor{‘Π’} \AgdaBound{A} \AgdaBound{B}\AgdaSymbol{)}\<%
\end{code}

  An easy way to check consistency of a syntactic theory which is
  weaker than the theory of the ambient proof assistant is to define
  an interpretation function, also commonly known as an unquoter, or a
  denotation function, from the syntax into the universe of types.
  This function gives a semantic model to the syntax.  Here is an
  example of such a function:

\begin{code}%
\> \AgdaKeyword{mutual}\<%
\\
\>[0]\AgdaIndent{2}{}\<[2]%
\>[2]\AgdaFunction{⟦\_⟧ᶜ} \AgdaSymbol{:} \AgdaDatatype{Context} \AgdaSymbol{→} \AgdaPrimitiveType{Set}\<%
\\
\>[0]\AgdaIndent{2}{}\<[2]%
\>[2]\AgdaFunction{⟦} \AgdaInductiveConstructor{ε} \AgdaFunction{⟧ᶜ} \<[13]%
\>[13]\AgdaSymbol{=} \AgdaRecord{⊤}\<%
\\
\>[0]\AgdaIndent{2}{}\<[2]%
\>[2]\AgdaFunction{⟦} \AgdaBound{Γ} \AgdaInductiveConstructor{▻} \AgdaBound{T} \AgdaFunction{⟧ᶜ} \AgdaSymbol{=} \AgdaRecord{Σ} \AgdaFunction{⟦} \AgdaBound{Γ} \AgdaFunction{⟧ᶜ} \AgdaFunction{⟦} \AgdaBound{T} \AgdaFunction{⟧ᵀ}\<%
\\
%
\\
\>[0]\AgdaIndent{2}{}\<[2]%
\>[2]\AgdaFunction{⟦\_⟧ᵀ} \AgdaSymbol{:} \AgdaSymbol{∀} \AgdaSymbol{\{}\AgdaBound{Γ}\AgdaSymbol{\}}\<%
\\
\>[2]\AgdaIndent{4}{}\<[4]%
\>[4]\AgdaSymbol{→} \AgdaDatatype{Type} \AgdaBound{Γ} \AgdaSymbol{→} \AgdaFunction{⟦} \AgdaBound{Γ} \AgdaFunction{⟧ᶜ} \AgdaSymbol{→} \AgdaPrimitiveType{Set}\<%
\\
\>[0]\AgdaIndent{2}{}\<[2]%
\>[2]\AgdaFunction{⟦} \AgdaInductiveConstructor{‘⊤’} \AgdaFunction{⟧ᵀ} \AgdaBound{Γ⇓} \<[18]%
\>[18]\AgdaSymbol{=} \AgdaRecord{⊤}\<%
\\
\>[0]\AgdaIndent{2}{}\<[2]%
\>[2]\AgdaFunction{⟦} \AgdaInductiveConstructor{‘⊥’} \AgdaFunction{⟧ᵀ} \AgdaBound{Γ⇓} \<[18]%
\>[18]\AgdaSymbol{=} \AgdaDatatype{⊥}\<%
\\
\>[0]\AgdaIndent{2}{}\<[2]%
\>[2]\AgdaFunction{⟦} \AgdaInductiveConstructor{‘Π’} \AgdaBound{A} \AgdaBound{B} \AgdaFunction{⟧ᵀ} \AgdaBound{Γ⇓} \AgdaSymbol{=} \AgdaSymbol{(}\AgdaBound{x} \AgdaSymbol{:} \AgdaFunction{⟦} \AgdaBound{A} \AgdaFunction{⟧ᵀ} \AgdaBound{Γ⇓}\AgdaSymbol{)} \AgdaSymbol{→} \AgdaFunction{⟦} \AgdaBound{B} \AgdaFunction{⟧ᵀ} \AgdaSymbol{(}\AgdaBound{Γ⇓} \AgdaInductiveConstructor{,} \AgdaBound{x}\AgdaSymbol{)}\<%
\\
%
\\
\>[0]\AgdaIndent{2}{}\<[2]%
\>[2]\AgdaFunction{⟦\_⟧ᵗ} \AgdaSymbol{:} \AgdaSymbol{∀} \AgdaSymbol{\{}\AgdaBound{Γ} \AgdaBound{T}\AgdaSymbol{\}}\<%
\\
\>[2]\AgdaIndent{4}{}\<[4]%
\>[4]\AgdaSymbol{→} \AgdaDatatype{Term} \AgdaSymbol{\{}\AgdaBound{Γ}\AgdaSymbol{\}} \AgdaBound{T} \AgdaSymbol{→} \AgdaSymbol{(}\AgdaBound{Γ⇓} \AgdaSymbol{:} \AgdaFunction{⟦} \AgdaBound{Γ} \AgdaFunction{⟧ᶜ}\AgdaSymbol{)} \AgdaSymbol{→} \AgdaFunction{⟦} \AgdaBound{T} \AgdaFunction{⟧ᵀ} \AgdaBound{Γ⇓}\<%
\\
\>[0]\AgdaIndent{2}{}\<[2]%
\>[2]\AgdaFunction{⟦} \AgdaInductiveConstructor{‘tt’} \AgdaFunction{⟧ᵗ} \AgdaBound{Γ⇓} \<[18]%
\>[18]\AgdaSymbol{=} \AgdaInductiveConstructor{tt}\<%
\\
\>[0]\AgdaIndent{2}{}\<[2]%
\>[2]\AgdaFunction{⟦} \AgdaInductiveConstructor{‘λ’} \AgdaBound{f} \AgdaFunction{⟧ᵗ} \AgdaBound{Γ⇓} \AgdaBound{x} \AgdaSymbol{=} \AgdaFunction{⟦} \AgdaBound{f} \AgdaFunction{⟧ᵗ} \AgdaSymbol{(}\AgdaBound{Γ⇓} \AgdaInductiveConstructor{,} \AgdaBound{x}\AgdaSymbol{)}\<%
\end{code}

  Note that this interpretation function has an essential property
  that we will call \emph{locality}: the interpretation of any given
  constructor does not require doing case analysis on any of its
  arguments.  By contrast, one could imagine an interpretation
  function that interpreted function types differently depending on
  their domain and codomain; for example, one might interpret
  \mintinline{Agda}|(‘⊥’ ‘→’ A)| as \mintinline{Agda}|⊤|, or one might
  interpret an equality type differently at each type, as in
  Observational Type Theory~\cite{Altenkirch:2007:OE:1292597.1292608}.

\section{This Paper}

 In this paper, we make extensive use of this trick for validating
 models.  In \autoref{sec:12-lines}, we formalize the simplest syntax
 that supports Lӧb's theorem and prove it sound relative to Agda in 12
 lines of code; the understanding is that this syntax could be
 extended to support basically anything you might want.  We then
 present in \autoref{sec:extended-trivial} an extended version of this
 solution, which supports enough operations that we can prove our
 syntax sound (consistent), incomplete, and nonempty.  In a hundred
 lines of code, we prove Lӧb's theorem in
 \autoref{sec:100-lines-quine} under the assumption that we are given
 a quine; this is basically the well-typed functional version of the
 program that uses \mintinline{Python}|open(__file__, 'r').read()|.
 After taking a digression for an application of Lӧb's theorem to the
 prisoner's dilemma in \autoref{sec:prisoner}, we sketch in
 \autoref{sec:only-add-quote} our implementation of Lӧb's theorem
 (code in the supplemental material) based on only the assumption that
 we can add a level of quotation to our syntax tree; this is the
 equivalent of letting the compiler implement
 \mintinline{Python}|repr|, rather than implementing it ourselves.  We
 close in \autoref{sec:future-work} with some discussion about avenues
 for removing the hard-coded \mintinline{Python}|repr|.

\section{Prior Work} \label{sec:prior-work-and-new}

 There exist a number of implementations or formalizations of various
 flavors of Lӧb's theorem in the literature.
 \Citeauthor{appel2007very} use Lӧb's theorem as an induction rule for
 program logics in Coq~\cite{appel2007very}.
 \Citeauthor{piponi-from-l-theorem-to-spreadsheet} formalizes a rule
 with the same shape as Lӧb's theorem in Haskell, and uses it for,
 among other things, spreadsheet
 evaluation~\cite{piponi-from-l-theorem-to-spreadsheet}.
 \Citeauthor{SimmonsToninho2012} formalize a constructive provability
 logic in Agda, and prove Lӧb's theorem within that
 logic~\cite{SimmonsToninho2012}.

 Gӧdel's incompleteness theorems, easy corollaries to Lӧb's theorem,
 have been formally verified numerous
 times~\cite{Shankar:1986:PM:913294,shankar1997metamathematics,DBLP:journals/corr/abs-cs-0505034,paulson2015mechanised}.

 To our knowledge, our twelve line proof is the shortest
 self-contained formally verified proof of the admissibility of Lӧb's
 theorem to date.  We are not aware of other formally verified proofs
 of Lӧb's theorem which interpret the modal □ operator as an
 inductively defined type of syntax trees of proofs of a given
 theorem, as we do in this formalization, though presumably the modal
 □ operator \citeauthor{SimmonsToninho2012} could be interpreted as
 such syntax trees.  Finally, we are not aware of other work which
 uses the trick of talking about a local interpretation function (as
 described at the end of \autoref{sec:local-interpretation}) to talk
 about consistent extensions to classes of encodings of type theory.

\section{Trivial Encoding} \label{sec:12-lines}
\AgdaHide{
  \begin{code}%
\>\AgdaKeyword{module} \AgdaModule{trivial-encoding} \AgdaKeyword{where}\<%
\\
\> \AgdaKeyword{infixr} \AgdaNumber{1} \AgdaFixityOp{\_‘→’\_}\<%
\\
\>[0]\AgdaIndent{2}{}\<[2]%
\>[2]\<%
\end{code}
}

 We begin with a language that supports almost nothing other than
 Lӧb's theorem.  We use \mintinline{Agda}|□ T| to denote the type of
 \mintinline{Agda}|Term|s of whose syntactic type is
 \mintinline{Agda}|T|.  We use \mintinline{Agda}|‘□’ T| to denote the
 syntactic type corresponding to the type of (syntactic) terms whose
 syntactic type is \mintinline{Agda}|T|.  For example, the type of a
 \mintinline{Python}|repr| which operated on syntax trees would be
 \mintinline{Agda}|□ T → □ (‘□’ T)|.

\begin{code}%
\> \AgdaKeyword{data} \AgdaDatatype{Type} \AgdaSymbol{:} \AgdaPrimitiveType{Set} \AgdaKeyword{where}\<%
\\
\>[2]\AgdaIndent{3}{}\<[3]%
\>[3]\AgdaInductiveConstructor{\_‘→’\_} \AgdaSymbol{:} \AgdaDatatype{Type} \AgdaSymbol{→} \AgdaDatatype{Type} \AgdaSymbol{→} \AgdaDatatype{Type}\<%
\\
\>[2]\AgdaIndent{3}{}\<[3]%
\>[3]\AgdaInductiveConstructor{‘□’} \AgdaSymbol{:} \AgdaDatatype{Type} \AgdaSymbol{→} \AgdaDatatype{Type}\<%
\\
%
\\
\> \AgdaKeyword{data} \AgdaDatatype{□} \AgdaSymbol{:} \AgdaDatatype{Type} \AgdaSymbol{→} \AgdaPrimitiveType{Set} \AgdaKeyword{where}\<%
\\
\>[2]\AgdaIndent{3}{}\<[3]%
\>[3]\AgdaInductiveConstructor{Lӧb} \AgdaSymbol{:} \AgdaSymbol{∀} \AgdaSymbol{\{}\AgdaBound{X}\AgdaSymbol{\}} \AgdaSymbol{→} \AgdaDatatype{□} \AgdaSymbol{(}\AgdaInductiveConstructor{‘□’} \AgdaBound{X} \AgdaInductiveConstructor{‘→’} \AgdaBound{X}\AgdaSymbol{)} \AgdaSymbol{→} \AgdaDatatype{□} \AgdaBound{X}\<%
\end{code}
 The only term supported by our term language is Lӧb's theorem.  We
 can prove this language consistent relative to Agda with an
 interpreter:

\begin{code}%
\> \AgdaFunction{⟦\_⟧ᵀ} \AgdaSymbol{:} \AgdaDatatype{Type} \AgdaSymbol{→} \AgdaPrimitiveType{Set}\<%
\\
\> \AgdaFunction{⟦} \AgdaBound{A} \AgdaInductiveConstructor{‘→’} \AgdaBound{B} \AgdaFunction{⟧ᵀ} \AgdaSymbol{=} \AgdaFunction{⟦} \AgdaBound{A} \AgdaFunction{⟧ᵀ} \AgdaSymbol{→} \AgdaFunction{⟦} \AgdaBound{B} \AgdaFunction{⟧ᵀ}\<%
\\
\> \AgdaFunction{⟦} \AgdaInductiveConstructor{‘□’} \AgdaBound{T} \AgdaFunction{⟧ᵀ} \<[14]%
\>[14]\AgdaSymbol{=} \AgdaDatatype{□} \AgdaBound{T}\<%
\\
%
\\
\> \AgdaFunction{⟦\_⟧ᵗ} \AgdaSymbol{:} \AgdaSymbol{∀} \AgdaSymbol{\{}\AgdaBound{T} \AgdaSymbol{:} \AgdaDatatype{Type}\AgdaSymbol{\}} \AgdaSymbol{→} \AgdaDatatype{□} \AgdaBound{T} \AgdaSymbol{→} \AgdaFunction{⟦} \AgdaBound{T} \AgdaFunction{⟧ᵀ}\<%
\\
\> \AgdaFunction{⟦} \AgdaInductiveConstructor{Lӧb} \AgdaBound{□‘X’→X} \AgdaFunction{⟧ᵗ} \AgdaSymbol{=} \AgdaFunction{⟦} \AgdaBound{□‘X’→X} \AgdaFunction{⟧ᵗ} \AgdaSymbol{(}\AgdaInductiveConstructor{Lӧb} \AgdaBound{□‘X’→X}\AgdaSymbol{)}\<%
\end{code}
 To interpret Lӧb's theorem applied to the syntax for a compiler $f$
 (\mintinline{Agda}|□‘X’→X| in the code above), we interpret $f$, and
 then apply this interpretation to the constructor
 \mintinline{Agda}|Lӧb| applied to $f$.

 Finally, we tie it all together:

\begin{code}%
\> \AgdaFunction{lӧb} \AgdaSymbol{:} \AgdaSymbol{∀} \AgdaSymbol{\{}\AgdaBound{‘X’}\AgdaSymbol{\}} \AgdaSymbol{→} \AgdaDatatype{□} \AgdaSymbol{(}\AgdaInductiveConstructor{‘□’} \AgdaBound{‘X’} \AgdaInductiveConstructor{‘→’} \AgdaBound{‘X’}\AgdaSymbol{)} \AgdaSymbol{→} \AgdaFunction{⟦} \AgdaBound{‘X’} \AgdaFunction{⟧ᵀ}\<%
\\
\> \AgdaFunction{lӧb} \AgdaBound{f} \AgdaSymbol{=} \AgdaFunction{⟦} \AgdaInductiveConstructor{Lӧb} \AgdaBound{f} \AgdaFunction{⟧ᵗ}\<%
\end{code}

 This code is deceptively short, with all of the interesting work
 happening in the interpretation of \mintinline{Agda}|Lӧb|.

 What have we actually proven, here?  It may seem as though we've
 proven absolutely nothing, or it may seem as though we've proven that
 Lӧb's theorem always holds.  Neither of these is the case.  The
 latter is ruled out, for example, by the existence of an
 self-interpreter for
 F$_\omega$~\cite{brown2016breaking}.\footnote{One may wonder how
 exactly the self-interpreter for F$_\omega$ does not contradict this
 theorem.  In private conversations with Matt Brown, we found that the
 F$_\omega$ self-interpreter does not have a separate syntax for
 types, instead indexing its terms over types in the metalanguage.
 This means that the type of Lӧb's theorem becomes either
 \mintinline{Agda}|□ (□ X → X) → □ X|, which is not strictly positive,
 or \mintinline{Agda}|□ (X → X) → □ X|, which, on interpretation, must
 be filled with a general fixpoint operator.  Such an operator is
 well-known to be inconsistent.}

 We have proven the following.  Suppose you have a formalization of
 type theory which has a syntax for types, and a syntax for terms
 indexed over those types.  If there is a ``local explanation'' for
 the system being sound, i.e., an interpretation function where each
 rule does not need to know about the full list of constructors, then
 it is consistent to add a constructor for Lӧb's theorem to your
 syntax.  This means that it is impossible to contradict Lӧb's theorem
 no matter what (consistent) constructors you add.  We will see in the
 next section how this gives incompleteness, and discuss in later
 sections how to \emph{prove Lӧb's theorem}, rather than simply
 proving that it is consistent to assume.

\section{Encoding with Soundness, Incompleteness, and Non-Emptiness} \label{sec:extended-trivial}

 By augmenting our representation with top (\mintinline{Agda}|‘⊤’|)
 and bottom (\mintinline{Agda}|‘⊥’|) types, and a unique inhabitant of
 \mintinline{Agda}|‘⊤’|, we can prove soundness, incompleteness, and
 non-emptiness.

\AgdaHide{
  \begin{code}%
\>\AgdaKeyword{module} \AgdaModule{sound-incomplete-nonempty} \AgdaKeyword{where}\<%
\\
\> \AgdaKeyword{infixr} \AgdaNumber{1} \AgdaFixityOp{\_‘→’\_}\<%
\\
\>[0]\AgdaIndent{2}{}\<[2]%
\>[2]\<%
\end{code}
}

\begin{code}%
\> \AgdaKeyword{data} \AgdaDatatype{Type} \AgdaSymbol{:} \AgdaPrimitiveType{Set} \AgdaKeyword{where}\<%
\\
\>[0]\AgdaIndent{2}{}\<[2]%
\>[2]\AgdaInductiveConstructor{\_‘→’\_} \AgdaSymbol{:} \AgdaDatatype{Type} \AgdaSymbol{→} \AgdaDatatype{Type} \AgdaSymbol{→} \AgdaDatatype{Type}\<%
\\
\>[0]\AgdaIndent{2}{}\<[2]%
\>[2]\AgdaInductiveConstructor{‘□’} \AgdaSymbol{:} \AgdaDatatype{Type} \AgdaSymbol{→} \AgdaDatatype{Type}\<%
\\
\>[0]\AgdaIndent{2}{}\<[2]%
\>[2]\AgdaInductiveConstructor{‘⊤’} \AgdaSymbol{:} \AgdaDatatype{Type}\<%
\\
\>[0]\AgdaIndent{2}{}\<[2]%
\>[2]\AgdaInductiveConstructor{‘⊥’} \AgdaSymbol{:} \AgdaDatatype{Type}\<%
\\
%
\\
\> \AgdaComment{---- "□" is sometimes written as "Term"}\<%
\\
\> \AgdaKeyword{data} \AgdaDatatype{□} \AgdaSymbol{:} \AgdaDatatype{Type} \AgdaSymbol{→} \AgdaPrimitiveType{Set} \AgdaKeyword{where}\<%
\\
\>[0]\AgdaIndent{2}{}\<[2]%
\>[2]\AgdaInductiveConstructor{Lӧb} \AgdaSymbol{:} \AgdaSymbol{∀} \AgdaSymbol{\{}\AgdaBound{X}\AgdaSymbol{\}} \AgdaSymbol{→} \AgdaDatatype{□} \AgdaSymbol{(}\AgdaInductiveConstructor{‘□’} \AgdaBound{X} \AgdaInductiveConstructor{‘→’} \AgdaBound{X}\AgdaSymbol{)} \AgdaSymbol{→} \AgdaDatatype{□} \AgdaBound{X}\<%
\\
\>[0]\AgdaIndent{2}{}\<[2]%
\>[2]\AgdaInductiveConstructor{‘tt’} \AgdaSymbol{:} \AgdaDatatype{□} \AgdaInductiveConstructor{‘⊤’}\<%
\\
%
\\
\> \AgdaFunction{⟦\_⟧ᵀ} \AgdaSymbol{:} \AgdaDatatype{Type} \AgdaSymbol{→} \AgdaPrimitiveType{Set}\<%
\\
\> \AgdaFunction{⟦} \AgdaBound{A} \AgdaInductiveConstructor{‘→’} \AgdaBound{B} \AgdaFunction{⟧ᵀ} \AgdaSymbol{=} \AgdaFunction{⟦} \AgdaBound{A} \AgdaFunction{⟧ᵀ} \AgdaSymbol{→} \AgdaFunction{⟦} \AgdaBound{B} \AgdaFunction{⟧ᵀ}\<%
\\
\> \AgdaFunction{⟦} \AgdaInductiveConstructor{‘□’} \AgdaBound{T} \AgdaFunction{⟧ᵀ} \<[14]%
\>[14]\AgdaSymbol{=} \AgdaDatatype{□} \AgdaBound{T}\<%
\\
\> \AgdaFunction{⟦} \AgdaInductiveConstructor{‘⊤’} \AgdaFunction{⟧ᵀ} \<[14]%
\>[14]\AgdaSymbol{=} \AgdaRecord{⊤}\<%
\\
\> \AgdaFunction{⟦} \AgdaInductiveConstructor{‘⊥’} \AgdaFunction{⟧ᵀ} \<[14]%
\>[14]\AgdaSymbol{=} \AgdaDatatype{⊥}\<%
\\
%
\\
\> \AgdaFunction{⟦\_⟧ᵗ} \AgdaSymbol{:} \AgdaSymbol{∀} \AgdaSymbol{\{}\AgdaBound{T} \AgdaSymbol{:} \AgdaDatatype{Type}\AgdaSymbol{\}} \AgdaSymbol{→} \AgdaDatatype{□} \AgdaBound{T} \AgdaSymbol{→} \AgdaFunction{⟦} \AgdaBound{T} \AgdaFunction{⟧ᵀ}\<%
\\
\> \AgdaFunction{⟦} \AgdaInductiveConstructor{Lӧb} \AgdaBound{□‘X’→X} \AgdaFunction{⟧ᵗ} \AgdaSymbol{=} \AgdaFunction{⟦} \AgdaBound{□‘X’→X} \AgdaFunction{⟧ᵗ} \AgdaSymbol{(}\AgdaInductiveConstructor{Lӧb} \AgdaBound{□‘X’→X}\AgdaSymbol{)}\<%
\\
\> \AgdaFunction{⟦} \AgdaInductiveConstructor{‘tt’} \AgdaFunction{⟧ᵗ} \AgdaSymbol{=} \AgdaInductiveConstructor{tt}\<%
\\
%
\\
\> \AgdaFunction{‘¬’\_} \AgdaSymbol{:} \AgdaDatatype{Type} \AgdaSymbol{→} \AgdaDatatype{Type}\<%
\\
\> \AgdaFunction{‘¬’} \AgdaBound{T} \AgdaSymbol{=} \AgdaBound{T} \AgdaInductiveConstructor{‘→’} \AgdaInductiveConstructor{‘⊥’}\<%
\\
%
\\
\> \AgdaFunction{lӧb} \AgdaSymbol{:} \AgdaSymbol{∀} \AgdaSymbol{\{}\AgdaBound{‘X’}\AgdaSymbol{\}} \AgdaSymbol{→} \AgdaDatatype{□} \AgdaSymbol{(}\AgdaInductiveConstructor{‘□’} \AgdaBound{‘X’} \AgdaInductiveConstructor{‘→’} \AgdaBound{‘X’}\AgdaSymbol{)} \AgdaSymbol{→} \AgdaFunction{⟦} \AgdaBound{‘X’} \AgdaFunction{⟧ᵀ}\<%
\\
\> \AgdaFunction{lӧb} \AgdaBound{f} \AgdaSymbol{=} \AgdaFunction{⟦} \AgdaInductiveConstructor{Lӧb} \AgdaBound{f} \AgdaFunction{⟧ᵗ}\<%
\\
%
\\
\> \AgdaComment{---- There is no syntactic proof of absurdity}\<%
\\
\> \AgdaFunction{soundness} \AgdaSymbol{:} \AgdaFunction{¬} \AgdaDatatype{□} \AgdaInductiveConstructor{‘⊥’}\<%
\\
\> \AgdaFunction{soundness} \AgdaBound{x} \AgdaSymbol{=} \AgdaFunction{⟦} \AgdaBound{x} \AgdaFunction{⟧ᵗ}\<%
\\
%
\\
\> \AgdaComment{---- but it would be absurd to have a syntactic}\<%
\\
\> \AgdaComment{---- proof of that fact}\<%
\\
\> \AgdaFunction{incompleteness} \AgdaSymbol{:} \AgdaFunction{¬} \AgdaDatatype{□} \AgdaSymbol{(}\AgdaFunction{‘¬’} \AgdaSymbol{(}\AgdaInductiveConstructor{‘□’} \AgdaInductiveConstructor{‘⊥’}\AgdaSymbol{))}\<%
\\
\> \AgdaFunction{incompleteness} \AgdaSymbol{=} \AgdaFunction{lӧb}\<%
\\
%
\\
\> \AgdaComment{---- However, there are syntactic proofs of some}\<%
\\
\> \AgdaComment{---- things (namely ⊤)}\<%
\\
\> \AgdaFunction{non-emptiness} \AgdaSymbol{:} \AgdaDatatype{□} \AgdaInductiveConstructor{‘⊤’}\<%
\\
\> \AgdaFunction{non-emptiness} \AgdaSymbol{=} \AgdaInductiveConstructor{‘tt’}\<%
\\
%
\\
\> \AgdaComment{---- There are no syntactic interpreters, things}\<%
\\
\> \AgdaComment{---- which, at any type, evaluate code at that}\<%
\\
\> \AgdaComment{---- type to produce its result.}\<%
\\
\> \AgdaFunction{no-interpreters} \AgdaSymbol{:} \AgdaFunction{¬} \AgdaSymbol{(∀} \AgdaSymbol{\{}\AgdaBound{‘X’}\AgdaSymbol{\}} \AgdaSymbol{→} \AgdaDatatype{□} \AgdaSymbol{(}\AgdaInductiveConstructor{‘□’} \AgdaBound{‘X’} \AgdaInductiveConstructor{‘→’} \AgdaBound{‘X’}\AgdaSymbol{))}\<%
\\
\> \AgdaFunction{no-interpreters} \AgdaBound{interp} \AgdaSymbol{=} \AgdaFunction{lӧb} \AgdaSymbol{(}\AgdaBound{interp} \AgdaSymbol{\{}\AgdaInductiveConstructor{‘⊥’}\AgdaSymbol{\})}\<%
\end{code}

  What is this incompleteness theorem?  Gӧdel's incompleteness theorem
  is typically interpreted as ``there exist true but unprovable
  statements.''  In intuitionistic logic, this is hardly surprising.
  A more accurate rendition of the theorem in Agda might be ``there
  exist true but inadmissible statements,'' i.e., there are statements
  which are provable meta-theoretically, but which lead to
  (meta-theoretically--provable) inconsistency if assumed at the
  object level.

  This may seem a bit esoteric, so let's back up a bit, and make it
  more concrete.  Let's begin by banishing ``truth''.  Sometimes it is
  useful to formalize a notion of provability.  For example, you might
  want to show neither assuming $T$ nor assuming $¬T$ yields a proof
  of contradiction.  You cannot phrase this is $¬T ∧ ¬¬T$, for that is
  absurd.  Instead, you want to say something like $(¬□T) ∧ ¬□(¬T)$,
  i.e., it would be absurd to have a proof object of either $T$ or of
  $¬T$.  After a while, you might find that meta-programming in this
  formal syntax is nice, and you might want it to be able to formalize
  every proof, so that you can do all of your solving reflectively.
  If you're like us, you might even want to reason about the
  reflective tactics themselves in a reflective manner; you'd want to
  be able to add levels of quotation to quoted things to talk about
  such tactics.

  The incompleteness theorem, then, is this: your reflective system,
  no matter how powerful, cannot formalize every proof.  For any fixed
  language of syntactic proofs which is powerful enough to represent
  itself, there will always be some valid proofs that you cannot
  reflect into your syntax.  In particular, you might be able to prove
  that your syntax has no proofs of ⊥ (by interpreting any such
  proof).  But you'll be unable to quote that proof.  This is what the
  incompleteness theorem stated above says.  Incompleteness,
  fundamentally, is a result about the limitations of formalizing
  provability.

\section{Encoding with Quines} \label{sec:100-lines-quine}

 \AgdaHide{
  \begin{code}%
\>\AgdaKeyword{module} \AgdaModule{lob-by-quines} \AgdaKeyword{where}\<%
\end{code}
}

 We now weaken our assumptions further.  Rather than assuming Lӧb's
 theorem, we instead assume only a type-level quine in our
 representation.  Recall that a \emph{quine} is a program that outputs
 some function of its own source code.  A \emph{type-level quine at ϕ}
 is program that outputs the result of evaluating the function ϕ on
 the abstract syntax tree of its own type.  Letting
 \mintinline{Agda}|Quine ϕ| denote the constructor for a type-level
 quine at ϕ, we have an isomorphism between \mintinline{Agda}|Quine ϕ|
 and \mintinline{Agda}|ϕ ⌜ Quine ϕ ⌝ᵀ|, where
 \mintinline{Agda}|⌜ Quine ϕ ⌝ᵀ| is the abstract syntax tree for the
 source code of \mintinline{Agda}|Quine ϕ|.  Note that we assume
 constructors for ``adding a level of quotation'', turning abstract
 syntax trees for programs of type $T$ into abstract syntax trees for
 abstract syntax trees for programs of type $T$; this corresponds to
 \mintinline{Python}|repr|.

\AgdaHide{
\begin{code}%
\> \AgdaKeyword{infixl} \AgdaNumber{3} \AgdaFixityOp{\_‘’ₐ\_}\<%
\\
\> \AgdaKeyword{infixl} \AgdaNumber{3} \AgdaFixityOp{\_w‘‘’’ₐ\_}\<%
\\
\> \AgdaKeyword{infixl} \AgdaNumber{3} \AgdaFixityOp{\_‘’\_}\<%
\\
\> \AgdaKeyword{infixl} \AgdaNumber{2} \AgdaFixityOp{\_▻\_}\<%
\\
\> \AgdaKeyword{infixr} \AgdaNumber{2} \AgdaFixityOp{\_‘∘’\_}\<%
\\
\> \AgdaKeyword{infixr} \AgdaNumber{1} \AgdaFixityOp{\_‘→’\_}\<%
\\
\>\<%
\end{code}
}

 We begin with an encoding of contexts and types, repeating from above
 the constructors of ‘→’, ‘□’, ‘⊤’, and ‘⊥’.  We add to this a
 constructor for quines (\mintinline{Agda}|Quine|), and a constructor
 for syntax trees of types in the empty context (‘Typeε’).  Finally,
 rather than proving weakening and substitution as mutually recursive
 definitions, we take the easier but more verbose route of adding
 constructors that allow adding and substituting extra terms in the
 context. Note that ‘□’ is now a function of the represented language,
 rather than a meta-level operator.

% \todo{Does this need more explanation?}
% \todo{\cite{mcbride2010outrageous}}

 Note that we use the infix operator \mintinline{Agda}|_‘’_| to denote
 substitution.

\begin{code}%
\> \AgdaKeyword{mutual}\<%
\\
\>[0]\AgdaIndent{2}{}\<[2]%
\>[2]\AgdaKeyword{data} \AgdaDatatype{Context} \AgdaSymbol{:} \AgdaPrimitiveType{Set} \AgdaKeyword{where}\<%
\\
\>[2]\AgdaIndent{3}{}\<[3]%
\>[3]\AgdaInductiveConstructor{ε} \AgdaSymbol{:} \AgdaDatatype{Context}\<%
\\
\>[2]\AgdaIndent{3}{}\<[3]%
\>[3]\AgdaInductiveConstructor{\_▻\_} \AgdaSymbol{:} \AgdaSymbol{(}\AgdaBound{Γ} \AgdaSymbol{:} \AgdaDatatype{Context}\AgdaSymbol{)} \AgdaSymbol{→} \AgdaDatatype{Type} \AgdaBound{Γ} \AgdaSymbol{→} \AgdaDatatype{Context}\<%
\\
%
\\
\>[0]\AgdaIndent{2}{}\<[2]%
\>[2]\AgdaKeyword{data} \AgdaDatatype{Type} \AgdaSymbol{:} \AgdaDatatype{Context} \AgdaSymbol{→} \AgdaPrimitiveType{Set} \AgdaKeyword{where}\<%
\\
\>[2]\AgdaIndent{3}{}\<[3]%
\>[3]\AgdaInductiveConstructor{\_‘→’\_} \AgdaSymbol{:} \AgdaSymbol{∀} \AgdaSymbol{\{}\AgdaBound{Γ}\AgdaSymbol{\}} \AgdaSymbol{→} \AgdaDatatype{Type} \AgdaBound{Γ} \AgdaSymbol{→} \AgdaDatatype{Type} \AgdaBound{Γ} \AgdaSymbol{→} \AgdaDatatype{Type} \AgdaBound{Γ}\<%
\\
\>[2]\AgdaIndent{3}{}\<[3]%
\>[3]\AgdaInductiveConstructor{‘⊤’} \AgdaSymbol{:} \AgdaSymbol{∀} \AgdaSymbol{\{}\AgdaBound{Γ}\AgdaSymbol{\}} \AgdaSymbol{→} \AgdaDatatype{Type} \AgdaBound{Γ}\<%
\\
\>[2]\AgdaIndent{3}{}\<[3]%
\>[3]\AgdaInductiveConstructor{‘⊥’} \AgdaSymbol{:} \AgdaSymbol{∀} \AgdaSymbol{\{}\AgdaBound{Γ}\AgdaSymbol{\}} \AgdaSymbol{→} \AgdaDatatype{Type} \AgdaBound{Γ}\<%
\\
\>[2]\AgdaIndent{3}{}\<[3]%
\>[3]\AgdaInductiveConstructor{‘Typeε’} \AgdaSymbol{:} \AgdaSymbol{∀} \AgdaSymbol{\{}\AgdaBound{Γ}\AgdaSymbol{\}} \AgdaSymbol{→} \AgdaDatatype{Type} \AgdaBound{Γ}\<%
\\
\>[2]\AgdaIndent{3}{}\<[3]%
\>[3]\AgdaInductiveConstructor{‘□’} \AgdaSymbol{:} \AgdaSymbol{∀} \AgdaSymbol{\{}\AgdaBound{Γ}\AgdaSymbol{\}} \AgdaSymbol{→} \AgdaDatatype{Type} \AgdaSymbol{(}\AgdaBound{Γ} \AgdaInductiveConstructor{▻} \AgdaInductiveConstructor{‘Typeε’}\AgdaSymbol{)}\<%
\\
\>[2]\AgdaIndent{3}{}\<[3]%
\>[3]\AgdaInductiveConstructor{Quine} \AgdaSymbol{:} \AgdaDatatype{Type} \AgdaSymbol{(}\AgdaInductiveConstructor{ε} \AgdaInductiveConstructor{▻} \AgdaInductiveConstructor{‘Typeε’}\AgdaSymbol{)} \AgdaSymbol{→} \AgdaDatatype{Type} \AgdaInductiveConstructor{ε}\<%
\\
\>[2]\AgdaIndent{3}{}\<[3]%
\>[3]\AgdaInductiveConstructor{W} \AgdaSymbol{:} \AgdaSymbol{∀} \AgdaSymbol{\{}\AgdaBound{Γ} \AgdaBound{A}\AgdaSymbol{\}}\<%
\\
\>[3]\AgdaIndent{5}{}\<[5]%
\>[5]\AgdaSymbol{→} \AgdaDatatype{Type} \AgdaBound{Γ} \AgdaSymbol{→} \AgdaDatatype{Type} \AgdaSymbol{(}\AgdaBound{Γ} \AgdaInductiveConstructor{▻} \AgdaBound{A}\AgdaSymbol{)}\<%
\\
\>[0]\AgdaIndent{3}{}\<[3]%
\>[3]\AgdaInductiveConstructor{W₁} \AgdaSymbol{:} \AgdaSymbol{∀} \AgdaSymbol{\{}\AgdaBound{Γ} \AgdaBound{A} \AgdaBound{B}\AgdaSymbol{\}}\<%
\\
\>[3]\AgdaIndent{5}{}\<[5]%
\>[5]\AgdaSymbol{→} \AgdaDatatype{Type} \AgdaSymbol{(}\AgdaBound{Γ} \AgdaInductiveConstructor{▻} \AgdaBound{B}\AgdaSymbol{)} \AgdaSymbol{→} \AgdaDatatype{Type} \AgdaSymbol{(}\AgdaBound{Γ} \AgdaInductiveConstructor{▻} \AgdaBound{A} \AgdaInductiveConstructor{▻} \AgdaSymbol{(}\AgdaInductiveConstructor{W} \AgdaBound{B}\AgdaSymbol{))}\<%
\\
\>[0]\AgdaIndent{3}{}\<[3]%
\>[3]\AgdaInductiveConstructor{\_‘’\_} \AgdaSymbol{:} \AgdaSymbol{∀} \AgdaSymbol{\{}\AgdaBound{Γ} \AgdaBound{A}\AgdaSymbol{\}}\<%
\\
\>[3]\AgdaIndent{5}{}\<[5]%
\>[5]\AgdaSymbol{→} \AgdaDatatype{Type} \AgdaSymbol{(}\AgdaBound{Γ} \AgdaInductiveConstructor{▻} \AgdaBound{A}\AgdaSymbol{)} \AgdaSymbol{→} \AgdaDatatype{Term} \AgdaBound{A} \AgdaSymbol{→} \AgdaDatatype{Type} \AgdaBound{Γ}\<%
\end{code}

  In addition to ‘λ’ and ‘tt’, we now have the AST-equivalents of
  Python's \mintinline{Python}|repr|, which we denote as
  \mintinline{Agda}|⌜_⌝ᵀ| for the type-level add-quote function, and
  \mintinline{Agda}|⌜_⌝ᵗ| for the term-level add-quote function.  We
  add constructors \mintinline{Agda}|quine→| and
  \mintinline{Agda}|quine←| that exhibit the isomorphism that defines
  our type-level quine constructor, though we elide a constructor
  declaring that these are inverses, as we found it unnecessary.

  To construct the proof of Lӧb's theorem, we need a few other
  standard constructors, such as \mintinline{Agda}|‘VAR₀’|, which
  references a term in the context; \mintinline{Agda}|_‘’ₐ_|, which we
  use to denote function application; \mintinline{Agda}|_‘∘’_|, a
  function composition operator; and \mintinline{Agda}|‘⌜‘VAR₀’⌝ᵗ’|,
  the variant of \mintinline{Agda}|‘VAR₀’| which adds an extra level
  of syntax-trees. We also include a number of constructors that
  handle weakening and substitution; this allows us to avoid both
  inductive-recursive definitions of weakening and substitution, and
  avoid defining a judgmental equality or conversion relation.

\begin{code}%
\>[0]\AgdaIndent{2}{}\<[2]%
\>[2]\AgdaKeyword{data} \AgdaDatatype{Term} \AgdaSymbol{:} \AgdaSymbol{\{}\AgdaBound{Γ} \AgdaSymbol{:} \AgdaDatatype{Context}\AgdaSymbol{\}} \AgdaSymbol{→} \AgdaDatatype{Type} \AgdaBound{Γ} \AgdaSymbol{→} \AgdaPrimitiveType{Set} \AgdaKeyword{where}\<%
\\
\>[2]\AgdaIndent{3}{}\<[3]%
\>[3]\AgdaInductiveConstructor{‘λ’} \AgdaSymbol{:} \AgdaSymbol{∀} \AgdaSymbol{\{}\AgdaBound{Γ} \AgdaBound{A} \AgdaBound{B}\AgdaSymbol{\}}\<%
\\
\>[3]\AgdaIndent{5}{}\<[5]%
\>[5]\AgdaSymbol{→} \AgdaDatatype{Term} \AgdaSymbol{\{}\AgdaBound{Γ} \AgdaInductiveConstructor{▻} \AgdaBound{A}\AgdaSymbol{\}} \AgdaSymbol{(}\AgdaInductiveConstructor{W} \AgdaBound{B}\AgdaSymbol{)} \AgdaSymbol{→} \AgdaDatatype{Term} \AgdaSymbol{(}\AgdaBound{A} \AgdaInductiveConstructor{‘→’} \AgdaBound{B}\AgdaSymbol{)}\<%
\\
\>[0]\AgdaIndent{3}{}\<[3]%
\>[3]\AgdaInductiveConstructor{‘tt’} \AgdaSymbol{:} \AgdaSymbol{∀} \AgdaSymbol{\{}\AgdaBound{Γ}\AgdaSymbol{\}}\<%
\\
\>[3]\AgdaIndent{5}{}\<[5]%
\>[5]\AgdaSymbol{→} \AgdaDatatype{Term} \AgdaSymbol{\{}\AgdaBound{Γ}\AgdaSymbol{\}} \AgdaInductiveConstructor{‘⊤’}\<%
\\
\>[0]\AgdaIndent{3}{}\<[3]%
\>[3]\AgdaInductiveConstructor{⌜\_⌝ᵀ} \AgdaSymbol{:} \AgdaSymbol{∀} \AgdaSymbol{\{}\AgdaBound{Γ}\AgdaSymbol{\}} \AgdaComment{---- type-level repr}\<%
\\
\>[3]\AgdaIndent{5}{}\<[5]%
\>[5]\AgdaSymbol{→} \AgdaDatatype{Type} \AgdaInductiveConstructor{ε}\<%
\\
\>[3]\AgdaIndent{5}{}\<[5]%
\>[5]\AgdaSymbol{→} \AgdaDatatype{Term} \AgdaSymbol{\{}\AgdaBound{Γ}\AgdaSymbol{\}} \AgdaInductiveConstructor{‘Typeε’}\<%
\\
\>[0]\AgdaIndent{3}{}\<[3]%
\>[3]\AgdaInductiveConstructor{⌜\_⌝ᵗ} \AgdaSymbol{:} \AgdaSymbol{∀} \AgdaSymbol{\{}\AgdaBound{Γ} \AgdaBound{T}\AgdaSymbol{\}} \AgdaComment{---- term-level repr}\<%
\\
\>[3]\AgdaIndent{5}{}\<[5]%
\>[5]\AgdaSymbol{→} \AgdaDatatype{Term} \AgdaSymbol{\{}\AgdaInductiveConstructor{ε}\AgdaSymbol{\}} \AgdaBound{T}\<%
\\
\>[3]\AgdaIndent{5}{}\<[5]%
\>[5]\AgdaSymbol{→} \AgdaDatatype{Term} \AgdaSymbol{\{}\AgdaBound{Γ}\AgdaSymbol{\}} \AgdaSymbol{(}\AgdaInductiveConstructor{‘□’} \AgdaInductiveConstructor{‘’} \AgdaInductiveConstructor{⌜} \AgdaBound{T} \AgdaInductiveConstructor{⌝ᵀ}\AgdaSymbol{)}\<%
\\
\>[0]\AgdaIndent{3}{}\<[3]%
\>[3]\AgdaInductiveConstructor{quine→} \AgdaSymbol{:} \AgdaSymbol{∀} \AgdaSymbol{\{}\AgdaBound{ϕ}\AgdaSymbol{\}}\<%
\\
\>[3]\AgdaIndent{5}{}\<[5]%
\>[5]\AgdaSymbol{→} \AgdaDatatype{Term} \AgdaSymbol{\{}\AgdaInductiveConstructor{ε}\AgdaSymbol{\}} \AgdaSymbol{(}\AgdaInductiveConstructor{Quine} \AgdaBound{ϕ} \<[35]%
\>[35]\AgdaInductiveConstructor{‘→’} \AgdaBound{ϕ} \AgdaInductiveConstructor{‘’} \AgdaInductiveConstructor{⌜} \AgdaInductiveConstructor{Quine} \AgdaBound{ϕ} \AgdaInductiveConstructor{⌝ᵀ}\AgdaSymbol{)}\<%
\\
\>[0]\AgdaIndent{3}{}\<[3]%
\>[3]\AgdaInductiveConstructor{quine←} \AgdaSymbol{:} \AgdaSymbol{∀} \AgdaSymbol{\{}\AgdaBound{ϕ}\AgdaSymbol{\}}\<%
\\
\>[3]\AgdaIndent{5}{}\<[5]%
\>[5]\AgdaSymbol{→} \AgdaDatatype{Term} \AgdaSymbol{\{}\AgdaInductiveConstructor{ε}\AgdaSymbol{\}} \AgdaSymbol{(}\AgdaBound{ϕ} \AgdaInductiveConstructor{‘’} \AgdaInductiveConstructor{⌜} \AgdaInductiveConstructor{Quine} \AgdaBound{ϕ} \AgdaInductiveConstructor{⌝ᵀ} \AgdaInductiveConstructor{‘→’} \AgdaInductiveConstructor{Quine} \AgdaBound{ϕ}\AgdaSymbol{)}\<%
\\
\>[0]\AgdaIndent{3}{}\<[3]%
\>[3]\AgdaComment{---- The constructors below here are for}\<%
\\
\>[0]\AgdaIndent{3}{}\<[3]%
\>[3]\AgdaComment{---- variables, weakening, and substitution}\<%
\\
\>[0]\AgdaIndent{3}{}\<[3]%
\>[3]\AgdaInductiveConstructor{‘VAR₀’} \AgdaSymbol{:} \AgdaSymbol{∀} \AgdaSymbol{\{}\AgdaBound{Γ} \AgdaBound{T}\AgdaSymbol{\}}\<%
\\
\>[3]\AgdaIndent{5}{}\<[5]%
\>[5]\AgdaSymbol{→} \AgdaDatatype{Term} \AgdaSymbol{\{}\AgdaBound{Γ} \AgdaInductiveConstructor{▻} \AgdaBound{T}\AgdaSymbol{\}} \AgdaSymbol{(}\AgdaInductiveConstructor{W} \AgdaBound{T}\AgdaSymbol{)}\<%
\\
\>[0]\AgdaIndent{3}{}\<[3]%
\>[3]\AgdaInductiveConstructor{\_‘’ₐ\_} \AgdaSymbol{:} \AgdaSymbol{∀} \AgdaSymbol{\{}\AgdaBound{Γ} \AgdaBound{A} \AgdaBound{B}\AgdaSymbol{\}}\<%
\\
\>[3]\AgdaIndent{4}{}\<[4]%
\>[4]\AgdaSymbol{→} \AgdaDatatype{Term} \AgdaSymbol{\{}\AgdaBound{Γ}\AgdaSymbol{\}} \AgdaSymbol{(}\AgdaBound{A} \AgdaInductiveConstructor{‘→’} \AgdaBound{B}\AgdaSymbol{)}\<%
\\
\>[3]\AgdaIndent{4}{}\<[4]%
\>[4]\AgdaSymbol{→} \AgdaDatatype{Term} \AgdaSymbol{\{}\AgdaBound{Γ}\AgdaSymbol{\}} \AgdaBound{A}\<%
\\
\>[3]\AgdaIndent{4}{}\<[4]%
\>[4]\AgdaSymbol{→} \AgdaDatatype{Term} \AgdaSymbol{\{}\AgdaBound{Γ}\AgdaSymbol{\}} \AgdaBound{B}\<%
\\
\>[0]\AgdaIndent{3}{}\<[3]%
\>[3]\AgdaInductiveConstructor{\_‘∘’\_} \AgdaSymbol{:} \AgdaSymbol{∀} \AgdaSymbol{\{}\AgdaBound{Γ} \AgdaBound{A} \AgdaBound{B} \AgdaBound{C}\AgdaSymbol{\}}\<%
\\
\>[3]\AgdaIndent{4}{}\<[4]%
\>[4]\AgdaSymbol{→} \AgdaDatatype{Term} \AgdaSymbol{\{}\AgdaBound{Γ}\AgdaSymbol{\}} \AgdaSymbol{(}\AgdaBound{B} \AgdaInductiveConstructor{‘→’} \AgdaBound{C}\AgdaSymbol{)}\<%
\\
\>[3]\AgdaIndent{4}{}\<[4]%
\>[4]\AgdaSymbol{→} \AgdaDatatype{Term} \AgdaSymbol{\{}\AgdaBound{Γ}\AgdaSymbol{\}} \AgdaSymbol{(}\AgdaBound{A} \AgdaInductiveConstructor{‘→’} \AgdaBound{B}\AgdaSymbol{)}\<%
\\
\>[3]\AgdaIndent{4}{}\<[4]%
\>[4]\AgdaSymbol{→} \AgdaDatatype{Term} \AgdaSymbol{\{}\AgdaBound{Γ}\AgdaSymbol{\}} \AgdaSymbol{(}\AgdaBound{A} \AgdaInductiveConstructor{‘→’} \AgdaBound{C}\AgdaSymbol{)}\<%
\\
\>[0]\AgdaIndent{3}{}\<[3]%
\>[3]\AgdaInductiveConstructor{‘⌜‘VAR₀’⌝ᵗ’} \AgdaSymbol{:} \AgdaSymbol{∀} \AgdaSymbol{\{}\AgdaBound{T}\AgdaSymbol{\}}\<%
\\
\>[3]\AgdaIndent{5}{}\<[5]%
\>[5]\AgdaSymbol{→} \AgdaDatatype{Term} \AgdaSymbol{\{}\AgdaInductiveConstructor{ε} \AgdaInductiveConstructor{▻} \AgdaInductiveConstructor{‘□’} \AgdaInductiveConstructor{‘’} \AgdaInductiveConstructor{⌜} \AgdaBound{T} \AgdaInductiveConstructor{⌝ᵀ}\AgdaSymbol{\}}\<%
\\
\>[5]\AgdaIndent{12}{}\<[12]%
\>[12]\AgdaSymbol{(}\AgdaInductiveConstructor{W} \AgdaSymbol{(}\AgdaInductiveConstructor{‘□’} \AgdaInductiveConstructor{‘’} \AgdaInductiveConstructor{⌜} \AgdaInductiveConstructor{‘□’} \AgdaInductiveConstructor{‘’} \AgdaInductiveConstructor{⌜} \AgdaBound{T} \AgdaInductiveConstructor{⌝ᵀ} \AgdaInductiveConstructor{⌝ᵀ}\AgdaSymbol{))}\<%
\\
\>[0]\AgdaIndent{3}{}\<[3]%
\>[3]\AgdaInductiveConstructor{→SW₁SV→W} \AgdaSymbol{:} \AgdaSymbol{∀} \AgdaSymbol{\{}\AgdaBound{Γ} \AgdaBound{T} \AgdaBound{X} \AgdaBound{A} \AgdaBound{B}\AgdaSymbol{\}} \AgdaSymbol{\{}\AgdaBound{x} \AgdaSymbol{:} \AgdaDatatype{Term} \AgdaSymbol{\{}\AgdaBound{Γ}\AgdaSymbol{\}} \AgdaBound{X}\AgdaSymbol{\}}\<%
\\
\>[3]\AgdaIndent{5}{}\<[5]%
\>[5]\AgdaSymbol{→} \AgdaDatatype{Term} \AgdaSymbol{(}\AgdaBound{T} \AgdaInductiveConstructor{‘→’} \AgdaSymbol{(}\AgdaInductiveConstructor{W₁} \AgdaBound{A} \AgdaInductiveConstructor{‘’} \AgdaInductiveConstructor{‘VAR₀’} \AgdaInductiveConstructor{‘→’} \AgdaInductiveConstructor{W} \AgdaBound{B}\AgdaSymbol{)} \AgdaInductiveConstructor{‘’} \AgdaBound{x}\AgdaSymbol{)}\<%
\\
\>[3]\AgdaIndent{5}{}\<[5]%
\>[5]\AgdaSymbol{→} \AgdaDatatype{Term} \AgdaSymbol{(}\AgdaBound{T} \AgdaInductiveConstructor{‘→’} \AgdaBound{A} \AgdaInductiveConstructor{‘’} \AgdaBound{x} \AgdaInductiveConstructor{‘→’} \AgdaBound{B}\AgdaSymbol{)}\<%
\\
\>[0]\AgdaIndent{3}{}\<[3]%
\>[3]\AgdaInductiveConstructor{←SW₁SV→W} \AgdaSymbol{:} \AgdaSymbol{∀} \AgdaSymbol{\{}\AgdaBound{Γ} \AgdaBound{T} \AgdaBound{X} \AgdaBound{A} \AgdaBound{B}\AgdaSymbol{\}} \AgdaSymbol{\{}\AgdaBound{x} \AgdaSymbol{:} \AgdaDatatype{Term} \AgdaSymbol{\{}\AgdaBound{Γ}\AgdaSymbol{\}} \AgdaBound{X}\AgdaSymbol{\}}\<%
\\
\>[3]\AgdaIndent{5}{}\<[5]%
\>[5]\AgdaSymbol{→} \AgdaDatatype{Term} \AgdaSymbol{((}\AgdaInductiveConstructor{W₁} \AgdaBound{A} \AgdaInductiveConstructor{‘’} \AgdaInductiveConstructor{‘VAR₀’} \AgdaInductiveConstructor{‘→’} \AgdaInductiveConstructor{W} \AgdaBound{B}\AgdaSymbol{)} \AgdaInductiveConstructor{‘’} \AgdaBound{x} \AgdaInductiveConstructor{‘→’} \AgdaBound{T}\AgdaSymbol{)}\<%
\\
\>[3]\AgdaIndent{5}{}\<[5]%
\>[5]\AgdaSymbol{→} \AgdaDatatype{Term} \AgdaSymbol{((}\AgdaBound{A} \AgdaInductiveConstructor{‘’} \AgdaBound{x} \AgdaInductiveConstructor{‘→’} \AgdaBound{B}\AgdaSymbol{)} \AgdaInductiveConstructor{‘→’} \AgdaBound{T}\AgdaSymbol{)}\<%
\\
\>[0]\AgdaIndent{3}{}\<[3]%
\>[3]\AgdaInductiveConstructor{w} \AgdaSymbol{:} \AgdaSymbol{∀} \AgdaSymbol{\{}\AgdaBound{Γ} \AgdaBound{A} \AgdaBound{T}\AgdaSymbol{\}} \AgdaSymbol{→} \AgdaDatatype{Term} \AgdaBound{A} \AgdaSymbol{→} \AgdaDatatype{Term} \AgdaSymbol{\{}\AgdaBound{Γ} \AgdaInductiveConstructor{▻} \AgdaBound{T}\AgdaSymbol{\}} \AgdaSymbol{(}\AgdaInductiveConstructor{W} \AgdaBound{A}\AgdaSymbol{)}\<%
\\
\>[0]\AgdaIndent{3}{}\<[3]%
\>[3]\AgdaInductiveConstructor{w→} \AgdaSymbol{:} \AgdaSymbol{∀} \AgdaSymbol{\{}\AgdaBound{Γ} \AgdaBound{A} \AgdaBound{B} \AgdaBound{X}\AgdaSymbol{\}}\<%
\\
\>[3]\AgdaIndent{4}{}\<[4]%
\>[4]\AgdaSymbol{→} \AgdaDatatype{Term} \AgdaSymbol{\{}\AgdaBound{Γ}\AgdaSymbol{\}} \AgdaSymbol{(}\AgdaBound{A} \AgdaInductiveConstructor{‘→’} \AgdaBound{B}\AgdaSymbol{)}\<%
\\
\>[3]\AgdaIndent{4}{}\<[4]%
\>[4]\AgdaSymbol{→} \AgdaDatatype{Term} \AgdaSymbol{\{}\AgdaBound{Γ} \AgdaInductiveConstructor{▻} \AgdaBound{X}\AgdaSymbol{\}} \AgdaSymbol{(}\AgdaInductiveConstructor{W} \AgdaBound{A} \AgdaInductiveConstructor{‘→’} \AgdaInductiveConstructor{W} \AgdaBound{B}\AgdaSymbol{)}\<%
\\
\>[0]\AgdaIndent{3}{}\<[3]%
\>[3]\AgdaInductiveConstructor{\_w‘‘’’ₐ\_} \AgdaSymbol{:} \AgdaSymbol{∀} \AgdaSymbol{\{}\AgdaBound{A} \AgdaBound{B} \AgdaBound{T}\AgdaSymbol{\}}\<%
\\
\>[3]\AgdaIndent{4}{}\<[4]%
\>[4]\AgdaSymbol{→} \AgdaDatatype{Term} \AgdaSymbol{\{}\AgdaInductiveConstructor{ε} \AgdaInductiveConstructor{▻} \AgdaBound{T}\AgdaSymbol{\}} \AgdaSymbol{(}\AgdaInductiveConstructor{W} \AgdaSymbol{(}\AgdaInductiveConstructor{‘□’} \AgdaInductiveConstructor{‘’} \AgdaInductiveConstructor{⌜} \AgdaBound{A} \AgdaInductiveConstructor{‘→’} \AgdaBound{B} \AgdaInductiveConstructor{⌝ᵀ}\AgdaSymbol{))}\<%
\\
\>[3]\AgdaIndent{4}{}\<[4]%
\>[4]\AgdaSymbol{→} \AgdaDatatype{Term} \AgdaSymbol{\{}\AgdaInductiveConstructor{ε} \AgdaInductiveConstructor{▻} \AgdaBound{T}\AgdaSymbol{\}} \AgdaSymbol{(}\AgdaInductiveConstructor{W} \AgdaSymbol{(}\AgdaInductiveConstructor{‘□’} \AgdaInductiveConstructor{‘’} \AgdaInductiveConstructor{⌜} \AgdaBound{A} \AgdaInductiveConstructor{⌝ᵀ}\AgdaSymbol{))}\<%
\\
\>[3]\AgdaIndent{4}{}\<[4]%
\>[4]\AgdaSymbol{→} \AgdaDatatype{Term} \AgdaSymbol{\{}\AgdaInductiveConstructor{ε} \AgdaInductiveConstructor{▻} \AgdaBound{T}\AgdaSymbol{\}} \AgdaSymbol{(}\AgdaInductiveConstructor{W} \AgdaSymbol{(}\AgdaInductiveConstructor{‘□’} \AgdaInductiveConstructor{‘’} \AgdaInductiveConstructor{⌜} \AgdaBound{B} \AgdaInductiveConstructor{⌝ᵀ}\AgdaSymbol{))}\<%
\\
%
\\
\> \AgdaFunction{□} \AgdaSymbol{:} \AgdaDatatype{Type} \AgdaInductiveConstructor{ε} \AgdaSymbol{→} \AgdaPrimitiveType{Set} \AgdaSymbol{\_}\<%
\\
\> \AgdaFunction{□} \AgdaSymbol{=} \AgdaDatatype{Term} \AgdaSymbol{\{}\AgdaInductiveConstructor{ε}\AgdaSymbol{\}}\<%
\end{code}

 To verify the soundness of our syntax, we provide a model for it and
 an interpretation into that model.  We call particular attention to
 the interpretation of \mintinline{Agda}|‘□’|, which is just
 \mintinline{Agda}|Term {ε}|; to \mintinline{Agda}|Quine ϕ|, which is
 the interpretation of \mintinline{Agda}|ϕ| applied to
 \mintinline{Agda}|Quine ϕ|; and to the interpretations of the quine
 isomorphism functions, which are just the identity functions.

\begin{code}%
\> \AgdaFunction{max-level} \AgdaSymbol{:} \AgdaPostulate{Level}\<%
\\
\> \AgdaFunction{max-level} \AgdaSymbol{=} \AgdaPrimitive{lzero} \<[21]%
\>[21]\AgdaComment{---- also works for higher levels}\<%
\\
%
\\
\> \AgdaKeyword{mutual}\<%
\\
\>[0]\AgdaIndent{2}{}\<[2]%
\>[2]\AgdaFunction{⟦\_⟧ᶜ} \AgdaSymbol{:} \AgdaSymbol{(}\AgdaBound{Γ} \AgdaSymbol{:} \AgdaDatatype{Context}\AgdaSymbol{)} \AgdaSymbol{→} \AgdaPrimitiveType{Set} \AgdaSymbol{(}\AgdaPrimitive{lsuc} \AgdaFunction{max-level}\AgdaSymbol{)}\<%
\\
\>[0]\AgdaIndent{2}{}\<[2]%
\>[2]\AgdaFunction{⟦} \AgdaInductiveConstructor{ε} \AgdaFunction{⟧ᶜ} \<[10]%
\>[10]\AgdaSymbol{=} \AgdaRecord{⊤}\<%
\\
\>[0]\AgdaIndent{2}{}\<[2]%
\>[2]\AgdaFunction{⟦} \AgdaBound{Γ} \AgdaInductiveConstructor{▻} \AgdaBound{T} \AgdaFunction{⟧ᶜ} \AgdaSymbol{=} \AgdaRecord{Σ} \AgdaFunction{⟦} \AgdaBound{Γ} \AgdaFunction{⟧ᶜ} \AgdaFunction{⟦} \AgdaBound{T} \AgdaFunction{⟧ᵀ}\<%
\\
%
\\
\>[0]\AgdaIndent{2}{}\<[2]%
\>[2]\AgdaFunction{⟦\_⟧ᵀ} \AgdaSymbol{:} \AgdaSymbol{∀} \AgdaSymbol{\{}\AgdaBound{Γ}\AgdaSymbol{\}}\<%
\\
\>[2]\AgdaIndent{4}{}\<[4]%
\>[4]\AgdaSymbol{→} \AgdaDatatype{Type} \AgdaBound{Γ} \AgdaSymbol{→} \AgdaFunction{⟦} \AgdaBound{Γ} \AgdaFunction{⟧ᶜ} \AgdaSymbol{→} \AgdaPrimitiveType{Set} \AgdaFunction{max-level}\<%
\\
\>[0]\AgdaIndent{2}{}\<[2]%
\>[2]\AgdaFunction{⟦} \AgdaInductiveConstructor{‘Typeε’} \AgdaFunction{⟧ᵀ} \AgdaBound{Γ⇓} \AgdaSymbol{=} \AgdaDatatype{Lifted} \AgdaSymbol{(}\AgdaDatatype{Type} \AgdaInductiveConstructor{ε}\AgdaSymbol{)}\<%
\\
\>[0]\AgdaIndent{2}{}\<[2]%
\>[2]\AgdaFunction{⟦} \AgdaInductiveConstructor{‘□’} \AgdaFunction{⟧ᵀ} \AgdaBound{Γ⇓} \AgdaSymbol{=} \AgdaDatatype{Lifted} \AgdaSymbol{(}\AgdaDatatype{Term} \AgdaSymbol{\{}\AgdaInductiveConstructor{ε}\AgdaSymbol{\}} \AgdaSymbol{(}\AgdaFunction{lower} \AgdaSymbol{(}\AgdaField{snd} \AgdaBound{Γ⇓}\AgdaSymbol{)))}\<%
\\
\>[0]\AgdaIndent{2}{}\<[2]%
\>[2]\AgdaFunction{⟦} \AgdaInductiveConstructor{Quine} \AgdaBound{ϕ} \AgdaFunction{⟧ᵀ} \AgdaBound{Γ⇓} \AgdaSymbol{=} \AgdaFunction{⟦} \AgdaBound{ϕ} \AgdaFunction{⟧ᵀ} \AgdaSymbol{(}\AgdaBound{Γ⇓} \AgdaInductiveConstructor{,} \AgdaInductiveConstructor{lift} \AgdaSymbol{(}\AgdaInductiveConstructor{Quine} \AgdaBound{ϕ}\AgdaSymbol{))}\<%
\\
\>[0]\AgdaIndent{2}{}\<[2]%
\>[2]\AgdaComment{---- The rest of the type-level interpretations}\<%
\\
\>[0]\AgdaIndent{2}{}\<[2]%
\>[2]\AgdaComment{---- are the obvious ones, if a bit obscured by}\<%
\\
\>[0]\AgdaIndent{2}{}\<[2]%
\>[2]\AgdaComment{---- carrying around the context.}\<%
\\
\>[0]\AgdaIndent{2}{}\<[2]%
\>[2]\AgdaFunction{⟦} \AgdaBound{A} \AgdaInductiveConstructor{‘→’} \AgdaBound{B} \AgdaFunction{⟧ᵀ} \AgdaBound{Γ⇓} \AgdaSymbol{=} \AgdaFunction{⟦} \AgdaBound{A} \AgdaFunction{⟧ᵀ} \AgdaBound{Γ⇓} \AgdaSymbol{→} \AgdaFunction{⟦} \AgdaBound{B} \AgdaFunction{⟧ᵀ} \AgdaBound{Γ⇓}\<%
\\
\>[0]\AgdaIndent{2}{}\<[2]%
\>[2]\AgdaFunction{⟦} \AgdaInductiveConstructor{‘⊤’} \AgdaFunction{⟧ᵀ} \AgdaBound{Γ⇓} \AgdaSymbol{=} \AgdaRecord{⊤}\<%
\\
\>[0]\AgdaIndent{2}{}\<[2]%
\>[2]\AgdaFunction{⟦} \AgdaInductiveConstructor{‘⊥’} \AgdaFunction{⟧ᵀ} \AgdaBound{Γ⇓} \AgdaSymbol{=} \AgdaDatatype{⊥}\<%
\\
\>[0]\AgdaIndent{2}{}\<[2]%
\>[2]\AgdaFunction{⟦} \AgdaInductiveConstructor{W} \AgdaBound{T} \AgdaFunction{⟧ᵀ} \AgdaBound{Γ⇓} \AgdaSymbol{=} \AgdaFunction{⟦} \AgdaBound{T} \AgdaFunction{⟧ᵀ} \AgdaSymbol{(}\AgdaField{fst} \AgdaBound{Γ⇓}\AgdaSymbol{)}\<%
\\
\>[0]\AgdaIndent{2}{}\<[2]%
\>[2]\AgdaFunction{⟦} \AgdaInductiveConstructor{W₁} \AgdaBound{T} \AgdaFunction{⟧ᵀ} \AgdaBound{Γ⇓} \AgdaSymbol{=} \AgdaFunction{⟦} \AgdaBound{T} \AgdaFunction{⟧ᵀ} \AgdaSymbol{(}\AgdaField{fst} \AgdaSymbol{(}\AgdaField{fst} \AgdaBound{Γ⇓}\AgdaSymbol{)} \AgdaInductiveConstructor{,} \AgdaField{snd} \AgdaBound{Γ⇓}\AgdaSymbol{)}\<%
\\
\>[0]\AgdaIndent{2}{}\<[2]%
\>[2]\AgdaFunction{⟦} \AgdaBound{T} \AgdaInductiveConstructor{‘’} \AgdaBound{x} \AgdaFunction{⟧ᵀ} \AgdaBound{Γ⇓} \AgdaSymbol{=} \AgdaFunction{⟦} \AgdaBound{T} \AgdaFunction{⟧ᵀ} \AgdaSymbol{(}\AgdaBound{Γ⇓} \AgdaInductiveConstructor{,} \AgdaFunction{⟦} \AgdaBound{x} \AgdaFunction{⟧ᵗ} \AgdaBound{Γ⇓}\AgdaSymbol{)}\<%
\\
%
\\
\>[0]\AgdaIndent{2}{}\<[2]%
\>[2]\AgdaFunction{⟦\_⟧ᵗ} \AgdaSymbol{:} \AgdaSymbol{∀} \AgdaSymbol{\{}\AgdaBound{Γ} \AgdaBound{T}\AgdaSymbol{\}}\<%
\\
\>[2]\AgdaIndent{4}{}\<[4]%
\>[4]\AgdaSymbol{→} \AgdaDatatype{Term} \AgdaSymbol{\{}\AgdaBound{Γ}\AgdaSymbol{\}} \AgdaBound{T} \AgdaSymbol{→} \AgdaSymbol{(}\AgdaBound{Γ⇓} \AgdaSymbol{:} \AgdaFunction{⟦} \AgdaBound{Γ} \AgdaFunction{⟧ᶜ}\AgdaSymbol{)} \AgdaSymbol{→} \AgdaFunction{⟦} \AgdaBound{T} \AgdaFunction{⟧ᵀ} \AgdaBound{Γ⇓}\<%
\\
\>[0]\AgdaIndent{2}{}\<[2]%
\>[2]\AgdaFunction{⟦} \AgdaInductiveConstructor{⌜} \AgdaBound{x} \AgdaInductiveConstructor{⌝ᵀ} \AgdaFunction{⟧ᵗ} \AgdaBound{Γ⇓} \AgdaSymbol{=} \AgdaInductiveConstructor{lift} \AgdaBound{x}\<%
\\
\>[0]\AgdaIndent{2}{}\<[2]%
\>[2]\AgdaFunction{⟦} \AgdaInductiveConstructor{⌜} \AgdaBound{x} \AgdaInductiveConstructor{⌝ᵗ} \AgdaFunction{⟧ᵗ} \AgdaBound{Γ⇓} \AgdaSymbol{=} \AgdaInductiveConstructor{lift} \AgdaBound{x}\<%
\\
\>[0]\AgdaIndent{2}{}\<[2]%
\>[2]\AgdaFunction{⟦} \AgdaInductiveConstructor{quine→} \AgdaFunction{⟧ᵗ} \AgdaBound{Γ⇓} \AgdaBound{x} \AgdaSymbol{=} \AgdaBound{x}\<%
\\
\>[0]\AgdaIndent{2}{}\<[2]%
\>[2]\AgdaFunction{⟦} \AgdaInductiveConstructor{quine←} \AgdaFunction{⟧ᵗ} \AgdaBound{Γ⇓} \AgdaBound{x} \AgdaSymbol{=} \AgdaBound{x}\<%
\\
\>[0]\AgdaIndent{2}{}\<[2]%
\>[2]\AgdaComment{---- The rest of the term-level interpretations}\<%
\\
\>[0]\AgdaIndent{2}{}\<[2]%
\>[2]\AgdaComment{---- are the obvious ones, if a bit obscured by}\<%
\\
\>[0]\AgdaIndent{2}{}\<[2]%
\>[2]\AgdaComment{---- carrying around the context.}\<%
\\
\>[0]\AgdaIndent{2}{}\<[2]%
\>[2]\AgdaFunction{⟦} \AgdaInductiveConstructor{‘λ’} \AgdaBound{f} \AgdaFunction{⟧ᵗ} \AgdaBound{Γ⇓} \AgdaBound{x} \AgdaSymbol{=} \AgdaFunction{⟦} \AgdaBound{f} \AgdaFunction{⟧ᵗ} \AgdaSymbol{(}\AgdaBound{Γ⇓} \AgdaInductiveConstructor{,} \AgdaBound{x}\AgdaSymbol{)}\<%
\\
\>[0]\AgdaIndent{2}{}\<[2]%
\>[2]\AgdaFunction{⟦} \AgdaInductiveConstructor{‘tt’} \AgdaFunction{⟧ᵗ} \<[13]%
\>[13]\AgdaBound{Γ⇓} \AgdaSymbol{=} \AgdaInductiveConstructor{tt}\<%
\\
\>[0]\AgdaIndent{2}{}\<[2]%
\>[2]\AgdaFunction{⟦} \AgdaInductiveConstructor{‘VAR₀’} \AgdaFunction{⟧ᵗ} \AgdaBound{Γ⇓} \AgdaSymbol{=} \AgdaField{snd} \AgdaBound{Γ⇓}\<%
\\
\>[0]\AgdaIndent{2}{}\<[2]%
\>[2]\AgdaFunction{⟦} \AgdaInductiveConstructor{‘⌜‘VAR₀’⌝ᵗ’} \AgdaFunction{⟧ᵗ} \AgdaBound{Γ⇓} \AgdaSymbol{=} \AgdaInductiveConstructor{lift} \AgdaInductiveConstructor{⌜} \AgdaFunction{lower} \AgdaSymbol{(}\AgdaField{snd} \AgdaBound{Γ⇓}\AgdaSymbol{)} \AgdaInductiveConstructor{⌝ᵗ}\<%
\\
\>[0]\AgdaIndent{2}{}\<[2]%
\>[2]\AgdaFunction{⟦} \AgdaBound{g} \AgdaInductiveConstructor{‘∘’} \AgdaBound{f} \AgdaFunction{⟧ᵗ} \AgdaBound{Γ⇓} \AgdaBound{x} \AgdaSymbol{=} \AgdaFunction{⟦} \AgdaBound{g} \AgdaFunction{⟧ᵗ} \AgdaBound{Γ⇓} \AgdaSymbol{(}\AgdaFunction{⟦} \AgdaBound{f} \AgdaFunction{⟧ᵗ} \AgdaBound{Γ⇓} \AgdaBound{x}\AgdaSymbol{)}\<%
\\
\>[0]\AgdaIndent{2}{}\<[2]%
\>[2]\AgdaFunction{⟦} \AgdaBound{f} \AgdaInductiveConstructor{‘’ₐ} \AgdaBound{x} \AgdaFunction{⟧ᵗ} \AgdaBound{Γ⇓} \AgdaSymbol{=} \AgdaFunction{⟦} \AgdaBound{f} \AgdaFunction{⟧ᵗ} \AgdaBound{Γ⇓} \AgdaSymbol{(}\AgdaFunction{⟦} \AgdaBound{x} \AgdaFunction{⟧ᵗ} \AgdaBound{Γ⇓}\AgdaSymbol{)}\<%
\\
\>[0]\AgdaIndent{2}{}\<[2]%
\>[2]\AgdaFunction{⟦} \AgdaInductiveConstructor{←SW₁SV→W} \AgdaBound{f} \AgdaFunction{⟧ᵗ} \AgdaSymbol{=} \AgdaFunction{⟦} \AgdaBound{f} \AgdaFunction{⟧ᵗ}\<%
\\
\>[0]\AgdaIndent{2}{}\<[2]%
\>[2]\AgdaFunction{⟦} \AgdaInductiveConstructor{→SW₁SV→W} \AgdaBound{f} \AgdaFunction{⟧ᵗ} \AgdaSymbol{=} \AgdaFunction{⟦} \AgdaBound{f} \AgdaFunction{⟧ᵗ}\<%
\\
\>[0]\AgdaIndent{2}{}\<[2]%
\>[2]\AgdaFunction{⟦} \AgdaInductiveConstructor{w} \AgdaBound{x} \AgdaFunction{⟧ᵗ} \AgdaBound{Γ⇓} \AgdaSymbol{=} \AgdaFunction{⟦} \AgdaBound{x} \AgdaFunction{⟧ᵗ} \AgdaSymbol{(}\AgdaField{fst} \AgdaBound{Γ⇓}\AgdaSymbol{)}\<%
\\
\>[0]\AgdaIndent{2}{}\<[2]%
\>[2]\AgdaFunction{⟦} \AgdaInductiveConstructor{w→} \AgdaBound{f} \AgdaFunction{⟧ᵗ} \AgdaBound{Γ⇓} \AgdaSymbol{=} \AgdaFunction{⟦} \AgdaBound{f} \AgdaFunction{⟧ᵗ} \AgdaSymbol{(}\AgdaField{fst} \AgdaBound{Γ⇓}\AgdaSymbol{)}\<%
\\
\>[0]\AgdaIndent{2}{}\<[2]%
\>[2]\AgdaFunction{⟦} \AgdaBound{f} \AgdaInductiveConstructor{w‘‘’’ₐ} \AgdaBound{x} \AgdaFunction{⟧ᵗ} \AgdaBound{Γ⇓}\<%
\\
\>[2]\AgdaIndent{4}{}\<[4]%
\>[4]\AgdaSymbol{=} \AgdaInductiveConstructor{lift} \AgdaSymbol{(}\AgdaFunction{lower} \AgdaSymbol{(}\AgdaFunction{⟦} \AgdaBound{f} \AgdaFunction{⟧ᵗ} \AgdaBound{Γ⇓}\AgdaSymbol{)} \AgdaInductiveConstructor{‘’ₐ} \AgdaFunction{lower} \AgdaSymbol{(}\AgdaFunction{⟦} \AgdaBound{x} \AgdaFunction{⟧ᵗ} \AgdaBound{Γ⇓}\AgdaSymbol{))}\<%
\end{code}

 To prove Lӧb's theorem, we must create the sentence ``if this
 sentence is provable, then $X$'', and then provide and inhabitant of
 that type.  We can define this sentence, which we call
 \mintinline{Agda}|‘H’|, as the type-level quine at the function
 $\lambda v.\ □ v → ‘X’$.  We can then convert back and forth between
 the types \mintinline{Agda}|□ ‘H’| and \mintinline{Agda}|□ ‘H’ → ‘X’|
 with our quine isomorphism functions, and a bit of quotation magic
 and function application gives us a term of type
 \mintinline{Agda}|□ ‘H’ → □ ‘X’|; this corresponds to the inference
 of the provability of Santa Claus' existence from the assumption that
 the sentence is provable.  We compose this with the assumption of
 Lӧb's theorem, that \mintinline{Agda}|□ ‘X’ → ‘X’|, to get a term of
 type \mintinline{Agda}|□ ‘H’ → ‘X’|, i.e., a term of type
 \mintinline{Agda}|‘H’|; this is the inference that when provability
 implies truth, Santa Claus exists, and hence that the sentence is
 provable.  Finally, we apply this to its own quotation, obtaining a
 term of type \mintinline{Agda}|□ ‘X’|, i.e., a proof that Santa Claus
 exists.

\begin{code}%
\> \AgdaKeyword{module} \AgdaModule{inner} \AgdaSymbol{(}\AgdaBound{‘X’} \AgdaSymbol{:} \AgdaDatatype{Type} \AgdaInductiveConstructor{ε}\AgdaSymbol{)}\<%
\\
\>[4]\AgdaIndent{14}{}\<[14]%
\>[14]\AgdaSymbol{(}\AgdaBound{‘f’} \AgdaSymbol{:} \AgdaDatatype{Term} \AgdaSymbol{\{}\AgdaInductiveConstructor{ε}\AgdaSymbol{\}} \AgdaSymbol{(}\AgdaInductiveConstructor{‘□’} \AgdaInductiveConstructor{‘’} \AgdaInductiveConstructor{⌜} \AgdaBound{‘X’} \AgdaInductiveConstructor{⌝ᵀ} \AgdaInductiveConstructor{‘→’} \AgdaBound{‘X’}\AgdaSymbol{))}\<%
\\
\>[0]\AgdaIndent{8}{}\<[8]%
\>[8]\AgdaKeyword{where}\<%
\\
\>[0]\AgdaIndent{2}{}\<[2]%
\>[2]\AgdaFunction{‘H’} \AgdaSymbol{:} \AgdaDatatype{Type} \AgdaInductiveConstructor{ε}\<%
\\
\>[0]\AgdaIndent{2}{}\<[2]%
\>[2]\AgdaFunction{‘H’} \AgdaSymbol{=} \AgdaInductiveConstructor{Quine} \AgdaSymbol{(}\AgdaInductiveConstructor{W₁} \AgdaInductiveConstructor{‘□’} \AgdaInductiveConstructor{‘’} \AgdaInductiveConstructor{‘VAR₀’} \AgdaInductiveConstructor{‘→’} \AgdaInductiveConstructor{W} \AgdaBound{‘X’}\AgdaSymbol{)}\<%
\\
%
\\
\>[0]\AgdaIndent{2}{}\<[2]%
\>[2]\AgdaFunction{‘toH’} \AgdaSymbol{:} \AgdaFunction{□} \AgdaSymbol{((}\AgdaInductiveConstructor{‘□’} \AgdaInductiveConstructor{‘’} \AgdaInductiveConstructor{⌜} \AgdaFunction{‘H’} \AgdaInductiveConstructor{⌝ᵀ} \AgdaInductiveConstructor{‘→’} \AgdaBound{‘X’}\AgdaSymbol{)} \AgdaInductiveConstructor{‘→’} \AgdaFunction{‘H’}\AgdaSymbol{)}\<%
\\
\>[0]\AgdaIndent{2}{}\<[2]%
\>[2]\AgdaFunction{‘toH’} \AgdaSymbol{=} \AgdaInductiveConstructor{←SW₁SV→W} \AgdaInductiveConstructor{quine←}\<%
\\
%
\\
\>[0]\AgdaIndent{2}{}\<[2]%
\>[2]\AgdaFunction{‘fromH’} \AgdaSymbol{:} \AgdaFunction{□} \AgdaSymbol{(}\AgdaFunction{‘H’} \AgdaInductiveConstructor{‘→’} \AgdaSymbol{(}\AgdaInductiveConstructor{‘□’} \AgdaInductiveConstructor{‘’} \AgdaInductiveConstructor{⌜} \AgdaFunction{‘H’} \AgdaInductiveConstructor{⌝ᵀ} \AgdaInductiveConstructor{‘→’} \AgdaBound{‘X’}\AgdaSymbol{))}\<%
\\
\>[0]\AgdaIndent{2}{}\<[2]%
\>[2]\AgdaFunction{‘fromH’} \AgdaSymbol{=} \AgdaInductiveConstructor{→SW₁SV→W} \AgdaInductiveConstructor{quine→}\<%
\\
%
\\
\>[0]\AgdaIndent{2}{}\<[2]%
\>[2]\AgdaFunction{‘□‘H’→□‘X’’} \AgdaSymbol{:} \AgdaFunction{□} \AgdaSymbol{(}\AgdaInductiveConstructor{‘□’} \AgdaInductiveConstructor{‘’} \AgdaInductiveConstructor{⌜} \AgdaFunction{‘H’} \AgdaInductiveConstructor{⌝ᵀ} \AgdaInductiveConstructor{‘→’} \AgdaInductiveConstructor{‘□’} \AgdaInductiveConstructor{‘’} \AgdaInductiveConstructor{⌜} \AgdaBound{‘X’} \AgdaInductiveConstructor{⌝ᵀ}\AgdaSymbol{)}\<%
\\
\>[0]\AgdaIndent{2}{}\<[2]%
\>[2]\AgdaFunction{‘□‘H’→□‘X’’}\<%
\\
\>[2]\AgdaIndent{4}{}\<[4]%
\>[4]\AgdaSymbol{=} \AgdaInductiveConstructor{‘λ’} \AgdaSymbol{(}\AgdaInductiveConstructor{w} \AgdaInductiveConstructor{⌜} \AgdaFunction{‘fromH’} \AgdaInductiveConstructor{⌝ᵗ}\<%
\\
\>[4]\AgdaIndent{10}{}\<[10]%
\>[10]\AgdaInductiveConstructor{w‘‘’’ₐ} \AgdaInductiveConstructor{‘VAR₀’}\<%
\\
\>[4]\AgdaIndent{10}{}\<[10]%
\>[10]\AgdaInductiveConstructor{w‘‘’’ₐ} \AgdaInductiveConstructor{‘⌜‘VAR₀’⌝ᵗ’}\AgdaSymbol{)}\<%
\\
%
\\
\>[0]\AgdaIndent{2}{}\<[2]%
\>[2]\AgdaFunction{‘h’} \AgdaSymbol{:} \AgdaDatatype{Term} \AgdaFunction{‘H’}\<%
\\
\>[0]\AgdaIndent{2}{}\<[2]%
\>[2]\AgdaFunction{‘h’} \AgdaSymbol{=} \AgdaFunction{‘toH’} \AgdaInductiveConstructor{‘’ₐ} \AgdaSymbol{(}\AgdaBound{‘f’} \AgdaInductiveConstructor{‘∘’} \AgdaFunction{‘□‘H’→□‘X’’}\AgdaSymbol{)}\<%
\\
%
\\
\>[0]\AgdaIndent{2}{}\<[2]%
\>[2]\AgdaFunction{Lӧb} \AgdaSymbol{:} \AgdaFunction{□} \AgdaBound{‘X’}\<%
\\
\>[0]\AgdaIndent{2}{}\<[2]%
\>[2]\AgdaFunction{Lӧb} \AgdaSymbol{=} \AgdaFunction{‘fromH’} \AgdaInductiveConstructor{‘’ₐ} \AgdaFunction{‘h’} \AgdaInductiveConstructor{‘’ₐ} \AgdaInductiveConstructor{⌜} \AgdaFunction{‘h’} \AgdaInductiveConstructor{⌝ᵗ}\<%
\\
%
\\
\> \AgdaFunction{Lӧb} \AgdaSymbol{:} \AgdaSymbol{∀} \AgdaSymbol{\{}\AgdaBound{X}\AgdaSymbol{\}} \AgdaSymbol{→} \AgdaFunction{□} \AgdaSymbol{(}\AgdaInductiveConstructor{‘□’} \AgdaInductiveConstructor{‘’} \AgdaInductiveConstructor{⌜} \AgdaBound{X} \AgdaInductiveConstructor{⌝ᵀ} \AgdaInductiveConstructor{‘→’} \AgdaBound{X}\AgdaSymbol{)} \AgdaSymbol{→} \AgdaFunction{□} \AgdaBound{X}\<%
\\
\> \AgdaFunction{Lӧb} \AgdaSymbol{\{}\AgdaBound{X}\AgdaSymbol{\}} \AgdaBound{f} \AgdaSymbol{=} \AgdaFunction{inner.Lӧb} \AgdaBound{X} \AgdaBound{f}\<%
\\
%
\\
\> \AgdaFunction{⟦\_⟧} \AgdaSymbol{:} \AgdaDatatype{Type} \AgdaInductiveConstructor{ε} \AgdaSymbol{→} \AgdaPrimitiveType{Set} \AgdaSymbol{\_}\<%
\\
\> \AgdaFunction{⟦} \AgdaBound{T} \AgdaFunction{⟧} \AgdaSymbol{=} \AgdaFunction{⟦} \AgdaBound{T} \AgdaFunction{⟧ᵀ} \AgdaInductiveConstructor{tt}\<%
\\
%
\\
\> \AgdaFunction{‘¬’\_} \AgdaSymbol{:} \AgdaSymbol{∀} \AgdaSymbol{\{}\AgdaBound{Γ}\AgdaSymbol{\}} \AgdaSymbol{→} \AgdaDatatype{Type} \AgdaBound{Γ} \AgdaSymbol{→} \AgdaDatatype{Type} \AgdaBound{Γ}\<%
\\
\> \AgdaFunction{‘¬’} \AgdaBound{T} \AgdaSymbol{=} \AgdaBound{T} \AgdaInductiveConstructor{‘→’} \AgdaInductiveConstructor{‘⊥’}\<%
\\
%
\\
\> \AgdaFunction{lӧb} \AgdaSymbol{:} \AgdaSymbol{∀} \AgdaSymbol{\{}\AgdaBound{‘X’}\AgdaSymbol{\}} \AgdaSymbol{→} \AgdaFunction{□} \AgdaSymbol{(}\AgdaInductiveConstructor{‘□’} \AgdaInductiveConstructor{‘’} \AgdaInductiveConstructor{⌜} \AgdaBound{‘X’} \AgdaInductiveConstructor{⌝ᵀ} \AgdaInductiveConstructor{‘→’} \AgdaBound{‘X’}\AgdaSymbol{)} \AgdaSymbol{→} \AgdaFunction{⟦} \AgdaBound{‘X’} \AgdaFunction{⟧}\<%
\\
\> \AgdaFunction{lӧb} \AgdaBound{f} \AgdaSymbol{=} \AgdaFunction{⟦\_⟧ᵗ} \AgdaSymbol{(}\AgdaFunction{Lӧb} \AgdaBound{f}\AgdaSymbol{)} \AgdaInductiveConstructor{tt}\<%
\end{code}

 As above, we can again prove soundness, incompleteness, and non-emptiness.

\begin{code}%
\> \AgdaFunction{incompleteness} \AgdaSymbol{:} \AgdaFunction{¬} \AgdaFunction{□} \AgdaSymbol{(}\AgdaFunction{‘¬’} \AgdaSymbol{(}\AgdaInductiveConstructor{‘□’} \AgdaInductiveConstructor{‘’} \AgdaInductiveConstructor{⌜} \AgdaInductiveConstructor{‘⊥’} \AgdaInductiveConstructor{⌝ᵀ}\AgdaSymbol{))}\<%
\\
\> \AgdaFunction{incompleteness} \AgdaSymbol{=} \AgdaFunction{lӧb}\<%
\\
%
\\
\> \AgdaFunction{soundness} \AgdaSymbol{:} \AgdaFunction{¬} \AgdaFunction{□} \AgdaInductiveConstructor{‘⊥’}\<%
\\
\> \AgdaFunction{soundness} \AgdaBound{x} \AgdaSymbol{=} \AgdaFunction{⟦} \AgdaBound{x} \AgdaFunction{⟧ᵗ} \AgdaInductiveConstructor{tt}\<%
\\
%
\\
\> \AgdaFunction{non-emptiness} \AgdaSymbol{:} \AgdaRecord{Σ} \AgdaSymbol{(}\AgdaDatatype{Type} \AgdaInductiveConstructor{ε}\AgdaSymbol{)} \AgdaSymbol{(λ} \AgdaBound{T} \AgdaSymbol{→} \AgdaFunction{□} \AgdaBound{T}\AgdaSymbol{)}\<%
\\
\> \AgdaFunction{non-emptiness} \AgdaSymbol{=} \AgdaInductiveConstructor{‘⊤’} \AgdaInductiveConstructor{,} \AgdaInductiveConstructor{‘tt’}\<%
\\
\>\<%
\end{code}

\section{Digression: Application of Quining to The Prisoner's Dilemma} \label{sec:prisoner}

  In this section, we use a slightly more enriched encoding of syntax;
  see \autoref{sec:prisoners-dilemma-lob-encoding} for details.

\AgdaHide{
  \begin{code}%
\>\AgdaKeyword{module} \AgdaModule{prisoners-dilemma} \AgdaKeyword{where}\<%
\\
\> \AgdaKeyword{open} \AgdaKeyword{import} \AgdaModule{prisoners-dilemma-lob} \AgdaKeyword{public}\<%
\end{code}
}

  \subsection{The Prisoner's Dilemma}

    The Prisoner's Dilemma is a classic problem in game theory.  Two
    people have been arrested as suspects in a crime and are being
    held in solitary confinement, with no means of communication.  The
    investigators offer each of them a plea bargain: a decreased
    sentence for ratting out the other person.  Each suspect can then
    choose to either cooperate with the other suspect by remaining
    silent, or defect by ratting out the other suspect.  The possible
    outcomes are summarized in~\autoref{tab:prisoner-payoff}.

\begin{table}
\begin{center}
\begin{tabular}{c|cc}
\backslashbox{$B$ Says}{$A$ Says} & Cooperate & Defect \\ \hline
Cooperate & (1 year, 1 year) & (0 years, 3 years) \\
Defect & (3 years, 0 years) & (2 years, 2 years)
\end{tabular}
\caption{The payoff matrix for the prisoner's dilemma; each cell contains (the years $A$ spends in prison, the years $B$ spends in prison).} \label{tab:prisoner-payoff}
\end{center}
\end{table}

    Suspect $A$ might reason thusly: ``Suppose the other suspect
    cooperates with me.  Then I'd get off with no prison time if I
    defected, while I'd have to spend a year in prison if I cooperate.
    Similarly, if the other suspect defects, then I'd get two years in
    prison for defecting, and three for cooperating.  In all cases, I
    do better by defecting.''  If suspect $B$ reasons similarly, then
    both decide to defect, and both get two years in prison, despite
    the fact that both prefer the (Cooperate, Cooperate) outcome over
    the (Defect, Defect) outcome!

  \subsection{Adding Source Code}

    We have the intuition that if both suspects are good at reasoning,
    and both know that they'll reason the same way, then they should
    be able to mutually cooperate.  One way to formalize this is to
    talk about programs (rather than people) playing the prisoner's
    dilemma, and to allow each program access to its own source code
    and its opponent's source
    code~\cite{BaraszChristianoFallensteinEtAl2014}.

    We have formalized this framework in Agda: we use
    \mintinline{Agda}|‘Bot’| to denote the type of programs that can
    play in such a prisoner's dilemma; each one takes in source code
    for two \mintinline{Agda}|‘Bot’|s and outputs a proposition which
    is true (a type which is inhabited) if and only if it cooperates
    with its opponent.  Said another way, the output of each bot is a
    proposition describing the assertion that it cooperates with its
    opponent.

\begin{code}%
\> \AgdaKeyword{open} \AgdaModule{lob}\<%
\\
%
\\
\> \AgdaComment{---- ‘Bot’ is defined as the fixed point of}\<%
\\
\> \AgdaComment{---- ‘Bot’}\<%
\\
\> \AgdaComment{----   ↔ (Term ‘Bot’ → Term ‘Bot’ → ‘Type’)}\<%
\\
\> \AgdaFunction{‘Bot’} \AgdaSymbol{:} \AgdaSymbol{∀} \AgdaSymbol{\{}\AgdaBound{Γ}\AgdaSymbol{\}} \AgdaSymbol{→} \AgdaDatatype{Type} \AgdaBound{Γ}\<%
\\
\> \AgdaFunction{‘Bot’} \AgdaSymbol{\{}\AgdaBound{Γ}\AgdaSymbol{\}}\<%
\\
\>[2]\AgdaIndent{3}{}\<[3]%
\>[3]\AgdaSymbol{=} \AgdaInductiveConstructor{Quine} \AgdaSymbol{(}\AgdaInductiveConstructor{W₁} \AgdaInductiveConstructor{‘Term’} \AgdaInductiveConstructor{‘’} \AgdaInductiveConstructor{‘VAR₀’}\<%
\\
\>[3]\AgdaIndent{12}{}\<[12]%
\>[12]\AgdaInductiveConstructor{‘→’} \AgdaInductiveConstructor{W₁} \AgdaInductiveConstructor{‘Term’} \AgdaInductiveConstructor{‘’} \AgdaInductiveConstructor{‘VAR₀’}\<%
\\
\>[3]\AgdaIndent{12}{}\<[12]%
\>[12]\AgdaInductiveConstructor{‘→’} \AgdaInductiveConstructor{W} \AgdaSymbol{(}\AgdaInductiveConstructor{‘Type’} \AgdaBound{Γ}\AgdaSymbol{))}\<%
\end{code}

  To construct an executable bot, we could do a bounded search for
  proofs of this proposition; one useful method described in
  \cite{BaraszChristianoFallensteinEtAl2014} is to use Kripke frames.
  This computation is, however, beyond the scope of this paper.

  The assertion that a bot \mintinline{Agda}|b₁| cooperates with a bot
  \mintinline{Agda}|b₂| is the result of interpreting the source code
  for the bot, and feeding the resulting function the source code for
  \mintinline{Agda}|b₁| and \mintinline{Agda}|b₂|.

\begin{code}%
\> \AgdaComment{---- N.B. "□" means "Term \{ε\}", i.e., a term in}\<%
\\
\> \AgdaComment{---- the empty context}\<%
\\
\> \AgdaFunction{\_cooperates-with\_} \AgdaSymbol{:} \AgdaFunction{□} \AgdaFunction{‘Bot’} \AgdaSymbol{→} \AgdaFunction{□} \AgdaFunction{‘Bot’} \AgdaSymbol{→} \AgdaDatatype{Type} \AgdaInductiveConstructor{ε}\<%
\\
\> \AgdaBound{b₁} \AgdaFunction{cooperates-with} \AgdaBound{b₂} \AgdaSymbol{=} \AgdaFunction{lower} \AgdaSymbol{(}\AgdaFunction{⟦} \AgdaBound{b₁} \AgdaFunction{⟧ᵗ} \AgdaInductiveConstructor{tt} \AgdaSymbol{(}\AgdaInductiveConstructor{lift} \AgdaBound{b₁}\AgdaSymbol{)} \AgdaSymbol{(}\AgdaInductiveConstructor{lift} \AgdaBound{b₂}\AgdaSymbol{))}\<%
\end{code}

  We now provide a convenience constructor for building bots, based on
  the definition of quines, and present three relatively simple bots:
  DefectBot, CooperateBot, and FairBot.

\begin{code}%
\> \AgdaFunction{make-bot} \AgdaSymbol{:} \AgdaSymbol{∀} \AgdaSymbol{\{}\AgdaBound{Γ}\AgdaSymbol{\}}\<%
\\
\>[0]\AgdaIndent{3}{}\<[3]%
\>[3]\AgdaSymbol{→} \AgdaDatatype{Term} \AgdaSymbol{\{}\AgdaBound{Γ} \AgdaInductiveConstructor{▻} \AgdaFunction{‘□’} \AgdaFunction{‘Bot’} \AgdaInductiveConstructor{▻} \AgdaInductiveConstructor{W} \AgdaSymbol{(}\AgdaFunction{‘□’} \AgdaFunction{‘Bot’}\AgdaSymbol{)\}}\<%
\\
\>[3]\AgdaIndent{10}{}\<[10]%
\>[10]\AgdaSymbol{(}\AgdaInductiveConstructor{W} \AgdaSymbol{(}\AgdaInductiveConstructor{W} \AgdaSymbol{(}\AgdaInductiveConstructor{‘Type’} \AgdaBound{Γ}\AgdaSymbol{)))}\<%
\\
\>[0]\AgdaIndent{3}{}\<[3]%
\>[3]\AgdaSymbol{→} \AgdaDatatype{Term} \AgdaSymbol{\{}\AgdaBound{Γ}\AgdaSymbol{\}} \AgdaFunction{‘Bot’}\<%
\\
\> \AgdaFunction{make-bot} \AgdaBound{t}\<%
\\
\>[0]\AgdaIndent{3}{}\<[3]%
\>[3]\AgdaSymbol{=} \AgdaInductiveConstructor{←SW₁SV→SW₁SV→W}\<%
\\
\>[3]\AgdaIndent{5}{}\<[5]%
\>[5]\AgdaInductiveConstructor{quine←} \AgdaInductiveConstructor{‘’ₐ} \AgdaInductiveConstructor{‘λ’} \AgdaSymbol{(}\AgdaInductiveConstructor{→w} \AgdaSymbol{(}\AgdaInductiveConstructor{‘λ’} \AgdaBound{t}\AgdaSymbol{))}\<%
\\
%
\\
\> \AgdaFunction{‘DefectBot’} \<[16]%
\>[16]\AgdaSymbol{:} \AgdaFunction{□} \AgdaFunction{‘Bot’}\<%
\\
\> \AgdaFunction{‘CooperateBot’} \AgdaSymbol{:} \AgdaFunction{□} \AgdaFunction{‘Bot’}\<%
\\
\> \AgdaFunction{‘FairBot’} \<[16]%
\>[16]\AgdaSymbol{:} \AgdaFunction{□} \AgdaFunction{‘Bot’}\<%
\end{code}

  The first two bots are very simple: DefectBot never cooperates (the
  assertion that DefectBot cooperates is a contradiction), while
  CooperateBot always cooperates.  We define these bots, and prove
  that DefectBot never cooperates and CooperateBot always cooperates.

\begin{code}%
\> \AgdaFunction{‘DefectBot’} \<[16]%
\>[16]\AgdaSymbol{=} \AgdaFunction{make-bot} \AgdaSymbol{(}\AgdaInductiveConstructor{w} \AgdaSymbol{(}\AgdaInductiveConstructor{w} \AgdaInductiveConstructor{⌜} \AgdaInductiveConstructor{‘⊥’} \AgdaInductiveConstructor{⌝ᵀ}\AgdaSymbol{))}\<%
\\
\> \AgdaFunction{‘CooperateBot’} \AgdaSymbol{=} \AgdaFunction{make-bot} \AgdaSymbol{(}\AgdaInductiveConstructor{w} \AgdaSymbol{(}\AgdaInductiveConstructor{w} \AgdaInductiveConstructor{⌜} \AgdaInductiveConstructor{‘⊤’} \AgdaInductiveConstructor{⌝ᵀ}\AgdaSymbol{))}\<%
\\
%
\\
\> \AgdaFunction{DB-defects} \AgdaSymbol{:} \AgdaSymbol{∀} \AgdaSymbol{\{}\AgdaBound{b}\AgdaSymbol{\}}\<%
\\
\>[0]\AgdaIndent{3}{}\<[3]%
\>[3]\AgdaSymbol{→} \AgdaFunction{¬} \AgdaFunction{⟦} \AgdaFunction{‘DefectBot’} \AgdaFunction{cooperates-with} \AgdaBound{b} \AgdaFunction{⟧}\<%
\\
\> \AgdaFunction{DB-defects} \AgdaSymbol{\{}\AgdaBound{b}\AgdaSymbol{\}} \AgdaBound{pf} \AgdaSymbol{=} \AgdaBound{pf}\<%
\\
%
\\
\> \AgdaFunction{CB-cooperates} \AgdaSymbol{:} \AgdaSymbol{∀} \AgdaSymbol{\{}\AgdaBound{b}\AgdaSymbol{\}}\<%
\\
\>[0]\AgdaIndent{3}{}\<[3]%
\>[3]\AgdaSymbol{→} \AgdaFunction{⟦} \AgdaFunction{‘CooperateBot’} \AgdaFunction{cooperates-with} \AgdaBound{b} \AgdaFunction{⟧}\<%
\\
\> \AgdaFunction{CB-cooperates} \AgdaSymbol{\{}\AgdaBound{b}\AgdaSymbol{\}} \AgdaSymbol{=} \AgdaInductiveConstructor{tt}\<%
\end{code}

  We can do better than DefectBot, though, now that we have source
  code.  FairBot cooperates with you if and only if it can find a
  proof that you cooperate with FairBot.  By Lӧb's theorem, to prove
  that FairBot cooperates with itself, it suffices to prove that if
  there is a proof that FairBot cooperates with itself, then FairBot
  does, in fact, cooperate with itself.  This is obvious, though:
  FairBot decides whether or not to cooperate with itself by searching
  for a proof that it does, in fact, cooperate with itself.

  To define FairBot, we first define what it means for the other bot
  to cooperate with some particular bot.

\begin{code}%
\> \AgdaComment{---- We can "evaluate" a bot to turn it into a}\<%
\\
\> \AgdaComment{---- function accepting the source code of two}\<%
\\
\> \AgdaComment{---- bots.}\<%
\\
\> \AgdaFunction{‘eval-bot’} \AgdaSymbol{:} \AgdaSymbol{∀} \AgdaSymbol{\{}\AgdaBound{Γ}\AgdaSymbol{\}}\<%
\\
\>[0]\AgdaIndent{3}{}\<[3]%
\>[3]\AgdaSymbol{→} \AgdaDatatype{Term} \AgdaSymbol{\{}\AgdaBound{Γ}\AgdaSymbol{\}} \AgdaSymbol{(}\AgdaFunction{‘Bot’}\<%
\\
\>[3]\AgdaIndent{14}{}\<[14]%
\>[14]\AgdaInductiveConstructor{‘→’} \AgdaSymbol{(}\AgdaFunction{‘□’} \AgdaFunction{‘Bot’} \AgdaInductiveConstructor{‘→’} \AgdaFunction{‘□’} \AgdaFunction{‘Bot’} \AgdaInductiveConstructor{‘→’} \AgdaInductiveConstructor{‘Type’} \AgdaBound{Γ}\AgdaSymbol{))}\<%
\\
\> \AgdaFunction{‘eval-bot’} \AgdaSymbol{=} \AgdaInductiveConstructor{→SW₁SV→SW₁SV→W} \AgdaInductiveConstructor{quine→}\<%
\\
%
\\
\> \AgdaComment{---- We can quote this, and get a function that}\<%
\\
\> \AgdaComment{---- takes the source code for a bot, and}\<%
\\
\> \AgdaComment{---- outputs the source code for a function that}\<%
\\
\> \AgdaComment{---- takes (the source code for) that bot's}\<%
\\
\> \AgdaComment{---- opponent, and returns an assertion of}\<%
\\
\> \AgdaComment{---- cooperation with that opponent}\<%
\\
\> \AgdaFunction{‘‘eval-bot’’} \AgdaSymbol{:} \AgdaSymbol{∀} \AgdaSymbol{\{}\AgdaBound{Γ}\AgdaSymbol{\}}\<%
\\
\>[0]\AgdaIndent{3}{}\<[3]%
\>[3]\AgdaSymbol{→} \AgdaDatatype{Term} \AgdaSymbol{\{}\AgdaBound{Γ}\AgdaSymbol{\}} \AgdaSymbol{(}\AgdaFunction{‘□’} \AgdaFunction{‘Bot’}\<%
\\
\>[3]\AgdaIndent{5}{}\<[5]%
\>[5]\AgdaInductiveConstructor{‘→’} \AgdaFunction{‘□’} \AgdaSymbol{(}\AgdaComment{\{- other -\}} \AgdaFunction{‘□’} \AgdaFunction{‘Bot’} \AgdaInductiveConstructor{‘→’} \AgdaInductiveConstructor{‘Type’} \AgdaBound{Γ}\AgdaSymbol{))}\<%
\\
\> \AgdaFunction{‘‘eval-bot’’}\<%
\\
\>[0]\AgdaIndent{3}{}\<[3]%
\>[3]\AgdaSymbol{=} \AgdaInductiveConstructor{‘λ’} \AgdaSymbol{(}\AgdaInductiveConstructor{w} \AgdaInductiveConstructor{⌜} \AgdaFunction{‘eval-bot’} \AgdaInductiveConstructor{⌝ᵗ}\<%
\\
\>[3]\AgdaIndent{9}{}\<[9]%
\>[9]\AgdaInductiveConstructor{w‘‘’’ₐ} \AgdaInductiveConstructor{‘VAR₀’}\<%
\\
\>[3]\AgdaIndent{9}{}\<[9]%
\>[9]\AgdaInductiveConstructor{w‘‘’’ₐ} \AgdaInductiveConstructor{‘⌜‘VAR₀’⌝ᵗ’}\AgdaSymbol{)}\<%
\\
%
\\
\> \AgdaComment{---- The assertion "our opponent cooperates with}\<%
\\
\> \AgdaComment{---- a bot b" is equivalent to the evaluation of}\<%
\\
\> \AgdaComment{---- our opponent, applied to b.  Most of the}\<%
\\
\> \AgdaComment{---- noise in this statement is manipulation of}\<%
\\
\> \AgdaComment{---- weakening and substitution.}\<%
\\
\> \AgdaFunction{‘other-cooperates-with’} \AgdaSymbol{:} \AgdaSymbol{∀} \AgdaSymbol{\{}\AgdaBound{Γ}\AgdaSymbol{\}}\<%
\\
\>[0]\AgdaIndent{3}{}\<[3]%
\>[3]\AgdaSymbol{→} \AgdaDatatype{Term} \AgdaSymbol{\{}\AgdaBound{Γ}\<%
\\
\>[3]\AgdaIndent{6}{}\<[6]%
\>[6]\AgdaInductiveConstructor{▻} \AgdaFunction{‘□’} \AgdaFunction{‘Bot’}\<%
\\
\>[3]\AgdaIndent{6}{}\<[6]%
\>[6]\AgdaInductiveConstructor{▻} \AgdaInductiveConstructor{W} \AgdaSymbol{(}\AgdaFunction{‘□’} \AgdaFunction{‘Bot’}\AgdaSymbol{)\}}\<%
\\
\>[0]\AgdaIndent{5}{}\<[5]%
\>[5]\AgdaSymbol{(}\AgdaInductiveConstructor{W} \AgdaSymbol{(}\AgdaInductiveConstructor{W} \AgdaSymbol{(}\AgdaFunction{‘□’} \AgdaFunction{‘Bot’}\AgdaSymbol{))} \AgdaInductiveConstructor{‘→’} \AgdaInductiveConstructor{W} \AgdaSymbol{(}\AgdaInductiveConstructor{W} \AgdaSymbol{(}\AgdaFunction{‘□’} \AgdaSymbol{(}\AgdaInductiveConstructor{‘Type’} \AgdaBound{Γ}\AgdaSymbol{))))}\<%
\\
\> \AgdaFunction{‘other-cooperates-with’} \AgdaSymbol{\{}\AgdaBound{Γ}\AgdaSymbol{\}}\<%
\\
\>[0]\AgdaIndent{3}{}\<[3]%
\>[3]\AgdaSymbol{=} \AgdaFunction{‘eval-other'’}\<%
\\
\>[3]\AgdaIndent{5}{}\<[5]%
\>[5]\AgdaInductiveConstructor{‘∘’} \AgdaInductiveConstructor{w→} \AgdaSymbol{(}\AgdaInductiveConstructor{w} \AgdaSymbol{(}\AgdaInductiveConstructor{w→} \AgdaSymbol{(}\AgdaInductiveConstructor{w} \AgdaSymbol{(}\AgdaInductiveConstructor{‘λ’} \AgdaInductiveConstructor{‘⌜‘VAR₀’⌝ᵗ’}\AgdaSymbol{))))}\<%
\\
\>[0]\AgdaIndent{2}{}\<[2]%
\>[2]\AgdaKeyword{where}\<%
\\
\>[2]\AgdaIndent{3}{}\<[3]%
\>[3]\AgdaFunction{‘eval-other’}\<%
\\
\>[3]\AgdaIndent{5}{}\<[5]%
\>[5]\AgdaSymbol{:} \AgdaDatatype{Term} \AgdaSymbol{\{}\AgdaBound{Γ} \AgdaInductiveConstructor{▻} \AgdaFunction{‘□’} \AgdaFunction{‘Bot’} \AgdaInductiveConstructor{▻} \AgdaInductiveConstructor{W} \AgdaSymbol{(}\AgdaFunction{‘□’} \AgdaFunction{‘Bot’}\AgdaSymbol{)\}}\<%
\\
\>[5]\AgdaIndent{12}{}\<[12]%
\>[12]\AgdaSymbol{(}\AgdaInductiveConstructor{W} \AgdaSymbol{(}\AgdaInductiveConstructor{W} \AgdaSymbol{(}\AgdaFunction{‘□’} \AgdaSymbol{(}\AgdaFunction{‘□’} \AgdaFunction{‘Bot’} \AgdaInductiveConstructor{‘→’} \AgdaInductiveConstructor{‘Type’} \AgdaBound{Γ}\AgdaSymbol{))))}\<%
\\
\>[0]\AgdaIndent{3}{}\<[3]%
\>[3]\AgdaFunction{‘eval-other’}\<%
\\
\>[3]\AgdaIndent{5}{}\<[5]%
\>[5]\AgdaSymbol{=} \AgdaInductiveConstructor{w→} \AgdaSymbol{(}\AgdaInductiveConstructor{w} \AgdaSymbol{(}\AgdaInductiveConstructor{w→} \AgdaSymbol{(}\AgdaInductiveConstructor{w} \AgdaFunction{‘‘eval-bot’’}\AgdaSymbol{)))} \AgdaInductiveConstructor{‘’ₐ} \AgdaInductiveConstructor{‘VAR₀’}\<%
\\
%
\\
\>[0]\AgdaIndent{3}{}\<[3]%
\>[3]\AgdaFunction{‘eval-other'’}\<%
\\
\>[3]\AgdaIndent{5}{}\<[5]%
\>[5]\AgdaSymbol{:} \AgdaDatatype{Term} \AgdaSymbol{(}\AgdaInductiveConstructor{W} \AgdaSymbol{(}\AgdaInductiveConstructor{W} \AgdaSymbol{(}\AgdaFunction{‘□’} \AgdaSymbol{(}\AgdaFunction{‘□’} \AgdaFunction{‘Bot’}\AgdaSymbol{)))}\<%
\\
\>[5]\AgdaIndent{13}{}\<[13]%
\>[13]\AgdaInductiveConstructor{‘→’} \AgdaInductiveConstructor{W} \AgdaSymbol{(}\AgdaInductiveConstructor{W} \AgdaSymbol{(}\AgdaFunction{‘□’} \AgdaSymbol{(}\AgdaInductiveConstructor{‘Type’} \AgdaBound{Γ}\AgdaSymbol{))))}\<%
\\
\>[0]\AgdaIndent{3}{}\<[3]%
\>[3]\AgdaFunction{‘eval-other'’}\<%
\\
\>[3]\AgdaIndent{5}{}\<[5]%
\>[5]\AgdaSymbol{=} \AgdaInductiveConstructor{ww→} \AgdaSymbol{(}\AgdaInductiveConstructor{w→} \AgdaSymbol{(}\AgdaInductiveConstructor{w} \AgdaSymbol{(}\AgdaInductiveConstructor{w→} \AgdaSymbol{(}\AgdaInductiveConstructor{w} \AgdaInductiveConstructor{‘‘’ₐ’}\AgdaSymbol{)))} \AgdaInductiveConstructor{‘’ₐ} \AgdaFunction{‘eval-other’}\AgdaSymbol{)}\<%
\\
%
\\
\> \AgdaComment{---- A bot gets its own source code as the first}\<%
\\
\> \AgdaComment{---- argument (of two)}\<%
\\
\> \AgdaFunction{‘self’} \AgdaSymbol{:} \AgdaSymbol{∀} \AgdaSymbol{\{}\AgdaBound{Γ}\AgdaSymbol{\}}\<%
\\
\>[0]\AgdaIndent{3}{}\<[3]%
\>[3]\AgdaSymbol{→} \AgdaDatatype{Term} \AgdaSymbol{\{}\AgdaBound{Γ} \AgdaInductiveConstructor{▻} \AgdaFunction{‘□’} \AgdaFunction{‘Bot’} \AgdaInductiveConstructor{▻} \AgdaInductiveConstructor{W} \AgdaSymbol{(}\AgdaFunction{‘□’} \AgdaFunction{‘Bot’}\AgdaSymbol{)\}}\<%
\\
\>[3]\AgdaIndent{10}{}\<[10]%
\>[10]\AgdaSymbol{(}\AgdaInductiveConstructor{W} \AgdaSymbol{(}\AgdaInductiveConstructor{W} \AgdaSymbol{(}\AgdaFunction{‘□’} \AgdaFunction{‘Bot’}\AgdaSymbol{)))}\<%
\\
\> \AgdaFunction{‘self’} \AgdaSymbol{=} \AgdaInductiveConstructor{w} \AgdaInductiveConstructor{‘VAR₀’}\<%
\\
%
\\
\> \AgdaComment{---- A bot gets its opponent's source code as}\<%
\\
\> \AgdaComment{---- the second argument (of two)}\<%
\\
\> \AgdaFunction{‘other’} \AgdaSymbol{:} \AgdaSymbol{∀} \AgdaSymbol{\{}\AgdaBound{Γ}\AgdaSymbol{\}}\<%
\\
\>[0]\AgdaIndent{3}{}\<[3]%
\>[3]\AgdaSymbol{→} \AgdaDatatype{Term} \AgdaSymbol{\{}\AgdaBound{Γ} \AgdaInductiveConstructor{▻} \AgdaFunction{‘□’} \AgdaFunction{‘Bot’} \AgdaInductiveConstructor{▻} \AgdaInductiveConstructor{W} \AgdaSymbol{(}\AgdaFunction{‘□’} \AgdaFunction{‘Bot’}\AgdaSymbol{)\}}\<%
\\
\>[3]\AgdaIndent{10}{}\<[10]%
\>[10]\AgdaSymbol{(}\AgdaInductiveConstructor{W} \AgdaSymbol{(}\AgdaInductiveConstructor{W} \AgdaSymbol{(}\AgdaFunction{‘□’} \AgdaFunction{‘Bot’}\AgdaSymbol{)))}\<%
\\
\> \AgdaFunction{‘other’} \AgdaSymbol{=} \AgdaInductiveConstructor{‘VAR₀’}\<%
\\
%
\\
\> \AgdaComment{---- FairBot is the bot that cooperates iff its}\<%
\\
\> \AgdaComment{---- opponent cooperates with it}\<%
\\
\> \AgdaFunction{‘FairBot’}\<%
\\
\>[0]\AgdaIndent{3}{}\<[3]%
\>[3]\AgdaSymbol{=} \AgdaFunction{make-bot} \AgdaSymbol{(}\AgdaInductiveConstructor{‘‘□’’} \AgdaSymbol{(}\AgdaFunction{‘other-cooperates-with’} \AgdaInductiveConstructor{‘’ₐ} \AgdaFunction{‘self’}\AgdaSymbol{))}\<%
\end{code}

  We leave the proof that this formalization of FairBot cooperates
  with itself as an exercise for the reader. In
  \autoref{sec:fair-bot-self-cooperates}, we present an alternative
  formalization with a simple proof that FairBot cooperates with
  itself, but with no general definition of the type of bots; we
  relegate this code to an appendix so as to not confuse the reader by
  introducing a different way of handling contexts and weakening in
  the middle of this paper.

\section{Encoding with an Add-Quote Function} \label{sec:only-add-quote}

  Now we return to our proving of Lӧb's theorem.  Included in the
  artifact for this paper\footnote{In \texttt{lob-build-quine.lagda}.}
  is code that replaces the \mintinline{Agda}|Quine| constructor with
  simpler constructors.  Because the lack of β-reduction in the syntax
  clouds the main points and makes the code rather verbose, we do not
  include the code in the paper, and instead describe the most
  interesting and central points.

  Recall our Python quine from \autoref{sec:python-quine}:
\begin{minted}[gobble=1]{Python}
 (lambda T: Π(□(T % repr(T)), X))
  ('(lambda T: Π(□(T %% repr(T)), X))\n (%s)')
\end{minted}

  To translate this into Agda, we need to give a type to
  \mintinline{Agda}|T|. Clearly, \mintinline{Agda}|T| needs to be of
  type \mintinline{Agda}|Type ???| for some context
  \mintinline{Agda}|???|.  Since we need to be able to substitute
  something into that context, we must have
  \mintinline{Agda}|T : Type (Γ ▻ ???)|, i.e., \mintinline{Agda}|T|
  must be a syntax tree for a type, with a hole in it.

  What's the shape of the thing being substituted?  Well, it's a
  syntax tree for a type with a hole in it.  What shape does that hole
  have?  The shape is that of a syntax tree with a hole in
  it\ldots\space Uh-oh.  Our quine's type, na\"ively, is infinite!

  We know of two ways to work around this.  Classical mathematics,
  which uses Gӧdel codes instead of abstract syntax trees, uses an
  untyped representation of proofs.  It's only later in the proof of
  Lӧb's theorem that a notion of a formula being ``well-formed'' is
  introduced.

  Here, we describe an alternate approach.  Rather than giving up
  types all-together, we can ``box'' the type of the hole, to hide it.
  Using \mintinline{Agda}|fst| and \mintinline{Agda}|snd| to denote
  projections from a Σ type, using \mintinline{Agda}|⌜ A ⌝| to denote
  the abstract syntax tree for \mintinline{Agda}|A|,\footnote{Note
  that \mintinline{Agda}|⌜_⌝| would not be a function in the language,
  but a meta-level operation.} and using \mintinline{Agda}|%s| to
  denote the first variable in the context (written as
  \mintinline{Agda}|‘VAR₀’| in previous formalizations above), we can
  write:

  \begin{minted}[gobble=1]{Agda}
 dummy : Type (ε ▻ ⌜Σ Context Type⌝)
 repr : Σ Context Type → Term {ε} ⌜Σ Context Type⌝

 cast-fst
   : Σ Context Type → Type (ε ▻ ⌜Σ Context Type⌝)
 cast-fst (ε ▻ ⌜Σ Context Type⌝ , T) = T
 cast-fst (_ , _) = dummy

 LӧbSentence : Type ε
 LӧbSentence
   = (λ (T : Σ Context Type)
        → □ (cast-fst T % repr T) ‘→’ X)
       ( ε ▻ ⌜Σ Context Type⌝
       , ⌜ (λ (T : Σ Context Type)
              → □ (cast-fst T % repr T) ‘→’ X)
             (%s) ⌝
  \end{minted}

  In this pseudo-Agda code, \mintinline{Agda}|cast-fst| unboxes the
  sentence that it gets, and returns it if it is the right type.
  Since the sentence is, in fact, always the right type, what we do in
  the other cases doesn't matter.

  Summing up, the key ingredients to this construction are:
  \begin{itemize}

    \item A type of syntactic terms indexed over a type of syntactic
      types (and contexts)

    \item Decidable equality on syntactic contexts at a particular
      point (in particular, at \mintinline{Agda}|Σ Context Type|),
      with appropriate reduction on equal things

    \item Σ types, projections, and appropriate reduction on their
      projections

    \item Function types

    \item A function \mintinline{Agda}|repr| which adds a level of
      quotation to any syntax tree

    \item Syntax trees for all of the above

  \end{itemize}

  In any formalization of dependent type theory with all of these
  ingredients, we can prove Lӧb's theorem.

\section{Conclusion} \label{sec:future-work}

  What remains to be done is formalizing Martin--Lӧf type theory
  without assuming \mintinline{Agda}|repr| and without assuming a
  constructor for the type of syntax trees
  (\mintinline{Agda}|‘Context’|, \mintinline{Agda}|‘Type’|, and
  \mintinline{Agda}|‘Term’| or \mintinline{Agda}|‘□’| in our
  formalizations).  We would instead support inductive types, and
  construct these operators as inductive types and as folds over
  inductive types.

  If you take away only three things from this paper, take away these:
  \begin{enumerate}
    \item There will always be some true things which are not possible
      to say, no matter how good you are at talking in type theory
      about type theory.

    \item Giving meaning to syntax in a way that doesn't use cases
      inside cases allows you to talk about when it's okay to add new
      syntax.

    % \todo{Find an Up-Goer Five way to say "syntax" that doesn't give away the game}

    \item If believing in something is enough to make it true, then
      it already is.  Dream big.
  \end{enumerate}

\appendix
\section{Standard Data-Type Declarations} \label{sec:common}
 \AgdaHide{
  \begin{code}%
\>\AgdaKeyword{module} \AgdaModule{common} \AgdaKeyword{where}\<%
\end{code}
}
\begin{code}%
\>\AgdaKeyword{open} \AgdaKeyword{import} \AgdaModule{Agda.Primitive} \AgdaKeyword{public}\<%
\\
\>[0]\AgdaIndent{2}{}\<[2]%
\>[2]\AgdaKeyword{using} \<[11]%
\>[11]\AgdaSymbol{(}Level\AgdaSymbol{;} \_⊔\_\AgdaSymbol{;} lzero\AgdaSymbol{;} lsuc\AgdaSymbol{)}\<%
\\
%
\\
\>\AgdaKeyword{infixl} \AgdaNumber{1} \AgdaFixityOp{\_,\_}\<%
\\
\>\AgdaKeyword{infixr} \AgdaNumber{2} \AgdaFixityOp{\_×\_}\<%
\\
\>\AgdaKeyword{infixl} \AgdaNumber{1} \AgdaFixityOp{\_≡\_}\<%
\\
%
\\
\>\AgdaKeyword{record} \AgdaRecord{⊤} \AgdaSymbol{\{}\AgdaBound{ℓ}\AgdaSymbol{\}} \AgdaSymbol{:} \AgdaPrimitiveType{Set} \AgdaBound{ℓ} \AgdaKeyword{where}\<%
\\
\>[0]\AgdaIndent{2}{}\<[2]%
\>[2]\AgdaKeyword{constructor} \AgdaInductiveConstructor{tt}\<%
\\
%
\\
\>\AgdaKeyword{data} \AgdaDatatype{⊥} \AgdaSymbol{\{}\AgdaBound{ℓ}\AgdaSymbol{\}} \AgdaSymbol{:} \AgdaPrimitiveType{Set} \AgdaBound{ℓ} \AgdaKeyword{where}\<%
\\
%
\\
\>\AgdaFunction{¬\_} \AgdaSymbol{:} \AgdaSymbol{∀} \AgdaSymbol{\{}\AgdaBound{ℓ} \AgdaBound{ℓ′}\AgdaSymbol{\}} \AgdaSymbol{→} \AgdaPrimitiveType{Set} \AgdaBound{ℓ} \AgdaSymbol{→} \AgdaPrimitiveType{Set} \AgdaSymbol{(}\AgdaBound{ℓ} \AgdaPrimitive{⊔} \AgdaBound{ℓ′}\AgdaSymbol{)}\<%
\\
\>\AgdaFunction{¬\_} \AgdaSymbol{\{}\AgdaBound{ℓ}\AgdaSymbol{\}} \AgdaSymbol{\{}\AgdaBound{ℓ′}\AgdaSymbol{\}} \AgdaBound{T} \AgdaSymbol{=} \AgdaBound{T} \AgdaSymbol{→} \AgdaDatatype{⊥} \AgdaSymbol{\{}\AgdaBound{ℓ′}\AgdaSymbol{\}}\<%
\\
%
\\
\>\AgdaKeyword{record} \AgdaRecord{Σ} \AgdaSymbol{\{}\AgdaBound{a} \AgdaBound{p}\AgdaSymbol{\}} \AgdaSymbol{(}\AgdaBound{A} \AgdaSymbol{:} \AgdaPrimitiveType{Set} \AgdaBound{a}\AgdaSymbol{)} \AgdaSymbol{(}\AgdaBound{P} \AgdaSymbol{:} \AgdaBound{A} \AgdaSymbol{→} \AgdaPrimitiveType{Set} \AgdaBound{p}\AgdaSymbol{)}\<%
\\
\>[2]\AgdaIndent{9}{}\<[9]%
\>[9]\AgdaSymbol{:} \AgdaPrimitiveType{Set} \AgdaSymbol{(}\AgdaBound{a} \AgdaPrimitive{⊔} \AgdaBound{p}\AgdaSymbol{)}\<%
\\
\>[0]\AgdaIndent{7}{}\<[7]%
\>[7]\AgdaKeyword{where}\<%
\\
\>[0]\AgdaIndent{2}{}\<[2]%
\>[2]\AgdaKeyword{constructor} \AgdaInductiveConstructor{\_,\_}\<%
\\
\>[0]\AgdaIndent{2}{}\<[2]%
\>[2]\AgdaKeyword{field}\<%
\\
\>[2]\AgdaIndent{4}{}\<[4]%
\>[4]\AgdaField{fst} \AgdaSymbol{:} \AgdaBound{A}\<%
\\
\>[2]\AgdaIndent{4}{}\<[4]%
\>[4]\AgdaField{snd} \AgdaSymbol{:} \AgdaBound{P} \AgdaBound{fst}\<%
\\
%
\\
\>\AgdaKeyword{open} \AgdaModule{Σ} \AgdaKeyword{public}\<%
\\
%
\\
\>\AgdaKeyword{data} \AgdaDatatype{Lifted} \AgdaSymbol{\{}\AgdaBound{a} \AgdaBound{b}\AgdaSymbol{\}} \AgdaSymbol{(}\AgdaBound{A} \AgdaSymbol{:} \AgdaPrimitiveType{Set} \AgdaBound{a}\AgdaSymbol{)} \AgdaSymbol{:} \AgdaPrimitiveType{Set} \AgdaSymbol{(}\AgdaBound{b} \AgdaPrimitive{⊔} \AgdaBound{a}\AgdaSymbol{)} \AgdaKeyword{where}\<%
\\
\>[0]\AgdaIndent{2}{}\<[2]%
\>[2]\AgdaInductiveConstructor{lift} \AgdaSymbol{:} \AgdaBound{A} \AgdaSymbol{→} \AgdaDatatype{Lifted} \AgdaBound{A}\<%
\\
%
\\
\>\AgdaFunction{lower} \AgdaSymbol{:} \AgdaSymbol{∀} \AgdaSymbol{\{}\AgdaBound{a} \AgdaBound{b} \AgdaBound{A}\AgdaSymbol{\}} \AgdaSymbol{→} \AgdaDatatype{Lifted} \AgdaSymbol{\{}\AgdaBound{a}\AgdaSymbol{\}} \AgdaSymbol{\{}\AgdaBound{b}\AgdaSymbol{\}} \AgdaBound{A} \AgdaSymbol{→} \AgdaBound{A}\<%
\\
\>\AgdaFunction{lower} \AgdaSymbol{(}\AgdaInductiveConstructor{lift} \AgdaBound{x}\AgdaSymbol{)} \AgdaSymbol{=} \AgdaBound{x}\<%
\\
%
\\
\>\AgdaFunction{\_×\_} \AgdaSymbol{:} \AgdaSymbol{∀} \AgdaSymbol{\{}\AgdaBound{ℓ} \AgdaBound{ℓ′}\AgdaSymbol{\}} \AgdaSymbol{(}\AgdaBound{A} \AgdaSymbol{:} \AgdaPrimitiveType{Set} \AgdaBound{ℓ}\AgdaSymbol{)} \AgdaSymbol{(}\AgdaBound{B} \AgdaSymbol{:} \AgdaPrimitiveType{Set} \AgdaBound{ℓ′}\AgdaSymbol{)} \AgdaSymbol{→} \AgdaPrimitiveType{Set} \AgdaSymbol{(}\AgdaBound{ℓ} \AgdaPrimitive{⊔} \AgdaBound{ℓ′}\AgdaSymbol{)}\<%
\\
\>\AgdaBound{A} \AgdaFunction{×} \AgdaBound{B} \AgdaSymbol{=} \AgdaRecord{Σ} \AgdaBound{A} \AgdaSymbol{(λ} \AgdaBound{\_} \AgdaSymbol{→} \AgdaBound{B}\AgdaSymbol{)}\<%
\\
%
\\
\>\AgdaKeyword{data} \AgdaDatatype{\_≡\_} \AgdaSymbol{\{}\AgdaBound{ℓ}\AgdaSymbol{\}} \AgdaSymbol{\{}\AgdaBound{A} \AgdaSymbol{:} \AgdaPrimitiveType{Set} \AgdaBound{ℓ}\AgdaSymbol{\}} \AgdaSymbol{(}\AgdaBound{x} \AgdaSymbol{:} \AgdaBound{A}\AgdaSymbol{)} \AgdaSymbol{:} \AgdaBound{A} \AgdaSymbol{→} \AgdaPrimitiveType{Set} \AgdaBound{ℓ} \AgdaKeyword{where}\<%
\\
\>[0]\AgdaIndent{2}{}\<[2]%
\>[2]\AgdaInductiveConstructor{refl} \AgdaSymbol{:} \AgdaBound{x} \AgdaDatatype{≡} \AgdaBound{x}\<%
\\
%
\\
\>\AgdaFunction{sym} \AgdaSymbol{:} \AgdaSymbol{\{}\AgdaBound{A} \AgdaSymbol{:} \AgdaPrimitiveType{Set}\AgdaSymbol{\}} \AgdaSymbol{→} \AgdaSymbol{\{}\AgdaBound{x} \AgdaSymbol{:} \AgdaBound{A}\AgdaSymbol{\}} \AgdaSymbol{→} \AgdaSymbol{\{}\AgdaBound{y} \AgdaSymbol{:} \AgdaBound{A}\AgdaSymbol{\}} \AgdaSymbol{→} \AgdaBound{x} \AgdaDatatype{≡} \AgdaBound{y} \AgdaSymbol{→} \AgdaBound{y} \AgdaDatatype{≡} \AgdaBound{x}\<%
\\
\>\AgdaFunction{sym} \AgdaInductiveConstructor{refl} \AgdaSymbol{=} \AgdaInductiveConstructor{refl}\<%
\\
%
\\
\>\AgdaFunction{trans} \AgdaSymbol{:} \AgdaSymbol{\{}\AgdaBound{A} \AgdaSymbol{:} \AgdaPrimitiveType{Set}\AgdaSymbol{\}} \AgdaSymbol{→} \AgdaSymbol{\{}\AgdaBound{x} \AgdaBound{y} \AgdaBound{z} \AgdaSymbol{:} \AgdaBound{A}\AgdaSymbol{\}} \AgdaSymbol{→} \AgdaBound{x} \AgdaDatatype{≡} \AgdaBound{y} \AgdaSymbol{→} \AgdaBound{y} \AgdaDatatype{≡} \AgdaBound{z} \AgdaSymbol{→} \AgdaBound{x} \AgdaDatatype{≡} \AgdaBound{z}\<%
\\
\>\AgdaFunction{trans} \AgdaInductiveConstructor{refl} \AgdaInductiveConstructor{refl} \AgdaSymbol{=} \AgdaInductiveConstructor{refl}\<%
\\
%
\\
\>\AgdaFunction{transport} \AgdaSymbol{:} \AgdaSymbol{∀} \AgdaSymbol{\{}\AgdaBound{A} \AgdaSymbol{:} \AgdaPrimitiveType{Set}\AgdaSymbol{\}} \AgdaSymbol{\{}\AgdaBound{x} \AgdaSymbol{:} \AgdaBound{A}\AgdaSymbol{\}} \AgdaSymbol{\{}\AgdaBound{y} \AgdaSymbol{:} \AgdaBound{A}\AgdaSymbol{\}} \AgdaSymbol{→} \AgdaSymbol{(}\AgdaBound{P} \AgdaSymbol{:} \AgdaBound{A} \AgdaSymbol{→} \AgdaPrimitiveType{Set}\AgdaSymbol{)}\<%
\\
\>[0]\AgdaIndent{2}{}\<[2]%
\>[2]\AgdaSymbol{→} \AgdaBound{x} \AgdaDatatype{≡} \AgdaBound{y} \AgdaSymbol{→} \AgdaBound{P} \AgdaBound{x} \AgdaSymbol{→} \AgdaBound{P} \AgdaBound{y}\<%
\\
\>\AgdaFunction{transport} \AgdaBound{P} \AgdaInductiveConstructor{refl} \AgdaBound{v} \AgdaSymbol{=} \AgdaBound{v}\<%
\\
%
\\
\>\AgdaKeyword{data} \AgdaDatatype{List} \AgdaSymbol{(}\AgdaBound{A} \AgdaSymbol{:} \AgdaPrimitiveType{Set}\AgdaSymbol{)} \AgdaSymbol{:} \AgdaPrimitiveType{Set} \AgdaKeyword{where}\<%
\\
\>[0]\AgdaIndent{2}{}\<[2]%
\>[2]\AgdaInductiveConstructor{ε} \AgdaSymbol{:} \AgdaDatatype{List} \AgdaBound{A}\<%
\\
\>[0]\AgdaIndent{2}{}\<[2]%
\>[2]\AgdaInductiveConstructor{\_::\_} \AgdaSymbol{:} \AgdaBound{A} \AgdaSymbol{→} \AgdaDatatype{List} \AgdaBound{A} \AgdaSymbol{→} \AgdaDatatype{List} \AgdaBound{A}\<%
\\
%
\\
\>\AgdaFunction{\_++\_} \AgdaSymbol{:} \AgdaSymbol{∀\{}\AgdaBound{A}\AgdaSymbol{\}} \AgdaSymbol{→} \AgdaDatatype{List} \AgdaBound{A} \AgdaSymbol{→} \AgdaDatatype{List} \AgdaBound{A} \AgdaSymbol{→} \AgdaDatatype{List} \AgdaBound{A}\<%
\\
\>\AgdaInductiveConstructor{ε} \AgdaFunction{++} \AgdaBound{ys} \AgdaSymbol{=} \AgdaBound{ys}\<%
\\
\>\AgdaSymbol{(}\AgdaBound{x} \AgdaInductiveConstructor{::} \AgdaBound{xs}\AgdaSymbol{)} \AgdaFunction{++} \AgdaBound{ys} \AgdaSymbol{=} \AgdaBound{x} \AgdaInductiveConstructor{::} \AgdaSymbol{(}\AgdaBound{xs} \AgdaFunction{++} \AgdaBound{ys}\AgdaSymbol{)}\<%
\\
\>\<%
\end{code}

\section{Encoding of \texorpdfstring{L\"ob}{Lӧb}'s Theorem for the Prisoner's Dilemma} \label{sec:prisoners-dilemma-lob-encoding}
\AgdaHide{
  \begin{code}%
\>\AgdaKeyword{module} \AgdaModule{prisoners-dilemma-lob} \AgdaKeyword{where}\<%
\\
\> \AgdaKeyword{open} \AgdaKeyword{import} \AgdaModule{common}\<%
\end{code}
}
\begin{code}%
\> \AgdaKeyword{module} \AgdaModule{lob} \AgdaKeyword{where}\<%
\\
\>[0]\AgdaIndent{2}{}\<[2]%
\>[2]\AgdaKeyword{infixl} \AgdaNumber{2} \AgdaFixityOp{\_▻\_}\<%
\\
\>[0]\AgdaIndent{2}{}\<[2]%
\>[2]\AgdaKeyword{infixl} \AgdaNumber{3} \AgdaFixityOp{\_‘’\_}\<%
\\
\>[0]\AgdaIndent{2}{}\<[2]%
\>[2]\AgdaKeyword{infixr} \AgdaNumber{1} \AgdaFixityOp{\_‘→’\_}\<%
\\
\>[0]\AgdaIndent{2}{}\<[2]%
\>[2]\AgdaKeyword{infixr} \AgdaNumber{1} \AgdaFixityOp{\_‘‘→’’\_}\<%
\\
\>[0]\AgdaIndent{2}{}\<[2]%
\>[2]\AgdaKeyword{infixr} \AgdaNumber{1} \AgdaFixityOp{\_‘“→”’\_}\<%
\\
\>[0]\AgdaIndent{2}{}\<[2]%
\>[2]\AgdaKeyword{infixl} \AgdaNumber{3} \AgdaFixityOp{\_‘’ₐ\_}\<%
\\
\>[0]\AgdaIndent{2}{}\<[2]%
\>[2]\AgdaKeyword{infixl} \AgdaNumber{3} \AgdaFixityOp{\_w‘‘’’ₐ\_}\<%
\\
\>[0]\AgdaIndent{2}{}\<[2]%
\>[2]\AgdaKeyword{infixr} \AgdaNumber{2} \AgdaFixityOp{\_‘∘’\_}\<%
\\
\>[0]\AgdaIndent{2}{}\<[2]%
\>[2]\AgdaKeyword{infixr} \AgdaNumber{2} \AgdaFixityOp{\_‘×’\_}\<%
\\
\>[0]\AgdaIndent{2}{}\<[2]%
\>[2]\AgdaKeyword{infixr} \AgdaNumber{2} \AgdaFixityOp{\_‘‘×’’\_}\<%
\\
\>[0]\AgdaIndent{2}{}\<[2]%
\>[2]\AgdaKeyword{infixr} \AgdaNumber{2} \AgdaFixityOp{\_w‘‘×’’\_}\<%
\\
%
\\
\>[0]\AgdaIndent{2}{}\<[2]%
\>[2]\AgdaKeyword{mutual}\<%
\\
\>[2]\AgdaIndent{3}{}\<[3]%
\>[3]\AgdaKeyword{data} \AgdaDatatype{Context} \AgdaSymbol{:} \AgdaPrimitiveType{Set} \AgdaKeyword{where}\<%
\\
\>[3]\AgdaIndent{5}{}\<[5]%
\>[5]\AgdaInductiveConstructor{ε} \AgdaSymbol{:} \AgdaDatatype{Context}\<%
\\
\>[3]\AgdaIndent{5}{}\<[5]%
\>[5]\AgdaInductiveConstructor{\_▻\_} \AgdaSymbol{:} \AgdaSymbol{(}\AgdaBound{Γ} \AgdaSymbol{:} \AgdaDatatype{Context}\AgdaSymbol{)} \AgdaSymbol{→} \AgdaDatatype{Type} \AgdaBound{Γ} \AgdaSymbol{→} \AgdaDatatype{Context}\<%
\\
%
\\
\>[0]\AgdaIndent{3}{}\<[3]%
\>[3]\AgdaKeyword{data} \AgdaDatatype{Type} \AgdaSymbol{:} \AgdaDatatype{Context} \AgdaSymbol{→} \AgdaPrimitiveType{Set} \AgdaKeyword{where}\<%
\\
\>[3]\AgdaIndent{5}{}\<[5]%
\>[5]\AgdaInductiveConstructor{‘⊤’} \AgdaSymbol{:} \AgdaSymbol{∀} \AgdaSymbol{\{}\AgdaBound{Γ}\AgdaSymbol{\}} \AgdaSymbol{→} \AgdaDatatype{Type} \AgdaBound{Γ}\<%
\\
\>[3]\AgdaIndent{5}{}\<[5]%
\>[5]\AgdaInductiveConstructor{‘⊥’} \AgdaSymbol{:} \AgdaSymbol{∀} \AgdaSymbol{\{}\AgdaBound{Γ}\AgdaSymbol{\}} \AgdaSymbol{→} \AgdaDatatype{Type} \AgdaBound{Γ}\<%
\\
\>[3]\AgdaIndent{5}{}\<[5]%
\>[5]\AgdaInductiveConstructor{\_‘→’\_} \AgdaSymbol{:} \AgdaSymbol{∀} \AgdaSymbol{\{}\AgdaBound{Γ}\AgdaSymbol{\}} \AgdaSymbol{→} \AgdaDatatype{Type} \AgdaBound{Γ} \AgdaSymbol{→} \AgdaDatatype{Type} \AgdaBound{Γ} \AgdaSymbol{→} \AgdaDatatype{Type} \AgdaBound{Γ}\<%
\\
\>[3]\AgdaIndent{5}{}\<[5]%
\>[5]\AgdaInductiveConstructor{\_‘×’\_} \AgdaSymbol{:} \AgdaSymbol{∀} \AgdaSymbol{\{}\AgdaBound{Γ}\AgdaSymbol{\}} \AgdaSymbol{→} \AgdaDatatype{Type} \AgdaBound{Γ} \AgdaSymbol{→} \AgdaDatatype{Type} \AgdaBound{Γ} \AgdaSymbol{→} \AgdaDatatype{Type} \AgdaBound{Γ}\<%
\\
\>[3]\AgdaIndent{5}{}\<[5]%
\>[5]\AgdaInductiveConstructor{‘Type’} \AgdaSymbol{:} \AgdaSymbol{∀} \AgdaBound{Γ} \AgdaSymbol{→} \AgdaDatatype{Type} \AgdaBound{Γ}\<%
\\
\>[3]\AgdaIndent{5}{}\<[5]%
\>[5]\AgdaInductiveConstructor{‘Term’} \AgdaSymbol{:} \AgdaSymbol{∀} \AgdaSymbol{\{}\AgdaBound{Γ}\AgdaSymbol{\}} \AgdaSymbol{→} \AgdaDatatype{Type} \AgdaSymbol{(}\AgdaBound{Γ} \AgdaInductiveConstructor{▻} \AgdaInductiveConstructor{‘Type’} \AgdaBound{Γ}\AgdaSymbol{)}\<%
\\
\>[3]\AgdaIndent{5}{}\<[5]%
\>[5]\AgdaInductiveConstructor{Quine} \AgdaSymbol{:} \AgdaSymbol{∀} \AgdaSymbol{\{}\AgdaBound{Γ}\AgdaSymbol{\}} \AgdaSymbol{→} \AgdaDatatype{Type} \AgdaSymbol{(}\AgdaBound{Γ} \AgdaInductiveConstructor{▻} \AgdaInductiveConstructor{‘Type’} \AgdaBound{Γ}\AgdaSymbol{)} \AgdaSymbol{→} \AgdaDatatype{Type} \AgdaBound{Γ}\<%
\\
\>[3]\AgdaIndent{5}{}\<[5]%
\>[5]\AgdaInductiveConstructor{W} \AgdaSymbol{:} \AgdaSymbol{∀} \AgdaSymbol{\{}\AgdaBound{Γ} \AgdaBound{A}\AgdaSymbol{\}} \AgdaSymbol{→} \AgdaDatatype{Type} \AgdaBound{Γ} \AgdaSymbol{→} \AgdaDatatype{Type} \AgdaSymbol{(}\AgdaBound{Γ} \AgdaInductiveConstructor{▻} \AgdaBound{A}\AgdaSymbol{)}\<%
\\
\>[3]\AgdaIndent{5}{}\<[5]%
\>[5]\AgdaInductiveConstructor{W₁} \AgdaSymbol{:} \AgdaSymbol{∀} \AgdaSymbol{\{}\AgdaBound{Γ} \AgdaBound{A} \AgdaBound{B}\AgdaSymbol{\}}\<%
\\
\>[5]\AgdaIndent{7}{}\<[7]%
\>[7]\AgdaSymbol{→} \AgdaDatatype{Type} \AgdaSymbol{(}\AgdaBound{Γ} \AgdaInductiveConstructor{▻} \AgdaBound{B}\AgdaSymbol{)}\<%
\\
\>[5]\AgdaIndent{7}{}\<[7]%
\>[7]\AgdaSymbol{→} \AgdaDatatype{Type} \AgdaSymbol{(}\AgdaBound{Γ} \AgdaInductiveConstructor{▻} \AgdaBound{A} \AgdaInductiveConstructor{▻} \AgdaInductiveConstructor{W} \AgdaBound{B}\AgdaSymbol{)}\<%
\\
\>[0]\AgdaIndent{5}{}\<[5]%
\>[5]\AgdaInductiveConstructor{\_‘’\_} \AgdaSymbol{:} \AgdaSymbol{∀} \AgdaSymbol{\{}\AgdaBound{Γ} \AgdaBound{A}\AgdaSymbol{\}}\<%
\\
\>[5]\AgdaIndent{7}{}\<[7]%
\>[7]\AgdaSymbol{→} \AgdaDatatype{Type} \AgdaSymbol{(}\AgdaBound{Γ} \AgdaInductiveConstructor{▻} \AgdaBound{A}\AgdaSymbol{)}\<%
\\
\>[5]\AgdaIndent{7}{}\<[7]%
\>[7]\AgdaSymbol{→} \AgdaDatatype{Term} \AgdaBound{A}\<%
\\
\>[5]\AgdaIndent{7}{}\<[7]%
\>[7]\AgdaSymbol{→} \AgdaDatatype{Type} \AgdaBound{Γ}\<%
\\
%
\\
\>[0]\AgdaIndent{3}{}\<[3]%
\>[3]\AgdaKeyword{data} \AgdaDatatype{Term} \AgdaSymbol{:} \AgdaSymbol{\{}\AgdaBound{Γ} \AgdaSymbol{:} \AgdaDatatype{Context}\AgdaSymbol{\}} \AgdaSymbol{→} \AgdaDatatype{Type} \AgdaBound{Γ} \AgdaSymbol{→} \AgdaPrimitiveType{Set} \AgdaKeyword{where}\<%
\\
\>[3]\AgdaIndent{5}{}\<[5]%
\>[5]\AgdaInductiveConstructor{‘tt’} \AgdaSymbol{:} \AgdaSymbol{∀} \AgdaSymbol{\{}\AgdaBound{Γ}\AgdaSymbol{\}} \AgdaSymbol{→} \AgdaDatatype{Term} \AgdaSymbol{\{}\AgdaBound{Γ}\AgdaSymbol{\}} \AgdaInductiveConstructor{‘⊤’}\<%
\\
\>[3]\AgdaIndent{5}{}\<[5]%
\>[5]\AgdaInductiveConstructor{‘λ’} \AgdaSymbol{:} \AgdaSymbol{∀} \AgdaSymbol{\{}\AgdaBound{Γ} \AgdaBound{A} \AgdaBound{B}\AgdaSymbol{\}}\<%
\\
\>[5]\AgdaIndent{7}{}\<[7]%
\>[7]\AgdaSymbol{→} \AgdaDatatype{Term} \AgdaSymbol{\{}\AgdaBound{Γ} \AgdaInductiveConstructor{▻} \AgdaBound{A}\AgdaSymbol{\}} \AgdaSymbol{(}\AgdaInductiveConstructor{W} \AgdaBound{B}\AgdaSymbol{)}\<%
\\
\>[5]\AgdaIndent{7}{}\<[7]%
\>[7]\AgdaSymbol{→} \AgdaDatatype{Term} \AgdaSymbol{(}\AgdaBound{A} \AgdaInductiveConstructor{‘→’} \AgdaBound{B}\AgdaSymbol{)}\<%
\\
\>[0]\AgdaIndent{5}{}\<[5]%
\>[5]\AgdaInductiveConstructor{‘VAR₀’} \AgdaSymbol{:} \AgdaSymbol{∀} \AgdaSymbol{\{}\AgdaBound{Γ} \AgdaBound{T}\AgdaSymbol{\}} \AgdaSymbol{→} \AgdaDatatype{Term} \AgdaSymbol{\{}\AgdaBound{Γ} \AgdaInductiveConstructor{▻} \AgdaBound{T}\AgdaSymbol{\}} \AgdaSymbol{(}\AgdaInductiveConstructor{W} \AgdaBound{T}\AgdaSymbol{)}\<%
\\
\>[0]\AgdaIndent{5}{}\<[5]%
\>[5]\AgdaInductiveConstructor{⌜\_⌝ᵀ} \AgdaSymbol{:} \AgdaSymbol{∀} \AgdaSymbol{\{}\AgdaBound{Γ}\AgdaSymbol{\}}\<%
\\
\>[5]\AgdaIndent{7}{}\<[7]%
\>[7]\AgdaSymbol{→} \AgdaDatatype{Type} \AgdaBound{Γ}\<%
\\
\>[5]\AgdaIndent{7}{}\<[7]%
\>[7]\AgdaSymbol{→} \AgdaDatatype{Term} \AgdaSymbol{\{}\AgdaBound{Γ}\AgdaSymbol{\}} \AgdaSymbol{(}\AgdaInductiveConstructor{‘Type’} \AgdaBound{Γ}\AgdaSymbol{)}\<%
\\
\>[0]\AgdaIndent{5}{}\<[5]%
\>[5]\AgdaInductiveConstructor{⌜\_⌝ᵗ} \AgdaSymbol{:} \AgdaSymbol{∀} \AgdaSymbol{\{}\AgdaBound{Γ} \AgdaBound{T}\AgdaSymbol{\}}\<%
\\
\>[5]\AgdaIndent{7}{}\<[7]%
\>[7]\AgdaSymbol{→} \AgdaDatatype{Term} \AgdaSymbol{\{}\AgdaBound{Γ}\AgdaSymbol{\}} \AgdaBound{T}\<%
\\
\>[5]\AgdaIndent{7}{}\<[7]%
\>[7]\AgdaSymbol{→} \AgdaDatatype{Term} \AgdaSymbol{\{}\AgdaBound{Γ}\AgdaSymbol{\}} \AgdaSymbol{(}\AgdaInductiveConstructor{‘Term’} \AgdaInductiveConstructor{‘’} \AgdaInductiveConstructor{⌜} \AgdaBound{T} \AgdaInductiveConstructor{⌝ᵀ}\AgdaSymbol{)}\<%
\\
\>[0]\AgdaIndent{5}{}\<[5]%
\>[5]\AgdaInductiveConstructor{‘⌜‘VAR₀’⌝ᵗ’} \AgdaSymbol{:} \AgdaSymbol{∀} \AgdaSymbol{\{}\AgdaBound{Γ} \AgdaBound{T}\AgdaSymbol{\}}\<%
\\
\>[5]\AgdaIndent{7}{}\<[7]%
\>[7]\AgdaSymbol{→} \AgdaDatatype{Term} \AgdaSymbol{\{}\AgdaBound{Γ} \AgdaInductiveConstructor{▻} \AgdaInductiveConstructor{‘Term’} \AgdaInductiveConstructor{‘’} \AgdaInductiveConstructor{⌜} \AgdaBound{T} \AgdaInductiveConstructor{⌝ᵀ}\AgdaSymbol{\}}\<%
\\
\>[7]\AgdaIndent{14}{}\<[14]%
\>[14]\AgdaSymbol{(}\AgdaInductiveConstructor{W} \AgdaSymbol{(}\AgdaInductiveConstructor{‘Term’} \AgdaInductiveConstructor{‘’} \AgdaInductiveConstructor{⌜} \AgdaInductiveConstructor{‘Term’} \AgdaInductiveConstructor{‘’} \AgdaInductiveConstructor{⌜} \AgdaBound{T} \AgdaInductiveConstructor{⌝ᵀ} \AgdaInductiveConstructor{⌝ᵀ}\AgdaSymbol{))}\<%
\\
\>[0]\AgdaIndent{5}{}\<[5]%
\>[5]\AgdaInductiveConstructor{‘⌜‘VAR₀’⌝ᵀ’} \AgdaSymbol{:} \AgdaSymbol{∀} \AgdaSymbol{\{}\AgdaBound{Γ}\AgdaSymbol{\}}\<%
\\
\>[5]\AgdaIndent{7}{}\<[7]%
\>[7]\AgdaSymbol{→} \AgdaDatatype{Term} \AgdaSymbol{\{}\AgdaBound{Γ} \AgdaInductiveConstructor{▻} \AgdaInductiveConstructor{‘Type’} \AgdaBound{Γ}\AgdaSymbol{\}}\<%
\\
\>[7]\AgdaIndent{14}{}\<[14]%
\>[14]\AgdaSymbol{(}\AgdaInductiveConstructor{W} \AgdaSymbol{(}\AgdaInductiveConstructor{‘Term’} \AgdaInductiveConstructor{‘’} \AgdaInductiveConstructor{⌜} \AgdaInductiveConstructor{‘Type’} \AgdaBound{Γ} \AgdaInductiveConstructor{⌝ᵀ}\AgdaSymbol{))}\<%
\\
\>[0]\AgdaIndent{5}{}\<[5]%
\>[5]\AgdaInductiveConstructor{\_‘’ₐ\_} \AgdaSymbol{:} \AgdaSymbol{∀} \AgdaSymbol{\{}\AgdaBound{Γ} \AgdaBound{A} \AgdaBound{B}\AgdaSymbol{\}}\<%
\\
\>[5]\AgdaIndent{7}{}\<[7]%
\>[7]\AgdaSymbol{→} \AgdaDatatype{Term} \AgdaSymbol{\{}\AgdaBound{Γ}\AgdaSymbol{\}} \AgdaSymbol{(}\AgdaBound{A} \AgdaInductiveConstructor{‘→’} \AgdaBound{B}\AgdaSymbol{)}\<%
\\
\>[5]\AgdaIndent{7}{}\<[7]%
\>[7]\AgdaSymbol{→} \AgdaDatatype{Term} \AgdaSymbol{\{}\AgdaBound{Γ}\AgdaSymbol{\}} \AgdaBound{A}\<%
\\
\>[5]\AgdaIndent{7}{}\<[7]%
\>[7]\AgdaSymbol{→} \AgdaDatatype{Term} \AgdaSymbol{\{}\AgdaBound{Γ}\AgdaSymbol{\}} \AgdaBound{B}\<%
\\
\>[0]\AgdaIndent{5}{}\<[5]%
\>[5]\AgdaInductiveConstructor{‘‘×'’’} \AgdaSymbol{:} \AgdaSymbol{∀} \AgdaSymbol{\{}\AgdaBound{Γ}\AgdaSymbol{\}}\<%
\\
\>[5]\AgdaIndent{7}{}\<[7]%
\>[7]\AgdaSymbol{→} \AgdaDatatype{Term} \AgdaSymbol{\{}\AgdaBound{Γ}\AgdaSymbol{\}} \AgdaSymbol{(}\AgdaInductiveConstructor{‘Type’} \AgdaBound{Γ}\<%
\\
\>[7]\AgdaIndent{18}{}\<[18]%
\>[18]\AgdaInductiveConstructor{‘→’} \AgdaInductiveConstructor{‘Type’} \AgdaBound{Γ}\<%
\\
\>[7]\AgdaIndent{18}{}\<[18]%
\>[18]\AgdaInductiveConstructor{‘→’} \AgdaInductiveConstructor{‘Type’} \AgdaBound{Γ}\AgdaSymbol{)}\<%
\\
\>[0]\AgdaIndent{5}{}\<[5]%
\>[5]\AgdaInductiveConstructor{quine→} \AgdaSymbol{:} \AgdaSymbol{∀} \AgdaSymbol{\{}\AgdaBound{Γ} \AgdaBound{φ}\AgdaSymbol{\}}\<%
\\
\>[5]\AgdaIndent{7}{}\<[7]%
\>[7]\AgdaSymbol{→} \AgdaDatatype{Term} \AgdaSymbol{\{}\AgdaBound{Γ}\AgdaSymbol{\}}\<%
\\
\>[7]\AgdaIndent{14}{}\<[14]%
\>[14]\AgdaSymbol{(}\AgdaInductiveConstructor{Quine} \AgdaBound{φ} \<[33]%
\>[33]\AgdaInductiveConstructor{‘→’} \AgdaBound{φ} \AgdaInductiveConstructor{‘’} \AgdaInductiveConstructor{⌜} \AgdaInductiveConstructor{Quine} \AgdaBound{φ} \AgdaInductiveConstructor{⌝ᵀ}\AgdaSymbol{)}\<%
\\
\>[0]\AgdaIndent{5}{}\<[5]%
\>[5]\AgdaInductiveConstructor{quine←} \AgdaSymbol{:} \AgdaSymbol{∀} \AgdaSymbol{\{}\AgdaBound{Γ} \AgdaBound{φ}\AgdaSymbol{\}}\<%
\\
\>[5]\AgdaIndent{7}{}\<[7]%
\>[7]\AgdaSymbol{→} \AgdaDatatype{Term} \AgdaSymbol{\{}\AgdaBound{Γ}\AgdaSymbol{\}}\<%
\\
\>[7]\AgdaIndent{14}{}\<[14]%
\>[14]\AgdaSymbol{(}\AgdaBound{φ} \AgdaInductiveConstructor{‘’} \AgdaInductiveConstructor{⌜} \AgdaInductiveConstructor{Quine} \AgdaBound{φ} \AgdaInductiveConstructor{⌝ᵀ} \AgdaInductiveConstructor{‘→’} \AgdaInductiveConstructor{Quine} \AgdaBound{φ}\AgdaSymbol{)}\<%
\\
\>[0]\AgdaIndent{5}{}\<[5]%
\>[5]\AgdaInductiveConstructor{SW} \AgdaSymbol{:} \AgdaSymbol{∀} \AgdaSymbol{\{}\AgdaBound{Γ} \AgdaBound{X} \AgdaBound{A}\AgdaSymbol{\}} \AgdaSymbol{\{}\AgdaBound{a} \AgdaSymbol{:} \AgdaDatatype{Term} \AgdaBound{A}\AgdaSymbol{\}}\<%
\\
\>[5]\AgdaIndent{7}{}\<[7]%
\>[7]\AgdaSymbol{→} \AgdaDatatype{Term} \AgdaSymbol{\{}\AgdaBound{Γ}\AgdaSymbol{\}} \AgdaSymbol{(}\AgdaInductiveConstructor{W} \AgdaBound{X} \AgdaInductiveConstructor{‘’} \AgdaBound{a}\AgdaSymbol{)}\<%
\\
\>[5]\AgdaIndent{7}{}\<[7]%
\>[7]\AgdaSymbol{→} \AgdaDatatype{Term} \AgdaBound{X}\<%
\\
\>[0]\AgdaIndent{5}{}\<[5]%
\>[5]\AgdaInductiveConstructor{→SW₁SV→W}\<%
\\
\>[5]\AgdaIndent{7}{}\<[7]%
\>[7]\AgdaSymbol{:} \AgdaSymbol{∀} \AgdaSymbol{\{}\AgdaBound{Γ} \AgdaBound{T} \AgdaBound{X} \AgdaBound{A} \AgdaBound{B}\AgdaSymbol{\}} \AgdaSymbol{\{}\AgdaBound{x} \AgdaSymbol{:} \AgdaDatatype{Term} \AgdaBound{X}\AgdaSymbol{\}}\<%
\\
\>[5]\AgdaIndent{7}{}\<[7]%
\>[7]\AgdaSymbol{→} \AgdaDatatype{Term} \AgdaSymbol{\{}\AgdaBound{Γ}\AgdaSymbol{\}}\<%
\\
\>[7]\AgdaIndent{14}{}\<[14]%
\>[14]\AgdaSymbol{(}\AgdaBound{T} \AgdaInductiveConstructor{‘→’} \AgdaSymbol{(}\AgdaInductiveConstructor{W₁} \AgdaBound{A} \AgdaInductiveConstructor{‘’} \AgdaInductiveConstructor{‘VAR₀’} \AgdaInductiveConstructor{‘→’} \AgdaInductiveConstructor{W} \AgdaBound{B}\AgdaSymbol{)} \AgdaInductiveConstructor{‘’} \AgdaBound{x}\AgdaSymbol{)}\<%
\\
\>[0]\AgdaIndent{7}{}\<[7]%
\>[7]\AgdaSymbol{→} \AgdaDatatype{Term} \AgdaSymbol{\{}\AgdaBound{Γ}\AgdaSymbol{\}}\<%
\\
\>[7]\AgdaIndent{14}{}\<[14]%
\>[14]\AgdaSymbol{(}\AgdaBound{T} \AgdaInductiveConstructor{‘→’} \AgdaBound{A} \AgdaInductiveConstructor{‘’} \AgdaBound{x} \AgdaInductiveConstructor{‘→’} \AgdaBound{B}\AgdaSymbol{)}\<%
\\
\>[0]\AgdaIndent{5}{}\<[5]%
\>[5]\AgdaInductiveConstructor{←SW₁SV→W}\<%
\\
\>[5]\AgdaIndent{7}{}\<[7]%
\>[7]\AgdaSymbol{:} \AgdaSymbol{∀} \AgdaSymbol{\{}\AgdaBound{Γ} \AgdaBound{T} \AgdaBound{X} \AgdaBound{A} \AgdaBound{B}\AgdaSymbol{\}} \AgdaSymbol{\{}\AgdaBound{x} \AgdaSymbol{:} \AgdaDatatype{Term} \AgdaBound{X}\AgdaSymbol{\}}\<%
\\
\>[5]\AgdaIndent{7}{}\<[7]%
\>[7]\AgdaSymbol{→} \AgdaDatatype{Term} \AgdaSymbol{\{}\AgdaBound{Γ}\AgdaSymbol{\}}\<%
\\
\>[7]\AgdaIndent{14}{}\<[14]%
\>[14]\AgdaSymbol{((}\AgdaInductiveConstructor{W₁} \AgdaBound{A} \AgdaInductiveConstructor{‘’} \AgdaInductiveConstructor{‘VAR₀’} \AgdaInductiveConstructor{‘→’} \AgdaInductiveConstructor{W} \AgdaBound{B}\AgdaSymbol{)} \AgdaInductiveConstructor{‘’} \AgdaBound{x} \AgdaInductiveConstructor{‘→’} \AgdaBound{T}\AgdaSymbol{)}\<%
\\
\>[0]\AgdaIndent{7}{}\<[7]%
\>[7]\AgdaSymbol{→} \AgdaDatatype{Term} \AgdaSymbol{\{}\AgdaBound{Γ}\AgdaSymbol{\}}\<%
\\
\>[7]\AgdaIndent{14}{}\<[14]%
\>[14]\AgdaSymbol{((}\AgdaBound{A} \AgdaInductiveConstructor{‘’} \AgdaBound{x} \AgdaInductiveConstructor{‘→’} \AgdaBound{B}\AgdaSymbol{)} \AgdaInductiveConstructor{‘→’} \AgdaBound{T}\AgdaSymbol{)}\<%
\\
\>[0]\AgdaIndent{5}{}\<[5]%
\>[5]\AgdaInductiveConstructor{→SW₁SV→SW₁SV→W}\<%
\\
\>[5]\AgdaIndent{7}{}\<[7]%
\>[7]\AgdaSymbol{:} \AgdaSymbol{∀} \AgdaSymbol{\{}\AgdaBound{Γ} \AgdaBound{T} \AgdaBound{X} \AgdaBound{A} \AgdaBound{B}\AgdaSymbol{\}} \AgdaSymbol{\{}\AgdaBound{x} \AgdaSymbol{:} \AgdaDatatype{Term} \AgdaBound{X}\AgdaSymbol{\}}\<%
\\
\>[5]\AgdaIndent{7}{}\<[7]%
\>[7]\AgdaSymbol{→} \AgdaDatatype{Term} \AgdaSymbol{\{}\AgdaBound{Γ}\AgdaSymbol{\}} \AgdaSymbol{(}\AgdaBound{T} \AgdaInductiveConstructor{‘→’} \AgdaSymbol{(}\AgdaInductiveConstructor{W₁} \AgdaBound{A} \AgdaInductiveConstructor{‘’} \AgdaInductiveConstructor{‘VAR₀’}\<%
\\
\>[7]\AgdaIndent{25}{}\<[25]%
\>[25]\AgdaInductiveConstructor{‘→’} \AgdaInductiveConstructor{W₁} \AgdaBound{A} \AgdaInductiveConstructor{‘’} \AgdaInductiveConstructor{‘VAR₀’}\<%
\\
\>[7]\AgdaIndent{25}{}\<[25]%
\>[25]\AgdaInductiveConstructor{‘→’} \AgdaInductiveConstructor{W} \AgdaBound{B}\AgdaSymbol{)} \AgdaInductiveConstructor{‘’} \AgdaBound{x}\AgdaSymbol{)}\<%
\\
\>[0]\AgdaIndent{7}{}\<[7]%
\>[7]\AgdaSymbol{→} \AgdaDatatype{Term} \AgdaSymbol{\{}\AgdaBound{Γ}\AgdaSymbol{\}} \AgdaSymbol{(}\AgdaBound{T} \AgdaInductiveConstructor{‘→’} \AgdaBound{A} \AgdaInductiveConstructor{‘’} \AgdaBound{x} \AgdaInductiveConstructor{‘→’} \AgdaBound{A} \AgdaInductiveConstructor{‘’} \AgdaBound{x} \AgdaInductiveConstructor{‘→’} \AgdaBound{B}\AgdaSymbol{)}\<%
\\
\>[0]\AgdaIndent{5}{}\<[5]%
\>[5]\AgdaInductiveConstructor{←SW₁SV→SW₁SV→W}\<%
\\
\>[5]\AgdaIndent{7}{}\<[7]%
\>[7]\AgdaSymbol{:} \AgdaSymbol{∀} \AgdaSymbol{\{}\AgdaBound{Γ} \AgdaBound{T} \AgdaBound{X} \AgdaBound{A} \AgdaBound{B}\AgdaSymbol{\}} \AgdaSymbol{\{}\AgdaBound{x} \AgdaSymbol{:} \AgdaDatatype{Term} \AgdaBound{X}\AgdaSymbol{\}}\<%
\\
\>[5]\AgdaIndent{7}{}\<[7]%
\>[7]\AgdaSymbol{→} \AgdaDatatype{Term} \AgdaSymbol{\{}\AgdaBound{Γ}\AgdaSymbol{\}} \AgdaSymbol{((}\AgdaInductiveConstructor{W₁} \AgdaBound{A} \AgdaInductiveConstructor{‘’} \AgdaInductiveConstructor{‘VAR₀’}\<%
\\
\>[7]\AgdaIndent{18}{}\<[18]%
\>[18]\AgdaInductiveConstructor{‘→’} \AgdaInductiveConstructor{W₁} \AgdaBound{A} \AgdaInductiveConstructor{‘’} \AgdaInductiveConstructor{‘VAR₀’}\<%
\\
\>[7]\AgdaIndent{18}{}\<[18]%
\>[18]\AgdaInductiveConstructor{‘→’} \AgdaInductiveConstructor{W} \AgdaBound{B}\AgdaSymbol{)} \AgdaInductiveConstructor{‘’} \AgdaBound{x}\<%
\\
\>[7]\AgdaIndent{18}{}\<[18]%
\>[18]\AgdaInductiveConstructor{‘→’} \AgdaBound{T}\AgdaSymbol{)}\<%
\\
\>[0]\AgdaIndent{7}{}\<[7]%
\>[7]\AgdaSymbol{→} \AgdaDatatype{Term} \AgdaSymbol{\{}\AgdaBound{Γ}\AgdaSymbol{\}} \AgdaSymbol{((}\AgdaBound{A} \AgdaInductiveConstructor{‘’} \AgdaBound{x} \AgdaInductiveConstructor{‘→’} \AgdaBound{A} \AgdaInductiveConstructor{‘’} \AgdaBound{x} \AgdaInductiveConstructor{‘→’} \AgdaBound{B}\AgdaSymbol{)} \AgdaInductiveConstructor{‘→’} \AgdaBound{T}\AgdaSymbol{)}\<%
\\
\>[0]\AgdaIndent{5}{}\<[5]%
\>[5]\AgdaInductiveConstructor{w} \AgdaSymbol{:} \AgdaSymbol{∀} \AgdaSymbol{\{}\AgdaBound{Γ} \AgdaBound{A} \AgdaBound{T}\AgdaSymbol{\}}\<%
\\
\>[5]\AgdaIndent{7}{}\<[7]%
\>[7]\AgdaSymbol{→} \AgdaDatatype{Term} \AgdaSymbol{\{}\AgdaBound{Γ}\AgdaSymbol{\}} \AgdaBound{A}\<%
\\
\>[5]\AgdaIndent{7}{}\<[7]%
\>[7]\AgdaSymbol{→} \AgdaDatatype{Term} \AgdaSymbol{\{}\AgdaBound{Γ} \AgdaInductiveConstructor{▻} \AgdaBound{T}\AgdaSymbol{\}} \AgdaSymbol{(}\AgdaInductiveConstructor{W} \AgdaBound{A}\AgdaSymbol{)}\<%
\\
\>[0]\AgdaIndent{5}{}\<[5]%
\>[5]\AgdaInductiveConstructor{w→} \AgdaSymbol{:} \AgdaSymbol{∀} \AgdaSymbol{\{}\AgdaBound{Γ} \AgdaBound{A} \AgdaBound{B} \AgdaBound{X}\AgdaSymbol{\}}\<%
\\
\>[5]\AgdaIndent{7}{}\<[7]%
\>[7]\AgdaSymbol{→} \AgdaDatatype{Term} \AgdaSymbol{\{}\AgdaBound{Γ} \AgdaInductiveConstructor{▻} \AgdaBound{X}\AgdaSymbol{\}} \AgdaSymbol{(}\AgdaInductiveConstructor{W} \AgdaSymbol{(}\AgdaBound{A} \AgdaInductiveConstructor{‘→’} \AgdaBound{B}\AgdaSymbol{))}\<%
\\
\>[5]\AgdaIndent{7}{}\<[7]%
\>[7]\AgdaSymbol{→} \AgdaDatatype{Term} \AgdaSymbol{\{}\AgdaBound{Γ} \AgdaInductiveConstructor{▻} \AgdaBound{X}\AgdaSymbol{\}} \AgdaSymbol{(}\AgdaInductiveConstructor{W} \AgdaBound{A} \AgdaInductiveConstructor{‘→’} \AgdaInductiveConstructor{W} \AgdaBound{B}\AgdaSymbol{)}\<%
\\
\>[0]\AgdaIndent{5}{}\<[5]%
\>[5]\AgdaInductiveConstructor{→w} \AgdaSymbol{:} \AgdaSymbol{∀} \AgdaSymbol{\{}\AgdaBound{Γ} \AgdaBound{A} \AgdaBound{B} \AgdaBound{X}\AgdaSymbol{\}}\<%
\\
\>[5]\AgdaIndent{7}{}\<[7]%
\>[7]\AgdaSymbol{→} \AgdaDatatype{Term} \AgdaSymbol{\{}\AgdaBound{Γ} \AgdaInductiveConstructor{▻} \AgdaBound{X}\AgdaSymbol{\}} \AgdaSymbol{(}\AgdaInductiveConstructor{W} \AgdaBound{A} \AgdaInductiveConstructor{‘→’} \AgdaInductiveConstructor{W} \AgdaBound{B}\AgdaSymbol{)}\<%
\\
\>[5]\AgdaIndent{7}{}\<[7]%
\>[7]\AgdaSymbol{→} \AgdaDatatype{Term} \AgdaSymbol{\{}\AgdaBound{Γ} \AgdaInductiveConstructor{▻} \AgdaBound{X}\AgdaSymbol{\}} \AgdaSymbol{(}\AgdaInductiveConstructor{W} \AgdaSymbol{(}\AgdaBound{A} \AgdaInductiveConstructor{‘→’} \AgdaBound{B}\AgdaSymbol{))}\<%
\\
\>[0]\AgdaIndent{5}{}\<[5]%
\>[5]\AgdaInductiveConstructor{ww→} \AgdaSymbol{:} \AgdaSymbol{∀} \AgdaSymbol{\{}\AgdaBound{Γ} \AgdaBound{A} \AgdaBound{B} \AgdaBound{X} \AgdaBound{Y}\AgdaSymbol{\}}\<%
\\
\>[5]\AgdaIndent{7}{}\<[7]%
\>[7]\AgdaSymbol{→} \AgdaDatatype{Term} \AgdaSymbol{\{}\AgdaBound{Γ} \AgdaInductiveConstructor{▻} \AgdaBound{X} \AgdaInductiveConstructor{▻} \AgdaBound{Y}\AgdaSymbol{\}} \AgdaSymbol{(}\AgdaInductiveConstructor{W} \AgdaSymbol{(}\AgdaInductiveConstructor{W} \AgdaSymbol{(}\AgdaBound{A} \AgdaInductiveConstructor{‘→’} \AgdaBound{B}\AgdaSymbol{)))}\<%
\\
\>[5]\AgdaIndent{7}{}\<[7]%
\>[7]\AgdaSymbol{→} \AgdaDatatype{Term} \AgdaSymbol{\{}\AgdaBound{Γ} \AgdaInductiveConstructor{▻} \AgdaBound{X} \AgdaInductiveConstructor{▻} \AgdaBound{Y}\AgdaSymbol{\}} \AgdaSymbol{(}\AgdaInductiveConstructor{W} \AgdaSymbol{(}\AgdaInductiveConstructor{W} \AgdaBound{A}\AgdaSymbol{)} \AgdaInductiveConstructor{‘→’} \AgdaInductiveConstructor{W} \AgdaSymbol{(}\AgdaInductiveConstructor{W} \AgdaBound{B}\AgdaSymbol{))}\<%
\\
\>[0]\AgdaIndent{5}{}\<[5]%
\>[5]\AgdaInductiveConstructor{→ww} \AgdaSymbol{:} \AgdaSymbol{∀} \AgdaSymbol{\{}\AgdaBound{Γ} \AgdaBound{A} \AgdaBound{B} \AgdaBound{X} \AgdaBound{Y}\AgdaSymbol{\}}\<%
\\
\>[5]\AgdaIndent{7}{}\<[7]%
\>[7]\AgdaSymbol{→} \AgdaDatatype{Term} \AgdaSymbol{\{}\AgdaBound{Γ} \AgdaInductiveConstructor{▻} \AgdaBound{X} \AgdaInductiveConstructor{▻} \AgdaBound{Y}\AgdaSymbol{\}} \AgdaSymbol{(}\AgdaInductiveConstructor{W} \AgdaSymbol{(}\AgdaInductiveConstructor{W} \AgdaBound{A}\AgdaSymbol{)} \AgdaInductiveConstructor{‘→’} \AgdaInductiveConstructor{W} \AgdaSymbol{(}\AgdaInductiveConstructor{W} \AgdaBound{B}\AgdaSymbol{))}\<%
\\
\>[5]\AgdaIndent{7}{}\<[7]%
\>[7]\AgdaSymbol{→} \AgdaDatatype{Term} \AgdaSymbol{\{}\AgdaBound{Γ} \AgdaInductiveConstructor{▻} \AgdaBound{X} \AgdaInductiveConstructor{▻} \AgdaBound{Y}\AgdaSymbol{\}} \AgdaSymbol{(}\AgdaInductiveConstructor{W} \AgdaSymbol{(}\AgdaInductiveConstructor{W} \AgdaSymbol{(}\AgdaBound{A} \AgdaInductiveConstructor{‘→’} \AgdaBound{B}\AgdaSymbol{)))}\<%
\\
\>[0]\AgdaIndent{5}{}\<[5]%
\>[5]\AgdaInductiveConstructor{\_‘∘’\_} \AgdaSymbol{:} \AgdaSymbol{∀} \AgdaSymbol{\{}\AgdaBound{Γ} \AgdaBound{A} \AgdaBound{B} \AgdaBound{C}\AgdaSymbol{\}}\<%
\\
\>[5]\AgdaIndent{7}{}\<[7]%
\>[7]\AgdaSymbol{→} \AgdaDatatype{Term} \AgdaSymbol{\{}\AgdaBound{Γ}\AgdaSymbol{\}} \AgdaSymbol{(}\AgdaBound{B} \AgdaInductiveConstructor{‘→’} \AgdaBound{C}\AgdaSymbol{)}\<%
\\
\>[5]\AgdaIndent{7}{}\<[7]%
\>[7]\AgdaSymbol{→} \AgdaDatatype{Term} \AgdaSymbol{\{}\AgdaBound{Γ}\AgdaSymbol{\}} \AgdaSymbol{(}\AgdaBound{A} \AgdaInductiveConstructor{‘→’} \AgdaBound{B}\AgdaSymbol{)}\<%
\\
\>[5]\AgdaIndent{7}{}\<[7]%
\>[7]\AgdaSymbol{→} \AgdaDatatype{Term} \AgdaSymbol{\{}\AgdaBound{Γ}\AgdaSymbol{\}} \AgdaSymbol{(}\AgdaBound{A} \AgdaInductiveConstructor{‘→’} \AgdaBound{C}\AgdaSymbol{)}\<%
\\
\>[0]\AgdaIndent{5}{}\<[5]%
\>[5]\AgdaInductiveConstructor{\_w‘‘’’ₐ\_} \AgdaSymbol{:} \AgdaSymbol{∀} \AgdaSymbol{\{}\AgdaBound{Γ} \AgdaBound{A} \AgdaBound{B} \AgdaBound{T}\AgdaSymbol{\}}\<%
\\
\>[5]\AgdaIndent{7}{}\<[7]%
\>[7]\AgdaSymbol{→} \AgdaDatatype{Term} \AgdaSymbol{\{}\AgdaBound{Γ} \AgdaInductiveConstructor{▻} \AgdaBound{T}\AgdaSymbol{\}} \AgdaSymbol{(}\AgdaInductiveConstructor{W} \AgdaSymbol{(}\AgdaInductiveConstructor{‘Term’} \AgdaInductiveConstructor{‘’} \AgdaInductiveConstructor{⌜} \AgdaBound{A} \AgdaInductiveConstructor{‘→’} \AgdaBound{B} \AgdaInductiveConstructor{⌝ᵀ}\AgdaSymbol{))}\<%
\\
\>[5]\AgdaIndent{7}{}\<[7]%
\>[7]\AgdaSymbol{→} \AgdaDatatype{Term} \AgdaSymbol{\{}\AgdaBound{Γ} \AgdaInductiveConstructor{▻} \AgdaBound{T}\AgdaSymbol{\}} \AgdaSymbol{(}\AgdaInductiveConstructor{W} \AgdaSymbol{(}\AgdaInductiveConstructor{‘Term’} \AgdaInductiveConstructor{‘’} \AgdaInductiveConstructor{⌜} \AgdaBound{A} \AgdaInductiveConstructor{⌝ᵀ}\AgdaSymbol{))}\<%
\\
\>[5]\AgdaIndent{7}{}\<[7]%
\>[7]\AgdaSymbol{→} \AgdaDatatype{Term} \AgdaSymbol{\{}\AgdaBound{Γ} \AgdaInductiveConstructor{▻} \AgdaBound{T}\AgdaSymbol{\}} \AgdaSymbol{(}\AgdaInductiveConstructor{W} \AgdaSymbol{(}\AgdaInductiveConstructor{‘Term’} \AgdaInductiveConstructor{‘’} \AgdaInductiveConstructor{⌜} \AgdaBound{B} \AgdaInductiveConstructor{⌝ᵀ}\AgdaSymbol{))}\<%
\\
\>[0]\AgdaIndent{5}{}\<[5]%
\>[5]\AgdaInductiveConstructor{‘‘’ₐ’} \AgdaSymbol{:} \AgdaSymbol{∀} \AgdaSymbol{\{}\AgdaBound{Γ} \AgdaBound{A} \AgdaBound{B}\AgdaSymbol{\}}\<%
\\
\>[5]\AgdaIndent{7}{}\<[7]%
\>[7]\AgdaSymbol{→} \AgdaDatatype{Term} \AgdaSymbol{\{}\AgdaBound{Γ}\AgdaSymbol{\}} \AgdaSymbol{(}\AgdaInductiveConstructor{‘Term’} \AgdaInductiveConstructor{‘’} \AgdaInductiveConstructor{⌜} \AgdaBound{A} \AgdaInductiveConstructor{‘→’} \AgdaBound{B} \AgdaInductiveConstructor{⌝ᵀ}\<%
\\
\>[7]\AgdaIndent{18}{}\<[18]%
\>[18]\AgdaInductiveConstructor{‘→’} \AgdaInductiveConstructor{‘Term’} \AgdaInductiveConstructor{‘’} \AgdaInductiveConstructor{⌜} \AgdaBound{A} \AgdaInductiveConstructor{⌝ᵀ}\<%
\\
\>[7]\AgdaIndent{18}{}\<[18]%
\>[18]\AgdaInductiveConstructor{‘→’} \AgdaInductiveConstructor{‘Term’} \AgdaInductiveConstructor{‘’} \AgdaInductiveConstructor{⌜} \AgdaBound{B} \AgdaInductiveConstructor{⌝ᵀ}\AgdaSymbol{)}\<%
\\
\>[0]\AgdaIndent{5}{}\<[5]%
\>[5]\AgdaInductiveConstructor{‘‘□’’} \AgdaSymbol{:} \AgdaSymbol{∀} \AgdaSymbol{\{}\AgdaBound{Γ} \AgdaBound{A} \AgdaBound{B}\AgdaSymbol{\}}\<%
\\
\>[5]\AgdaIndent{7}{}\<[7]%
\>[7]\AgdaSymbol{→} \AgdaDatatype{Term} \AgdaSymbol{\{}\AgdaBound{Γ} \AgdaInductiveConstructor{▻} \AgdaBound{A} \AgdaInductiveConstructor{▻} \AgdaBound{B}\AgdaSymbol{\}}\<%
\\
\>[7]\AgdaIndent{14}{}\<[14]%
\>[14]\AgdaSymbol{(}\AgdaInductiveConstructor{W} \AgdaSymbol{(}\AgdaInductiveConstructor{W} \AgdaSymbol{(}\AgdaInductiveConstructor{‘Term’} \AgdaInductiveConstructor{‘’} \AgdaInductiveConstructor{⌜} \AgdaInductiveConstructor{‘Type’} \AgdaBound{Γ} \AgdaInductiveConstructor{⌝ᵀ}\AgdaSymbol{)))}\<%
\\
\>[0]\AgdaIndent{7}{}\<[7]%
\>[7]\AgdaSymbol{→} \AgdaDatatype{Term} \AgdaSymbol{\{}\AgdaBound{Γ} \AgdaInductiveConstructor{▻} \AgdaBound{A} \AgdaInductiveConstructor{▻} \AgdaBound{B}\AgdaSymbol{\}}\<%
\\
\>[7]\AgdaIndent{14}{}\<[14]%
\>[14]\AgdaSymbol{(}\AgdaInductiveConstructor{W} \AgdaSymbol{(}\AgdaInductiveConstructor{W} \AgdaSymbol{(}\AgdaInductiveConstructor{‘Type’} \AgdaBound{Γ}\AgdaSymbol{)))}\<%
\\
\>[0]\AgdaIndent{5}{}\<[5]%
\>[5]\AgdaInductiveConstructor{\_‘‘→’’\_} \AgdaSymbol{:} \AgdaSymbol{∀} \AgdaSymbol{\{}\AgdaBound{Γ}\AgdaSymbol{\}}\<%
\\
\>[5]\AgdaIndent{7}{}\<[7]%
\>[7]\AgdaSymbol{→} \AgdaDatatype{Term} \AgdaSymbol{\{}\AgdaBound{Γ}\AgdaSymbol{\}} \AgdaSymbol{(}\AgdaInductiveConstructor{‘Type’} \AgdaBound{Γ}\AgdaSymbol{)}\<%
\\
\>[5]\AgdaIndent{7}{}\<[7]%
\>[7]\AgdaSymbol{→} \AgdaDatatype{Term} \AgdaSymbol{\{}\AgdaBound{Γ}\AgdaSymbol{\}} \AgdaSymbol{(}\AgdaInductiveConstructor{‘Type’} \AgdaBound{Γ}\AgdaSymbol{)}\<%
\\
\>[5]\AgdaIndent{7}{}\<[7]%
\>[7]\AgdaSymbol{→} \AgdaDatatype{Term} \AgdaSymbol{\{}\AgdaBound{Γ}\AgdaSymbol{\}} \AgdaSymbol{(}\AgdaInductiveConstructor{‘Type’} \AgdaBound{Γ}\AgdaSymbol{)}\<%
\\
\>[0]\AgdaIndent{5}{}\<[5]%
\>[5]\AgdaInductiveConstructor{\_‘“→”’\_} \AgdaSymbol{:} \AgdaSymbol{∀} \AgdaSymbol{\{}\AgdaBound{Γ} \AgdaBound{A} \AgdaBound{B}\AgdaSymbol{\}}\<%
\\
\>[5]\AgdaIndent{7}{}\<[7]%
\>[7]\AgdaSymbol{→} \AgdaDatatype{Term} \AgdaSymbol{\{}\AgdaBound{Γ} \AgdaInductiveConstructor{▻} \AgdaBound{A} \AgdaInductiveConstructor{▻} \AgdaBound{B}\AgdaSymbol{\}}\<%
\\
\>[7]\AgdaIndent{14}{}\<[14]%
\>[14]\AgdaSymbol{(}\AgdaInductiveConstructor{W} \AgdaSymbol{(}\AgdaInductiveConstructor{W} \AgdaSymbol{(}\AgdaInductiveConstructor{‘Term’} \AgdaInductiveConstructor{‘’} \AgdaInductiveConstructor{⌜} \AgdaInductiveConstructor{‘Type’} \AgdaBound{Γ} \AgdaInductiveConstructor{⌝ᵀ}\AgdaSymbol{)))}\<%
\\
\>[0]\AgdaIndent{7}{}\<[7]%
\>[7]\AgdaSymbol{→} \AgdaDatatype{Term} \AgdaSymbol{\{}\AgdaBound{Γ} \AgdaInductiveConstructor{▻} \AgdaBound{A} \AgdaInductiveConstructor{▻} \AgdaBound{B}\AgdaSymbol{\}}\<%
\\
\>[7]\AgdaIndent{14}{}\<[14]%
\>[14]\AgdaSymbol{(}\AgdaInductiveConstructor{W} \AgdaSymbol{(}\AgdaInductiveConstructor{W} \AgdaSymbol{(}\AgdaInductiveConstructor{‘Term’} \AgdaInductiveConstructor{‘’} \AgdaInductiveConstructor{⌜} \AgdaInductiveConstructor{‘Type’} \AgdaBound{Γ} \AgdaInductiveConstructor{⌝ᵀ}\AgdaSymbol{)))}\<%
\\
\>[0]\AgdaIndent{7}{}\<[7]%
\>[7]\AgdaSymbol{→} \AgdaDatatype{Term} \AgdaSymbol{\{}\AgdaBound{Γ} \AgdaInductiveConstructor{▻} \AgdaBound{A} \AgdaInductiveConstructor{▻} \AgdaBound{B}\AgdaSymbol{\}}\<%
\\
\>[7]\AgdaIndent{14}{}\<[14]%
\>[14]\AgdaSymbol{(}\AgdaInductiveConstructor{W} \AgdaSymbol{(}\AgdaInductiveConstructor{W} \AgdaSymbol{(}\AgdaInductiveConstructor{‘Term’} \AgdaInductiveConstructor{‘’} \AgdaInductiveConstructor{⌜} \AgdaInductiveConstructor{‘Type’} \AgdaBound{Γ} \AgdaInductiveConstructor{⌝ᵀ}\AgdaSymbol{)))}\<%
\\
\>[0]\AgdaIndent{5}{}\<[5]%
\>[5]\AgdaInductiveConstructor{\_‘“×”’\_} \AgdaSymbol{:} \AgdaSymbol{∀} \AgdaSymbol{\{}\AgdaBound{Γ} \AgdaBound{A} \AgdaBound{B}\AgdaSymbol{\}}\<%
\\
\>[5]\AgdaIndent{7}{}\<[7]%
\>[7]\AgdaSymbol{→} \AgdaDatatype{Term} \AgdaSymbol{\{}\AgdaBound{Γ} \AgdaInductiveConstructor{▻} \AgdaBound{A} \AgdaInductiveConstructor{▻} \AgdaBound{B}\AgdaSymbol{\}}\<%
\\
\>[7]\AgdaIndent{14}{}\<[14]%
\>[14]\AgdaSymbol{(}\AgdaInductiveConstructor{W} \AgdaSymbol{(}\AgdaInductiveConstructor{W} \AgdaSymbol{(}\AgdaInductiveConstructor{‘Term’} \AgdaInductiveConstructor{‘’} \AgdaInductiveConstructor{⌜} \AgdaInductiveConstructor{‘Type’} \AgdaBound{Γ} \AgdaInductiveConstructor{⌝ᵀ}\AgdaSymbol{)))}\<%
\\
\>[0]\AgdaIndent{7}{}\<[7]%
\>[7]\AgdaSymbol{→} \AgdaDatatype{Term} \AgdaSymbol{\{}\AgdaBound{Γ} \AgdaInductiveConstructor{▻} \AgdaBound{A} \AgdaInductiveConstructor{▻} \AgdaBound{B}\AgdaSymbol{\}}\<%
\\
\>[7]\AgdaIndent{14}{}\<[14]%
\>[14]\AgdaSymbol{(}\AgdaInductiveConstructor{W} \AgdaSymbol{(}\AgdaInductiveConstructor{W} \AgdaSymbol{(}\AgdaInductiveConstructor{‘Term’} \AgdaInductiveConstructor{‘’} \AgdaInductiveConstructor{⌜} \AgdaInductiveConstructor{‘Type’} \AgdaBound{Γ} \AgdaInductiveConstructor{⌝ᵀ}\AgdaSymbol{)))}\<%
\\
\>[0]\AgdaIndent{7}{}\<[7]%
\>[7]\AgdaSymbol{→} \AgdaDatatype{Term} \AgdaSymbol{\{}\AgdaBound{Γ} \AgdaInductiveConstructor{▻} \AgdaBound{A} \AgdaInductiveConstructor{▻} \AgdaBound{B}\AgdaSymbol{\}}\<%
\\
\>[7]\AgdaIndent{14}{}\<[14]%
\>[14]\AgdaSymbol{(}\AgdaInductiveConstructor{W} \AgdaSymbol{(}\AgdaInductiveConstructor{W} \AgdaSymbol{(}\AgdaInductiveConstructor{‘Term’} \AgdaInductiveConstructor{‘’} \AgdaInductiveConstructor{⌜} \AgdaInductiveConstructor{‘Type’} \AgdaBound{Γ} \AgdaInductiveConstructor{⌝ᵀ}\AgdaSymbol{)))}\<%
\\
%
\\
\>[0]\AgdaIndent{2}{}\<[2]%
\>[2]\AgdaFunction{□} \AgdaSymbol{:} \AgdaDatatype{Type} \AgdaInductiveConstructor{ε} \AgdaSymbol{→} \AgdaPrimitiveType{Set} \AgdaSymbol{\_}\<%
\\
\>[0]\AgdaIndent{2}{}\<[2]%
\>[2]\AgdaFunction{□} \AgdaSymbol{=} \AgdaDatatype{Term} \AgdaSymbol{\{}\AgdaInductiveConstructor{ε}\AgdaSymbol{\}}\<%
\\
%
\\
\>[0]\AgdaIndent{2}{}\<[2]%
\>[2]\AgdaFunction{‘□’} \AgdaSymbol{:} \AgdaSymbol{∀} \AgdaSymbol{\{}\AgdaBound{Γ}\AgdaSymbol{\}} \AgdaSymbol{→} \AgdaDatatype{Type} \AgdaBound{Γ} \AgdaSymbol{→} \AgdaDatatype{Type} \AgdaBound{Γ}\<%
\\
\>[0]\AgdaIndent{2}{}\<[2]%
\>[2]\AgdaFunction{‘□’} \AgdaBound{T} \AgdaSymbol{=} \AgdaInductiveConstructor{‘Term’} \AgdaInductiveConstructor{‘’} \AgdaInductiveConstructor{⌜} \AgdaBound{T} \AgdaInductiveConstructor{⌝ᵀ}\<%
\\
%
\\
\>[0]\AgdaIndent{2}{}\<[2]%
\>[2]\AgdaFunction{\_‘‘×’’\_} \AgdaSymbol{:} \AgdaSymbol{∀} \AgdaSymbol{\{}\AgdaBound{Γ}\AgdaSymbol{\}}\<%
\\
\>[2]\AgdaIndent{4}{}\<[4]%
\>[4]\AgdaSymbol{→} \AgdaDatatype{Term} \AgdaSymbol{\{}\AgdaBound{Γ}\AgdaSymbol{\}} \AgdaSymbol{(}\AgdaInductiveConstructor{‘Type’} \AgdaBound{Γ}\AgdaSymbol{)}\<%
\\
\>[2]\AgdaIndent{4}{}\<[4]%
\>[4]\AgdaSymbol{→} \AgdaDatatype{Term} \AgdaSymbol{\{}\AgdaBound{Γ}\AgdaSymbol{\}} \AgdaSymbol{(}\AgdaInductiveConstructor{‘Type’} \AgdaBound{Γ}\AgdaSymbol{)}\<%
\\
\>[2]\AgdaIndent{4}{}\<[4]%
\>[4]\AgdaSymbol{→} \AgdaDatatype{Term} \AgdaSymbol{\{}\AgdaBound{Γ}\AgdaSymbol{\}} \AgdaSymbol{(}\AgdaInductiveConstructor{‘Type’} \AgdaBound{Γ}\AgdaSymbol{)}\<%
\\
\>[0]\AgdaIndent{2}{}\<[2]%
\>[2]\AgdaBound{A} \AgdaFunction{‘‘×’’} \AgdaBound{B} \AgdaSymbol{=} \AgdaInductiveConstructor{‘‘×'’’} \AgdaInductiveConstructor{‘’ₐ} \AgdaBound{A} \AgdaInductiveConstructor{‘’ₐ} \AgdaBound{B}\<%
\\
%
\\
\>[0]\AgdaIndent{2}{}\<[2]%
\>[2]\AgdaFunction{max-level} \AgdaSymbol{:} \AgdaPostulate{Level}\<%
\\
\>[0]\AgdaIndent{2}{}\<[2]%
\>[2]\AgdaFunction{max-level} \AgdaSymbol{=} \AgdaPrimitive{lzero}\<%
\\
%
\\
\>[0]\AgdaIndent{2}{}\<[2]%
\>[2]\AgdaKeyword{mutual}\<%
\\
\>[2]\AgdaIndent{3}{}\<[3]%
\>[3]\AgdaFunction{⟦\_⟧ᶜ} \AgdaSymbol{:} \AgdaSymbol{(}\AgdaBound{Γ} \AgdaSymbol{:} \AgdaDatatype{Context}\AgdaSymbol{)} \AgdaSymbol{→} \AgdaPrimitiveType{Set} \AgdaSymbol{(}\AgdaPrimitive{lsuc} \AgdaFunction{max-level}\AgdaSymbol{)}\<%
\\
\>[2]\AgdaIndent{3}{}\<[3]%
\>[3]\AgdaFunction{⟦} \AgdaInductiveConstructor{ε} \AgdaFunction{⟧ᶜ} \AgdaSymbol{=} \AgdaRecord{⊤}\<%
\\
\>[2]\AgdaIndent{3}{}\<[3]%
\>[3]\AgdaFunction{⟦} \AgdaBound{Γ} \AgdaInductiveConstructor{▻} \AgdaBound{T} \AgdaFunction{⟧ᶜ} \AgdaSymbol{=} \AgdaRecord{Σ} \AgdaFunction{⟦} \AgdaBound{Γ} \AgdaFunction{⟧ᶜ} \AgdaFunction{⟦} \AgdaBound{T} \AgdaFunction{⟧ᵀ}\<%
\\
%
\\
\>[2]\AgdaIndent{3}{}\<[3]%
\>[3]\AgdaFunction{⟦\_⟧ᵀ} \AgdaSymbol{:} \AgdaSymbol{\{}\AgdaBound{Γ} \AgdaSymbol{:} \AgdaDatatype{Context}\AgdaSymbol{\}}\<%
\\
\>[3]\AgdaIndent{5}{}\<[5]%
\>[5]\AgdaSymbol{→} \AgdaDatatype{Type} \AgdaBound{Γ}\<%
\\
\>[3]\AgdaIndent{5}{}\<[5]%
\>[5]\AgdaSymbol{→} \AgdaFunction{⟦} \AgdaBound{Γ} \AgdaFunction{⟧ᶜ}\<%
\\
\>[3]\AgdaIndent{5}{}\<[5]%
\>[5]\AgdaSymbol{→} \AgdaPrimitiveType{Set} \AgdaFunction{max-level}\<%
\\
\>[0]\AgdaIndent{3}{}\<[3]%
\>[3]\AgdaFunction{⟦} \AgdaInductiveConstructor{W} \AgdaBound{T} \AgdaFunction{⟧ᵀ} \AgdaBound{⟦Γ⟧}\<%
\\
\>[3]\AgdaIndent{5}{}\<[5]%
\>[5]\AgdaSymbol{=} \AgdaFunction{⟦} \AgdaBound{T} \AgdaFunction{⟧ᵀ} \AgdaSymbol{(}\AgdaField{Σ.fst} \AgdaBound{⟦Γ⟧}\AgdaSymbol{)}\<%
\\
\>[0]\AgdaIndent{3}{}\<[3]%
\>[3]\AgdaFunction{⟦} \AgdaInductiveConstructor{W₁} \AgdaBound{T} \AgdaFunction{⟧ᵀ} \AgdaBound{⟦Γ⟧}\<%
\\
\>[3]\AgdaIndent{5}{}\<[5]%
\>[5]\AgdaSymbol{=} \AgdaFunction{⟦} \AgdaBound{T} \AgdaFunction{⟧ᵀ} \AgdaSymbol{(}\AgdaField{Σ.fst} \AgdaSymbol{(}\AgdaField{Σ.fst} \AgdaBound{⟦Γ⟧}\AgdaSymbol{)} \AgdaInductiveConstructor{,} \AgdaField{Σ.snd} \AgdaBound{⟦Γ⟧}\AgdaSymbol{)}\<%
\\
\>[0]\AgdaIndent{3}{}\<[3]%
\>[3]\AgdaFunction{⟦} \AgdaBound{T} \AgdaInductiveConstructor{‘’} \AgdaBound{x} \AgdaFunction{⟧ᵀ} \AgdaBound{⟦Γ⟧} \AgdaSymbol{=} \AgdaFunction{⟦} \AgdaBound{T} \AgdaFunction{⟧ᵀ} \AgdaSymbol{(}\AgdaBound{⟦Γ⟧} \AgdaInductiveConstructor{,} \AgdaFunction{⟦} \AgdaBound{x} \AgdaFunction{⟧ᵗ} \AgdaBound{⟦Γ⟧}\AgdaSymbol{)}\<%
\\
\>[0]\AgdaIndent{3}{}\<[3]%
\>[3]\AgdaFunction{⟦} \AgdaInductiveConstructor{‘Type’} \AgdaBound{Γ} \AgdaFunction{⟧ᵀ} \AgdaBound{⟦Γ⟧}\<%
\\
\>[3]\AgdaIndent{5}{}\<[5]%
\>[5]\AgdaSymbol{=} \AgdaDatatype{Lifted} \AgdaSymbol{(}\AgdaDatatype{Type} \AgdaBound{Γ}\AgdaSymbol{)}\<%
\\
\>[0]\AgdaIndent{3}{}\<[3]%
\>[3]\AgdaFunction{⟦} \AgdaInductiveConstructor{‘Term’} \AgdaFunction{⟧ᵀ} \AgdaBound{⟦Γ⟧}\<%
\\
\>[3]\AgdaIndent{5}{}\<[5]%
\>[5]\AgdaSymbol{=} \AgdaDatatype{Lifted} \AgdaSymbol{(}\AgdaDatatype{Term} \AgdaSymbol{(}\AgdaFunction{lower} \AgdaSymbol{(}\AgdaField{Σ.snd} \AgdaBound{⟦Γ⟧}\AgdaSymbol{)))}\<%
\\
\>[0]\AgdaIndent{3}{}\<[3]%
\>[3]\AgdaFunction{⟦} \AgdaBound{A} \AgdaInductiveConstructor{‘→’} \AgdaBound{B} \AgdaFunction{⟧ᵀ} \AgdaBound{⟦Γ⟧} \AgdaSymbol{=} \AgdaFunction{⟦} \AgdaBound{A} \AgdaFunction{⟧ᵀ} \AgdaBound{⟦Γ⟧} \AgdaSymbol{→} \AgdaFunction{⟦} \AgdaBound{B} \AgdaFunction{⟧ᵀ} \AgdaBound{⟦Γ⟧}\<%
\\
\>[0]\AgdaIndent{3}{}\<[3]%
\>[3]\AgdaFunction{⟦} \AgdaBound{A} \AgdaInductiveConstructor{‘×’} \AgdaBound{B} \AgdaFunction{⟧ᵀ} \AgdaBound{⟦Γ⟧} \AgdaSymbol{=} \AgdaFunction{⟦} \AgdaBound{A} \AgdaFunction{⟧ᵀ} \AgdaBound{⟦Γ⟧} \AgdaFunction{×} \AgdaFunction{⟦} \AgdaBound{B} \AgdaFunction{⟧ᵀ} \AgdaBound{⟦Γ⟧}\<%
\\
\>[0]\AgdaIndent{3}{}\<[3]%
\>[3]\AgdaFunction{⟦} \AgdaInductiveConstructor{‘⊤’} \AgdaFunction{⟧ᵀ} \AgdaBound{⟦Γ⟧} \AgdaSymbol{=} \AgdaRecord{⊤}\<%
\\
\>[0]\AgdaIndent{3}{}\<[3]%
\>[3]\AgdaFunction{⟦} \AgdaInductiveConstructor{‘⊥’} \AgdaFunction{⟧ᵀ} \AgdaBound{⟦Γ⟧} \AgdaSymbol{=} \AgdaDatatype{⊥}\<%
\\
\>[0]\AgdaIndent{3}{}\<[3]%
\>[3]\AgdaFunction{⟦} \AgdaInductiveConstructor{Quine} \AgdaBound{φ} \AgdaFunction{⟧ᵀ} \AgdaBound{⟦Γ⟧} \AgdaSymbol{=} \AgdaFunction{⟦} \AgdaBound{φ} \AgdaFunction{⟧ᵀ} \AgdaSymbol{(}\AgdaBound{⟦Γ⟧} \AgdaInductiveConstructor{,} \AgdaSymbol{(}\AgdaInductiveConstructor{lift} \AgdaSymbol{(}\AgdaInductiveConstructor{Quine} \AgdaBound{φ}\AgdaSymbol{)))}\<%
\\
%
\\
\>[0]\AgdaIndent{3}{}\<[3]%
\>[3]\AgdaFunction{⟦\_⟧ᵗ} \AgdaSymbol{:} \AgdaSymbol{∀} \AgdaSymbol{\{}\AgdaBound{Γ} \AgdaSymbol{:} \AgdaDatatype{Context}\AgdaSymbol{\}} \AgdaSymbol{\{}\AgdaBound{T} \AgdaSymbol{:} \AgdaDatatype{Type} \AgdaBound{Γ}\AgdaSymbol{\}}\<%
\\
\>[3]\AgdaIndent{5}{}\<[5]%
\>[5]\AgdaSymbol{→} \AgdaDatatype{Term} \AgdaBound{T}\<%
\\
\>[3]\AgdaIndent{5}{}\<[5]%
\>[5]\AgdaSymbol{→} \AgdaSymbol{(}\AgdaBound{⟦Γ⟧} \AgdaSymbol{:} \AgdaFunction{⟦} \AgdaBound{Γ} \AgdaFunction{⟧ᶜ}\AgdaSymbol{)}\<%
\\
\>[3]\AgdaIndent{5}{}\<[5]%
\>[5]\AgdaSymbol{→} \AgdaFunction{⟦} \AgdaBound{T} \AgdaFunction{⟧ᵀ} \AgdaBound{⟦Γ⟧}\<%
\\
\>[0]\AgdaIndent{3}{}\<[3]%
\>[3]\AgdaFunction{⟦} \AgdaInductiveConstructor{⌜} \AgdaBound{x} \AgdaInductiveConstructor{⌝ᵀ} \AgdaFunction{⟧ᵗ} \AgdaBound{⟦Γ⟧} \AgdaSymbol{=} \AgdaInductiveConstructor{lift} \AgdaBound{x}\<%
\\
\>[0]\AgdaIndent{3}{}\<[3]%
\>[3]\AgdaFunction{⟦} \AgdaInductiveConstructor{⌜} \AgdaBound{x} \AgdaInductiveConstructor{⌝ᵗ} \AgdaFunction{⟧ᵗ} \AgdaBound{⟦Γ⟧} \AgdaSymbol{=} \AgdaInductiveConstructor{lift} \AgdaBound{x}\<%
\\
\>[0]\AgdaIndent{3}{}\<[3]%
\>[3]\AgdaFunction{⟦} \AgdaInductiveConstructor{‘⌜‘VAR₀’⌝ᵗ’} \AgdaFunction{⟧ᵗ} \AgdaBound{⟦Γ⟧}\<%
\\
\>[3]\AgdaIndent{5}{}\<[5]%
\>[5]\AgdaSymbol{=} \AgdaInductiveConstructor{lift} \AgdaInductiveConstructor{⌜} \AgdaFunction{lower} \AgdaSymbol{(}\AgdaField{Σ.snd} \AgdaBound{⟦Γ⟧}\AgdaSymbol{)} \AgdaInductiveConstructor{⌝ᵗ}\<%
\\
\>[0]\AgdaIndent{3}{}\<[3]%
\>[3]\AgdaFunction{⟦} \AgdaInductiveConstructor{‘⌜‘VAR₀’⌝ᵀ’} \AgdaFunction{⟧ᵗ} \AgdaBound{⟦Γ⟧}\<%
\\
\>[3]\AgdaIndent{5}{}\<[5]%
\>[5]\AgdaSymbol{=} \AgdaInductiveConstructor{lift} \AgdaInductiveConstructor{⌜} \AgdaFunction{lower} \AgdaSymbol{(}\AgdaField{Σ.snd} \AgdaBound{⟦Γ⟧}\AgdaSymbol{)} \AgdaInductiveConstructor{⌝ᵀ}\<%
\\
\>[0]\AgdaIndent{3}{}\<[3]%
\>[3]\AgdaFunction{⟦} \AgdaBound{f} \AgdaInductiveConstructor{‘’ₐ} \AgdaBound{x} \AgdaFunction{⟧ᵗ} \AgdaBound{⟦Γ⟧} \AgdaSymbol{=} \AgdaFunction{⟦} \AgdaBound{f} \AgdaFunction{⟧ᵗ} \AgdaBound{⟦Γ⟧} \AgdaSymbol{(}\AgdaFunction{⟦} \AgdaBound{x} \AgdaFunction{⟧ᵗ} \AgdaBound{⟦Γ⟧}\AgdaSymbol{)}\<%
\\
\>[0]\AgdaIndent{3}{}\<[3]%
\>[3]\AgdaFunction{⟦} \AgdaInductiveConstructor{‘tt’} \AgdaFunction{⟧ᵗ} \AgdaBound{⟦Γ⟧} \AgdaSymbol{=} \AgdaInductiveConstructor{tt}\<%
\\
\>[0]\AgdaIndent{3}{}\<[3]%
\>[3]\AgdaFunction{⟦} \AgdaInductiveConstructor{quine→} \AgdaSymbol{\{}\AgdaBound{φ}\AgdaSymbol{\}} \AgdaFunction{⟧ᵗ} \AgdaBound{⟦Γ⟧} \AgdaBound{x} \AgdaSymbol{=} \AgdaBound{x}\<%
\\
\>[0]\AgdaIndent{3}{}\<[3]%
\>[3]\AgdaFunction{⟦} \AgdaInductiveConstructor{quine←} \AgdaSymbol{\{}\AgdaBound{φ}\AgdaSymbol{\}} \AgdaFunction{⟧ᵗ} \AgdaBound{⟦Γ⟧} \AgdaBound{x} \AgdaSymbol{=} \AgdaBound{x}\<%
\\
\>[0]\AgdaIndent{3}{}\<[3]%
\>[3]\AgdaFunction{⟦} \AgdaInductiveConstructor{‘λ’} \AgdaBound{f} \AgdaFunction{⟧ᵗ} \AgdaBound{⟦Γ⟧} \AgdaBound{x} \AgdaSymbol{=} \AgdaFunction{⟦} \AgdaBound{f} \AgdaFunction{⟧ᵗ} \AgdaSymbol{(}\AgdaBound{⟦Γ⟧} \AgdaInductiveConstructor{,} \AgdaBound{x}\AgdaSymbol{)}\<%
\\
\>[0]\AgdaIndent{3}{}\<[3]%
\>[3]\AgdaFunction{⟦} \AgdaInductiveConstructor{‘VAR₀’} \AgdaFunction{⟧ᵗ} \AgdaBound{⟦Γ⟧} \AgdaSymbol{=} \AgdaField{Σ.snd} \AgdaBound{⟦Γ⟧}\<%
\\
\>[0]\AgdaIndent{3}{}\<[3]%
\>[3]\AgdaFunction{⟦} \AgdaInductiveConstructor{SW} \AgdaBound{t} \AgdaFunction{⟧ᵗ} \AgdaSymbol{=} \AgdaFunction{⟦} \AgdaBound{t} \AgdaFunction{⟧ᵗ}\<%
\\
\>[0]\AgdaIndent{3}{}\<[3]%
\>[3]\AgdaFunction{⟦} \AgdaInductiveConstructor{←SW₁SV→W} \AgdaBound{f} \AgdaFunction{⟧ᵗ} \AgdaSymbol{=} \AgdaFunction{⟦} \AgdaBound{f} \AgdaFunction{⟧ᵗ}\<%
\\
\>[0]\AgdaIndent{3}{}\<[3]%
\>[3]\AgdaFunction{⟦} \AgdaInductiveConstructor{→SW₁SV→W} \AgdaBound{f} \AgdaFunction{⟧ᵗ} \AgdaSymbol{=} \AgdaFunction{⟦} \AgdaBound{f} \AgdaFunction{⟧ᵗ}\<%
\\
\>[0]\AgdaIndent{3}{}\<[3]%
\>[3]\AgdaFunction{⟦} \AgdaInductiveConstructor{←SW₁SV→SW₁SV→W} \AgdaBound{f} \AgdaFunction{⟧ᵗ} \AgdaSymbol{=} \AgdaFunction{⟦} \AgdaBound{f} \AgdaFunction{⟧ᵗ}\<%
\\
\>[0]\AgdaIndent{3}{}\<[3]%
\>[3]\AgdaFunction{⟦} \AgdaInductiveConstructor{→SW₁SV→SW₁SV→W} \AgdaBound{f} \AgdaFunction{⟧ᵗ} \AgdaSymbol{=} \AgdaFunction{⟦} \AgdaBound{f} \AgdaFunction{⟧ᵗ}\<%
\\
\>[0]\AgdaIndent{3}{}\<[3]%
\>[3]\AgdaFunction{⟦} \AgdaInductiveConstructor{w} \AgdaBound{x} \AgdaFunction{⟧ᵗ} \AgdaBound{⟦Γ⟧} \AgdaSymbol{=} \AgdaFunction{⟦} \AgdaBound{x} \AgdaFunction{⟧ᵗ} \AgdaSymbol{(}\AgdaField{Σ.fst} \AgdaBound{⟦Γ⟧}\AgdaSymbol{)}\<%
\\
\>[0]\AgdaIndent{3}{}\<[3]%
\>[3]\AgdaFunction{⟦} \AgdaInductiveConstructor{w→} \AgdaBound{f} \AgdaFunction{⟧ᵗ} \AgdaBound{⟦Γ⟧} \AgdaSymbol{=} \AgdaFunction{⟦} \AgdaBound{f} \AgdaFunction{⟧ᵗ} \AgdaBound{⟦Γ⟧}\<%
\\
\>[0]\AgdaIndent{3}{}\<[3]%
\>[3]\AgdaFunction{⟦} \AgdaInductiveConstructor{→w} \AgdaBound{f} \AgdaFunction{⟧ᵗ} \AgdaBound{⟦Γ⟧} \AgdaSymbol{=} \AgdaFunction{⟦} \AgdaBound{f} \AgdaFunction{⟧ᵗ} \AgdaBound{⟦Γ⟧}\<%
\\
\>[0]\AgdaIndent{3}{}\<[3]%
\>[3]\AgdaFunction{⟦} \AgdaInductiveConstructor{ww→} \AgdaBound{f} \AgdaFunction{⟧ᵗ} \AgdaBound{⟦Γ⟧} \AgdaSymbol{=} \AgdaFunction{⟦} \AgdaBound{f} \AgdaFunction{⟧ᵗ} \AgdaBound{⟦Γ⟧}\<%
\\
\>[0]\AgdaIndent{3}{}\<[3]%
\>[3]\AgdaFunction{⟦} \AgdaInductiveConstructor{→ww} \AgdaBound{f} \AgdaFunction{⟧ᵗ} \AgdaBound{⟦Γ⟧} \AgdaSymbol{=} \AgdaFunction{⟦} \AgdaBound{f} \AgdaFunction{⟧ᵗ} \AgdaBound{⟦Γ⟧}\<%
\\
\>[0]\AgdaIndent{3}{}\<[3]%
\>[3]\AgdaFunction{⟦} \AgdaInductiveConstructor{‘‘×'’’} \AgdaFunction{⟧ᵗ} \AgdaBound{⟦Γ⟧} \AgdaBound{A} \AgdaBound{B} \AgdaSymbol{=} \AgdaInductiveConstructor{lift} \AgdaSymbol{(}\AgdaFunction{lower} \AgdaBound{A} \AgdaInductiveConstructor{‘×’} \AgdaFunction{lower} \AgdaBound{B}\AgdaSymbol{)}\<%
\\
\>[0]\AgdaIndent{3}{}\<[3]%
\>[3]\AgdaFunction{⟦} \AgdaBound{g} \AgdaInductiveConstructor{‘∘’} \AgdaBound{f} \AgdaFunction{⟧ᵗ} \AgdaBound{⟦Γ⟧} \AgdaBound{x} \AgdaSymbol{=} \AgdaFunction{⟦} \AgdaBound{g} \AgdaFunction{⟧ᵗ} \AgdaBound{⟦Γ⟧} \AgdaSymbol{(}\AgdaFunction{⟦} \AgdaBound{f} \AgdaFunction{⟧ᵗ} \AgdaBound{⟦Γ⟧} \AgdaBound{x}\AgdaSymbol{)}\<%
\\
\>[0]\AgdaIndent{3}{}\<[3]%
\>[3]\AgdaFunction{⟦} \AgdaBound{f} \AgdaInductiveConstructor{w‘‘’’ₐ} \AgdaBound{x} \AgdaFunction{⟧ᵗ} \AgdaBound{⟦Γ⟧}\<%
\\
\>[3]\AgdaIndent{5}{}\<[5]%
\>[5]\AgdaSymbol{=} \AgdaInductiveConstructor{lift} \AgdaSymbol{(}\AgdaFunction{lower} \AgdaSymbol{(}\AgdaFunction{⟦} \AgdaBound{f} \AgdaFunction{⟧ᵗ} \AgdaBound{⟦Γ⟧}\AgdaSymbol{)} \AgdaInductiveConstructor{‘’ₐ} \AgdaFunction{lower} \AgdaSymbol{(}\AgdaFunction{⟦} \AgdaBound{x} \AgdaFunction{⟧ᵗ} \AgdaBound{⟦Γ⟧}\AgdaSymbol{))}\<%
\\
\>[0]\AgdaIndent{3}{}\<[3]%
\>[3]\AgdaFunction{⟦} \AgdaInductiveConstructor{‘‘’ₐ’} \AgdaFunction{⟧ᵗ} \AgdaBound{⟦Γ⟧} \AgdaBound{f} \AgdaBound{x}\<%
\\
\>[3]\AgdaIndent{5}{}\<[5]%
\>[5]\AgdaSymbol{=} \AgdaInductiveConstructor{lift} \AgdaSymbol{(}\AgdaFunction{lower} \AgdaBound{f} \AgdaInductiveConstructor{‘’ₐ} \AgdaFunction{lower} \AgdaBound{x}\AgdaSymbol{)}\<%
\\
\>[0]\AgdaIndent{3}{}\<[3]%
\>[3]\AgdaFunction{⟦} \AgdaInductiveConstructor{‘‘□’’} \AgdaSymbol{\{}\AgdaBound{Γ}\AgdaSymbol{\}} \AgdaBound{T} \AgdaFunction{⟧ᵗ} \AgdaBound{⟦Γ⟧}\<%
\\
\>[3]\AgdaIndent{5}{}\<[5]%
\>[5]\AgdaSymbol{=} \AgdaInductiveConstructor{lift} \AgdaSymbol{(}\AgdaInductiveConstructor{‘Term’} \AgdaInductiveConstructor{‘’} \AgdaFunction{lower} \AgdaSymbol{(}\AgdaFunction{⟦} \AgdaBound{T} \AgdaFunction{⟧ᵗ} \AgdaBound{⟦Γ⟧}\AgdaSymbol{))}\<%
\\
\>[0]\AgdaIndent{3}{}\<[3]%
\>[3]\AgdaFunction{⟦} \AgdaBound{A} \AgdaInductiveConstructor{‘‘→’’} \AgdaBound{B} \AgdaFunction{⟧ᵗ} \AgdaBound{⟦Γ⟧}\<%
\\
\>[3]\AgdaIndent{5}{}\<[5]%
\>[5]\AgdaSymbol{=} \AgdaInductiveConstructor{lift}\<%
\\
\>[5]\AgdaIndent{7}{}\<[7]%
\>[7]\AgdaSymbol{(}\AgdaFunction{lower} \AgdaSymbol{(}\AgdaFunction{⟦} \AgdaBound{A} \AgdaFunction{⟧ᵗ} \AgdaBound{⟦Γ⟧}\AgdaSymbol{)} \AgdaInductiveConstructor{‘→’} \AgdaFunction{lower} \AgdaSymbol{(}\AgdaFunction{⟦} \AgdaBound{B} \AgdaFunction{⟧ᵗ} \AgdaBound{⟦Γ⟧}\AgdaSymbol{))}\<%
\\
\>[0]\AgdaIndent{3}{}\<[3]%
\>[3]\AgdaFunction{⟦} \AgdaBound{A} \AgdaInductiveConstructor{‘“→”’} \AgdaBound{B} \AgdaFunction{⟧ᵗ} \AgdaBound{⟦Γ⟧}\<%
\\
\>[3]\AgdaIndent{5}{}\<[5]%
\>[5]\AgdaSymbol{=} \AgdaInductiveConstructor{lift}\<%
\\
\>[5]\AgdaIndent{7}{}\<[7]%
\>[7]\AgdaSymbol{(}\AgdaFunction{lower} \AgdaSymbol{(}\AgdaFunction{⟦} \AgdaBound{A} \AgdaFunction{⟧ᵗ} \AgdaBound{⟦Γ⟧}\AgdaSymbol{)} \AgdaInductiveConstructor{‘‘→’’} \AgdaFunction{lower} \AgdaSymbol{(}\AgdaFunction{⟦} \AgdaBound{B} \AgdaFunction{⟧ᵗ} \AgdaBound{⟦Γ⟧}\AgdaSymbol{))}\<%
\\
\>[0]\AgdaIndent{3}{}\<[3]%
\>[3]\AgdaFunction{⟦} \AgdaBound{A} \AgdaInductiveConstructor{‘“×”’} \AgdaBound{B} \AgdaFunction{⟧ᵗ} \AgdaBound{⟦Γ⟧}\<%
\\
\>[3]\AgdaIndent{5}{}\<[5]%
\>[5]\AgdaSymbol{=} \AgdaInductiveConstructor{lift}\<%
\\
\>[5]\AgdaIndent{7}{}\<[7]%
\>[7]\AgdaSymbol{(}\AgdaFunction{lower} \AgdaSymbol{(}\AgdaFunction{⟦} \AgdaBound{A} \AgdaFunction{⟧ᵗ} \AgdaBound{⟦Γ⟧}\AgdaSymbol{)} \AgdaFunction{‘‘×’’} \AgdaFunction{lower} \AgdaSymbol{(}\AgdaFunction{⟦} \AgdaBound{B} \AgdaFunction{⟧ᵗ} \AgdaBound{⟦Γ⟧}\AgdaSymbol{))}\<%
\\
%
\\
%
\\
\>[0]\AgdaIndent{2}{}\<[2]%
\>[2]\AgdaKeyword{module} \AgdaModule{inner} \AgdaSymbol{(}\AgdaBound{‘X’} \AgdaSymbol{:} \AgdaDatatype{Type} \AgdaInductiveConstructor{ε}\AgdaSymbol{)}\<%
\\
\>[2]\AgdaIndent{15}{}\<[15]%
\>[15]\AgdaSymbol{(}\AgdaBound{‘f’} \AgdaSymbol{:} \AgdaDatatype{Term} \AgdaSymbol{\{}\AgdaInductiveConstructor{ε}\AgdaSymbol{\}} \AgdaSymbol{(}\AgdaFunction{‘□’} \AgdaBound{‘X’} \AgdaInductiveConstructor{‘→’} \AgdaBound{‘X’}\AgdaSymbol{))}\<%
\\
\>[0]\AgdaIndent{9}{}\<[9]%
\>[9]\AgdaKeyword{where}\<%
\\
\>[0]\AgdaIndent{3}{}\<[3]%
\>[3]\AgdaFunction{‘H’} \AgdaSymbol{:} \AgdaDatatype{Type} \AgdaInductiveConstructor{ε}\<%
\\
\>[0]\AgdaIndent{3}{}\<[3]%
\>[3]\AgdaFunction{‘H’} \AgdaSymbol{=} \AgdaInductiveConstructor{Quine} \AgdaSymbol{(}\AgdaInductiveConstructor{W₁} \AgdaInductiveConstructor{‘Term’} \AgdaInductiveConstructor{‘’} \AgdaInductiveConstructor{‘VAR₀’} \AgdaInductiveConstructor{‘→’} \AgdaInductiveConstructor{W} \AgdaBound{‘X’}\AgdaSymbol{)}\<%
\\
%
\\
\>[0]\AgdaIndent{3}{}\<[3]%
\>[3]\AgdaFunction{‘toH’} \AgdaSymbol{:} \AgdaFunction{□} \AgdaSymbol{((}\AgdaFunction{‘□’} \AgdaFunction{‘H’} \AgdaInductiveConstructor{‘→’} \AgdaBound{‘X’}\AgdaSymbol{)} \AgdaInductiveConstructor{‘→’} \AgdaFunction{‘H’}\AgdaSymbol{)}\<%
\\
\>[0]\AgdaIndent{3}{}\<[3]%
\>[3]\AgdaFunction{‘toH’} \AgdaSymbol{=} \AgdaInductiveConstructor{←SW₁SV→W} \AgdaInductiveConstructor{quine←}\<%
\\
%
\\
\>[0]\AgdaIndent{3}{}\<[3]%
\>[3]\AgdaFunction{‘fromH’} \AgdaSymbol{:} \AgdaFunction{□} \AgdaSymbol{(}\AgdaFunction{‘H’} \AgdaInductiveConstructor{‘→’} \AgdaSymbol{(}\AgdaFunction{‘□’} \AgdaFunction{‘H’} \AgdaInductiveConstructor{‘→’} \AgdaBound{‘X’}\AgdaSymbol{))}\<%
\\
\>[0]\AgdaIndent{3}{}\<[3]%
\>[3]\AgdaFunction{‘fromH’} \AgdaSymbol{=} \AgdaInductiveConstructor{→SW₁SV→W} \AgdaInductiveConstructor{quine→}\<%
\\
%
\\
\>[0]\AgdaIndent{3}{}\<[3]%
\>[3]\AgdaFunction{‘□‘H’→□‘X’’} \AgdaSymbol{:} \AgdaFunction{□} \AgdaSymbol{(}\AgdaFunction{‘□’} \AgdaFunction{‘H’} \AgdaInductiveConstructor{‘→’} \AgdaFunction{‘□’} \AgdaBound{‘X’}\AgdaSymbol{)}\<%
\\
\>[0]\AgdaIndent{3}{}\<[3]%
\>[3]\AgdaFunction{‘□‘H’→□‘X’’}\<%
\\
\>[3]\AgdaIndent{5}{}\<[5]%
\>[5]\AgdaSymbol{=} \AgdaInductiveConstructor{‘λ’} \AgdaSymbol{(}\AgdaInductiveConstructor{w} \AgdaInductiveConstructor{⌜} \AgdaFunction{‘fromH’} \AgdaInductiveConstructor{⌝ᵗ}\<%
\\
\>[5]\AgdaIndent{11}{}\<[11]%
\>[11]\AgdaInductiveConstructor{w‘‘’’ₐ} \AgdaInductiveConstructor{‘VAR₀’}\<%
\\
\>[5]\AgdaIndent{11}{}\<[11]%
\>[11]\AgdaInductiveConstructor{w‘‘’’ₐ} \AgdaInductiveConstructor{‘⌜‘VAR₀’⌝ᵗ’}\AgdaSymbol{)}\<%
\\
%
\\
\>[0]\AgdaIndent{3}{}\<[3]%
\>[3]\AgdaFunction{‘h’} \AgdaSymbol{:} \AgdaDatatype{Term} \AgdaFunction{‘H’}\<%
\\
\>[0]\AgdaIndent{3}{}\<[3]%
\>[3]\AgdaFunction{‘h’} \AgdaSymbol{=} \AgdaFunction{‘toH’} \AgdaInductiveConstructor{‘’ₐ} \AgdaSymbol{(}\AgdaBound{‘f’} \AgdaInductiveConstructor{‘∘’} \AgdaFunction{‘□‘H’→□‘X’’}\AgdaSymbol{)}\<%
\\
%
\\
\>[0]\AgdaIndent{3}{}\<[3]%
\>[3]\AgdaFunction{Lӧb} \AgdaSymbol{:} \AgdaFunction{□} \AgdaBound{‘X’}\<%
\\
\>[0]\AgdaIndent{3}{}\<[3]%
\>[3]\AgdaFunction{Lӧb} \AgdaSymbol{=} \AgdaFunction{‘fromH’} \AgdaInductiveConstructor{‘’ₐ} \AgdaFunction{‘h’} \AgdaInductiveConstructor{‘’ₐ} \AgdaInductiveConstructor{⌜} \AgdaFunction{‘h’} \AgdaInductiveConstructor{⌝ᵗ}\<%
\\
%
\\
\>[0]\AgdaIndent{2}{}\<[2]%
\>[2]\AgdaFunction{Lӧb} \AgdaSymbol{:} \AgdaSymbol{∀} \AgdaSymbol{\{}\AgdaBound{X}\AgdaSymbol{\}}\<%
\\
\>[2]\AgdaIndent{4}{}\<[4]%
\>[4]\AgdaSymbol{→} \AgdaDatatype{Term} \AgdaSymbol{\{}\AgdaInductiveConstructor{ε}\AgdaSymbol{\}} \AgdaSymbol{(}\AgdaFunction{‘□’} \AgdaBound{X} \AgdaInductiveConstructor{‘→’} \AgdaBound{X}\AgdaSymbol{)} \AgdaSymbol{→} \AgdaDatatype{Term} \AgdaSymbol{\{}\AgdaInductiveConstructor{ε}\AgdaSymbol{\}} \AgdaBound{X}\<%
\\
\>[0]\AgdaIndent{2}{}\<[2]%
\>[2]\AgdaFunction{Lӧb} \AgdaSymbol{\{}\AgdaBound{X}\AgdaSymbol{\}} \AgdaBound{f} \AgdaSymbol{=} \AgdaFunction{inner.Lӧb} \AgdaBound{X} \AgdaBound{f}\<%
\\
%
\\
\>[0]\AgdaIndent{2}{}\<[2]%
\>[2]\AgdaFunction{⟦\_⟧} \AgdaSymbol{:} \AgdaDatatype{Type} \AgdaInductiveConstructor{ε} \AgdaSymbol{→} \AgdaPrimitiveType{Set} \AgdaSymbol{\_}\<%
\\
\>[0]\AgdaIndent{2}{}\<[2]%
\>[2]\AgdaFunction{⟦} \AgdaBound{T} \AgdaFunction{⟧} \AgdaSymbol{=} \AgdaFunction{⟦} \AgdaBound{T} \AgdaFunction{⟧ᵀ} \AgdaInductiveConstructor{tt}\<%
\\
%
\\
\>[0]\AgdaIndent{2}{}\<[2]%
\>[2]\AgdaFunction{‘¬’\_} \AgdaSymbol{:} \AgdaSymbol{∀} \AgdaSymbol{\{}\AgdaBound{Γ}\AgdaSymbol{\}} \AgdaSymbol{→} \AgdaDatatype{Type} \AgdaBound{Γ} \AgdaSymbol{→} \AgdaDatatype{Type} \AgdaBound{Γ}\<%
\\
\>[0]\AgdaIndent{2}{}\<[2]%
\>[2]\AgdaFunction{‘¬’} \AgdaBound{T} \AgdaSymbol{=} \AgdaBound{T} \AgdaInductiveConstructor{‘→’} \AgdaInductiveConstructor{‘⊥’}\<%
\\
%
\\
\>[0]\AgdaIndent{2}{}\<[2]%
\>[2]\AgdaFunction{\_w‘‘×’’\_} \AgdaSymbol{:} \AgdaSymbol{∀} \AgdaSymbol{\{}\AgdaBound{Γ} \AgdaBound{X}\AgdaSymbol{\}}\<%
\\
\>[2]\AgdaIndent{4}{}\<[4]%
\>[4]\AgdaSymbol{→} \AgdaDatatype{Term} \AgdaSymbol{\{}\AgdaBound{Γ} \AgdaInductiveConstructor{▻} \AgdaBound{X}\AgdaSymbol{\}} \AgdaSymbol{(}\AgdaInductiveConstructor{W} \AgdaSymbol{(}\AgdaInductiveConstructor{‘Type’} \AgdaBound{Γ}\AgdaSymbol{))}\<%
\\
\>[2]\AgdaIndent{4}{}\<[4]%
\>[4]\AgdaSymbol{→} \AgdaDatatype{Term} \AgdaSymbol{\{}\AgdaBound{Γ} \AgdaInductiveConstructor{▻} \AgdaBound{X}\AgdaSymbol{\}} \AgdaSymbol{(}\AgdaInductiveConstructor{W} \AgdaSymbol{(}\AgdaInductiveConstructor{‘Type’} \AgdaBound{Γ}\AgdaSymbol{))}\<%
\\
\>[2]\AgdaIndent{4}{}\<[4]%
\>[4]\AgdaSymbol{→} \AgdaDatatype{Term} \AgdaSymbol{\{}\AgdaBound{Γ} \AgdaInductiveConstructor{▻} \AgdaBound{X}\AgdaSymbol{\}} \AgdaSymbol{(}\AgdaInductiveConstructor{W} \AgdaSymbol{(}\AgdaInductiveConstructor{‘Type’} \AgdaBound{Γ}\AgdaSymbol{))}\<%
\\
\>[0]\AgdaIndent{2}{}\<[2]%
\>[2]\AgdaBound{A} \AgdaFunction{w‘‘×’’} \AgdaBound{B} \AgdaSymbol{=} \AgdaInductiveConstructor{w→} \AgdaSymbol{(}\AgdaInductiveConstructor{w→} \AgdaSymbol{(}\AgdaInductiveConstructor{w} \AgdaInductiveConstructor{‘‘×'’’}\AgdaSymbol{)} \AgdaInductiveConstructor{‘’ₐ} \AgdaBound{A}\AgdaSymbol{)} \AgdaInductiveConstructor{‘’ₐ} \AgdaBound{B}\<%
\\
%
\\
\>[0]\AgdaIndent{2}{}\<[2]%
\>[2]\AgdaFunction{lӧb} \AgdaSymbol{:} \AgdaSymbol{∀} \AgdaSymbol{\{}\AgdaBound{‘X’}\AgdaSymbol{\}} \AgdaSymbol{→} \AgdaFunction{□} \AgdaSymbol{(}\AgdaFunction{‘□’} \AgdaBound{‘X’} \AgdaInductiveConstructor{‘→’} \AgdaBound{‘X’}\AgdaSymbol{)} \AgdaSymbol{→} \AgdaFunction{⟦} \AgdaBound{‘X’} \AgdaFunction{⟧}\<%
\\
\>[0]\AgdaIndent{2}{}\<[2]%
\>[2]\AgdaFunction{lӧb} \AgdaBound{f} \AgdaSymbol{=} \AgdaFunction{⟦} \AgdaFunction{Lӧb} \AgdaBound{f} \AgdaFunction{⟧ᵗ} \AgdaInductiveConstructor{tt}\<%
\\
%
\\
\>[0]\AgdaIndent{2}{}\<[2]%
\>[2]\AgdaFunction{incompleteness} \AgdaSymbol{:} \AgdaFunction{¬} \AgdaFunction{□} \AgdaSymbol{(}\AgdaFunction{‘¬’} \AgdaSymbol{(}\AgdaFunction{‘□’} \AgdaInductiveConstructor{‘⊥’}\AgdaSymbol{))}\<%
\\
\>[0]\AgdaIndent{2}{}\<[2]%
\>[2]\AgdaFunction{incompleteness} \AgdaSymbol{=} \AgdaFunction{lӧb}\<%
\\
%
\\
\>[0]\AgdaIndent{2}{}\<[2]%
\>[2]\AgdaFunction{soundness} \AgdaSymbol{:} \AgdaFunction{¬} \AgdaFunction{□} \AgdaInductiveConstructor{‘⊥’}\<%
\\
\>[0]\AgdaIndent{2}{}\<[2]%
\>[2]\AgdaFunction{soundness} \AgdaBound{x} \AgdaSymbol{=} \AgdaFunction{⟦} \AgdaBound{x} \AgdaFunction{⟧ᵗ} \AgdaInductiveConstructor{tt}\<%
\\
%
\\
\>[0]\AgdaIndent{2}{}\<[2]%
\>[2]\AgdaFunction{non-emptiness} \AgdaSymbol{:} \AgdaRecord{Σ} \AgdaSymbol{(}\AgdaDatatype{Type} \AgdaInductiveConstructor{ε}\AgdaSymbol{)} \AgdaSymbol{(λ} \AgdaBound{T} \AgdaSymbol{→} \AgdaFunction{□} \AgdaBound{T}\AgdaSymbol{)}\<%
\\
\>[0]\AgdaIndent{2}{}\<[2]%
\>[2]\AgdaFunction{non-emptiness} \AgdaSymbol{=} \AgdaInductiveConstructor{‘⊤’} \AgdaInductiveConstructor{,} \AgdaInductiveConstructor{‘tt’}\<%
\end{code}

\AgdaHide{
  \begin{code}%
\>\AgdaKeyword{import} \AgdaModule{fair-bot-self-cooperates}\<%
\end{code}
}
\section{Proving that FairBot Cooperates with Itself} \label{sec:fair-bot-self-cooperates}

\AgdaHide{
  \begin{code}%
\>\AgdaKeyword{module} \AgdaModule{fair-bot-self-cooperates} \AgdaKeyword{where}\<%
\\
\>\AgdaKeyword{open} \AgdaKeyword{import} \AgdaModule{common}\<%
\end{code}
}

We begin with the definitions of a few particularly useful dependent
combinators:

\begin{code}%
\>\AgdaFunction{\_∘\_} \AgdaSymbol{:} \AgdaSymbol{∀} \AgdaSymbol{\{}\AgdaBound{A} \AgdaSymbol{:} \AgdaPrimitiveType{Set}\AgdaSymbol{\}}\<%
\\
\>[0]\AgdaIndent{2}{}\<[2]%
\>[2]\AgdaSymbol{\{}\AgdaBound{B} \AgdaSymbol{:} \AgdaBound{A} \AgdaSymbol{→} \AgdaPrimitiveType{Set}\AgdaSymbol{\}}\<%
\\
\>[0]\AgdaIndent{2}{}\<[2]%
\>[2]\AgdaSymbol{\{}\AgdaBound{C} \AgdaSymbol{:} \AgdaSymbol{\{}\AgdaBound{x} \AgdaSymbol{:} \AgdaBound{A}\AgdaSymbol{\}} \AgdaSymbol{→} \AgdaBound{B} \AgdaBound{x} \AgdaSymbol{→} \AgdaPrimitiveType{Set}\AgdaSymbol{\}}\<%
\\
\>[0]\AgdaIndent{2}{}\<[2]%
\>[2]\AgdaSymbol{→} \AgdaSymbol{(\{}\AgdaBound{x} \AgdaSymbol{:} \AgdaBound{A}\AgdaSymbol{\}} \AgdaSymbol{(}\AgdaBound{y} \AgdaSymbol{:} \AgdaBound{B} \AgdaBound{x}\AgdaSymbol{)} \AgdaSymbol{→} \AgdaBound{C} \AgdaBound{y}\AgdaSymbol{)}\<%
\\
\>[0]\AgdaIndent{2}{}\<[2]%
\>[2]\AgdaSymbol{→} \AgdaSymbol{(}\AgdaBound{g} \AgdaSymbol{:} \AgdaSymbol{(}\AgdaBound{x} \AgdaSymbol{:} \AgdaBound{A}\AgdaSymbol{)} \AgdaSymbol{→} \AgdaBound{B} \AgdaBound{x}\AgdaSymbol{)} \AgdaSymbol{(}\AgdaBound{x} \AgdaSymbol{:} \AgdaBound{A}\AgdaSymbol{)}\<%
\\
\>[0]\AgdaIndent{2}{}\<[2]%
\>[2]\AgdaSymbol{→} \AgdaBound{C} \AgdaSymbol{(}\AgdaBound{g} \AgdaBound{x}\AgdaSymbol{)}\<%
\\
\>\AgdaBound{f} \AgdaFunction{∘} \AgdaBound{g} \AgdaSymbol{=} \AgdaSymbol{λ} \AgdaBound{x} \AgdaSymbol{→} \AgdaBound{f} \AgdaSymbol{(}\AgdaBound{g} \AgdaBound{x}\AgdaSymbol{)}\<%
\\
%
\\
\>\AgdaKeyword{infixl} \AgdaNumber{8} \AgdaFixityOp{\_ˢ\_}\<%
\\
%
\\
\>\AgdaFunction{\_ˢ\_} \AgdaSymbol{:} \AgdaSymbol{∀} \AgdaSymbol{\{}\AgdaBound{A} \AgdaSymbol{:} \AgdaPrimitiveType{Set}\AgdaSymbol{\}}\<%
\\
\>[0]\AgdaIndent{2}{}\<[2]%
\>[2]\AgdaSymbol{\{}\AgdaBound{B} \AgdaSymbol{:} \AgdaBound{A} \AgdaSymbol{→} \AgdaPrimitiveType{Set}\AgdaSymbol{\}}\<%
\\
\>[0]\AgdaIndent{2}{}\<[2]%
\>[2]\AgdaSymbol{\{}\AgdaBound{C} \AgdaSymbol{:} \AgdaSymbol{(}\AgdaBound{x} \AgdaSymbol{:} \AgdaBound{A}\AgdaSymbol{)} \AgdaSymbol{→} \AgdaBound{B} \AgdaBound{x} \AgdaSymbol{→} \AgdaPrimitiveType{Set}\AgdaSymbol{\}}\<%
\\
\>[0]\AgdaIndent{2}{}\<[2]%
\>[2]\AgdaSymbol{→} \AgdaSymbol{((}\AgdaBound{x} \AgdaSymbol{:} \AgdaBound{A}\AgdaSymbol{)} \AgdaSymbol{(}\AgdaBound{y} \AgdaSymbol{:} \AgdaBound{B} \AgdaBound{x}\AgdaSymbol{)} \AgdaSymbol{→} \AgdaBound{C} \AgdaBound{x} \AgdaBound{y}\AgdaSymbol{)}\<%
\\
\>[0]\AgdaIndent{2}{}\<[2]%
\>[2]\AgdaSymbol{→} \AgdaSymbol{(}\AgdaBound{g} \AgdaSymbol{:} \AgdaSymbol{(}\AgdaBound{x} \AgdaSymbol{:} \AgdaBound{A}\AgdaSymbol{)} \AgdaSymbol{→} \AgdaBound{B} \AgdaBound{x}\AgdaSymbol{)} \AgdaSymbol{(}\AgdaBound{x} \AgdaSymbol{:} \AgdaBound{A}\AgdaSymbol{)}\<%
\\
\>[0]\AgdaIndent{2}{}\<[2]%
\>[2]\AgdaSymbol{→} \AgdaBound{C} \AgdaBound{x} \AgdaSymbol{(}\AgdaBound{g} \AgdaBound{x}\AgdaSymbol{)}\<%
\\
\>\AgdaBound{f} \AgdaFunction{ˢ} \AgdaBound{g} \AgdaSymbol{=} \AgdaSymbol{λ} \AgdaBound{x} \AgdaSymbol{→} \AgdaBound{f} \AgdaBound{x} \AgdaSymbol{(}\AgdaBound{g} \AgdaBound{x}\AgdaSymbol{)}\<%
\\
%
\\
\>\AgdaFunction{ᵏ} \AgdaSymbol{:} \AgdaSymbol{\{}\AgdaBound{A} \AgdaBound{B} \AgdaSymbol{:} \AgdaPrimitiveType{Set}\AgdaSymbol{\}} \AgdaSymbol{→} \AgdaBound{A} \AgdaSymbol{→} \AgdaBound{B} \AgdaSymbol{→} \AgdaBound{A}\<%
\\
\>\AgdaFunction{ᵏ} \AgdaBound{a} \AgdaBound{b} \AgdaSymbol{=} \AgdaBound{a}\<%
\\
%
\\
\>\AgdaFunction{\textasciicircum} \AgdaSymbol{:} \AgdaSymbol{∀} \AgdaSymbol{\{}\AgdaBound{S} \AgdaSymbol{:} \AgdaPrimitiveType{Set}\AgdaSymbol{\}} \AgdaSymbol{\{}\AgdaBound{T} \AgdaSymbol{:} \AgdaBound{S} \AgdaSymbol{→} \AgdaPrimitiveType{Set}\AgdaSymbol{\}} \AgdaSymbol{\{}\AgdaBound{P} \AgdaSymbol{:} \AgdaRecord{Σ} \AgdaBound{S} \AgdaBound{T} \AgdaSymbol{→} \AgdaPrimitiveType{Set}\AgdaSymbol{\}}\<%
\\
\>[2]\AgdaIndent{4}{}\<[4]%
\>[4]\AgdaSymbol{→} \AgdaSymbol{((}\AgdaBound{σ} \AgdaSymbol{:} \AgdaRecord{Σ} \AgdaBound{S} \AgdaBound{T}\AgdaSymbol{)} \AgdaSymbol{→} \AgdaBound{P} \AgdaBound{σ}\AgdaSymbol{)}\<%
\\
\>[2]\AgdaIndent{4}{}\<[4]%
\>[4]\AgdaSymbol{→} \AgdaSymbol{(}\AgdaBound{s} \AgdaSymbol{:} \AgdaBound{S}\AgdaSymbol{)} \AgdaSymbol{(}\AgdaBound{t} \AgdaSymbol{:} \AgdaBound{T} \AgdaBound{s}\AgdaSymbol{)} \AgdaSymbol{→} \AgdaBound{P} \AgdaSymbol{(}\AgdaBound{s} \AgdaInductiveConstructor{,} \AgdaBound{t}\AgdaSymbol{)}\<%
\\
\>\AgdaFunction{\textasciicircum} \AgdaBound{f} \AgdaBound{s} \AgdaBound{t} \AgdaSymbol{=} \AgdaBound{f} \AgdaSymbol{(}\AgdaBound{s} \AgdaInductiveConstructor{,} \AgdaBound{t}\AgdaSymbol{)}\<%
\end{code}

It turns out that we can define all the things we need for proving
self-cooperation of FairBot in a variant of the simply typed lambda
calculus (STLC).  In order to do this, we do not index types over
contexts.  Rather than using \mintinline{Agda}|Term {Γ} T|, we will
denote the type of terms in context \mintinline{Agda}|Γ| of type
\mintinline{Agda}|T| as \mintinline{Agda}|Γ ⊢ T|, the standard
notation for ``provability''.  Since our types are no longer indexed
over contexts, we can represent a context as a list of types.

\begin{code}%
\>\AgdaKeyword{infixr} \AgdaNumber{5} \AgdaFixityOp{\_⊢\_} \AgdaFixityOp{\_‘⊢’\_}\<%
\\
\>\AgdaKeyword{infixr} \AgdaNumber{10} \AgdaFixityOp{\_‘→’\_} \AgdaFixityOp{\_‘×’\_}\<%
\\
%
\\
\>\AgdaKeyword{data} \AgdaDatatype{Type} \AgdaSymbol{:} \AgdaPrimitiveType{Set} \AgdaKeyword{where}\<%
\\
\>[0]\AgdaIndent{2}{}\<[2]%
\>[2]\AgdaInductiveConstructor{\_‘⊢’\_} \AgdaSymbol{:} \AgdaDatatype{List} \AgdaDatatype{Type} \AgdaSymbol{→} \AgdaDatatype{Type} \AgdaSymbol{→} \AgdaDatatype{Type}\<%
\\
\>[0]\AgdaIndent{2}{}\<[2]%
\>[2]\AgdaInductiveConstructor{\_‘→’\_} \AgdaInductiveConstructor{\_‘×’\_} \AgdaSymbol{:} \AgdaDatatype{Type} \AgdaSymbol{→} \AgdaDatatype{Type} \AgdaSymbol{→} \AgdaDatatype{Type}\<%
\\
\>[0]\AgdaIndent{2}{}\<[2]%
\>[2]\AgdaInductiveConstructor{‘⊥’} \AgdaInductiveConstructor{‘⊤’} \AgdaSymbol{:} \AgdaDatatype{Type}\<%
\\
%
\\
\>\AgdaFunction{Context} \AgdaSymbol{=} \AgdaDatatype{List} \AgdaDatatype{Type}\<%
\end{code}

We will then need some way to handle binding.
For simplicity, we'll make use of a dependent form of DeBrujin variables.

\begin{code}%
\>\AgdaKeyword{data} \AgdaDatatype{\_∈\_} \AgdaSymbol{(}\AgdaBound{T} \AgdaSymbol{:} \AgdaDatatype{Type}\AgdaSymbol{)} \AgdaSymbol{:} \AgdaFunction{Context} \AgdaSymbol{→} \AgdaPrimitiveType{Set} \AgdaKeyword{where}\<%
\end{code}

First we want our ``variable zero'', which lets us pick off the ``top'' element of the context.

\begin{code}%
\>[0]\AgdaIndent{2}{}\<[2]%
\>[2]\AgdaInductiveConstructor{top} \AgdaSymbol{:} \AgdaSymbol{∀} \AgdaSymbol{\{}\AgdaBound{Γ}\AgdaSymbol{\}} \AgdaSymbol{→} \AgdaBound{T} \AgdaDatatype{∈} \AgdaSymbol{(}\AgdaBound{T} \AgdaInductiveConstructor{::} \AgdaBound{Γ}\AgdaSymbol{)}\<%
\end{code}

Then we want a way to extend variables to work in larger contexts.

\begin{code}%
\>[0]\AgdaIndent{2}{}\<[2]%
\>[2]\AgdaInductiveConstructor{pop} \AgdaSymbol{:} \AgdaSymbol{∀} \AgdaSymbol{\{}\AgdaBound{Γ} \AgdaBound{S}\AgdaSymbol{\}} \AgdaSymbol{→} \AgdaBound{T} \AgdaDatatype{∈} \AgdaBound{Γ} \AgdaSymbol{→} \AgdaBound{T} \AgdaDatatype{∈} \AgdaSymbol{(}\AgdaBound{S} \AgdaInductiveConstructor{::} \AgdaBound{Γ}\AgdaSymbol{)}\<%
\end{code}

And, finally, we are ready to define the term language for our extended STLC.

\begin{code}%
\>\AgdaKeyword{data} \AgdaDatatype{\_⊢\_} \AgdaSymbol{(}\AgdaBound{Γ} \AgdaSymbol{:} \AgdaFunction{Context}\AgdaSymbol{)} \AgdaSymbol{:} \AgdaDatatype{Type} \AgdaSymbol{→} \AgdaPrimitiveType{Set} \AgdaKeyword{where}\<%
\end{code}

The next few constructors are fairly standard.
Before anything else, we want to be able to lift bindings into terms.

\begin{code}%
\>[0]\AgdaIndent{2}{}\<[2]%
\>[2]\AgdaInductiveConstructor{var} \AgdaSymbol{:} \AgdaSymbol{∀} \AgdaSymbol{\{}\AgdaBound{T}\AgdaSymbol{\}} \AgdaSymbol{→} \AgdaBound{T} \AgdaDatatype{∈} \AgdaBound{Γ} \AgdaSymbol{→} \AgdaBound{Γ} \AgdaDatatype{⊢} \AgdaBound{T}\<%
\end{code}

Then the intro rules for all of our easier datatypes.

\begin{code}%
\>[0]\AgdaIndent{2}{}\<[2]%
\>[2]\AgdaInductiveConstructor{<>} \AgdaSymbol{:} \AgdaBound{Γ} \AgdaDatatype{⊢} \AgdaInductiveConstructor{‘⊤’}\<%
\\
\>[0]\AgdaIndent{2}{}\<[2]%
\>[2]\AgdaInductiveConstructor{\_,\_} \AgdaSymbol{:} \AgdaSymbol{∀} \AgdaSymbol{\{}\AgdaBound{A} \AgdaBound{B}\AgdaSymbol{\}} \AgdaSymbol{→} \AgdaBound{Γ} \AgdaDatatype{⊢} \AgdaBound{A} \AgdaSymbol{→} \AgdaBound{Γ} \AgdaDatatype{⊢} \AgdaBound{B} \AgdaSymbol{→} \AgdaBound{Γ} \AgdaDatatype{⊢} \AgdaBound{A} \AgdaInductiveConstructor{‘×’} \AgdaBound{B}\<%
\\
\>[0]\AgdaIndent{2}{}\<[2]%
\>[2]\AgdaInductiveConstructor{‘⊥’-elim} \AgdaSymbol{:} \AgdaSymbol{∀} \AgdaSymbol{\{}\AgdaBound{A}\AgdaSymbol{\}} \AgdaSymbol{→} \AgdaBound{Γ} \AgdaDatatype{⊢} \AgdaInductiveConstructor{‘⊥’} \AgdaSymbol{→} \AgdaBound{Γ} \AgdaDatatype{⊢} \AgdaBound{A}\<%
\\
\>[0]\AgdaIndent{2}{}\<[2]%
\>[2]\AgdaInductiveConstructor{π₁} \AgdaSymbol{:} \AgdaSymbol{∀} \AgdaSymbol{\{}\AgdaBound{A} \AgdaBound{B}\AgdaSymbol{\}} \AgdaSymbol{→} \AgdaBound{Γ} \AgdaDatatype{⊢} \AgdaBound{A} \AgdaInductiveConstructor{‘×’} \AgdaBound{B} \AgdaSymbol{→} \AgdaBound{Γ} \AgdaDatatype{⊢} \AgdaBound{A}\<%
\\
\>[0]\AgdaIndent{2}{}\<[2]%
\>[2]\AgdaInductiveConstructor{π₂} \AgdaSymbol{:} \AgdaSymbol{∀} \AgdaSymbol{\{}\AgdaBound{A} \AgdaBound{B}\AgdaSymbol{\}} \AgdaSymbol{→} \AgdaBound{Γ} \AgdaDatatype{⊢} \AgdaBound{A} \AgdaInductiveConstructor{‘×’} \AgdaBound{B} \AgdaSymbol{→} \AgdaBound{Γ} \AgdaDatatype{⊢} \AgdaBound{B}\<%
\\
\>[0]\AgdaIndent{2}{}\<[2]%
\>[2]\AgdaInductiveConstructor{‘λ’} \AgdaSymbol{:} \AgdaSymbol{∀} \AgdaSymbol{\{}\AgdaBound{A} \AgdaBound{B}\AgdaSymbol{\}} \AgdaSymbol{→} \AgdaSymbol{(}\AgdaBound{A} \AgdaInductiveConstructor{::} \AgdaBound{Γ}\AgdaSymbol{)} \AgdaDatatype{⊢} \AgdaBound{B} \AgdaSymbol{→} \AgdaBound{Γ} \AgdaDatatype{⊢} \AgdaSymbol{(}\AgdaBound{A} \AgdaInductiveConstructor{‘→’} \AgdaBound{B}\AgdaSymbol{)}\<%
\\
\>[0]\AgdaIndent{2}{}\<[2]%
\>[2]\AgdaInductiveConstructor{\_‘’ₐ\_} \AgdaSymbol{:} \AgdaSymbol{∀} \AgdaSymbol{\{}\AgdaBound{A} \AgdaBound{B}\AgdaSymbol{\}} \AgdaSymbol{→} \AgdaBound{Γ} \AgdaDatatype{⊢} \AgdaSymbol{(}\AgdaBound{A} \AgdaInductiveConstructor{‘→’} \AgdaBound{B}\AgdaSymbol{)} \AgdaSymbol{→} \AgdaBound{Γ} \AgdaDatatype{⊢} \AgdaBound{A} \AgdaSymbol{→} \AgdaBound{Γ} \AgdaDatatype{⊢} \AgdaBound{B}\<%
\end{code}

At this point things become more delicate.
To properly capture Gӧdel--Lӧb modal logic, abbreviated as GL, we want our theory to validate the rules

\begin{enumerate}
\item \mintinline{Agda}|⊢ A → ⊢ □ A|
\item \mintinline{Agda}|⊢ □ A ‘→’ □ □ A|
\end{enumerate}

However, it should \emph{not} validate \mintinline{Agda}|⊢ A ‘→’ □ A|.
If we only had the unary \mintinline{Agda}|□| operator we would run into difficulty later.
Crucially, we couldn't add the rule \mintinline{Agda}|Γ ⊢ A → Γ ⊢ □ A|, since this would let us prove \mintinline{Agda}|A ‘→’ □ A|.

We will use Gödel quotes to denote the constructor corresponding to rule 1:

\begin{code}%
\>[0]\AgdaIndent{2}{}\<[2]%
\>[2]\AgdaInductiveConstructor{⌜\_⌝} \AgdaSymbol{:} \AgdaSymbol{∀} \AgdaSymbol{\{}\AgdaBound{Δ} \AgdaBound{A}\AgdaSymbol{\}} \AgdaSymbol{→} \AgdaBound{Δ} \AgdaDatatype{⊢} \AgdaBound{A} \AgdaSymbol{→} \AgdaBound{Γ} \AgdaDatatype{⊢} \AgdaSymbol{(}\AgdaBound{Δ} \AgdaInductiveConstructor{‘⊢’} \AgdaBound{A}\AgdaSymbol{)}\<%
\end{code}

Similarly, we will write the rule validating \mintinline{Agda}|□ A ‘→’ □ □ A| as \mintinline{Agda}|repr|.

\begin{code}%
\>[0]\AgdaIndent{2}{}\<[2]%
\>[2]\AgdaInductiveConstructor{repr} \AgdaSymbol{:} \AgdaSymbol{∀} \AgdaSymbol{\{}\AgdaBound{Δ} \AgdaBound{A}\AgdaSymbol{\}} \AgdaSymbol{→} \AgdaBound{Γ} \AgdaDatatype{⊢} \AgdaSymbol{(}\AgdaBound{Δ} \AgdaInductiveConstructor{‘⊢’} \AgdaBound{A}\AgdaSymbol{)} \AgdaSymbol{→} \AgdaBound{Γ} \AgdaDatatype{⊢} \AgdaSymbol{(}\AgdaBound{Δ} \AgdaInductiveConstructor{‘⊢’} \AgdaSymbol{(}\AgdaBound{Δ} \AgdaInductiveConstructor{‘⊢’} \AgdaBound{A}\AgdaSymbol{))}\<%
\end{code}

We would like to be able to apply functions under \mintinline{Agda}|□|, and for this we introduce the so-called ``distribution'' rule.
In GL, it takes the form \mintinline{Agda}|⊢ □ (A ‘→’ B) → ⊢ (□ A ‘→’ □ B)|.
For us it is not much more complicated.

\begin{code}%
\>[0]\AgdaIndent{2}{}\<[2]%
\>[2]\AgdaInductiveConstructor{dist} \AgdaSymbol{:} \AgdaSymbol{∀} \AgdaSymbol{\{}\AgdaBound{Δ} \AgdaBound{A} \AgdaBound{B}\AgdaSymbol{\}}\<%
\\
\>[2]\AgdaIndent{4}{}\<[4]%
\>[4]\AgdaSymbol{→} \AgdaBound{Γ} \AgdaDatatype{⊢} \AgdaSymbol{(}\AgdaBound{Δ} \AgdaInductiveConstructor{‘⊢’} \AgdaSymbol{(}\AgdaBound{A} \AgdaInductiveConstructor{‘→’} \AgdaBound{B}\AgdaSymbol{))}\<%
\\
\>[2]\AgdaIndent{4}{}\<[4]%
\>[4]\AgdaSymbol{→} \AgdaBound{Γ} \AgdaDatatype{⊢} \AgdaSymbol{(}\AgdaBound{Δ} \AgdaInductiveConstructor{‘⊢’} \AgdaBound{A}\AgdaSymbol{)}\<%
\\
\>[2]\AgdaIndent{4}{}\<[4]%
\>[4]\AgdaSymbol{→} \AgdaBound{Γ} \AgdaDatatype{⊢} \AgdaSymbol{(}\AgdaBound{Δ} \AgdaInductiveConstructor{‘⊢’} \AgdaBound{B}\AgdaSymbol{)}\<%
\end{code}

And, finally, we include the Löbian axiom.

\begin{code}%
\>[0]\AgdaIndent{2}{}\<[2]%
\>[2]\AgdaInductiveConstructor{Lӧb} \AgdaSymbol{:} \AgdaSymbol{∀} \AgdaSymbol{\{}\AgdaBound{Δ} \AgdaBound{A}\AgdaSymbol{\}}\<%
\\
\>[2]\AgdaIndent{4}{}\<[4]%
\>[4]\AgdaSymbol{→} \AgdaBound{Γ} \AgdaDatatype{⊢} \AgdaSymbol{(}\AgdaBound{Δ} \AgdaInductiveConstructor{‘⊢’} \AgdaSymbol{((}\AgdaBound{Δ} \AgdaInductiveConstructor{‘⊢’} \AgdaBound{A}\AgdaSymbol{)} \AgdaInductiveConstructor{‘→’} \AgdaBound{A}\AgdaSymbol{))}\<%
\\
\>[2]\AgdaIndent{4}{}\<[4]%
\>[4]\AgdaSymbol{→} \AgdaBound{Γ} \AgdaDatatype{⊢} \AgdaSymbol{(}\AgdaBound{Δ} \AgdaInductiveConstructor{‘⊢’} \AgdaBound{A}\AgdaSymbol{)}\<%
\end{code}

\AgdaHide{
\begin{code}%
\>\AgdaKeyword{infixl} \AgdaNumber{50} \AgdaFixityOp{\_‘’ₐ\_}\<%
\\
\>\<%
\end{code}
}

From these constructors we can prove the simpler form of the Lӧb rule.

\begin{code}%
\>\AgdaFunction{lӧb} \AgdaSymbol{:} \AgdaSymbol{∀} \AgdaSymbol{\{}\AgdaBound{Γ} \AgdaBound{A}\AgdaSymbol{\}} \AgdaSymbol{→} \AgdaBound{Γ} \AgdaDatatype{⊢} \AgdaSymbol{((}\AgdaBound{Γ} \AgdaInductiveConstructor{‘⊢’} \AgdaBound{A}\AgdaSymbol{)} \AgdaInductiveConstructor{‘→’} \AgdaBound{A}\AgdaSymbol{)} \AgdaSymbol{→} \AgdaBound{Γ} \AgdaDatatype{⊢} \AgdaBound{A}\<%
\\
\>\AgdaFunction{lӧb} \AgdaBound{t} \AgdaSymbol{=} \AgdaBound{t} \AgdaInductiveConstructor{‘’ₐ} \AgdaInductiveConstructor{Lӧb} \AgdaInductiveConstructor{⌜} \AgdaBound{t} \AgdaInductiveConstructor{⌝}\<%
\end{code}

Of course, because we are using DeBrujin indices, before we can do too much we'll need to give an account of lifting.
Thankfully, unlike when we were dealing with dependent type theory, we can define these computationally, and get for free all the congruences we had to add as axioms before.

Our definition of weakening is unremarkable, and sufficiently simple
that Agsy, Agda's automatic proof-finder, was able to fill in all of
the code; we include it in the artifact and elide all but the type
signature from the paper.

\AgdaHide{
\begin{code}%
\>\AgdaFunction{lift-var} \AgdaSymbol{:} \AgdaSymbol{∀} \AgdaSymbol{\{}\AgdaBound{Γ} \AgdaBound{A}\AgdaSymbol{\}} \AgdaBound{T} \AgdaBound{Δ} \AgdaSymbol{→} \AgdaBound{A} \AgdaDatatype{∈} \AgdaSymbol{(}\AgdaBound{Δ} \AgdaFunction{++} \AgdaBound{Γ}\AgdaSymbol{)} \AgdaSymbol{→} \AgdaBound{A} \AgdaDatatype{∈} \AgdaSymbol{(}\AgdaBound{Δ} \AgdaFunction{++} \AgdaSymbol{(}\AgdaBound{T} \AgdaInductiveConstructor{::} \AgdaBound{Γ}\AgdaSymbol{))}\<%
\\
\>\AgdaFunction{lift-var} \AgdaBound{T} \AgdaInductiveConstructor{ε} \AgdaBound{v} \AgdaSymbol{=} \AgdaInductiveConstructor{pop} \AgdaBound{v}\<%
\\
\>\AgdaFunction{lift-var} \AgdaBound{T} \AgdaSymbol{(}\AgdaBound{A} \AgdaInductiveConstructor{::} \AgdaBound{Δ}\AgdaSymbol{)} \AgdaInductiveConstructor{top} \AgdaSymbol{=} \AgdaInductiveConstructor{top}\<%
\\
\>\AgdaFunction{lift-var} \AgdaBound{T} \AgdaSymbol{(}\AgdaBound{x} \AgdaInductiveConstructor{::} \AgdaBound{Δ}\AgdaSymbol{)} \AgdaSymbol{(}\AgdaInductiveConstructor{pop} \AgdaBound{v}\AgdaSymbol{)} \AgdaSymbol{=} \AgdaInductiveConstructor{pop} \AgdaSymbol{(}\AgdaFunction{lift-var} \AgdaBound{T} \AgdaBound{Δ} \AgdaBound{v}\AgdaSymbol{)}\<%
\end{code}
}

\begin{code}%
\>\AgdaFunction{lift-tm}\<%
\\
\>[0]\AgdaIndent{2}{}\<[2]%
\>[2]\AgdaSymbol{:} \AgdaSymbol{∀} \AgdaSymbol{\{}\AgdaBound{Γ} \AgdaBound{A}\AgdaSymbol{\}} \AgdaBound{T} \AgdaBound{Δ} \AgdaSymbol{→} \AgdaSymbol{(}\AgdaBound{Δ} \AgdaFunction{++} \AgdaBound{Γ}\AgdaSymbol{)} \AgdaDatatype{⊢} \AgdaBound{A} \AgdaSymbol{→} \AgdaSymbol{(}\AgdaBound{Δ} \AgdaFunction{++} \AgdaSymbol{(}\AgdaBound{T} \AgdaInductiveConstructor{::} \AgdaBound{Γ}\AgdaSymbol{))} \AgdaDatatype{⊢} \AgdaBound{A}\<%
\end{code}
\AgdaHide{
\begin{code}%
\>\AgdaFunction{lift-tm} \AgdaBound{T} \AgdaBound{Δ} \AgdaSymbol{(}\AgdaInductiveConstructor{var} \AgdaBound{x}\AgdaSymbol{)} \AgdaSymbol{=} \AgdaInductiveConstructor{var} \AgdaSymbol{(}\AgdaFunction{lift-var} \AgdaBound{T} \AgdaBound{Δ} \AgdaBound{x}\AgdaSymbol{)}\<%
\\
\>\AgdaFunction{lift-tm} \AgdaBound{T} \AgdaBound{Δ} \AgdaInductiveConstructor{<>} \AgdaSymbol{=} \AgdaInductiveConstructor{<>}\<%
\\
\>\AgdaFunction{lift-tm} \AgdaBound{T} \AgdaBound{Δ} \AgdaSymbol{(}\AgdaBound{a} \AgdaInductiveConstructor{,} \AgdaBound{b}\AgdaSymbol{)} \AgdaSymbol{=} \AgdaFunction{lift-tm} \AgdaBound{T} \AgdaBound{Δ} \AgdaBound{a} \AgdaInductiveConstructor{,} \AgdaFunction{lift-tm} \AgdaBound{T} \AgdaBound{Δ} \AgdaBound{b}\<%
\\
\>\AgdaFunction{lift-tm} \AgdaBound{T} \AgdaBound{Δ} \AgdaSymbol{(}\AgdaInductiveConstructor{‘⊥’-elim} \AgdaBound{t}\AgdaSymbol{)} \AgdaSymbol{=} \AgdaInductiveConstructor{‘⊥’-elim} \AgdaSymbol{(}\AgdaFunction{lift-tm} \AgdaBound{T} \AgdaBound{Δ} \AgdaBound{t}\AgdaSymbol{)}\<%
\\
\>\AgdaFunction{lift-tm} \AgdaBound{T} \AgdaBound{Δ} \AgdaSymbol{(}\AgdaInductiveConstructor{π₁} \AgdaBound{t}\AgdaSymbol{)} \AgdaSymbol{=} \AgdaInductiveConstructor{π₁} \AgdaSymbol{(}\AgdaFunction{lift-tm} \AgdaBound{T} \AgdaBound{Δ} \AgdaBound{t}\AgdaSymbol{)}\<%
\\
\>\AgdaFunction{lift-tm} \AgdaBound{T} \AgdaBound{Δ} \AgdaSymbol{(}\AgdaInductiveConstructor{π₂} \AgdaBound{t}\AgdaSymbol{)} \AgdaSymbol{=} \AgdaInductiveConstructor{π₂} \AgdaSymbol{(}\AgdaFunction{lift-tm} \AgdaBound{T} \AgdaBound{Δ} \AgdaBound{t}\AgdaSymbol{)}\<%
\\
\>\AgdaFunction{lift-tm} \AgdaBound{T} \AgdaBound{Δ} \AgdaSymbol{(}\AgdaInductiveConstructor{‘λ’} \AgdaBound{t}\AgdaSymbol{)} \AgdaSymbol{=} \AgdaInductiveConstructor{‘λ’} \AgdaSymbol{(}\AgdaFunction{lift-tm} \AgdaBound{T} \AgdaSymbol{(\_} \AgdaInductiveConstructor{::} \AgdaBound{Δ}\AgdaSymbol{)} \AgdaBound{t}\AgdaSymbol{)}\<%
\\
\>\AgdaFunction{lift-tm} \AgdaBound{T} \AgdaBound{Δ} \AgdaSymbol{(}\AgdaBound{t} \AgdaInductiveConstructor{‘’ₐ} \AgdaBound{t₁}\AgdaSymbol{)} \AgdaSymbol{=} \AgdaFunction{lift-tm} \AgdaBound{T} \AgdaBound{Δ} \AgdaBound{t} \AgdaInductiveConstructor{‘’ₐ} \AgdaFunction{lift-tm} \AgdaBound{T} \AgdaBound{Δ} \AgdaBound{t₁}\<%
\\
\>\AgdaFunction{lift-tm} \AgdaBound{T} \AgdaBound{Δ} \AgdaInductiveConstructor{⌜} \AgdaBound{t} \AgdaInductiveConstructor{⌝} \AgdaSymbol{=} \AgdaInductiveConstructor{⌜} \AgdaBound{t} \AgdaInductiveConstructor{⌝}\<%
\\
\>\AgdaFunction{lift-tm} \AgdaBound{T} \AgdaBound{Δ} \AgdaSymbol{(}\AgdaInductiveConstructor{repr} \AgdaBound{t}\AgdaSymbol{)} \AgdaSymbol{=} \AgdaInductiveConstructor{repr} \AgdaSymbol{(}\AgdaFunction{lift-tm} \AgdaBound{T} \AgdaBound{Δ} \AgdaBound{t}\AgdaSymbol{)}\<%
\\
\>\AgdaFunction{lift-tm} \AgdaBound{T} \AgdaBound{Δ} \AgdaSymbol{(}\AgdaInductiveConstructor{dist} \AgdaBound{t} \AgdaBound{t₁}\AgdaSymbol{)} \AgdaSymbol{=} \AgdaInductiveConstructor{dist} \AgdaSymbol{(}\AgdaFunction{lift-tm} \AgdaBound{T} \AgdaBound{Δ} \AgdaBound{t}\AgdaSymbol{)} \AgdaSymbol{(}\AgdaFunction{lift-tm} \AgdaBound{T} \AgdaBound{Δ} \AgdaBound{t₁}\AgdaSymbol{)}\<%
\\
\>\AgdaFunction{lift-tm} \AgdaBound{T} \AgdaBound{Δ} \AgdaSymbol{(}\AgdaInductiveConstructor{Lӧb} \AgdaBound{t}\AgdaSymbol{)} \AgdaSymbol{=} \AgdaInductiveConstructor{Lӧb} \AgdaSymbol{(}\AgdaFunction{lift-tm} \AgdaBound{T} \AgdaBound{Δ} \AgdaBound{t}\AgdaSymbol{)}\<%
\end{code}
}

Weakening is a special case of \mintinline{Agda}|lift-tm|.

\begin{code}%
\>\AgdaFunction{wk} \AgdaSymbol{:} \AgdaSymbol{∀} \AgdaSymbol{\{}\AgdaBound{Γ} \AgdaBound{A} \AgdaBound{B}\AgdaSymbol{\}} \AgdaSymbol{→} \AgdaBound{Γ} \AgdaDatatype{⊢} \AgdaBound{A} \AgdaSymbol{→} \AgdaSymbol{(}\AgdaBound{B} \AgdaInductiveConstructor{::} \AgdaBound{Γ}\AgdaSymbol{)} \AgdaDatatype{⊢} \AgdaBound{A}\<%
\\
\>\AgdaFunction{wk} \AgdaSymbol{=} \AgdaFunction{lift-tm} \AgdaSymbol{\_} \AgdaInductiveConstructor{ε}\<%
\end{code}

Finally, we define function composition for our internal language.

\begin{code}%
\>\AgdaKeyword{infixl} \AgdaNumber{10} \AgdaFixityOp{\_∘'\_}\<%
\\
\>\AgdaFunction{\_∘'\_} \AgdaSymbol{:} \AgdaSymbol{∀} \AgdaSymbol{\{}\AgdaBound{Γ} \AgdaBound{A} \AgdaBound{B} \AgdaBound{C}\AgdaSymbol{\}}\<%
\\
\>[0]\AgdaIndent{2}{}\<[2]%
\>[2]\AgdaSymbol{→} \AgdaBound{Γ} \AgdaDatatype{⊢} \AgdaSymbol{(}\AgdaBound{B} \AgdaInductiveConstructor{‘→’} \AgdaBound{C}\AgdaSymbol{)}\<%
\\
\>[0]\AgdaIndent{2}{}\<[2]%
\>[2]\AgdaSymbol{→} \AgdaBound{Γ} \AgdaDatatype{⊢} \AgdaSymbol{(}\AgdaBound{A} \AgdaInductiveConstructor{‘→’} \AgdaBound{B}\AgdaSymbol{)}\<%
\\
\>[0]\AgdaIndent{2}{}\<[2]%
\>[2]\AgdaSymbol{→} \AgdaBound{Γ} \AgdaDatatype{⊢} \AgdaSymbol{(}\AgdaBound{A} \AgdaInductiveConstructor{‘→’} \AgdaBound{C}\AgdaSymbol{)}\<%
\\
\>\AgdaBound{f} \AgdaFunction{∘'} \AgdaBound{g} \AgdaSymbol{=} \AgdaInductiveConstructor{‘λ’} \AgdaSymbol{(}\AgdaFunction{wk} \AgdaBound{f} \AgdaInductiveConstructor{‘’ₐ} \AgdaSymbol{(}\AgdaFunction{wk} \AgdaBound{g} \AgdaInductiveConstructor{‘’ₐ} \AgdaInductiveConstructor{var} \AgdaInductiveConstructor{top}\AgdaSymbol{))}\<%
\end{code}

Now we are ready to prove that FairBot cooperates with itself.
Sadly, our type system isn't expressive enough to give a general type of bots, but we can still prove things about the interactions of particular bots if we substitute their types by hand.
For example, we can state the desired theorem (that FairBot cooperates with itself) as:

\begin{code}%
\>\AgdaFunction{distf} \AgdaSymbol{:} \AgdaSymbol{∀} \AgdaSymbol{\{}\AgdaBound{Γ} \AgdaBound{Δ} \AgdaBound{A} \AgdaBound{B}\AgdaSymbol{\}}\<%
\\
\>[0]\AgdaIndent{2}{}\<[2]%
\>[2]\AgdaSymbol{→} \AgdaBound{Γ} \AgdaDatatype{⊢} \AgdaSymbol{(}\AgdaBound{Δ} \AgdaInductiveConstructor{‘⊢’} \AgdaBound{A} \AgdaInductiveConstructor{‘→’} \AgdaBound{B}\AgdaSymbol{)}\<%
\\
\>[0]\AgdaIndent{2}{}\<[2]%
\>[2]\AgdaSymbol{→} \AgdaBound{Γ} \AgdaDatatype{⊢} \AgdaSymbol{(}\AgdaBound{Δ} \AgdaInductiveConstructor{‘⊢’} \AgdaBound{A}\AgdaSymbol{)} \AgdaInductiveConstructor{‘→’} \AgdaSymbol{(}\AgdaBound{Δ} \AgdaInductiveConstructor{‘⊢’} \AgdaBound{B}\AgdaSymbol{)}\<%
\\
\>\AgdaFunction{distf} \AgdaBound{bf} \AgdaSymbol{=} \AgdaInductiveConstructor{‘λ’} \AgdaSymbol{(}\AgdaInductiveConstructor{dist} \AgdaSymbol{(}\AgdaFunction{wk} \AgdaBound{bf}\AgdaSymbol{)} \AgdaSymbol{(}\AgdaInductiveConstructor{var} \AgdaInductiveConstructor{top}\AgdaSymbol{))}\<%
\\
%
\\
\>\AgdaFunction{evf} \AgdaSymbol{:} \AgdaSymbol{∀} \AgdaSymbol{\{}\AgdaBound{Γ} \AgdaBound{Δ} \AgdaBound{A}\AgdaSymbol{\}}\<%
\\
\>[0]\AgdaIndent{2}{}\<[2]%
\>[2]\AgdaSymbol{→} \AgdaBound{Γ} \AgdaDatatype{⊢} \AgdaSymbol{(}\AgdaBound{Δ} \AgdaInductiveConstructor{‘⊢’} \AgdaBound{A}\AgdaSymbol{)} \AgdaInductiveConstructor{‘→’} \AgdaSymbol{(}\AgdaBound{Δ} \AgdaInductiveConstructor{‘⊢’} \AgdaSymbol{(}\AgdaBound{Δ} \AgdaInductiveConstructor{‘⊢’} \AgdaBound{A}\AgdaSymbol{))}\<%
\\
\>\AgdaFunction{evf} \AgdaSymbol{=} \AgdaInductiveConstructor{‘λ’} \AgdaSymbol{(}\AgdaInductiveConstructor{repr} \AgdaSymbol{(}\AgdaInductiveConstructor{var} \AgdaInductiveConstructor{top}\AgdaSymbol{))}\<%
\\
%
\\
\>\AgdaFunction{fb-fb-cooperate} \AgdaSymbol{:} \AgdaSymbol{∀} \AgdaSymbol{\{}\AgdaBound{Γ} \AgdaBound{A} \AgdaBound{B}\AgdaSymbol{\}}\<%
\\
\>[0]\AgdaIndent{2}{}\<[2]%
\>[2]\AgdaSymbol{→} \AgdaBound{Γ} \AgdaDatatype{⊢} \AgdaSymbol{(}\AgdaBound{Γ} \AgdaInductiveConstructor{‘⊢’} \AgdaBound{A}\AgdaSymbol{)} \AgdaInductiveConstructor{‘→’} \AgdaBound{B}\<%
\\
\>[0]\AgdaIndent{2}{}\<[2]%
\>[2]\AgdaSymbol{→} \AgdaBound{Γ} \AgdaDatatype{⊢}\AgdaSymbol{(}\AgdaBound{Γ} \AgdaInductiveConstructor{‘⊢’} \AgdaBound{B}\AgdaSymbol{)} \AgdaInductiveConstructor{‘→’} \AgdaBound{A}\<%
\\
\>[0]\AgdaIndent{2}{}\<[2]%
\>[2]\AgdaSymbol{→} \AgdaBound{Γ} \AgdaDatatype{⊢} \AgdaSymbol{(}\AgdaBound{A} \AgdaInductiveConstructor{‘×’} \AgdaBound{B}\AgdaSymbol{)}\<%
\\
\>\AgdaFunction{fb-fb-cooperate} \AgdaBound{a} \AgdaBound{b}\<%
\\
\>[0]\AgdaIndent{2}{}\<[2]%
\>[2]\AgdaSymbol{=} \<[6]%
\>[6]\AgdaFunction{lӧb} \AgdaSymbol{(}\AgdaBound{b} \AgdaFunction{∘'} \AgdaFunction{distf} \AgdaInductiveConstructor{⌜} \AgdaBound{a} \AgdaInductiveConstructor{⌝} \AgdaFunction{∘'} \AgdaFunction{evf}\AgdaSymbol{)}\<%
\\
\>[2]\AgdaIndent{4}{}\<[4]%
\>[4]\AgdaInductiveConstructor{,} \AgdaFunction{lӧb} \AgdaSymbol{(}\AgdaBound{a} \AgdaFunction{∘'} \AgdaFunction{distf} \AgdaInductiveConstructor{⌜} \AgdaBound{b} \AgdaInductiveConstructor{⌝} \AgdaFunction{∘'} \AgdaFunction{evf}\AgdaSymbol{)}\<%
\end{code}

We can also state the theorem in a more familiar form with a couple abbreviations

\begin{code}%
\>\AgdaFunction{‘□’} \AgdaSymbol{=} \AgdaInductiveConstructor{\_‘⊢’\_} \AgdaInductiveConstructor{ε}\<%
\\
\>\AgdaFunction{□} \AgdaSymbol{=} \AgdaDatatype{\_⊢\_} \AgdaInductiveConstructor{ε}\<%
\\
%
\\
\>\AgdaFunction{fb-fb-cooperate'} \AgdaSymbol{:} \AgdaSymbol{∀} \AgdaSymbol{\{}\AgdaBound{A} \AgdaBound{B}\AgdaSymbol{\}}\<%
\\
\>[0]\AgdaIndent{2}{}\<[2]%
\>[2]\AgdaSymbol{→} \AgdaFunction{□} \AgdaSymbol{(}\AgdaFunction{‘□’} \AgdaBound{A} \AgdaInductiveConstructor{‘→’} \AgdaBound{B}\AgdaSymbol{)}\<%
\\
\>[0]\AgdaIndent{2}{}\<[2]%
\>[2]\AgdaSymbol{→} \AgdaFunction{□} \AgdaSymbol{(}\AgdaFunction{‘□’} \AgdaBound{B} \AgdaInductiveConstructor{‘→’} \AgdaBound{A}\AgdaSymbol{)}\<%
\\
\>[0]\AgdaIndent{2}{}\<[2]%
\>[2]\AgdaSymbol{→} \AgdaFunction{□} \AgdaSymbol{(}\AgdaBound{A} \AgdaInductiveConstructor{‘×’} \AgdaBound{B}\AgdaSymbol{)}\<%
\\
\>\AgdaFunction{fb-fb-cooperate'} \AgdaSymbol{=} \AgdaFunction{fb-fb-cooperate}\<%
\end{code}

In the file \texttt{fair-bot-self-cooperates.lagda} in the artifact, we show all the meta-theoretic properties we had before: soundness, inhabitedness, and incompleteness.
\AgdaHide{
We can show inhabitedness immediately in several different ways. We'll take the easiest one.

\begin{code}%
\>\AgdaFunction{inhabited} \AgdaSymbol{:} \AgdaRecord{Σ} \AgdaDatatype{Type} \AgdaSymbol{(λ} \AgdaBound{T} \AgdaSymbol{→} \AgdaInductiveConstructor{ε} \AgdaDatatype{⊢} \AgdaBound{T}\AgdaSymbol{)}\<%
\\
\>\AgdaFunction{inhabited} \AgdaSymbol{=} \AgdaInductiveConstructor{‘⊤’} \AgdaInductiveConstructor{,} \AgdaInductiveConstructor{<>}\<%
\end{code}

We now prove soundness and incompleteness of this system,
and give it a semantic model via an interpretation function.

First, we'll first need to give the standard interpretation.
Again, the simplicity of our system makes our lives easier.
We define the interpreter for types as follows:

\begin{code}%
\>\AgdaFunction{⟦\_⟧ᵀ} \AgdaSymbol{:} \AgdaDatatype{Type} \AgdaSymbol{→} \AgdaPrimitiveType{Set}\<%
\\
\>\AgdaFunction{⟦} \AgdaBound{Δ} \AgdaInductiveConstructor{‘⊢’} \AgdaBound{T} \AgdaFunction{⟧ᵀ} \AgdaSymbol{=} \AgdaBound{Δ} \AgdaDatatype{⊢} \AgdaBound{T}\<%
\\
\>\AgdaFunction{⟦} \AgdaBound{A} \AgdaInductiveConstructor{‘→’} \AgdaBound{B} \AgdaFunction{⟧ᵀ} \AgdaSymbol{=} \AgdaFunction{⟦} \AgdaBound{A} \AgdaFunction{⟧ᵀ} \AgdaSymbol{→} \AgdaFunction{⟦} \AgdaBound{B} \AgdaFunction{⟧ᵀ}\<%
\\
\>\AgdaFunction{⟦} \AgdaBound{A} \AgdaInductiveConstructor{‘×’} \AgdaBound{B} \AgdaFunction{⟧ᵀ} \AgdaSymbol{=} \AgdaFunction{⟦} \AgdaBound{A} \AgdaFunction{⟧ᵀ} \AgdaFunction{×} \AgdaFunction{⟦} \AgdaBound{B} \AgdaFunction{⟧ᵀ}\<%
\\
\>\AgdaFunction{⟦} \AgdaInductiveConstructor{‘⊥’} \AgdaFunction{⟧ᵀ} \AgdaSymbol{=} \AgdaDatatype{⊥}\<%
\\
\>\AgdaFunction{⟦} \AgdaInductiveConstructor{‘⊤’} \AgdaFunction{⟧ᵀ} \AgdaSymbol{=} \AgdaRecord{⊤}\<%
\end{code}

The interpreter for contexts is simplified - we only need simple products to interpret simple contexts.

\begin{code}%
\>\AgdaFunction{⟦\_⟧ᶜ} \AgdaSymbol{:} \AgdaFunction{Context} \AgdaSymbol{→} \AgdaPrimitiveType{Set}\<%
\\
\>\AgdaFunction{⟦} \AgdaInductiveConstructor{ε} \AgdaFunction{⟧ᶜ} \AgdaSymbol{=} \AgdaRecord{⊤}\<%
\\
\>\AgdaFunction{⟦} \AgdaBound{x} \AgdaInductiveConstructor{::} \AgdaBound{Γ} \AgdaFunction{⟧ᶜ} \AgdaSymbol{=} \AgdaFunction{⟦} \AgdaBound{Γ} \AgdaFunction{⟧ᶜ} \AgdaFunction{×} \AgdaFunction{⟦} \AgdaBound{x} \AgdaFunction{⟧ᵀ}\<%
\end{code}

We can then interpret variables in any interpretable context.

\begin{code}%
\>\AgdaFunction{⟦\_⟧v} \AgdaSymbol{:} \AgdaSymbol{∀} \AgdaSymbol{\{}\AgdaBound{Γ} \AgdaBound{A}\AgdaSymbol{\}} \AgdaSymbol{→} \AgdaBound{A} \AgdaDatatype{∈} \AgdaBound{Γ} \AgdaSymbol{→} \AgdaFunction{⟦} \AgdaBound{Γ} \AgdaFunction{⟧ᶜ} \AgdaSymbol{→} \AgdaFunction{⟦} \AgdaBound{A} \AgdaFunction{⟧ᵀ}\<%
\\
\>\AgdaFunction{⟦} \AgdaInductiveConstructor{top} \AgdaFunction{⟧v} \AgdaSymbol{=} \AgdaField{snd}\<%
\\
\>\AgdaFunction{⟦} \AgdaInductiveConstructor{pop} \AgdaBound{v} \AgdaFunction{⟧v} \AgdaSymbol{=} \AgdaFunction{⟦} \AgdaBound{v} \AgdaFunction{⟧v} \AgdaFunction{∘} \AgdaField{fst}\<%
\end{code}

And now we can interpret terms.

\begin{code}%
\>\AgdaFunction{⟦\_⟧ᵗ} \AgdaSymbol{:} \AgdaSymbol{∀} \AgdaSymbol{\{}\AgdaBound{Γ} \AgdaBound{A}\AgdaSymbol{\}} \AgdaSymbol{→} \AgdaBound{Γ} \AgdaDatatype{⊢} \AgdaBound{A} \AgdaSymbol{→} \AgdaFunction{⟦} \AgdaBound{Γ} \AgdaFunction{⟧ᶜ} \AgdaSymbol{→} \AgdaFunction{⟦} \AgdaBound{A} \AgdaFunction{⟧ᵀ}\<%
\\
\>\AgdaFunction{⟦} \AgdaInductiveConstructor{var} \AgdaBound{v} \AgdaFunction{⟧ᵗ} \AgdaSymbol{=} \AgdaFunction{⟦} \AgdaBound{v} \AgdaFunction{⟧v}\<%
\\
\>\AgdaFunction{⟦} \AgdaInductiveConstructor{<>} \AgdaFunction{⟧ᵗ} \AgdaSymbol{=} \AgdaFunction{ᵏ} \AgdaSymbol{\_}\<%
\\
\>\AgdaFunction{⟦} \AgdaBound{a} \AgdaInductiveConstructor{,} \AgdaBound{b} \AgdaFunction{⟧ᵗ} \AgdaSymbol{=} \AgdaFunction{ᵏ} \AgdaInductiveConstructor{\_,\_} \AgdaFunction{ˢ} \AgdaFunction{⟦} \AgdaBound{a} \AgdaFunction{⟧ᵗ} \AgdaFunction{ˢ} \AgdaFunction{⟦} \AgdaBound{b} \AgdaFunction{⟧ᵗ}\<%
\\
\>\AgdaFunction{⟦} \AgdaInductiveConstructor{‘⊥’-elim} \AgdaBound{t} \AgdaFunction{⟧ᵗ} \AgdaSymbol{=} \AgdaFunction{ᵏ} \AgdaSymbol{(λ} \AgdaSymbol{())} \AgdaFunction{ˢ} \AgdaFunction{⟦} \AgdaBound{t} \AgdaFunction{⟧ᵗ}\<%
\\
\>\AgdaFunction{⟦} \AgdaInductiveConstructor{π₁} \AgdaBound{t} \AgdaFunction{⟧ᵗ} \AgdaSymbol{=} \AgdaFunction{ᵏ} \AgdaField{fst} \AgdaFunction{ˢ} \AgdaFunction{⟦} \AgdaBound{t} \AgdaFunction{⟧ᵗ}\<%
\\
\>\AgdaFunction{⟦} \AgdaInductiveConstructor{π₂} \AgdaBound{t} \AgdaFunction{⟧ᵗ} \AgdaSymbol{=} \AgdaFunction{ᵏ} \AgdaField{snd} \AgdaFunction{ˢ} \AgdaFunction{⟦} \AgdaBound{t} \AgdaFunction{⟧ᵗ}\<%
\\
\>\AgdaFunction{⟦} \AgdaInductiveConstructor{‘λ’} \AgdaBound{b} \AgdaFunction{⟧ᵗ} \AgdaSymbol{=} \AgdaFunction{\textasciicircum} \AgdaFunction{⟦} \AgdaBound{b} \AgdaFunction{⟧ᵗ}\<%
\\
\>\AgdaFunction{⟦} \AgdaBound{f} \AgdaInductiveConstructor{‘’ₐ} \AgdaBound{x} \AgdaFunction{⟧ᵗ} \AgdaSymbol{=} \AgdaFunction{⟦} \AgdaBound{f} \AgdaFunction{⟧ᵗ} \AgdaFunction{ˢ} \AgdaFunction{⟦} \AgdaBound{x} \AgdaFunction{⟧ᵗ}\<%
\\
\>\AgdaFunction{⟦} \AgdaInductiveConstructor{⌜} \AgdaBound{t} \AgdaInductiveConstructor{⌝} \AgdaFunction{⟧ᵗ} \AgdaSymbol{=} \AgdaFunction{ᵏ} \AgdaBound{t}\<%
\\
\>\AgdaFunction{⟦} \AgdaInductiveConstructor{repr} \AgdaBound{t} \AgdaFunction{⟧ᵗ} \AgdaSymbol{=} \AgdaFunction{ᵏ} \AgdaInductiveConstructor{⌜\_⌝} \AgdaFunction{ˢ} \AgdaFunction{⟦} \AgdaBound{t} \AgdaFunction{⟧ᵗ}\<%
\\
\>\AgdaFunction{⟦} \AgdaInductiveConstructor{dist} \AgdaBound{f} \AgdaBound{x} \AgdaFunction{⟧ᵗ} \AgdaSymbol{=} \AgdaFunction{ᵏ} \AgdaInductiveConstructor{\_‘’ₐ\_} \AgdaFunction{ˢ} \AgdaFunction{⟦} \AgdaBound{f} \AgdaFunction{⟧ᵗ} \AgdaFunction{ˢ} \AgdaFunction{⟦} \AgdaBound{x} \AgdaFunction{⟧ᵗ}\<%
\\
\>\AgdaFunction{⟦} \AgdaInductiveConstructor{Lӧb} \AgdaBound{l} \AgdaFunction{⟧ᵗ} \AgdaSymbol{=} \AgdaFunction{ᵏ} \AgdaFunction{lӧb} \AgdaFunction{ˢ} \AgdaFunction{⟦} \AgdaBound{l} \AgdaFunction{⟧ᵗ}\<%
\end{code}

Which lets us prove all our sanity checks.

\begin{code}%
\>\AgdaFunction{‘¬’\_} \AgdaSymbol{:} \AgdaDatatype{Type} \AgdaSymbol{→} \AgdaDatatype{Type}\<%
\\
\>\AgdaFunction{‘¬’} \AgdaBound{T} \AgdaSymbol{=} \AgdaBound{T} \AgdaInductiveConstructor{‘→’} \AgdaInductiveConstructor{‘⊥’}\<%
\\
%
\\
\>\AgdaFunction{consistency} \AgdaSymbol{:} \AgdaFunction{¬} \AgdaSymbol{(}\AgdaFunction{□} \AgdaInductiveConstructor{‘⊥’}\AgdaSymbol{)}\<%
\\
\>\AgdaFunction{consistency} \AgdaBound{f} \AgdaSymbol{=} \AgdaFunction{⟦} \AgdaBound{f} \AgdaFunction{⟧ᵗ} \AgdaInductiveConstructor{tt}\<%
\\
%
\\
\>\AgdaFunction{incompleteness} \AgdaSymbol{:} \AgdaFunction{¬} \AgdaSymbol{(}\AgdaFunction{□} \AgdaSymbol{(}\AgdaFunction{‘¬’} \AgdaFunction{‘□’} \AgdaInductiveConstructor{‘⊥’}\AgdaSymbol{))}\<%
\\
\>\AgdaFunction{incompleteness} \AgdaBound{t} \AgdaSymbol{=} \AgdaFunction{⟦} \AgdaFunction{lӧb} \AgdaBound{t} \AgdaFunction{⟧ᵗ} \AgdaInductiveConstructor{tt}\<%
\\
%
\\
\>\AgdaFunction{soundness} \AgdaSymbol{:} \AgdaSymbol{∀} \AgdaSymbol{\{}\AgdaBound{A}\AgdaSymbol{\}} \AgdaSymbol{→} \AgdaFunction{□} \AgdaBound{A} \AgdaSymbol{→} \AgdaFunction{⟦} \AgdaBound{A} \AgdaFunction{⟧ᵀ}\<%
\\
\>\AgdaFunction{soundness} \AgdaBound{a} \AgdaSymbol{=} \AgdaFunction{⟦} \AgdaBound{a} \AgdaFunction{⟧ᵗ} \AgdaInductiveConstructor{tt}\<%
\end{code}

}

%\section{Encoding with Add-Quote Function}

\AgdaHide{
  \begin{code}%
\>\AgdaKeyword{module} \AgdaModule{lob-build-quine} \AgdaKeyword{where}\<%
\\
\>\AgdaKeyword{open} \AgdaKeyword{import} \AgdaModule{lob}\<%
\end{code}
}


\begin{code}%
\>\AgdaKeyword{module} \AgdaModule{lob-by-repr} \AgdaKeyword{where}\<%
\\
\> \AgdaKeyword{module} \AgdaModule{well-typed-syntax} \AgdaKeyword{where}\<%
\\
%
\\
\>[0]\AgdaIndent{2}{}\<[2]%
\>[2]\AgdaKeyword{infixl} \AgdaNumber{1} \AgdaFixityOp{\_‘,’\_}\<%
\\
\>[0]\AgdaIndent{2}{}\<[2]%
\>[2]\AgdaKeyword{infixl} \AgdaNumber{2} \AgdaFixityOp{\_▻\_}\<%
\\
\>[0]\AgdaIndent{2}{}\<[2]%
\>[2]\AgdaKeyword{infixl} \AgdaNumber{3} \AgdaFixityOp{\_‘’\_}\<%
\\
\>[0]\AgdaIndent{2}{}\<[2]%
\>[2]\AgdaKeyword{infixl} \AgdaNumber{3} \AgdaFixityOp{\_‘’₁\_}\<%
\\
\>[0]\AgdaIndent{2}{}\<[2]%
\>[2]\AgdaKeyword{infixl} \AgdaNumber{3} \AgdaFixityOp{\_‘’₂\_}\<%
\\
\>[0]\AgdaIndent{2}{}\<[2]%
\>[2]\AgdaKeyword{infixl} \AgdaNumber{3} \AgdaFixityOp{\_‘’₃\_}\<%
\\
\>[0]\AgdaIndent{2}{}\<[2]%
\>[2]\AgdaKeyword{infixl} \AgdaNumber{3} \AgdaFixityOp{\_‘’ₐ\_}\<%
\\
\>[0]\AgdaIndent{2}{}\<[2]%
\>[2]\AgdaKeyword{infixr} \AgdaNumber{1} \AgdaFixityOp{\_‘→’\_}\<%
\\
\>[0]\AgdaIndent{2}{}\<[2]%
\>[2]\AgdaKeyword{infixl} \AgdaNumber{3} \AgdaFixityOp{\_‘‘’’\_}\<%
\\
\>[0]\AgdaIndent{2}{}\<[2]%
\>[2]\AgdaKeyword{infixl} \AgdaNumber{3} \AgdaFixityOp{\_w‘‘’’\_}\<%
\\
\>[0]\AgdaIndent{2}{}\<[2]%
\>[2]\AgdaKeyword{infixr} \AgdaNumber{1} \AgdaFixityOp{\_‘‘→'’’\_}\<%
\\
\>[0]\AgdaIndent{2}{}\<[2]%
\>[2]\AgdaKeyword{infixr} \AgdaNumber{1} \AgdaFixityOp{\_w‘‘→'’’\_}\<%
\\
%
\\
\>[0]\AgdaIndent{2}{}\<[2]%
\>[2]\AgdaKeyword{mutual}\<%
\\
\>[2]\AgdaIndent{3}{}\<[3]%
\>[3]\AgdaKeyword{data} \AgdaDatatype{Context} \AgdaSymbol{:} \AgdaPrimitiveType{Set} \AgdaKeyword{where}\<%
\\
\>[3]\AgdaIndent{4}{}\<[4]%
\>[4]\AgdaInductiveConstructor{ε} \AgdaSymbol{:} \AgdaDatatype{Context}\<%
\\
\>[3]\AgdaIndent{4}{}\<[4]%
\>[4]\AgdaInductiveConstructor{\_▻\_} \AgdaSymbol{:} \AgdaSymbol{(}\AgdaBound{Γ} \AgdaSymbol{:} \AgdaDatatype{Context}\AgdaSymbol{)} \AgdaSymbol{→} \AgdaDatatype{Type} \AgdaBound{Γ} \AgdaSymbol{→} \AgdaDatatype{Context}\<%
\\
%
\\
\>[0]\AgdaIndent{3}{}\<[3]%
\>[3]\AgdaKeyword{data} \AgdaDatatype{Type} \AgdaSymbol{:} \AgdaDatatype{Context} \AgdaSymbol{→} \AgdaPrimitiveType{Set} \AgdaKeyword{where}\<%
\\
\>[3]\AgdaIndent{4}{}\<[4]%
\>[4]\AgdaInductiveConstructor{\_‘→’\_} \AgdaSymbol{:} \AgdaSymbol{∀} \AgdaSymbol{\{}\AgdaBound{Γ}\AgdaSymbol{\}} \AgdaSymbol{(}\AgdaBound{A} \AgdaSymbol{:} \AgdaDatatype{Type} \AgdaBound{Γ}\AgdaSymbol{)} \AgdaSymbol{→} \AgdaDatatype{Type} \AgdaSymbol{(}\AgdaBound{Γ} \AgdaInductiveConstructor{▻} \AgdaBound{A}\AgdaSymbol{)} \AgdaSymbol{→} \AgdaDatatype{Type} \AgdaBound{Γ}\<%
\\
\>[3]\AgdaIndent{4}{}\<[4]%
\>[4]\AgdaInductiveConstructor{‘Σ’} \AgdaSymbol{:} \AgdaSymbol{∀} \AgdaSymbol{\{}\AgdaBound{Γ}\AgdaSymbol{\}} \AgdaSymbol{(}\AgdaBound{T} \AgdaSymbol{:} \AgdaDatatype{Type} \AgdaBound{Γ}\AgdaSymbol{)} \AgdaSymbol{→} \AgdaDatatype{Type} \AgdaSymbol{(}\AgdaBound{Γ} \AgdaInductiveConstructor{▻} \AgdaBound{T}\AgdaSymbol{)} \AgdaSymbol{→} \AgdaDatatype{Type} \AgdaBound{Γ}\<%
\\
\>[3]\AgdaIndent{4}{}\<[4]%
\>[4]\AgdaInductiveConstructor{‘Context’} \AgdaSymbol{:} \AgdaSymbol{∀} \AgdaSymbol{\{}\AgdaBound{Γ}\AgdaSymbol{\}} \AgdaSymbol{→} \AgdaDatatype{Type} \AgdaBound{Γ}\<%
\\
\>[3]\AgdaIndent{4}{}\<[4]%
\>[4]\AgdaInductiveConstructor{‘Type’} \AgdaSymbol{:} \AgdaSymbol{∀} \AgdaSymbol{\{}\AgdaBound{Γ}\AgdaSymbol{\}} \AgdaSymbol{→} \AgdaDatatype{Type} \AgdaSymbol{(}\AgdaBound{Γ} \AgdaInductiveConstructor{▻} \AgdaInductiveConstructor{‘Context’}\AgdaSymbol{)}\<%
\\
\>[3]\AgdaIndent{4}{}\<[4]%
\>[4]\AgdaInductiveConstructor{‘Term’} \AgdaSymbol{:} \AgdaSymbol{∀} \AgdaSymbol{\{}\AgdaBound{Γ}\AgdaSymbol{\}} \AgdaSymbol{→} \AgdaDatatype{Type} \AgdaSymbol{(}\AgdaBound{Γ} \AgdaInductiveConstructor{▻} \AgdaInductiveConstructor{‘Context’} \AgdaInductiveConstructor{▻} \AgdaInductiveConstructor{‘Type’}\AgdaSymbol{)}\<%
\\
\>[3]\AgdaIndent{4}{}\<[4]%
\>[4]\AgdaInductiveConstructor{\_‘’\_} \AgdaSymbol{:} \AgdaSymbol{∀} \AgdaSymbol{\{}\AgdaBound{Γ} \AgdaBound{A}\AgdaSymbol{\}} \AgdaSymbol{→} \AgdaDatatype{Type} \AgdaSymbol{(}\AgdaBound{Γ} \AgdaInductiveConstructor{▻} \AgdaBound{A}\AgdaSymbol{)} \AgdaSymbol{→} \AgdaDatatype{Term} \AgdaBound{A} \AgdaSymbol{→} \AgdaDatatype{Type} \AgdaBound{Γ}\<%
\\
\>[3]\AgdaIndent{4}{}\<[4]%
\>[4]\AgdaInductiveConstructor{\_‘’₁\_} \AgdaSymbol{:} \AgdaSymbol{∀} \AgdaSymbol{\{}\AgdaBound{Γ} \AgdaBound{A} \AgdaBound{B}\AgdaSymbol{\}} \AgdaSymbol{→} \AgdaSymbol{(}\AgdaBound{C} \AgdaSymbol{:} \AgdaDatatype{Type} \AgdaSymbol{(}\AgdaBound{Γ} \AgdaInductiveConstructor{▻} \AgdaBound{A} \AgdaInductiveConstructor{▻} \AgdaBound{B}\AgdaSymbol{))} \AgdaSymbol{→} \AgdaSymbol{(}\AgdaBound{a} \AgdaSymbol{:} \AgdaDatatype{Term} \AgdaBound{A}\AgdaSymbol{)} \AgdaSymbol{→} \AgdaDatatype{Type} \AgdaSymbol{(}\AgdaBound{Γ} \AgdaInductiveConstructor{▻} \AgdaBound{B} \AgdaInductiveConstructor{‘’} \AgdaBound{a}\AgdaSymbol{)}\<%
\\
\>[3]\AgdaIndent{4}{}\<[4]%
\>[4]\AgdaInductiveConstructor{\_‘’₂\_} \AgdaSymbol{:} \AgdaSymbol{∀} \AgdaSymbol{\{}\AgdaBound{Γ} \AgdaBound{A} \AgdaBound{B} \AgdaBound{C}\AgdaSymbol{\}} \AgdaSymbol{→} \AgdaSymbol{(}\AgdaBound{D} \AgdaSymbol{:} \AgdaDatatype{Type} \AgdaSymbol{(}\AgdaBound{Γ} \AgdaInductiveConstructor{▻} \AgdaBound{A} \AgdaInductiveConstructor{▻} \AgdaBound{B} \AgdaInductiveConstructor{▻} \AgdaBound{C}\AgdaSymbol{))} \AgdaSymbol{→} \AgdaSymbol{(}\AgdaBound{a} \AgdaSymbol{:} \AgdaDatatype{Term} \AgdaBound{A}\AgdaSymbol{)} \AgdaSymbol{→} \AgdaDatatype{Type} \AgdaSymbol{(}\AgdaBound{Γ} \AgdaInductiveConstructor{▻} \AgdaBound{B} \AgdaInductiveConstructor{‘’} \AgdaBound{a} \AgdaInductiveConstructor{▻} \AgdaBound{C} \AgdaInductiveConstructor{‘’₁} \AgdaBound{a}\AgdaSymbol{)}\<%
\\
\>[3]\AgdaIndent{4}{}\<[4]%
\>[4]\AgdaInductiveConstructor{\_‘’₃\_} \AgdaSymbol{:} \AgdaSymbol{∀} \AgdaSymbol{\{}\AgdaBound{Γ} \AgdaBound{A} \AgdaBound{B} \AgdaBound{C} \AgdaBound{D}\AgdaSymbol{\}} \AgdaSymbol{→} \AgdaSymbol{(}\AgdaBound{E} \AgdaSymbol{:} \AgdaDatatype{Type} \AgdaSymbol{(}\AgdaBound{Γ} \AgdaInductiveConstructor{▻} \AgdaBound{A} \AgdaInductiveConstructor{▻} \AgdaBound{B} \AgdaInductiveConstructor{▻} \AgdaBound{C} \AgdaInductiveConstructor{▻} \AgdaBound{D}\AgdaSymbol{))} \AgdaSymbol{→} \AgdaSymbol{(}\AgdaBound{a} \AgdaSymbol{:} \AgdaDatatype{Term} \AgdaBound{A}\AgdaSymbol{)} \AgdaSymbol{→} \AgdaDatatype{Type} \AgdaSymbol{(}\AgdaBound{Γ} \AgdaInductiveConstructor{▻} \AgdaBound{B} \AgdaInductiveConstructor{‘’} \AgdaBound{a} \AgdaInductiveConstructor{▻} \AgdaBound{C} \AgdaInductiveConstructor{‘’₁} \AgdaBound{a} \AgdaInductiveConstructor{▻} \AgdaBound{D} \AgdaInductiveConstructor{‘’₂} \AgdaBound{a}\AgdaSymbol{)}\<%
\\
\>[3]\AgdaIndent{4}{}\<[4]%
\>[4]\AgdaInductiveConstructor{W} \AgdaSymbol{:} \AgdaSymbol{∀} \AgdaSymbol{\{}\AgdaBound{Γ} \AgdaBound{A}\AgdaSymbol{\}} \AgdaSymbol{→} \AgdaDatatype{Type} \AgdaBound{Γ} \AgdaSymbol{→} \AgdaDatatype{Type} \AgdaSymbol{(}\AgdaBound{Γ} \AgdaInductiveConstructor{▻} \AgdaBound{A}\AgdaSymbol{)}\<%
\\
\>[3]\AgdaIndent{4}{}\<[4]%
\>[4]\AgdaInductiveConstructor{W₁} \AgdaSymbol{:} \AgdaSymbol{∀} \AgdaSymbol{\{}\AgdaBound{Γ} \AgdaBound{A} \AgdaBound{B}\AgdaSymbol{\}} \AgdaSymbol{→} \AgdaDatatype{Type} \AgdaSymbol{(}\AgdaBound{Γ} \AgdaInductiveConstructor{▻} \AgdaBound{B}\AgdaSymbol{)} \AgdaSymbol{→} \AgdaDatatype{Type} \AgdaSymbol{(}\AgdaBound{Γ} \AgdaInductiveConstructor{▻} \AgdaBound{A} \AgdaInductiveConstructor{▻} \AgdaSymbol{(}\AgdaInductiveConstructor{W} \AgdaSymbol{\{}\AgdaBound{Γ}\AgdaSymbol{\}} \AgdaSymbol{\{}\AgdaBound{A}\AgdaSymbol{\}} \AgdaBound{B}\AgdaSymbol{))}\<%
\\
\>[3]\AgdaIndent{4}{}\<[4]%
\>[4]\AgdaInductiveConstructor{W₂} \AgdaSymbol{:} \AgdaSymbol{∀} \AgdaSymbol{\{}\AgdaBound{Γ} \AgdaBound{A} \AgdaBound{B} \AgdaBound{C}\AgdaSymbol{\}} \AgdaSymbol{→} \AgdaDatatype{Type} \AgdaSymbol{(}\AgdaBound{Γ} \AgdaInductiveConstructor{▻} \AgdaBound{B} \AgdaInductiveConstructor{▻} \AgdaBound{C}\AgdaSymbol{)} \AgdaSymbol{→} \AgdaDatatype{Type} \AgdaSymbol{(}\AgdaBound{Γ} \AgdaInductiveConstructor{▻} \AgdaBound{A} \AgdaInductiveConstructor{▻} \AgdaInductiveConstructor{W} \AgdaBound{B} \AgdaInductiveConstructor{▻} \AgdaInductiveConstructor{W₁} \AgdaBound{C}\AgdaSymbol{)}\<%
\\
%
\\
\>[0]\AgdaIndent{3}{}\<[3]%
\>[3]\AgdaKeyword{data} \AgdaDatatype{Term} \AgdaSymbol{:} \AgdaSymbol{∀} \AgdaSymbol{\{}\AgdaBound{Γ}\AgdaSymbol{\}} \AgdaSymbol{→} \AgdaDatatype{Type} \AgdaBound{Γ} \AgdaSymbol{→} \AgdaPrimitiveType{Set} \AgdaKeyword{where}\<%
\\
\>[3]\AgdaIndent{4}{}\<[4]%
\>[4]\AgdaInductiveConstructor{w} \AgdaSymbol{:} \AgdaSymbol{∀} \AgdaSymbol{\{}\AgdaBound{Γ} \AgdaBound{A} \AgdaBound{B}\AgdaSymbol{\}} \AgdaSymbol{→} \AgdaDatatype{Term} \AgdaSymbol{\{}\AgdaBound{Γ}\AgdaSymbol{\}} \AgdaBound{B} \AgdaSymbol{→} \AgdaDatatype{Term} \AgdaSymbol{\{}\AgdaBound{Γ} \AgdaInductiveConstructor{▻} \AgdaBound{A}\AgdaSymbol{\}} \AgdaSymbol{(}\AgdaInductiveConstructor{W} \AgdaSymbol{\{}\AgdaBound{Γ}\AgdaSymbol{\}} \AgdaSymbol{\{}\AgdaBound{A}\AgdaSymbol{\}} \AgdaBound{B}\AgdaSymbol{)}\<%
\\
\>[3]\AgdaIndent{4}{}\<[4]%
\>[4]\AgdaInductiveConstructor{‘λ’} \AgdaSymbol{:} \AgdaSymbol{∀} \AgdaSymbol{\{}\AgdaBound{Γ} \AgdaBound{A} \AgdaBound{B}\AgdaSymbol{\}} \AgdaSymbol{→} \AgdaDatatype{Term} \AgdaSymbol{\{(}\AgdaBound{Γ} \AgdaInductiveConstructor{▻} \AgdaBound{A}\AgdaSymbol{)\}} \AgdaBound{B} \AgdaSymbol{→} \AgdaDatatype{Term} \AgdaSymbol{\{}\AgdaBound{Γ}\AgdaSymbol{\}} \AgdaSymbol{(}\AgdaBound{A} \AgdaInductiveConstructor{‘→’} \AgdaBound{B}\AgdaSymbol{)}\<%
\\
\>[3]\AgdaIndent{4}{}\<[4]%
\>[4]\AgdaInductiveConstructor{\_‘’ₐ\_} \AgdaSymbol{:} \AgdaSymbol{∀} \AgdaSymbol{\{}\AgdaBound{Γ} \AgdaBound{A} \AgdaBound{B}\AgdaSymbol{\}} \AgdaSymbol{→} \AgdaSymbol{(}\AgdaBound{f} \AgdaSymbol{:} \AgdaDatatype{Term} \AgdaSymbol{\{}\AgdaBound{Γ}\AgdaSymbol{\}} \AgdaSymbol{(}\AgdaBound{A} \AgdaInductiveConstructor{‘→’} \AgdaBound{B}\AgdaSymbol{))} \AgdaSymbol{→} \AgdaSymbol{(}\AgdaBound{x} \AgdaSymbol{:} \AgdaDatatype{Term} \AgdaSymbol{\{}\AgdaBound{Γ}\AgdaSymbol{\}} \AgdaBound{A}\AgdaSymbol{)} \AgdaSymbol{→} \AgdaDatatype{Term} \AgdaSymbol{\{}\AgdaBound{Γ}\AgdaSymbol{\}} \AgdaSymbol{(}\AgdaBound{B} \AgdaInductiveConstructor{‘’} \AgdaBound{x}\AgdaSymbol{)}\<%
\\
\>[3]\AgdaIndent{4}{}\<[4]%
\>[4]\AgdaInductiveConstructor{‘VAR₀’} \AgdaSymbol{:} \AgdaSymbol{∀} \AgdaSymbol{\{}\AgdaBound{Γ} \AgdaBound{T}\AgdaSymbol{\}} \AgdaSymbol{→} \AgdaDatatype{Term} \AgdaSymbol{\{}\AgdaBound{Γ} \AgdaInductiveConstructor{▻} \AgdaBound{T}\AgdaSymbol{\}} \AgdaSymbol{(}\AgdaInductiveConstructor{W} \AgdaBound{T}\AgdaSymbol{)}\<%
\\
\>[3]\AgdaIndent{4}{}\<[4]%
\>[4]\AgdaInductiveConstructor{⌜\_⌝ᶜ} \AgdaSymbol{:} \AgdaSymbol{∀} \AgdaSymbol{\{}\AgdaBound{Γ}\AgdaSymbol{\}} \AgdaSymbol{→} \AgdaDatatype{Context} \AgdaSymbol{→} \AgdaDatatype{Term} \AgdaSymbol{\{}\AgdaBound{Γ}\AgdaSymbol{\}} \AgdaInductiveConstructor{‘Context’}\<%
\\
\>[3]\AgdaIndent{4}{}\<[4]%
\>[4]\AgdaInductiveConstructor{⌜\_⌝ᵀ} \AgdaSymbol{:} \AgdaSymbol{∀} \AgdaSymbol{\{}\AgdaBound{Γ} \AgdaBound{Γ'}\AgdaSymbol{\}} \AgdaSymbol{→} \AgdaDatatype{Type} \AgdaBound{Γ'} \AgdaSymbol{→} \AgdaDatatype{Term} \AgdaSymbol{\{}\AgdaBound{Γ}\AgdaSymbol{\}} \AgdaSymbol{(}\AgdaInductiveConstructor{‘Type’} \AgdaInductiveConstructor{‘’} \AgdaInductiveConstructor{⌜} \AgdaBound{Γ'} \AgdaInductiveConstructor{⌝ᶜ}\AgdaSymbol{)}\<%
\\
\>[3]\AgdaIndent{4}{}\<[4]%
\>[4]\AgdaInductiveConstructor{⌜\_⌝ᵗ} \AgdaSymbol{:} \AgdaSymbol{∀} \AgdaSymbol{\{}\AgdaBound{Γ} \AgdaBound{Γ'}\AgdaSymbol{\}} \AgdaSymbol{\{}\AgdaBound{T} \AgdaSymbol{:} \AgdaDatatype{Type} \AgdaBound{Γ'}\AgdaSymbol{\}} \AgdaSymbol{→} \AgdaDatatype{Term} \AgdaBound{T} \AgdaSymbol{→} \AgdaDatatype{Term} \AgdaSymbol{\{}\AgdaBound{Γ}\AgdaSymbol{\}} \AgdaSymbol{(}\AgdaInductiveConstructor{‘Term’} \AgdaInductiveConstructor{‘’₁} \AgdaInductiveConstructor{⌜} \AgdaBound{Γ'} \AgdaInductiveConstructor{⌝ᶜ} \AgdaInductiveConstructor{‘’} \AgdaInductiveConstructor{⌜} \AgdaBound{T} \AgdaInductiveConstructor{⌝ᵀ}\AgdaSymbol{)}\<%
\\
\>[3]\AgdaIndent{4}{}\<[4]%
\>[4]\AgdaInductiveConstructor{‘⌜\_⌝ᵗ’} \AgdaSymbol{:} \AgdaSymbol{∀} \AgdaSymbol{\{}\AgdaBound{Γ} \AgdaBound{Γ'}\AgdaSymbol{\}} \AgdaSymbol{\{}\AgdaBound{A} \AgdaSymbol{:} \AgdaDatatype{Type} \AgdaBound{Γ'}\AgdaSymbol{\}} \AgdaSymbol{→} \AgdaDatatype{Term} \AgdaSymbol{\{}\AgdaBound{Γ}\AgdaSymbol{\}} \AgdaSymbol{(}\AgdaInductiveConstructor{‘Term’} \AgdaInductiveConstructor{‘’₁} \AgdaInductiveConstructor{⌜} \AgdaBound{Γ'} \AgdaInductiveConstructor{⌝ᶜ} \AgdaInductiveConstructor{‘’} \AgdaInductiveConstructor{⌜} \AgdaBound{A} \AgdaInductiveConstructor{⌝ᵀ} \AgdaInductiveConstructor{‘→’} \AgdaInductiveConstructor{W} \AgdaSymbol{(}\AgdaInductiveConstructor{‘Term’} \AgdaInductiveConstructor{‘’₁} \AgdaInductiveConstructor{⌜} \AgdaBound{Γ} \AgdaInductiveConstructor{⌝ᶜ} \AgdaInductiveConstructor{‘’} \AgdaInductiveConstructor{⌜} \AgdaInductiveConstructor{‘Term’} \AgdaInductiveConstructor{‘’₁} \AgdaInductiveConstructor{⌜} \AgdaBound{Γ'} \AgdaInductiveConstructor{⌝ᶜ} \AgdaInductiveConstructor{‘’} \AgdaInductiveConstructor{⌜} \AgdaBound{A} \AgdaInductiveConstructor{⌝ᵀ} \AgdaInductiveConstructor{⌝ᵀ}\AgdaSymbol{))}\<%
\\
\>[3]\AgdaIndent{4}{}\<[4]%
\>[4]\AgdaInductiveConstructor{‘quote-Σ’} \AgdaSymbol{:} \AgdaSymbol{∀} \AgdaSymbol{\{}\AgdaBound{Γ} \AgdaBound{Γ'}\AgdaSymbol{\}} \AgdaSymbol{→} \AgdaDatatype{Term} \AgdaSymbol{\{}\AgdaBound{Γ}\AgdaSymbol{\}} \AgdaSymbol{(}\AgdaInductiveConstructor{‘Σ’} \AgdaInductiveConstructor{‘Context’} \AgdaInductiveConstructor{‘Type’} \AgdaInductiveConstructor{‘→’} \AgdaInductiveConstructor{W} \AgdaSymbol{(}\AgdaInductiveConstructor{‘Term’} \AgdaInductiveConstructor{‘’₁} \AgdaInductiveConstructor{⌜} \AgdaBound{Γ'} \AgdaInductiveConstructor{⌝ᶜ} \AgdaInductiveConstructor{‘’} \AgdaInductiveConstructor{⌜} \AgdaInductiveConstructor{‘Σ’} \AgdaInductiveConstructor{‘Context’} \AgdaInductiveConstructor{‘Type’} \AgdaInductiveConstructor{⌝ᵀ}\AgdaSymbol{))}\<%
\\
\>[3]\AgdaIndent{4}{}\<[4]%
\>[4]\AgdaInductiveConstructor{‘cast’} \AgdaSymbol{:} \AgdaDatatype{Term} \AgdaSymbol{\{}\AgdaInductiveConstructor{ε}\AgdaSymbol{\}} \AgdaSymbol{(}\AgdaInductiveConstructor{‘Σ’} \AgdaInductiveConstructor{‘Context’} \AgdaInductiveConstructor{‘Type’} \AgdaInductiveConstructor{‘→’} \AgdaInductiveConstructor{W} \AgdaSymbol{(}\AgdaInductiveConstructor{‘Type’} \AgdaInductiveConstructor{‘’} \AgdaInductiveConstructor{⌜} \AgdaInductiveConstructor{ε} \AgdaInductiveConstructor{▻} \AgdaInductiveConstructor{‘Σ’} \AgdaInductiveConstructor{‘Context’} \AgdaInductiveConstructor{‘Type’} \AgdaInductiveConstructor{⌝ᶜ}\AgdaSymbol{))}\<%
\\
\>[3]\AgdaIndent{4}{}\<[4]%
\>[4]\AgdaInductiveConstructor{SW} \AgdaSymbol{:} \AgdaSymbol{∀} \AgdaSymbol{\{}\AgdaBound{Γ} \AgdaBound{A} \AgdaBound{B}\AgdaSymbol{\}} \AgdaSymbol{\{}\AgdaBound{a} \AgdaSymbol{:} \AgdaDatatype{Term} \AgdaSymbol{\{}\AgdaBound{Γ}\AgdaSymbol{\}} \AgdaBound{A}\AgdaSymbol{\}} \AgdaSymbol{→} \AgdaDatatype{Term} \AgdaSymbol{\{}\AgdaBound{Γ}\AgdaSymbol{\}} \AgdaSymbol{(}\AgdaInductiveConstructor{W} \AgdaBound{B} \AgdaInductiveConstructor{‘’} \AgdaBound{a}\AgdaSymbol{)} \AgdaSymbol{→} \AgdaDatatype{Term} \AgdaSymbol{\{}\AgdaBound{Γ}\AgdaSymbol{\}} \AgdaBound{B}\<%
\\
\>[3]\AgdaIndent{4}{}\<[4]%
\>[4]\AgdaInductiveConstructor{WS∀} \AgdaSymbol{:} \AgdaSymbol{∀} \AgdaSymbol{\{}\AgdaBound{Γ} \AgdaBound{T} \AgdaBound{T'} \AgdaBound{A} \AgdaBound{B}\AgdaSymbol{\}} \AgdaSymbol{\{}\AgdaBound{a} \AgdaSymbol{:} \AgdaDatatype{Term} \AgdaSymbol{\{}\AgdaBound{Γ}\AgdaSymbol{\}} \AgdaBound{T}\AgdaSymbol{\}} \AgdaSymbol{→} \AgdaDatatype{Term} \AgdaSymbol{\{}\AgdaBound{Γ} \AgdaInductiveConstructor{▻} \AgdaBound{T'}\AgdaSymbol{\}} \AgdaSymbol{(}\AgdaInductiveConstructor{W} \AgdaSymbol{((}\AgdaBound{A} \AgdaInductiveConstructor{‘→’} \AgdaBound{B}\AgdaSymbol{)} \AgdaInductiveConstructor{‘’} \AgdaBound{a}\AgdaSymbol{))} \AgdaSymbol{→} \AgdaDatatype{Term} \AgdaSymbol{\{}\AgdaBound{Γ} \AgdaInductiveConstructor{▻} \AgdaBound{T'}\AgdaSymbol{\}} \AgdaSymbol{(}\AgdaInductiveConstructor{W} \AgdaSymbol{((}\AgdaBound{A} \AgdaInductiveConstructor{‘’} \AgdaBound{a}\AgdaSymbol{)} \AgdaInductiveConstructor{‘→’} \AgdaSymbol{(}\AgdaBound{B} \AgdaInductiveConstructor{‘’₁} \AgdaBound{a}\AgdaSymbol{)))}\<%
\\
\>[3]\AgdaIndent{4}{}\<[4]%
\>[4]\AgdaInductiveConstructor{SW₁V} \AgdaSymbol{:} \AgdaSymbol{∀} \AgdaSymbol{\{}\AgdaBound{Γ} \AgdaBound{A} \AgdaBound{T}\AgdaSymbol{\}} \AgdaSymbol{→} \AgdaDatatype{Term} \AgdaSymbol{\{}\AgdaBound{Γ} \AgdaInductiveConstructor{▻} \AgdaBound{A}\AgdaSymbol{\}} \AgdaSymbol{(}\AgdaInductiveConstructor{W₁} \AgdaBound{T} \AgdaInductiveConstructor{‘’} \AgdaInductiveConstructor{‘VAR₀’}\AgdaSymbol{)} \AgdaSymbol{→} \AgdaDatatype{Term} \AgdaSymbol{\{}\AgdaBound{Γ} \AgdaInductiveConstructor{▻} \AgdaBound{A}\AgdaSymbol{\}} \AgdaBound{T}\<%
\\
\>[3]\AgdaIndent{4}{}\<[4]%
\>[4]\AgdaInductiveConstructor{W∀} \AgdaSymbol{:} \AgdaSymbol{∀} \AgdaSymbol{\{}\AgdaBound{Γ} \AgdaBound{A} \AgdaBound{B} \AgdaBound{C}\AgdaSymbol{\}} \AgdaSymbol{→} \AgdaDatatype{Term} \AgdaSymbol{\{}\AgdaBound{Γ} \AgdaInductiveConstructor{▻} \AgdaBound{C}\AgdaSymbol{\}} \AgdaSymbol{(}\AgdaInductiveConstructor{W} \AgdaSymbol{(}\AgdaBound{A} \AgdaInductiveConstructor{‘→’} \AgdaBound{B}\AgdaSymbol{))} \AgdaSymbol{→} \AgdaDatatype{Term} \AgdaSymbol{\{}\AgdaBound{Γ} \AgdaInductiveConstructor{▻} \AgdaBound{C}\AgdaSymbol{\}} \AgdaSymbol{(}\AgdaInductiveConstructor{W} \AgdaBound{A} \AgdaInductiveConstructor{‘→’} \AgdaInductiveConstructor{W₁} \AgdaBound{B}\AgdaSymbol{)}\<%
\\
\>[3]\AgdaIndent{4}{}\<[4]%
\>[4]\AgdaInductiveConstructor{W∀⁻¹} \AgdaSymbol{:} \AgdaSymbol{∀} \AgdaSymbol{\{}\AgdaBound{Γ} \AgdaBound{A} \AgdaBound{B} \AgdaBound{C}\AgdaSymbol{\}} \AgdaSymbol{→} \AgdaDatatype{Term} \AgdaSymbol{\{}\AgdaBound{Γ} \AgdaInductiveConstructor{▻} \AgdaBound{C}\AgdaSymbol{\}} \AgdaSymbol{(}\AgdaInductiveConstructor{W} \AgdaBound{A} \AgdaInductiveConstructor{‘→’} \AgdaInductiveConstructor{W₁} \AgdaBound{B}\AgdaSymbol{)} \AgdaSymbol{→} \AgdaDatatype{Term} \AgdaSymbol{\{}\AgdaBound{Γ} \AgdaInductiveConstructor{▻} \AgdaBound{C}\AgdaSymbol{\}} \AgdaSymbol{(}\AgdaInductiveConstructor{W} \AgdaSymbol{(}\AgdaBound{A} \AgdaInductiveConstructor{‘→’} \AgdaBound{B}\AgdaSymbol{))}\<%
\\
\>[3]\AgdaIndent{4}{}\<[4]%
\>[4]\AgdaInductiveConstructor{WW∀} \AgdaSymbol{:} \AgdaSymbol{∀} \AgdaSymbol{\{}\AgdaBound{Γ} \AgdaBound{A} \AgdaBound{B} \AgdaBound{C} \AgdaBound{D}\AgdaSymbol{\}} \AgdaSymbol{→} \AgdaDatatype{Term} \AgdaSymbol{\{}\AgdaBound{Γ} \AgdaInductiveConstructor{▻} \AgdaBound{C} \AgdaInductiveConstructor{▻} \AgdaBound{D}\AgdaSymbol{\}} \AgdaSymbol{(}\AgdaInductiveConstructor{W} \AgdaSymbol{(}\AgdaInductiveConstructor{W} \AgdaSymbol{(}\AgdaBound{A} \AgdaInductiveConstructor{‘→’} \AgdaBound{B}\AgdaSymbol{)))} \AgdaSymbol{→} \AgdaDatatype{Term} \AgdaSymbol{\{}\AgdaBound{Γ} \AgdaInductiveConstructor{▻} \AgdaBound{C} \AgdaInductiveConstructor{▻} \AgdaBound{D}\AgdaSymbol{\}} \AgdaSymbol{(}\AgdaInductiveConstructor{W} \AgdaSymbol{(}\AgdaInductiveConstructor{W} \AgdaBound{A} \AgdaInductiveConstructor{‘→’} \AgdaInductiveConstructor{W₁} \AgdaBound{B}\AgdaSymbol{))}\<%
\\
\>[3]\AgdaIndent{4}{}\<[4]%
\>[4]\AgdaInductiveConstructor{S₁∀} \AgdaSymbol{:} \AgdaSymbol{∀} \AgdaSymbol{\{}\AgdaBound{Γ} \AgdaBound{T} \AgdaBound{T'} \AgdaBound{A} \AgdaBound{B}\AgdaSymbol{\}} \AgdaSymbol{\{}\AgdaBound{a} \AgdaSymbol{:} \AgdaDatatype{Term} \AgdaSymbol{\{}\AgdaBound{Γ}\AgdaSymbol{\}} \AgdaBound{T}\AgdaSymbol{\}} \AgdaSymbol{→} \AgdaDatatype{Term} \AgdaSymbol{\{}\AgdaBound{Γ} \AgdaInductiveConstructor{▻} \AgdaBound{T'} \AgdaInductiveConstructor{‘’} \AgdaBound{a}\AgdaSymbol{\}} \AgdaSymbol{((}\AgdaBound{A} \AgdaInductiveConstructor{‘→’} \AgdaBound{B}\AgdaSymbol{)} \AgdaInductiveConstructor{‘’₁} \AgdaBound{a}\AgdaSymbol{)} \AgdaSymbol{→} \AgdaDatatype{Term} \AgdaSymbol{\{}\AgdaBound{Γ} \AgdaInductiveConstructor{▻} \AgdaBound{T'} \AgdaInductiveConstructor{‘’} \AgdaBound{a}\AgdaSymbol{\}} \AgdaSymbol{((}\AgdaBound{A} \AgdaInductiveConstructor{‘’₁} \AgdaBound{a}\AgdaSymbol{)} \AgdaInductiveConstructor{‘→’} \AgdaSymbol{(}\AgdaBound{B} \AgdaInductiveConstructor{‘’₂} \AgdaBound{a}\AgdaSymbol{))}\<%
\\
\>[3]\AgdaIndent{4}{}\<[4]%
\>[4]\AgdaInductiveConstructor{S₁₀W⁻¹} \AgdaSymbol{:} \AgdaSymbol{∀} \AgdaSymbol{\{}\AgdaBound{Γ} \AgdaBound{C} \AgdaBound{T} \AgdaBound{A}\AgdaSymbol{\}} \AgdaSymbol{\{}\AgdaBound{a} \AgdaSymbol{:} \AgdaDatatype{Term} \AgdaSymbol{\{}\AgdaBound{Γ}\AgdaSymbol{\}} \AgdaBound{C}\AgdaSymbol{\}} \AgdaSymbol{\{}\AgdaBound{b} \AgdaSymbol{:} \AgdaDatatype{Term} \AgdaSymbol{\{}\AgdaBound{Γ}\AgdaSymbol{\}} \AgdaSymbol{(}\AgdaBound{T} \AgdaInductiveConstructor{‘’} \AgdaBound{a}\AgdaSymbol{)\}} \AgdaSymbol{→} \AgdaDatatype{Term} \AgdaSymbol{\{}\AgdaBound{Γ}\AgdaSymbol{\}} \AgdaSymbol{(}\AgdaBound{A} \AgdaInductiveConstructor{‘’} \AgdaBound{a}\AgdaSymbol{)} \AgdaSymbol{→} \AgdaDatatype{Term} \AgdaSymbol{\{}\AgdaBound{Γ}\AgdaSymbol{\}} \AgdaSymbol{(}\AgdaInductiveConstructor{W} \AgdaBound{A} \AgdaInductiveConstructor{‘’₁} \AgdaBound{a} \AgdaInductiveConstructor{‘’} \AgdaBound{b}\AgdaSymbol{)}\<%
\\
\>[3]\AgdaIndent{4}{}\<[4]%
\>[4]\AgdaInductiveConstructor{S₁₀W} \AgdaSymbol{:} \AgdaSymbol{∀} \AgdaSymbol{\{}\AgdaBound{Γ} \AgdaBound{C} \AgdaBound{T} \AgdaBound{A}\AgdaSymbol{\}} \AgdaSymbol{\{}\AgdaBound{a} \AgdaSymbol{:} \AgdaDatatype{Term} \AgdaSymbol{\{}\AgdaBound{Γ}\AgdaSymbol{\}} \AgdaBound{C}\AgdaSymbol{\}} \AgdaSymbol{\{}\AgdaBound{b} \AgdaSymbol{:} \AgdaDatatype{Term} \AgdaSymbol{\{}\AgdaBound{Γ}\AgdaSymbol{\}} \AgdaSymbol{(}\AgdaBound{T} \AgdaInductiveConstructor{‘’} \AgdaBound{a}\AgdaSymbol{)\}} \AgdaSymbol{→} \AgdaDatatype{Term} \AgdaSymbol{\{}\AgdaBound{Γ}\AgdaSymbol{\}} \AgdaSymbol{(}\AgdaInductiveConstructor{W} \AgdaBound{A} \AgdaInductiveConstructor{‘’₁} \AgdaBound{a} \AgdaInductiveConstructor{‘’} \AgdaBound{b}\AgdaSymbol{)} \AgdaSymbol{→} \AgdaDatatype{Term} \AgdaSymbol{\{}\AgdaBound{Γ}\AgdaSymbol{\}} \AgdaSymbol{(}\AgdaBound{A} \AgdaInductiveConstructor{‘’} \AgdaBound{a}\AgdaSymbol{)}\<%
\\
\>[3]\AgdaIndent{4}{}\<[4]%
\>[4]\AgdaInductiveConstructor{WWS₁₀W} \AgdaSymbol{:} \AgdaSymbol{∀} \AgdaSymbol{\{}\AgdaBound{Γ} \AgdaBound{C} \AgdaBound{T} \AgdaBound{A} \AgdaBound{D} \AgdaBound{E}\AgdaSymbol{\}} \AgdaSymbol{\{}\AgdaBound{a} \AgdaSymbol{:} \AgdaDatatype{Term} \AgdaSymbol{\{}\AgdaBound{Γ}\AgdaSymbol{\}} \AgdaBound{C}\AgdaSymbol{\}} \AgdaSymbol{\{}\AgdaBound{b} \AgdaSymbol{:} \AgdaDatatype{Term} \AgdaSymbol{\{}\AgdaBound{Γ}\AgdaSymbol{\}} \AgdaSymbol{(}\AgdaBound{T} \AgdaInductiveConstructor{‘’} \AgdaBound{a}\AgdaSymbol{)\}} \AgdaSymbol{→} \AgdaDatatype{Term} \AgdaSymbol{\{}\AgdaBound{Γ} \AgdaInductiveConstructor{▻} \AgdaBound{D} \AgdaInductiveConstructor{▻} \AgdaBound{E}\AgdaSymbol{\}} \AgdaSymbol{(}\AgdaInductiveConstructor{W} \AgdaSymbol{(}\AgdaInductiveConstructor{W} \AgdaSymbol{(}\AgdaInductiveConstructor{W} \AgdaBound{A} \AgdaInductiveConstructor{‘’₁} \AgdaBound{a} \AgdaInductiveConstructor{‘’} \AgdaBound{b}\AgdaSymbol{)))} \AgdaSymbol{→} \AgdaDatatype{Term} \AgdaSymbol{\{}\AgdaBound{Γ} \AgdaInductiveConstructor{▻} \AgdaBound{D} \AgdaInductiveConstructor{▻} \AgdaBound{E}\AgdaSymbol{\}} \AgdaSymbol{(}\AgdaInductiveConstructor{W} \AgdaSymbol{(}\AgdaInductiveConstructor{W} \AgdaSymbol{(}\AgdaBound{A} \AgdaInductiveConstructor{‘’} \AgdaBound{a}\AgdaSymbol{)))}\<%
\\
\>[3]\AgdaIndent{4}{}\<[4]%
\>[4]\AgdaInductiveConstructor{WS₂₁₀W⁻¹} \AgdaSymbol{:} \AgdaSymbol{∀} \AgdaSymbol{\{}\AgdaBound{Γ} \AgdaBound{A} \AgdaBound{B} \AgdaBound{C} \AgdaBound{T} \AgdaBound{T'}\AgdaSymbol{\}} \AgdaSymbol{\{}\AgdaBound{a} \AgdaSymbol{:} \AgdaDatatype{Term} \AgdaSymbol{\{}\AgdaBound{Γ}\AgdaSymbol{\}} \AgdaBound{A}\AgdaSymbol{\}} \AgdaSymbol{\{}\AgdaBound{b} \AgdaSymbol{:} \AgdaDatatype{Term} \AgdaSymbol{\{}\AgdaBound{Γ}\AgdaSymbol{\}} \AgdaSymbol{(}\AgdaBound{B} \AgdaInductiveConstructor{‘’} \AgdaBound{a}\AgdaSymbol{)\}} \AgdaSymbol{\{}\AgdaBound{c} \AgdaSymbol{:} \AgdaDatatype{Term} \AgdaSymbol{\{}\AgdaBound{Γ}\AgdaSymbol{\}} \AgdaSymbol{(}\AgdaBound{C} \AgdaInductiveConstructor{‘’₁} \AgdaBound{a} \AgdaInductiveConstructor{‘’} \AgdaBound{b}\AgdaSymbol{)\}}\<%
\\
\>[4]\AgdaIndent{6}{}\<[6]%
\>[6]\AgdaSymbol{→} \AgdaDatatype{Term} \AgdaSymbol{\{}\AgdaBound{Γ} \AgdaInductiveConstructor{▻} \AgdaBound{T'}\AgdaSymbol{\}} \AgdaSymbol{(}\AgdaInductiveConstructor{W} \AgdaSymbol{(}\AgdaBound{T} \AgdaInductiveConstructor{‘’₁} \AgdaBound{a} \AgdaInductiveConstructor{‘’} \AgdaBound{b}\AgdaSymbol{))}\<%
\\
\>[4]\AgdaIndent{6}{}\<[6]%
\>[6]\AgdaSymbol{→} \AgdaDatatype{Term} \AgdaSymbol{\{}\AgdaBound{Γ} \AgdaInductiveConstructor{▻} \AgdaBound{T'}\AgdaSymbol{\}} \AgdaSymbol{(}\AgdaInductiveConstructor{W} \AgdaSymbol{(}\AgdaInductiveConstructor{W} \AgdaBound{T} \AgdaInductiveConstructor{‘’₂} \AgdaBound{a} \AgdaInductiveConstructor{‘’₁} \AgdaBound{b} \AgdaInductiveConstructor{‘’} \AgdaBound{c}\AgdaSymbol{))}\<%
\\
\>[0]\AgdaIndent{4}{}\<[4]%
\>[4]\AgdaInductiveConstructor{S₂₁₀W} \AgdaSymbol{:} \AgdaSymbol{∀} \AgdaSymbol{\{}\AgdaBound{Γ} \AgdaBound{A} \AgdaBound{B} \AgdaBound{C} \AgdaBound{T}\AgdaSymbol{\}} \AgdaSymbol{\{}\AgdaBound{a} \AgdaSymbol{:} \AgdaDatatype{Term} \AgdaSymbol{\{}\AgdaBound{Γ}\AgdaSymbol{\}} \AgdaBound{A}\AgdaSymbol{\}} \AgdaSymbol{\{}\AgdaBound{b} \AgdaSymbol{:} \AgdaDatatype{Term} \AgdaSymbol{\{}\AgdaBound{Γ}\AgdaSymbol{\}} \AgdaSymbol{(}\AgdaBound{B} \AgdaInductiveConstructor{‘’} \AgdaBound{a}\AgdaSymbol{)\}} \AgdaSymbol{\{}\AgdaBound{c} \AgdaSymbol{:} \AgdaDatatype{Term} \AgdaSymbol{\{}\AgdaBound{Γ}\AgdaSymbol{\}} \AgdaSymbol{(}\AgdaBound{C} \AgdaInductiveConstructor{‘’₁} \AgdaBound{a} \AgdaInductiveConstructor{‘’} \AgdaBound{b}\AgdaSymbol{)\}}\<%
\\
\>[4]\AgdaIndent{6}{}\<[6]%
\>[6]\AgdaSymbol{→} \AgdaDatatype{Term} \AgdaSymbol{\{}\AgdaBound{Γ}\AgdaSymbol{\}} \AgdaSymbol{(}\AgdaInductiveConstructor{W} \AgdaBound{T} \AgdaInductiveConstructor{‘’₂} \AgdaBound{a} \AgdaInductiveConstructor{‘’₁} \AgdaBound{b} \AgdaInductiveConstructor{‘’} \AgdaBound{c}\AgdaSymbol{)}\<%
\\
\>[4]\AgdaIndent{6}{}\<[6]%
\>[6]\AgdaSymbol{→} \AgdaDatatype{Term} \AgdaSymbol{\{}\AgdaBound{Γ}\AgdaSymbol{\}} \AgdaSymbol{(}\AgdaBound{T} \AgdaInductiveConstructor{‘’₁} \AgdaBound{a} \AgdaInductiveConstructor{‘’} \AgdaBound{b}\AgdaSymbol{)}\<%
\\
\>[0]\AgdaIndent{4}{}\<[4]%
\>[4]\AgdaInductiveConstructor{W₁₀} \AgdaSymbol{:} \AgdaSymbol{∀} \AgdaSymbol{\{}\AgdaBound{Γ} \AgdaBound{A} \AgdaBound{B} \AgdaBound{C}\AgdaSymbol{\}} \AgdaSymbol{→} \AgdaDatatype{Term} \AgdaSymbol{\{}\AgdaBound{Γ} \AgdaInductiveConstructor{▻} \AgdaBound{A} \AgdaInductiveConstructor{▻} \AgdaInductiveConstructor{W} \AgdaBound{B}\AgdaSymbol{\}} \AgdaSymbol{(}\AgdaInductiveConstructor{W₁} \AgdaSymbol{(}\AgdaInductiveConstructor{W} \AgdaBound{C}\AgdaSymbol{))} \AgdaSymbol{→} \AgdaDatatype{Term} \AgdaSymbol{\{}\AgdaBound{Γ} \AgdaInductiveConstructor{▻} \AgdaBound{A} \AgdaInductiveConstructor{▻} \AgdaInductiveConstructor{W} \AgdaBound{B}\AgdaSymbol{\}} \AgdaSymbol{(}\AgdaInductiveConstructor{W} \AgdaSymbol{(}\AgdaInductiveConstructor{W} \AgdaBound{C}\AgdaSymbol{))}\<%
\\
\>[0]\AgdaIndent{4}{}\<[4]%
\>[4]\AgdaInductiveConstructor{W₁₀⁻¹} \AgdaSymbol{:} \AgdaSymbol{∀} \AgdaSymbol{\{}\AgdaBound{Γ} \AgdaBound{A} \AgdaBound{B} \AgdaBound{C}\AgdaSymbol{\}} \AgdaSymbol{→} \AgdaDatatype{Term} \AgdaSymbol{\{}\AgdaBound{Γ} \AgdaInductiveConstructor{▻} \AgdaBound{A} \AgdaInductiveConstructor{▻} \AgdaInductiveConstructor{W} \AgdaBound{B}\AgdaSymbol{\}} \AgdaSymbol{(}\AgdaInductiveConstructor{W} \AgdaSymbol{(}\AgdaInductiveConstructor{W} \AgdaBound{C}\AgdaSymbol{))} \AgdaSymbol{→} \AgdaDatatype{Term} \AgdaSymbol{\{}\AgdaBound{Γ} \AgdaInductiveConstructor{▻} \AgdaBound{A} \AgdaInductiveConstructor{▻} \AgdaInductiveConstructor{W} \AgdaBound{B}\AgdaSymbol{\}} \AgdaSymbol{(}\AgdaInductiveConstructor{W₁} \AgdaSymbol{(}\AgdaInductiveConstructor{W} \AgdaBound{C}\AgdaSymbol{))}\<%
\\
\>[0]\AgdaIndent{4}{}\<[4]%
\>[4]\AgdaInductiveConstructor{W₁₀₁₀} \AgdaSymbol{:} \AgdaSymbol{∀} \AgdaSymbol{\{}\AgdaBound{Γ} \AgdaBound{A} \AgdaBound{B} \AgdaBound{C} \AgdaBound{T}\AgdaSymbol{\}} \AgdaSymbol{→} \AgdaDatatype{Term} \AgdaSymbol{\{}\AgdaBound{Γ} \AgdaInductiveConstructor{▻} \AgdaBound{A} \AgdaInductiveConstructor{▻} \AgdaBound{B} \AgdaInductiveConstructor{▻} \AgdaInductiveConstructor{W} \AgdaSymbol{(}\AgdaInductiveConstructor{W} \AgdaBound{C}\AgdaSymbol{)\}} \AgdaSymbol{(}\AgdaInductiveConstructor{W₁} \AgdaSymbol{(}\AgdaInductiveConstructor{W₁} \AgdaSymbol{(}\AgdaInductiveConstructor{W} \AgdaBound{T}\AgdaSymbol{)))} \AgdaSymbol{→} \AgdaDatatype{Term} \AgdaSymbol{\{}\AgdaBound{Γ} \AgdaInductiveConstructor{▻} \AgdaBound{A} \AgdaInductiveConstructor{▻} \AgdaBound{B} \AgdaInductiveConstructor{▻} \AgdaInductiveConstructor{W} \AgdaSymbol{(}\AgdaInductiveConstructor{W} \AgdaBound{C}\AgdaSymbol{)\}} \AgdaSymbol{(}\AgdaInductiveConstructor{W₁} \AgdaSymbol{(}\AgdaInductiveConstructor{W} \AgdaSymbol{(}\AgdaInductiveConstructor{W} \AgdaBound{T}\AgdaSymbol{)))}\<%
\\
\>[0]\AgdaIndent{4}{}\<[4]%
\>[4]\AgdaInductiveConstructor{S₁W₁} \AgdaSymbol{:} \AgdaSymbol{∀} \AgdaSymbol{\{}\AgdaBound{Γ} \AgdaBound{A} \AgdaBound{B} \AgdaBound{C}\AgdaSymbol{\}} \AgdaSymbol{\{}\AgdaBound{a} \AgdaSymbol{:} \AgdaDatatype{Term} \AgdaSymbol{\{}\AgdaBound{Γ}\AgdaSymbol{\}} \AgdaBound{A}\AgdaSymbol{\}} \AgdaSymbol{→} \AgdaDatatype{Term} \AgdaSymbol{\{}\AgdaBound{Γ} \AgdaInductiveConstructor{▻} \AgdaInductiveConstructor{W} \AgdaBound{B} \AgdaInductiveConstructor{‘’} \AgdaBound{a}\AgdaSymbol{\}} \AgdaSymbol{(}\AgdaInductiveConstructor{W₁} \AgdaBound{C} \AgdaInductiveConstructor{‘’₁} \AgdaBound{a}\AgdaSymbol{)} \AgdaSymbol{→} \AgdaDatatype{Term} \AgdaSymbol{\{}\AgdaBound{Γ} \AgdaInductiveConstructor{▻} \AgdaBound{B}\AgdaSymbol{\}} \AgdaBound{C}\<%
\\
\>[0]\AgdaIndent{4}{}\<[4]%
\>[4]\AgdaInductiveConstructor{W₁SW₁⁻¹} \AgdaSymbol{:} \AgdaSymbol{∀} \AgdaSymbol{\{}\AgdaBound{Γ} \AgdaBound{A} \AgdaBound{T''} \AgdaBound{T'} \AgdaBound{T}\AgdaSymbol{\}} \AgdaSymbol{\{}\AgdaBound{a} \AgdaSymbol{:} \AgdaDatatype{Term} \AgdaSymbol{\{}\AgdaBound{Γ}\AgdaSymbol{\}} \AgdaBound{A}\AgdaSymbol{\}}\<%
\\
\>[4]\AgdaIndent{6}{}\<[6]%
\>[6]\AgdaSymbol{→} \AgdaDatatype{Term} \AgdaSymbol{\{}\AgdaBound{Γ} \AgdaInductiveConstructor{▻} \AgdaBound{T''} \AgdaInductiveConstructor{▻} \AgdaInductiveConstructor{W} \AgdaSymbol{(}\AgdaBound{T'} \AgdaInductiveConstructor{‘’} \AgdaBound{a}\AgdaSymbol{)\}} \AgdaSymbol{(}\AgdaInductiveConstructor{W₁} \AgdaSymbol{(}\AgdaInductiveConstructor{W} \AgdaSymbol{(}\AgdaBound{T} \AgdaInductiveConstructor{‘’} \AgdaBound{a}\AgdaSymbol{)))}\<%
\\
\>[4]\AgdaIndent{6}{}\<[6]%
\>[6]\AgdaSymbol{→} \AgdaDatatype{Term} \AgdaSymbol{\{}\AgdaBound{Γ} \AgdaInductiveConstructor{▻} \AgdaBound{T''} \AgdaInductiveConstructor{▻} \AgdaInductiveConstructor{W} \AgdaSymbol{(}\AgdaBound{T'} \AgdaInductiveConstructor{‘’} \AgdaBound{a}\AgdaSymbol{)\}} \AgdaSymbol{(}\AgdaInductiveConstructor{W₁} \AgdaSymbol{(}\AgdaInductiveConstructor{W} \AgdaBound{T} \AgdaInductiveConstructor{‘’₁} \AgdaBound{a}\AgdaSymbol{))}\<%
\\
\>[0]\AgdaIndent{4}{}\<[4]%
\>[4]\AgdaInductiveConstructor{W₁SW₁} \AgdaSymbol{:} \AgdaSymbol{∀} \AgdaSymbol{\{}\AgdaBound{Γ} \AgdaBound{A} \AgdaBound{T''} \AgdaBound{T'} \AgdaBound{T}\AgdaSymbol{\}} \AgdaSymbol{\{}\AgdaBound{a} \AgdaSymbol{:} \AgdaDatatype{Term} \AgdaSymbol{\{}\AgdaBound{Γ}\AgdaSymbol{\}} \AgdaBound{A}\AgdaSymbol{\}}\<%
\\
\>[4]\AgdaIndent{6}{}\<[6]%
\>[6]\AgdaSymbol{→} \AgdaDatatype{Term} \AgdaSymbol{\{}\AgdaBound{Γ} \AgdaInductiveConstructor{▻} \AgdaBound{T''} \AgdaInductiveConstructor{▻} \AgdaInductiveConstructor{W} \AgdaSymbol{(}\AgdaBound{T'} \AgdaInductiveConstructor{‘’} \AgdaBound{a}\AgdaSymbol{)\}} \AgdaSymbol{(}\AgdaInductiveConstructor{W₁} \AgdaSymbol{(}\AgdaInductiveConstructor{W} \AgdaBound{T} \AgdaInductiveConstructor{‘’₁} \AgdaBound{a}\AgdaSymbol{))}\<%
\\
\>[4]\AgdaIndent{6}{}\<[6]%
\>[6]\AgdaSymbol{→} \AgdaDatatype{Term} \AgdaSymbol{\{}\AgdaBound{Γ} \AgdaInductiveConstructor{▻} \AgdaBound{T''} \AgdaInductiveConstructor{▻} \AgdaInductiveConstructor{W} \AgdaSymbol{(}\AgdaBound{T'} \AgdaInductiveConstructor{‘’} \AgdaBound{a}\AgdaSymbol{)\}} \AgdaSymbol{(}\AgdaInductiveConstructor{W₁} \AgdaSymbol{(}\AgdaInductiveConstructor{W} \AgdaSymbol{(}\AgdaBound{T} \AgdaInductiveConstructor{‘’} \AgdaBound{a}\AgdaSymbol{)))}\<%
\\
\>[0]\AgdaIndent{4}{}\<[4]%
\>[4]\AgdaInductiveConstructor{WSSW₁} \AgdaSymbol{:} \AgdaSymbol{∀} \AgdaSymbol{\{}\AgdaBound{Γ} \AgdaBound{T'} \AgdaBound{B} \AgdaBound{A}\AgdaSymbol{\}} \AgdaSymbol{\{}\AgdaBound{b} \AgdaSymbol{:} \AgdaDatatype{Term} \AgdaSymbol{\{}\AgdaBound{Γ}\AgdaSymbol{\}} \AgdaBound{B}\AgdaSymbol{\}} \AgdaSymbol{\{}\AgdaBound{a} \AgdaSymbol{:} \AgdaDatatype{Term} \AgdaSymbol{\{}\AgdaBound{Γ} \AgdaInductiveConstructor{▻} \AgdaBound{B}\AgdaSymbol{\}} \AgdaSymbol{(}\AgdaInductiveConstructor{W} \AgdaBound{A}\AgdaSymbol{)\}} \AgdaSymbol{\{}\AgdaBound{T} \AgdaSymbol{:} \AgdaDatatype{Type} \AgdaSymbol{(}\AgdaBound{Γ} \AgdaInductiveConstructor{▻} \AgdaBound{A}\AgdaSymbol{)\}}\<%
\\
\>[4]\AgdaIndent{6}{}\<[6]%
\>[6]\AgdaSymbol{→} \AgdaDatatype{Term} \AgdaSymbol{\{}\AgdaBound{Γ} \AgdaInductiveConstructor{▻} \AgdaBound{T'}\AgdaSymbol{\}} \AgdaSymbol{(}\AgdaInductiveConstructor{W} \AgdaSymbol{(}\AgdaInductiveConstructor{W₁} \AgdaBound{T} \AgdaInductiveConstructor{‘’} \AgdaBound{a} \AgdaInductiveConstructor{‘’} \AgdaBound{b}\AgdaSymbol{))}\<%
\\
\>[4]\AgdaIndent{6}{}\<[6]%
\>[6]\AgdaSymbol{→} \AgdaDatatype{Term} \AgdaSymbol{\{}\AgdaBound{Γ} \AgdaInductiveConstructor{▻} \AgdaBound{T'}\AgdaSymbol{\}} \AgdaSymbol{(}\AgdaInductiveConstructor{W} \AgdaSymbol{(}\AgdaBound{T} \AgdaInductiveConstructor{‘’} \AgdaSymbol{(}\AgdaInductiveConstructor{SW} \AgdaSymbol{((}\AgdaInductiveConstructor{‘λ’} \AgdaBound{a}\AgdaSymbol{)} \AgdaInductiveConstructor{‘’ₐ} \AgdaBound{b}\AgdaSymbol{))))}\<%
\\
\>[0]\AgdaIndent{4}{}\<[4]%
\>[4]\AgdaInductiveConstructor{WSSW₁⁻¹} \AgdaSymbol{:} \AgdaSymbol{∀} \AgdaSymbol{\{}\AgdaBound{Γ} \AgdaBound{T'} \AgdaBound{B} \AgdaBound{A}\AgdaSymbol{\}} \AgdaSymbol{\{}\AgdaBound{b} \AgdaSymbol{:} \AgdaDatatype{Term} \AgdaSymbol{\{}\AgdaBound{Γ}\AgdaSymbol{\}} \AgdaBound{B}\AgdaSymbol{\}} \AgdaSymbol{\{}\AgdaBound{a} \AgdaSymbol{:} \AgdaDatatype{Term} \AgdaSymbol{\{}\AgdaBound{Γ} \AgdaInductiveConstructor{▻} \AgdaBound{B}\AgdaSymbol{\}} \AgdaSymbol{(}\AgdaInductiveConstructor{W} \AgdaBound{A}\AgdaSymbol{)\}} \AgdaSymbol{\{}\AgdaBound{T} \AgdaSymbol{:} \AgdaDatatype{Type} \AgdaSymbol{(}\AgdaBound{Γ} \AgdaInductiveConstructor{▻} \AgdaBound{A}\AgdaSymbol{)\}}\<%
\\
\>[4]\AgdaIndent{6}{}\<[6]%
\>[6]\AgdaSymbol{→} \AgdaDatatype{Term} \AgdaSymbol{\{}\AgdaBound{Γ} \AgdaInductiveConstructor{▻} \AgdaBound{T'}\AgdaSymbol{\}} \AgdaSymbol{(}\AgdaInductiveConstructor{W} \AgdaSymbol{(}\AgdaBound{T} \AgdaInductiveConstructor{‘’} \AgdaSymbol{(}\AgdaInductiveConstructor{SW} \AgdaSymbol{((}\AgdaInductiveConstructor{‘λ’} \AgdaBound{a}\AgdaSymbol{)} \AgdaInductiveConstructor{‘’ₐ} \AgdaBound{b}\AgdaSymbol{))))}\<%
\\
\>[4]\AgdaIndent{6}{}\<[6]%
\>[6]\AgdaSymbol{→} \AgdaDatatype{Term} \AgdaSymbol{\{}\AgdaBound{Γ} \AgdaInductiveConstructor{▻} \AgdaBound{T'}\AgdaSymbol{\}} \AgdaSymbol{(}\AgdaInductiveConstructor{W} \AgdaSymbol{(}\AgdaInductiveConstructor{W₁} \AgdaBound{T} \AgdaInductiveConstructor{‘’} \AgdaBound{a} \AgdaInductiveConstructor{‘’} \AgdaBound{b}\AgdaSymbol{))}\<%
\\
\>[0]\AgdaIndent{4}{}\<[4]%
\>[4]\AgdaInductiveConstructor{SW₁₀} \AgdaSymbol{:} \AgdaSymbol{∀} \AgdaSymbol{\{}\AgdaBound{Γ} \AgdaBound{T}\AgdaSymbol{\}} \AgdaSymbol{\{}\AgdaBound{A} \AgdaSymbol{:} \AgdaDatatype{Type} \AgdaBound{Γ}\AgdaSymbol{\}} \AgdaSymbol{\{}\AgdaBound{B} \AgdaSymbol{:} \AgdaDatatype{Type} \AgdaBound{Γ}\AgdaSymbol{\}}\<%
\\
\>[4]\AgdaIndent{6}{}\<[6]%
\>[6]\AgdaSymbol{→} \AgdaSymbol{\{}\AgdaBound{a} \AgdaSymbol{:} \AgdaDatatype{Term} \AgdaSymbol{\{}\AgdaBound{Γ} \AgdaInductiveConstructor{▻} \AgdaBound{T}\AgdaSymbol{\}} \AgdaSymbol{(}\AgdaInductiveConstructor{W} \AgdaSymbol{\{}\AgdaBound{Γ}\AgdaSymbol{\}} \AgdaSymbol{\{}\AgdaBound{T}\AgdaSymbol{\}} \AgdaBound{B}\AgdaSymbol{)\}}\<%
\\
\>[4]\AgdaIndent{6}{}\<[6]%
\>[6]\AgdaSymbol{→} \AgdaDatatype{Term} \AgdaSymbol{\{}\AgdaBound{Γ} \AgdaInductiveConstructor{▻} \AgdaBound{T}\AgdaSymbol{\}} \AgdaSymbol{(}\AgdaInductiveConstructor{W₁} \AgdaSymbol{(}\AgdaInductiveConstructor{W} \AgdaBound{A}\AgdaSymbol{)} \AgdaInductiveConstructor{‘’} \AgdaBound{a}\AgdaSymbol{)}\<%
\\
\>[4]\AgdaIndent{6}{}\<[6]%
\>[6]\AgdaSymbol{→} \AgdaDatatype{Term} \AgdaSymbol{\{}\AgdaBound{Γ} \AgdaInductiveConstructor{▻} \AgdaBound{T}\AgdaSymbol{\}} \AgdaSymbol{(}\AgdaInductiveConstructor{W} \AgdaBound{A}\AgdaSymbol{)}\<%
\\
\>[0]\AgdaIndent{4}{}\<[4]%
\>[4]\AgdaInductiveConstructor{WS₂₁₀W₁} \AgdaSymbol{:} \AgdaSymbol{∀} \AgdaSymbol{\{}\AgdaBound{Γ} \AgdaBound{A} \AgdaBound{B} \AgdaBound{C} \AgdaBound{T} \AgdaBound{T'}\AgdaSymbol{\}} \AgdaSymbol{\{}\AgdaBound{a} \AgdaSymbol{:} \AgdaDatatype{Term} \AgdaSymbol{\{}\AgdaBound{Γ}\AgdaSymbol{\}} \AgdaBound{A}\AgdaSymbol{\}} \AgdaSymbol{\{}\AgdaBound{b} \AgdaSymbol{:} \AgdaDatatype{Term} \AgdaSymbol{(}\AgdaBound{B} \AgdaInductiveConstructor{‘’} \AgdaBound{a}\AgdaSymbol{)\}} \AgdaSymbol{\{}\AgdaBound{c} \AgdaSymbol{:} \AgdaDatatype{Term} \AgdaSymbol{(}\AgdaBound{C} \AgdaInductiveConstructor{‘’} \AgdaBound{a}\AgdaSymbol{)\}}\<%
\\
\>[4]\AgdaIndent{6}{}\<[6]%
\>[6]\AgdaSymbol{→} \AgdaDatatype{Term} \AgdaSymbol{\{(}\AgdaBound{Γ} \AgdaInductiveConstructor{▻} \AgdaBound{T'}\AgdaSymbol{)\}} \AgdaSymbol{(}\AgdaInductiveConstructor{W} \AgdaSymbol{(}\AgdaInductiveConstructor{W₁} \AgdaBound{T} \AgdaInductiveConstructor{‘’₂} \AgdaBound{a} \AgdaInductiveConstructor{‘’₁} \AgdaBound{b} \AgdaInductiveConstructor{‘’} \AgdaInductiveConstructor{S₁₀W⁻¹} \AgdaBound{c}\AgdaSymbol{))}\<%
\\
\>[4]\AgdaIndent{6}{}\<[6]%
\>[6]\AgdaSymbol{→} \AgdaDatatype{Term} \AgdaSymbol{\{(}\AgdaBound{Γ} \AgdaInductiveConstructor{▻} \AgdaBound{T'}\AgdaSymbol{)\}} \AgdaSymbol{(}\AgdaInductiveConstructor{W} \AgdaSymbol{(}\AgdaBound{T} \AgdaInductiveConstructor{‘’₁} \AgdaBound{a} \AgdaInductiveConstructor{‘’} \AgdaBound{c}\AgdaSymbol{))}\<%
\\
\>[0]\AgdaIndent{4}{}\<[4]%
\>[4]\AgdaInductiveConstructor{S₁₀∀} \AgdaSymbol{:} \AgdaSymbol{∀} \AgdaSymbol{\{}\AgdaBound{Γ} \AgdaBound{T} \AgdaBound{T'} \AgdaBound{A} \AgdaBound{B} \AgdaBound{a} \AgdaBound{b}\AgdaSymbol{\}} \AgdaSymbol{→} \AgdaDatatype{Term} \AgdaSymbol{((}\AgdaInductiveConstructor{\_‘→’\_} \AgdaSymbol{\{}\AgdaBound{Γ} \AgdaInductiveConstructor{▻} \AgdaBound{T} \AgdaInductiveConstructor{▻} \AgdaBound{T'}\AgdaSymbol{\}} \AgdaBound{A} \AgdaBound{B}\AgdaSymbol{)} \AgdaInductiveConstructor{‘’₁} \AgdaBound{a} \AgdaInductiveConstructor{‘’} \AgdaBound{b}\AgdaSymbol{)} \AgdaSymbol{→} \AgdaDatatype{Term} \AgdaSymbol{(}\AgdaInductiveConstructor{\_‘→’\_} \AgdaSymbol{\{}\AgdaBound{Γ}\AgdaSymbol{\}} \AgdaSymbol{(}\AgdaBound{A} \AgdaInductiveConstructor{‘’₁} \AgdaBound{a} \AgdaInductiveConstructor{‘’} \AgdaBound{b}\AgdaSymbol{)} \AgdaSymbol{(}\AgdaBound{B} \AgdaInductiveConstructor{‘’₂} \AgdaBound{a} \AgdaInductiveConstructor{‘’₁} \AgdaBound{b}\AgdaSymbol{))}\<%
\\
\>[0]\AgdaIndent{4}{}\<[4]%
\>[4]\AgdaInductiveConstructor{S₂₀₀W₁₀₀} \AgdaSymbol{:} \AgdaSymbol{∀} \AgdaSymbol{\{}\AgdaBound{Γ} \AgdaBound{A}\AgdaSymbol{\}} \AgdaSymbol{\{}\AgdaBound{T} \AgdaSymbol{:} \AgdaDatatype{Type} \AgdaSymbol{(}\AgdaBound{Γ} \AgdaInductiveConstructor{▻} \AgdaBound{A}\AgdaSymbol{)\}} \AgdaSymbol{\{}\AgdaBound{T'} \AgdaBound{C} \AgdaBound{B}\AgdaSymbol{\}} \AgdaSymbol{\{}\AgdaBound{a} \AgdaSymbol{:} \AgdaDatatype{Term} \AgdaSymbol{\{}\AgdaBound{Γ}\AgdaSymbol{\}} \AgdaBound{A}\AgdaSymbol{\}} \AgdaSymbol{\{}\AgdaBound{b} \AgdaSymbol{:} \AgdaDatatype{Term} \AgdaSymbol{\{(}\AgdaBound{Γ} \AgdaInductiveConstructor{▻} \AgdaBound{C} \AgdaInductiveConstructor{‘’} \AgdaBound{a}\AgdaSymbol{)\}} \AgdaSymbol{(}\AgdaBound{B} \AgdaInductiveConstructor{‘’₁} \AgdaBound{a}\AgdaSymbol{)\}}\<%
\\
\>[4]\AgdaIndent{9}{}\<[9]%
\>[9]\AgdaSymbol{\{}\AgdaBound{c} \AgdaSymbol{:} \AgdaDatatype{Term} \AgdaSymbol{\{(}\AgdaBound{Γ} \AgdaInductiveConstructor{▻} \AgdaBound{T'}\AgdaSymbol{)\}} \AgdaSymbol{(}\AgdaInductiveConstructor{W} \AgdaSymbol{(}\AgdaBound{C} \AgdaInductiveConstructor{‘’} \AgdaBound{a}\AgdaSymbol{))\}}\<%
\\
\>[4]\AgdaIndent{9}{}\<[9]%
\>[9]\AgdaSymbol{→} \AgdaDatatype{Term} \AgdaSymbol{\{(}\AgdaBound{Γ} \AgdaInductiveConstructor{▻} \AgdaBound{T'}\AgdaSymbol{)\}} \AgdaSymbol{(}\AgdaInductiveConstructor{W₁} \AgdaSymbol{(}\AgdaInductiveConstructor{W} \AgdaSymbol{(}\AgdaInductiveConstructor{W} \AgdaBound{T}\AgdaSymbol{)} \AgdaInductiveConstructor{‘’₂} \AgdaBound{a} \AgdaInductiveConstructor{‘’} \AgdaBound{b}\AgdaSymbol{)} \AgdaInductiveConstructor{‘’} \AgdaBound{c}\AgdaSymbol{)}\<%
\\
\>[4]\AgdaIndent{9}{}\<[9]%
\>[9]\AgdaSymbol{→} \AgdaDatatype{Term} \AgdaSymbol{\{(}\AgdaBound{Γ} \AgdaInductiveConstructor{▻} \AgdaBound{T'}\AgdaSymbol{)\}} \AgdaSymbol{(}\AgdaInductiveConstructor{W} \AgdaSymbol{(}\AgdaBound{T} \AgdaInductiveConstructor{‘’} \AgdaBound{a}\AgdaSymbol{))}\<%
\\
\>[0]\AgdaIndent{4}{}\<[4]%
\>[4]\AgdaInductiveConstructor{S₁₀W₂₀} \AgdaSymbol{:} \AgdaSymbol{∀} \AgdaSymbol{\{}\AgdaBound{Γ} \AgdaBound{T'} \AgdaBound{A} \AgdaBound{B} \AgdaBound{T}\AgdaSymbol{\}} \AgdaSymbol{\{}\AgdaBound{a} \AgdaSymbol{:} \AgdaDatatype{Term} \AgdaSymbol{\{}\AgdaBound{Γ} \AgdaInductiveConstructor{▻} \AgdaBound{T'}\AgdaSymbol{\}} \AgdaSymbol{(}\AgdaInductiveConstructor{W} \AgdaBound{A}\AgdaSymbol{)\}} \AgdaSymbol{\{}\AgdaBound{b} \AgdaSymbol{:} \AgdaDatatype{Term} \AgdaSymbol{\{}\AgdaBound{Γ} \AgdaInductiveConstructor{▻} \AgdaBound{T'}\AgdaSymbol{\}} \AgdaSymbol{(}\AgdaInductiveConstructor{W₁} \AgdaBound{B} \AgdaInductiveConstructor{‘’} \AgdaBound{a}\AgdaSymbol{)\}}\<%
\\
\>[4]\AgdaIndent{6}{}\<[6]%
\>[6]\AgdaSymbol{→} \AgdaDatatype{Term} \AgdaSymbol{\{}\AgdaBound{Γ} \AgdaInductiveConstructor{▻} \AgdaBound{T'}\AgdaSymbol{\}} \AgdaSymbol{(}\AgdaInductiveConstructor{W₂} \AgdaSymbol{(}\AgdaInductiveConstructor{W} \AgdaBound{T}\AgdaSymbol{)} \AgdaInductiveConstructor{‘’₁} \AgdaBound{a} \AgdaInductiveConstructor{‘’} \AgdaBound{b}\AgdaSymbol{)}\<%
\\
\>[4]\AgdaIndent{6}{}\<[6]%
\>[6]\AgdaSymbol{→} \AgdaDatatype{Term} \AgdaSymbol{\{}\AgdaBound{Γ} \AgdaInductiveConstructor{▻} \AgdaBound{T'}\AgdaSymbol{\}} \AgdaSymbol{(}\AgdaInductiveConstructor{W₁} \AgdaBound{T} \AgdaInductiveConstructor{‘’} \AgdaBound{a}\AgdaSymbol{)}\<%
\\
\>[0]\AgdaIndent{4}{}\<[4]%
\>[4]\AgdaInductiveConstructor{W₀₁₀} \AgdaSymbol{:} \AgdaSymbol{∀} \AgdaSymbol{\{}\AgdaBound{Γ} \AgdaBound{A} \AgdaBound{B} \AgdaBound{C} \AgdaBound{D}\AgdaSymbol{\}} \AgdaSymbol{→} \AgdaDatatype{Term} \AgdaSymbol{\{}\AgdaBound{Γ} \AgdaInductiveConstructor{▻} \AgdaBound{A} \AgdaInductiveConstructor{▻} \AgdaInductiveConstructor{W} \AgdaBound{B} \AgdaInductiveConstructor{▻} \AgdaInductiveConstructor{W₁} \AgdaBound{C}\AgdaSymbol{\}} \AgdaSymbol{(}\AgdaInductiveConstructor{W} \AgdaSymbol{(}\AgdaInductiveConstructor{W₁} \AgdaSymbol{(}\AgdaInductiveConstructor{W} \AgdaBound{D}\AgdaSymbol{)))} \AgdaSymbol{→} \AgdaDatatype{Term} \AgdaSymbol{\{}\AgdaBound{Γ} \AgdaInductiveConstructor{▻} \AgdaBound{A} \AgdaInductiveConstructor{▻} \AgdaInductiveConstructor{W} \AgdaBound{B} \AgdaInductiveConstructor{▻} \AgdaInductiveConstructor{W₁} \AgdaBound{C}\AgdaSymbol{\}} \AgdaSymbol{(}\AgdaInductiveConstructor{W} \AgdaSymbol{(}\AgdaInductiveConstructor{W} \AgdaSymbol{(}\AgdaInductiveConstructor{W} \AgdaBound{D}\AgdaSymbol{)))}\<%
\\
\>[0]\AgdaIndent{4}{}\<[4]%
\>[4]\AgdaInductiveConstructor{β-under-S} \AgdaSymbol{:} \AgdaSymbol{∀} \AgdaSymbol{\{}\AgdaBound{Γ} \AgdaBound{A} \AgdaBound{B} \AgdaBound{B'}\AgdaSymbol{\}} \AgdaSymbol{\{}\AgdaBound{g} \AgdaSymbol{:} \AgdaDatatype{Term} \AgdaSymbol{\{}\AgdaBound{Γ}\AgdaSymbol{\}} \AgdaSymbol{(}\AgdaBound{A} \AgdaInductiveConstructor{‘→’} \AgdaInductiveConstructor{W} \AgdaBound{B}\AgdaSymbol{)\}} \AgdaSymbol{\{}\AgdaBound{x} \AgdaSymbol{:} \AgdaDatatype{Term} \AgdaSymbol{\{}\AgdaBound{Γ}\AgdaSymbol{\}} \AgdaBound{A}\AgdaSymbol{\}}\<%
\\
\>[4]\AgdaIndent{6}{}\<[6]%
\>[6]\AgdaSymbol{→} \AgdaDatatype{Term} \AgdaSymbol{(}\AgdaBound{B'} \AgdaInductiveConstructor{‘’} \AgdaInductiveConstructor{SW} \AgdaSymbol{(}\AgdaInductiveConstructor{‘λ’} \AgdaSymbol{(}\AgdaInductiveConstructor{SW} \AgdaSymbol{(}\AgdaInductiveConstructor{‘λ’} \AgdaSymbol{(}\AgdaInductiveConstructor{W₁₀} \AgdaSymbol{(}\AgdaInductiveConstructor{SW₁V} \AgdaSymbol{(}\AgdaInductiveConstructor{W∀} \AgdaSymbol{(}\AgdaInductiveConstructor{w} \AgdaSymbol{(}\AgdaInductiveConstructor{W∀} \AgdaSymbol{(}\AgdaInductiveConstructor{w} \AgdaBound{g}\AgdaSymbol{)))} \AgdaInductiveConstructor{‘’ₐ} \AgdaInductiveConstructor{‘VAR₀’}\AgdaSymbol{)))} \AgdaInductiveConstructor{‘’ₐ} \AgdaInductiveConstructor{‘VAR₀’}\AgdaSymbol{))} \AgdaInductiveConstructor{‘’ₐ} \AgdaBound{x}\AgdaSymbol{))}\<%
\\
\>[4]\AgdaIndent{6}{}\<[6]%
\>[6]\AgdaSymbol{→} \AgdaDatatype{Term} \AgdaSymbol{(}\AgdaBound{B'} \AgdaInductiveConstructor{‘’} \AgdaInductiveConstructor{SW} \AgdaSymbol{(}\AgdaBound{g} \AgdaInductiveConstructor{‘’ₐ} \AgdaBound{x}\AgdaSymbol{))}\<%
\\
\>[0]\AgdaIndent{4}{}\<[4]%
\>[4]\AgdaInductiveConstructor{‘fst'’} \AgdaSymbol{:} \AgdaSymbol{∀} \AgdaSymbol{\{}\AgdaBound{Γ}\AgdaSymbol{\}} \AgdaSymbol{\{}\AgdaBound{T} \AgdaSymbol{:} \AgdaDatatype{Type} \AgdaBound{Γ}\AgdaSymbol{\}} \AgdaSymbol{\{}\AgdaBound{P} \AgdaSymbol{:} \AgdaDatatype{Type} \AgdaSymbol{(}\AgdaBound{Γ} \AgdaInductiveConstructor{▻} \AgdaBound{T}\AgdaSymbol{)\}} \AgdaSymbol{→} \AgdaDatatype{Term} \AgdaSymbol{(}\AgdaInductiveConstructor{‘Σ’} \AgdaBound{T} \AgdaBound{P} \AgdaInductiveConstructor{‘→’} \AgdaInductiveConstructor{W} \AgdaBound{T}\AgdaSymbol{)}\<%
\\
\>[0]\AgdaIndent{4}{}\<[4]%
\>[4]\AgdaInductiveConstructor{‘snd'’} \AgdaSymbol{:} \AgdaSymbol{∀} \AgdaSymbol{\{}\AgdaBound{Γ}\AgdaSymbol{\}} \AgdaSymbol{\{}\AgdaBound{T} \AgdaSymbol{:} \AgdaDatatype{Type} \AgdaBound{Γ}\AgdaSymbol{\}} \AgdaSymbol{\{}\AgdaBound{P} \AgdaSymbol{:} \AgdaDatatype{Type} \AgdaSymbol{(}\AgdaBound{Γ} \AgdaInductiveConstructor{▻} \AgdaBound{T}\AgdaSymbol{)\}} \AgdaSymbol{→} \AgdaDatatype{Term} \AgdaSymbol{\{}\AgdaBound{Γ} \AgdaInductiveConstructor{▻} \AgdaInductiveConstructor{‘Σ’} \AgdaBound{T} \AgdaBound{P}\AgdaSymbol{\}} \AgdaSymbol{(}\AgdaInductiveConstructor{W₁} \AgdaBound{P} \AgdaInductiveConstructor{‘’} \AgdaInductiveConstructor{SW} \AgdaSymbol{(}\AgdaInductiveConstructor{‘λ’} \AgdaSymbol{(}\AgdaInductiveConstructor{W₁₀} \AgdaSymbol{(}\AgdaInductiveConstructor{SW₁V} \AgdaSymbol{(}\AgdaInductiveConstructor{W∀} \AgdaSymbol{(}\AgdaInductiveConstructor{w} \AgdaSymbol{(}\AgdaInductiveConstructor{W∀} \AgdaSymbol{(}\AgdaInductiveConstructor{w} \AgdaInductiveConstructor{‘fst'’}\AgdaSymbol{)))} \AgdaInductiveConstructor{‘’ₐ} \AgdaInductiveConstructor{‘VAR₀’}\AgdaSymbol{)))} \AgdaInductiveConstructor{‘’ₐ} \AgdaInductiveConstructor{‘VAR₀’}\AgdaSymbol{))}\<%
\\
\>[0]\AgdaIndent{4}{}\<[4]%
\>[4]\AgdaInductiveConstructor{\_‘,’\_} \AgdaSymbol{:} \AgdaSymbol{∀} \AgdaSymbol{\{}\AgdaBound{Γ} \AgdaBound{T} \AgdaBound{P}\AgdaSymbol{\}} \AgdaSymbol{(}\AgdaBound{x} \AgdaSymbol{:} \AgdaDatatype{Term} \AgdaSymbol{\{}\AgdaBound{Γ}\AgdaSymbol{\}} \AgdaBound{T}\AgdaSymbol{)} \AgdaSymbol{(}\AgdaBound{p} \AgdaSymbol{:} \AgdaDatatype{Term} \AgdaSymbol{(}\AgdaBound{P} \AgdaInductiveConstructor{‘’} \AgdaBound{x}\AgdaSymbol{))} \AgdaSymbol{→} \AgdaDatatype{Term} \AgdaSymbol{(}\AgdaInductiveConstructor{‘Σ’} \AgdaBound{T} \AgdaBound{P}\AgdaSymbol{)}\<%
\\
\>[0]\AgdaIndent{4}{}\<[4]%
\>[4]\AgdaComment{\{- these are redundant, but useful for not having to normalize the subsequent ones -\}}\<%
\\
\>[0]\AgdaIndent{4}{}\<[4]%
\>[4]\AgdaInductiveConstructor{\_‘‘’’\_} \AgdaSymbol{:} \AgdaSymbol{∀} \AgdaSymbol{\{}\AgdaBound{Γ}\AgdaSymbol{\}} \AgdaSymbol{\{}\AgdaBound{A} \AgdaSymbol{:} \AgdaDatatype{Type} \AgdaBound{Γ}\AgdaSymbol{\}}\<%
\\
\>[4]\AgdaIndent{6}{}\<[6]%
\>[6]\AgdaSymbol{→} \AgdaDatatype{Term} \AgdaSymbol{\{}\AgdaInductiveConstructor{ε}\AgdaSymbol{\}} \AgdaSymbol{(}\AgdaInductiveConstructor{‘Type’} \AgdaInductiveConstructor{‘’} \AgdaInductiveConstructor{⌜} \AgdaBound{Γ} \AgdaInductiveConstructor{▻} \AgdaBound{A} \AgdaInductiveConstructor{⌝ᶜ}\AgdaSymbol{)}\<%
\\
\>[4]\AgdaIndent{6}{}\<[6]%
\>[6]\AgdaSymbol{→} \AgdaDatatype{Term} \AgdaSymbol{\{}\AgdaInductiveConstructor{ε}\AgdaSymbol{\}} \AgdaSymbol{(}\AgdaInductiveConstructor{‘Term’} \AgdaInductiveConstructor{‘’₁} \AgdaInductiveConstructor{⌜} \AgdaBound{Γ} \AgdaInductiveConstructor{⌝ᶜ} \AgdaInductiveConstructor{‘’} \AgdaInductiveConstructor{⌜} \AgdaBound{A} \AgdaInductiveConstructor{⌝ᵀ}\AgdaSymbol{)}\<%
\\
\>[4]\AgdaIndent{6}{}\<[6]%
\>[6]\AgdaSymbol{→} \AgdaDatatype{Term} \AgdaSymbol{\{}\AgdaInductiveConstructor{ε}\AgdaSymbol{\}} \AgdaSymbol{(}\AgdaInductiveConstructor{‘Type’} \AgdaInductiveConstructor{‘’} \AgdaInductiveConstructor{⌜} \AgdaBound{Γ} \AgdaInductiveConstructor{⌝ᶜ}\AgdaSymbol{)}\<%
\\
\>[0]\AgdaIndent{4}{}\<[4]%
\>[4]\AgdaInductiveConstructor{\_w‘‘’’\_} \AgdaSymbol{:} \AgdaSymbol{∀} \AgdaSymbol{\{}\AgdaBound{X} \AgdaBound{Γ}\AgdaSymbol{\}} \AgdaSymbol{\{}\AgdaBound{A} \AgdaSymbol{:} \AgdaDatatype{Type} \AgdaBound{Γ}\AgdaSymbol{\}}\<%
\\
\>[4]\AgdaIndent{6}{}\<[6]%
\>[6]\AgdaSymbol{→} \AgdaDatatype{Term} \AgdaSymbol{\{}\AgdaInductiveConstructor{ε} \AgdaInductiveConstructor{▻} \AgdaBound{X}\AgdaSymbol{\}} \AgdaSymbol{(}\AgdaInductiveConstructor{W} \AgdaSymbol{(}\AgdaInductiveConstructor{‘Type’} \AgdaInductiveConstructor{‘’} \AgdaInductiveConstructor{⌜} \AgdaBound{Γ} \AgdaInductiveConstructor{▻} \AgdaBound{A} \AgdaInductiveConstructor{⌝ᶜ}\AgdaSymbol{))}\<%
\\
\>[4]\AgdaIndent{6}{}\<[6]%
\>[6]\AgdaSymbol{→} \AgdaDatatype{Term} \AgdaSymbol{\{}\AgdaInductiveConstructor{ε} \AgdaInductiveConstructor{▻} \AgdaBound{X}\AgdaSymbol{\}} \AgdaSymbol{(}\AgdaInductiveConstructor{W} \AgdaSymbol{(}\AgdaInductiveConstructor{‘Term’} \AgdaInductiveConstructor{‘’₁} \AgdaInductiveConstructor{⌜} \AgdaBound{Γ} \AgdaInductiveConstructor{⌝ᶜ} \AgdaInductiveConstructor{‘’} \AgdaInductiveConstructor{⌜} \AgdaBound{A} \AgdaInductiveConstructor{⌝ᵀ}\AgdaSymbol{))}\<%
\\
\>[4]\AgdaIndent{6}{}\<[6]%
\>[6]\AgdaSymbol{→} \AgdaDatatype{Term} \AgdaSymbol{\{}\AgdaInductiveConstructor{ε} \AgdaInductiveConstructor{▻} \AgdaBound{X}\AgdaSymbol{\}} \AgdaSymbol{(}\AgdaInductiveConstructor{W} \AgdaSymbol{(}\AgdaInductiveConstructor{‘Type’} \AgdaInductiveConstructor{‘’} \AgdaInductiveConstructor{⌜} \AgdaBound{Γ} \AgdaInductiveConstructor{⌝ᶜ}\AgdaSymbol{))}\<%
\\
\>[0]\AgdaIndent{4}{}\<[4]%
\>[4]\AgdaInductiveConstructor{\_‘‘→'’’\_} \AgdaSymbol{:} \AgdaSymbol{∀} \AgdaSymbol{\{}\AgdaBound{Γ}\AgdaSymbol{\}}\<%
\\
\>[4]\AgdaIndent{6}{}\<[6]%
\>[6]\AgdaSymbol{→} \AgdaDatatype{Term} \AgdaSymbol{\{}\AgdaInductiveConstructor{ε}\AgdaSymbol{\}} \AgdaSymbol{(}\AgdaInductiveConstructor{‘Type’} \AgdaInductiveConstructor{‘’} \AgdaBound{Γ}\AgdaSymbol{)}\<%
\\
\>[4]\AgdaIndent{6}{}\<[6]%
\>[6]\AgdaSymbol{→} \AgdaDatatype{Term} \AgdaSymbol{\{}\AgdaInductiveConstructor{ε}\AgdaSymbol{\}} \AgdaSymbol{(}\AgdaInductiveConstructor{‘Type’} \AgdaInductiveConstructor{‘’} \AgdaBound{Γ}\AgdaSymbol{)}\<%
\\
\>[4]\AgdaIndent{6}{}\<[6]%
\>[6]\AgdaSymbol{→} \AgdaDatatype{Term} \AgdaSymbol{\{}\AgdaInductiveConstructor{ε}\AgdaSymbol{\}} \AgdaSymbol{(}\AgdaInductiveConstructor{‘Type’} \AgdaInductiveConstructor{‘’} \AgdaBound{Γ}\AgdaSymbol{)}\<%
\\
\>[0]\AgdaIndent{4}{}\<[4]%
\>[4]\AgdaInductiveConstructor{\_w‘‘→'’’\_} \AgdaSymbol{:} \AgdaSymbol{∀} \AgdaSymbol{\{}\AgdaBound{X} \AgdaBound{Γ}\AgdaSymbol{\}}\<%
\\
\>[4]\AgdaIndent{6}{}\<[6]%
\>[6]\AgdaSymbol{→} \AgdaDatatype{Term} \AgdaSymbol{\{}\AgdaInductiveConstructor{ε} \AgdaInductiveConstructor{▻} \AgdaBound{X}\AgdaSymbol{\}} \AgdaSymbol{(}\AgdaInductiveConstructor{W} \AgdaSymbol{(}\AgdaInductiveConstructor{‘Type’} \AgdaInductiveConstructor{‘’} \AgdaBound{Γ}\AgdaSymbol{))}\<%
\\
\>[4]\AgdaIndent{6}{}\<[6]%
\>[6]\AgdaSymbol{→} \AgdaDatatype{Term} \AgdaSymbol{\{}\AgdaInductiveConstructor{ε} \AgdaInductiveConstructor{▻} \AgdaBound{X}\AgdaSymbol{\}} \AgdaSymbol{(}\AgdaInductiveConstructor{W} \AgdaSymbol{(}\AgdaInductiveConstructor{‘Type’} \AgdaInductiveConstructor{‘’} \AgdaBound{Γ}\AgdaSymbol{))}\<%
\\
\>[4]\AgdaIndent{6}{}\<[6]%
\>[6]\AgdaSymbol{→} \AgdaDatatype{Term} \AgdaSymbol{\{}\AgdaInductiveConstructor{ε} \AgdaInductiveConstructor{▻} \AgdaBound{X}\AgdaSymbol{\}} \AgdaSymbol{(}\AgdaInductiveConstructor{W} \AgdaSymbol{(}\AgdaInductiveConstructor{‘Type’} \AgdaInductiveConstructor{‘’} \AgdaBound{Γ}\AgdaSymbol{))}\<%
\\
\>[0]\AgdaIndent{4}{}\<[4]%
\>[4]\AgdaInductiveConstructor{w→} \AgdaSymbol{:} \AgdaSymbol{∀} \AgdaSymbol{\{}\AgdaBound{Γ} \AgdaBound{A} \AgdaBound{B} \AgdaBound{C}\AgdaSymbol{\}} \AgdaSymbol{→} \AgdaDatatype{Term} \AgdaSymbol{(}\AgdaBound{A} \AgdaInductiveConstructor{‘→’} \AgdaInductiveConstructor{W} \AgdaBound{B}\AgdaSymbol{)} \AgdaSymbol{→} \AgdaDatatype{Term} \AgdaSymbol{\{}\AgdaBound{Γ} \AgdaInductiveConstructor{▻} \AgdaBound{C}\AgdaSymbol{\}} \AgdaSymbol{(}\AgdaInductiveConstructor{W} \AgdaBound{A} \AgdaInductiveConstructor{‘→’} \AgdaInductiveConstructor{W} \AgdaSymbol{(}\AgdaInductiveConstructor{W} \AgdaBound{B}\AgdaSymbol{))}\<%
\\
\>[0]\AgdaIndent{4}{}\<[4]%
\>[4]\AgdaComment{\{- things that were postulates, but are no longer -\}}\<%
\\
\>[0]\AgdaIndent{4}{}\<[4]%
\>[4]\AgdaInductiveConstructor{‘‘→'’’→w‘‘→'’’} \AgdaSymbol{:} \AgdaSymbol{∀} \AgdaSymbol{\{}\AgdaBound{T'}\AgdaSymbol{\}}\<%
\\
\>[4]\AgdaIndent{9}{}\<[9]%
\>[9]\AgdaSymbol{\{}\AgdaBound{b} \AgdaSymbol{:} \AgdaDatatype{Term} \AgdaSymbol{\{}\AgdaInductiveConstructor{ε}\AgdaSymbol{\}} \AgdaSymbol{(}\AgdaInductiveConstructor{‘Type’} \AgdaInductiveConstructor{‘’} \AgdaInductiveConstructor{⌜} \AgdaInductiveConstructor{ε} \AgdaInductiveConstructor{⌝ᶜ}\AgdaSymbol{)\}}\<%
\\
\>[4]\AgdaIndent{9}{}\<[9]%
\>[9]\AgdaSymbol{\{}\AgdaBound{c} \AgdaSymbol{:} \AgdaDatatype{Term} \AgdaSymbol{\{}\AgdaInductiveConstructor{ε} \AgdaInductiveConstructor{▻} \AgdaBound{T'}\AgdaSymbol{\}} \AgdaSymbol{(}\AgdaInductiveConstructor{W} \AgdaSymbol{(}\AgdaInductiveConstructor{‘Type’} \AgdaInductiveConstructor{‘’} \AgdaInductiveConstructor{⌜} \AgdaInductiveConstructor{ε} \AgdaInductiveConstructor{⌝ᶜ}\AgdaSymbol{))\}}\<%
\\
\>[4]\AgdaIndent{9}{}\<[9]%
\>[9]\AgdaSymbol{\{}\AgdaBound{e} \AgdaSymbol{:} \AgdaDatatype{Term} \AgdaSymbol{\{}\AgdaInductiveConstructor{ε}\AgdaSymbol{\}} \AgdaBound{T'}\AgdaSymbol{\}}\<%
\\
\>[4]\AgdaIndent{9}{}\<[9]%
\>[9]\AgdaSymbol{→} \AgdaDatatype{Term} \AgdaSymbol{\{}\AgdaInductiveConstructor{ε}\AgdaSymbol{\}} \AgdaSymbol{(}\AgdaInductiveConstructor{‘Term’} \AgdaInductiveConstructor{‘’₁} \AgdaInductiveConstructor{⌜} \AgdaInductiveConstructor{ε} \AgdaInductiveConstructor{⌝ᶜ} \AgdaInductiveConstructor{‘’} \AgdaSymbol{(}\AgdaInductiveConstructor{SW} \AgdaSymbol{(}\AgdaInductiveConstructor{‘λ’} \AgdaSymbol{(}\AgdaBound{c} \AgdaInductiveConstructor{w‘‘→'’’} \AgdaInductiveConstructor{w} \AgdaBound{b}\AgdaSymbol{)} \AgdaInductiveConstructor{‘’ₐ} \AgdaBound{e}\AgdaSymbol{))}\<%
\\
\>[9]\AgdaIndent{20}{}\<[20]%
\>[20]\AgdaInductiveConstructor{‘→’} \AgdaInductiveConstructor{W} \AgdaSymbol{(}\AgdaInductiveConstructor{‘Term’} \AgdaInductiveConstructor{‘’₁} \AgdaInductiveConstructor{⌜} \AgdaInductiveConstructor{ε} \AgdaInductiveConstructor{⌝ᶜ} \AgdaInductiveConstructor{‘’} \AgdaSymbol{(}\AgdaInductiveConstructor{SW} \AgdaSymbol{(}\AgdaInductiveConstructor{‘λ’} \AgdaBound{c} \AgdaInductiveConstructor{‘’ₐ} \AgdaBound{e}\AgdaSymbol{)} \AgdaInductiveConstructor{‘‘→'’’} \AgdaBound{b}\AgdaSymbol{)))}\<%
\\
\>[0]\AgdaIndent{4}{}\<[4]%
\>[4]\AgdaInductiveConstructor{w‘‘→'’’→‘‘→'’’} \AgdaSymbol{:} \AgdaSymbol{∀} \AgdaSymbol{\{}\AgdaBound{T'}\AgdaSymbol{\}}\<%
\\
\>[4]\AgdaIndent{9}{}\<[9]%
\>[9]\AgdaSymbol{\{}\AgdaBound{b} \AgdaSymbol{:} \AgdaDatatype{Term} \AgdaSymbol{\{}\AgdaInductiveConstructor{ε}\AgdaSymbol{\}} \AgdaSymbol{(}\AgdaInductiveConstructor{‘Type’} \AgdaInductiveConstructor{‘’} \AgdaInductiveConstructor{⌜} \AgdaInductiveConstructor{ε} \AgdaInductiveConstructor{⌝ᶜ}\AgdaSymbol{)\}}\<%
\\
\>[4]\AgdaIndent{9}{}\<[9]%
\>[9]\AgdaSymbol{\{}\AgdaBound{c} \AgdaSymbol{:} \AgdaDatatype{Term} \AgdaSymbol{\{}\AgdaInductiveConstructor{ε} \AgdaInductiveConstructor{▻} \AgdaBound{T'}\AgdaSymbol{\}} \AgdaSymbol{(}\AgdaInductiveConstructor{W} \AgdaSymbol{(}\AgdaInductiveConstructor{‘Type’} \AgdaInductiveConstructor{‘’} \AgdaInductiveConstructor{⌜} \AgdaInductiveConstructor{ε} \AgdaInductiveConstructor{⌝ᶜ}\AgdaSymbol{))\}}\<%
\\
\>[4]\AgdaIndent{9}{}\<[9]%
\>[9]\AgdaSymbol{\{}\AgdaBound{e} \AgdaSymbol{:} \AgdaDatatype{Term} \AgdaSymbol{\{}\AgdaInductiveConstructor{ε}\AgdaSymbol{\}} \AgdaBound{T'}\AgdaSymbol{\}}\<%
\\
\>[4]\AgdaIndent{9}{}\<[9]%
\>[9]\AgdaSymbol{→} \AgdaDatatype{Term} \AgdaSymbol{\{}\AgdaInductiveConstructor{ε}\AgdaSymbol{\}} \AgdaSymbol{(}\AgdaInductiveConstructor{‘Term’} \AgdaInductiveConstructor{‘’₁} \AgdaInductiveConstructor{⌜} \AgdaInductiveConstructor{ε} \AgdaInductiveConstructor{⌝ᶜ} \AgdaInductiveConstructor{‘’} \AgdaSymbol{(}\AgdaInductiveConstructor{SW} \AgdaSymbol{(}\AgdaInductiveConstructor{‘λ’} \AgdaBound{c} \AgdaInductiveConstructor{‘’ₐ} \AgdaBound{e}\AgdaSymbol{)} \AgdaInductiveConstructor{‘‘→'’’} \AgdaBound{b}\AgdaSymbol{)}\<%
\\
\>[9]\AgdaIndent{20}{}\<[20]%
\>[20]\AgdaInductiveConstructor{‘→’} \AgdaInductiveConstructor{W} \AgdaSymbol{(}\AgdaInductiveConstructor{‘Term’} \AgdaInductiveConstructor{‘’₁} \AgdaInductiveConstructor{⌜} \AgdaInductiveConstructor{ε} \AgdaInductiveConstructor{⌝ᶜ} \AgdaInductiveConstructor{‘’} \AgdaSymbol{(}\AgdaInductiveConstructor{SW} \AgdaSymbol{(}\AgdaInductiveConstructor{‘λ’} \AgdaSymbol{(}\AgdaBound{c} \AgdaInductiveConstructor{w‘‘→'’’} \AgdaInductiveConstructor{w} \AgdaBound{b}\AgdaSymbol{)} \AgdaInductiveConstructor{‘’ₐ} \AgdaBound{e}\AgdaSymbol{))))}\<%
\\
\>[0]\AgdaIndent{4}{}\<[4]%
\>[4]\AgdaInductiveConstructor{‘tApp-nd’} \AgdaSymbol{:} \AgdaSymbol{∀} \AgdaSymbol{\{}\AgdaBound{Γ}\AgdaSymbol{\}} \AgdaSymbol{\{}\AgdaBound{A} \AgdaSymbol{:} \AgdaDatatype{Term} \AgdaSymbol{\{}\AgdaInductiveConstructor{ε}\AgdaSymbol{\}} \AgdaSymbol{(}\AgdaInductiveConstructor{‘Type’} \AgdaInductiveConstructor{‘’} \AgdaBound{Γ}\AgdaSymbol{)\}} \AgdaSymbol{\{}\AgdaBound{B} \AgdaSymbol{:} \AgdaDatatype{Term} \AgdaSymbol{\{}\AgdaInductiveConstructor{ε}\AgdaSymbol{\}} \AgdaSymbol{(}\AgdaInductiveConstructor{‘Type’} \AgdaInductiveConstructor{‘’} \AgdaBound{Γ}\AgdaSymbol{)\}} \AgdaSymbol{→}\<%
\\
\>[4]\AgdaIndent{6}{}\<[6]%
\>[6]\AgdaDatatype{Term} \AgdaSymbol{\{}\AgdaInductiveConstructor{ε}\AgdaSymbol{\}} \AgdaSymbol{(}\AgdaInductiveConstructor{‘Term’} \AgdaInductiveConstructor{‘’₁} \AgdaBound{Γ} \AgdaInductiveConstructor{‘’} \AgdaSymbol{(}\AgdaBound{A} \AgdaInductiveConstructor{‘‘→'’’} \AgdaBound{B}\AgdaSymbol{)}\<%
\\
\>[6]\AgdaIndent{11}{}\<[11]%
\>[11]\AgdaInductiveConstructor{‘→’} \AgdaInductiveConstructor{W} \AgdaSymbol{(}\AgdaInductiveConstructor{‘Term’} \AgdaInductiveConstructor{‘’₁} \AgdaBound{Γ} \AgdaInductiveConstructor{‘’} \AgdaBound{A}\<%
\\
\>[6]\AgdaIndent{11}{}\<[11]%
\>[11]\AgdaInductiveConstructor{‘→’} \AgdaInductiveConstructor{W} \AgdaSymbol{(}\AgdaInductiveConstructor{‘Term’} \AgdaInductiveConstructor{‘’₁} \AgdaBound{Γ} \AgdaInductiveConstructor{‘’} \AgdaBound{B}\AgdaSymbol{)))}\<%
\\
\>[0]\AgdaIndent{4}{}\<[4]%
\>[4]\AgdaInductiveConstructor{⌜←'⌝} \AgdaSymbol{:} \AgdaSymbol{∀} \AgdaSymbol{\{}\AgdaBound{H} \AgdaBound{X}\AgdaSymbol{\}} \AgdaSymbol{→}\<%
\\
\>[4]\AgdaIndent{6}{}\<[6]%
\>[6]\AgdaDatatype{Term} \AgdaSymbol{\{}\AgdaInductiveConstructor{ε}\AgdaSymbol{\}} \AgdaSymbol{(}\AgdaInductiveConstructor{‘Term’} \AgdaInductiveConstructor{‘’₁} \AgdaInductiveConstructor{⌜} \AgdaInductiveConstructor{ε} \AgdaInductiveConstructor{⌝ᶜ} \AgdaInductiveConstructor{‘’} \AgdaSymbol{(}\AgdaInductiveConstructor{⌜} \AgdaBound{H} \AgdaInductiveConstructor{⌝ᵀ} \AgdaInductiveConstructor{‘‘→'’’} \AgdaInductiveConstructor{⌜} \AgdaBound{X} \AgdaInductiveConstructor{⌝ᵀ}\AgdaSymbol{)}\<%
\\
\>[6]\AgdaIndent{11}{}\<[11]%
\>[11]\AgdaInductiveConstructor{‘→’} \AgdaInductiveConstructor{W} \AgdaSymbol{(}\AgdaInductiveConstructor{‘Term’} \AgdaInductiveConstructor{‘’₁} \AgdaInductiveConstructor{⌜} \AgdaInductiveConstructor{ε} \AgdaInductiveConstructor{⌝ᶜ} \AgdaInductiveConstructor{‘’} \AgdaInductiveConstructor{⌜} \AgdaBound{H} \AgdaInductiveConstructor{‘→’} \AgdaInductiveConstructor{W} \AgdaBound{X} \AgdaInductiveConstructor{⌝ᵀ}\AgdaSymbol{))}\<%
\\
\>[0]\AgdaIndent{4}{}\<[4]%
\>[4]\AgdaInductiveConstructor{⌜→'⌝} \AgdaSymbol{:} \AgdaSymbol{∀} \AgdaSymbol{\{}\AgdaBound{H} \AgdaBound{X}\AgdaSymbol{\}} \AgdaSymbol{→}\<%
\\
\>[4]\AgdaIndent{6}{}\<[6]%
\>[6]\AgdaDatatype{Term} \AgdaSymbol{\{}\AgdaInductiveConstructor{ε}\AgdaSymbol{\}} \AgdaSymbol{(}\AgdaInductiveConstructor{‘Term’} \AgdaInductiveConstructor{‘’₁} \AgdaInductiveConstructor{⌜} \AgdaInductiveConstructor{ε} \AgdaInductiveConstructor{⌝ᶜ} \AgdaInductiveConstructor{‘’} \AgdaInductiveConstructor{⌜} \AgdaBound{H} \AgdaInductiveConstructor{‘→’} \AgdaInductiveConstructor{W} \AgdaBound{X} \AgdaInductiveConstructor{⌝ᵀ}\<%
\\
\>[6]\AgdaIndent{11}{}\<[11]%
\>[11]\AgdaInductiveConstructor{‘→’} \AgdaInductiveConstructor{W} \AgdaSymbol{(}\AgdaInductiveConstructor{‘Term’} \AgdaInductiveConstructor{‘’₁} \AgdaInductiveConstructor{⌜} \AgdaInductiveConstructor{ε} \AgdaInductiveConstructor{⌝ᶜ} \AgdaInductiveConstructor{‘’} \AgdaSymbol{(}\AgdaInductiveConstructor{⌜} \AgdaBound{H} \AgdaInductiveConstructor{⌝ᵀ} \AgdaInductiveConstructor{‘‘→'’’} \AgdaInductiveConstructor{⌜} \AgdaBound{X} \AgdaInductiveConstructor{⌝ᵀ}\AgdaSymbol{)))}\<%
\\
\>[0]\AgdaIndent{4}{}\<[4]%
\>[4]\AgdaInductiveConstructor{‘‘∘-nd’’} \AgdaSymbol{:} \AgdaSymbol{∀} \AgdaSymbol{\{}\AgdaBound{A} \AgdaBound{B} \AgdaBound{C}\AgdaSymbol{\}} \AgdaSymbol{→}\<%
\\
\>[4]\AgdaIndent{6}{}\<[6]%
\>[6]\AgdaDatatype{Term} \AgdaSymbol{\{}\AgdaInductiveConstructor{ε}\AgdaSymbol{\}} \AgdaSymbol{(}\AgdaInductiveConstructor{‘Term’} \AgdaInductiveConstructor{‘’₁} \AgdaInductiveConstructor{⌜} \AgdaInductiveConstructor{ε} \AgdaInductiveConstructor{⌝ᶜ} \AgdaInductiveConstructor{‘’} \AgdaSymbol{(}\AgdaBound{A} \AgdaInductiveConstructor{‘‘→'’’} \AgdaBound{C}\AgdaSymbol{)}\<%
\\
\>[6]\AgdaIndent{11}{}\<[11]%
\>[11]\AgdaInductiveConstructor{‘→’} \AgdaInductiveConstructor{W} \AgdaSymbol{(}\AgdaInductiveConstructor{‘Term’} \AgdaInductiveConstructor{‘’₁} \AgdaInductiveConstructor{⌜} \AgdaInductiveConstructor{ε} \AgdaInductiveConstructor{⌝ᶜ} \AgdaInductiveConstructor{‘’} \AgdaSymbol{(}\AgdaBound{C} \AgdaInductiveConstructor{‘‘→'’’} \AgdaBound{B}\AgdaSymbol{)}\<%
\\
\>[6]\AgdaIndent{11}{}\<[11]%
\>[11]\AgdaInductiveConstructor{‘→’} \AgdaInductiveConstructor{W} \AgdaSymbol{(}\AgdaInductiveConstructor{‘Term’} \AgdaInductiveConstructor{‘’₁} \AgdaInductiveConstructor{⌜} \AgdaInductiveConstructor{ε} \AgdaInductiveConstructor{⌝ᶜ} \AgdaInductiveConstructor{‘’} \AgdaSymbol{(}\AgdaBound{A} \AgdaInductiveConstructor{‘‘→'’’} \AgdaBound{B}\AgdaSymbol{))))}\<%
\\
\>[0]\AgdaIndent{4}{}\<[4]%
\>[4]\AgdaInductiveConstructor{⌜‘’⌝} \AgdaSymbol{:} \AgdaSymbol{∀} \AgdaSymbol{\{}\AgdaBound{B} \AgdaBound{A}\AgdaSymbol{\}} \AgdaSymbol{\{}\AgdaBound{b} \AgdaSymbol{:} \AgdaDatatype{Term} \AgdaSymbol{\{}\AgdaInductiveConstructor{ε}\AgdaSymbol{\}} \AgdaBound{B}\AgdaSymbol{\}} \AgdaSymbol{→}\<%
\\
\>[4]\AgdaIndent{8}{}\<[8]%
\>[8]\AgdaDatatype{Term} \AgdaSymbol{\{}\AgdaInductiveConstructor{ε}\AgdaSymbol{\}} \AgdaSymbol{(}\AgdaInductiveConstructor{‘Term’} \AgdaInductiveConstructor{‘’₁} \AgdaInductiveConstructor{⌜} \AgdaInductiveConstructor{ε} \AgdaInductiveConstructor{⌝ᶜ} \AgdaInductiveConstructor{‘’}\<%
\\
\>[8]\AgdaIndent{12}{}\<[12]%
\>[12]\AgdaSymbol{(}\AgdaInductiveConstructor{⌜} \AgdaBound{A} \AgdaInductiveConstructor{‘’} \AgdaBound{b} \AgdaInductiveConstructor{⌝ᵀ} \AgdaInductiveConstructor{‘‘→'’’} \AgdaInductiveConstructor{⌜} \AgdaBound{A} \AgdaInductiveConstructor{⌝ᵀ} \AgdaInductiveConstructor{‘‘’’} \AgdaInductiveConstructor{⌜} \AgdaBound{b} \AgdaInductiveConstructor{⌝ᵗ}\AgdaSymbol{))}\<%
\\
\>[0]\AgdaIndent{4}{}\<[4]%
\>[4]\AgdaInductiveConstructor{⌜‘’⌝'} \AgdaSymbol{:} \AgdaSymbol{∀} \AgdaSymbol{\{}\AgdaBound{B} \AgdaBound{A}\AgdaSymbol{\}} \AgdaSymbol{\{}\AgdaBound{b} \AgdaSymbol{:} \AgdaDatatype{Term} \AgdaSymbol{\{}\AgdaInductiveConstructor{ε}\AgdaSymbol{\}} \AgdaBound{B}\AgdaSymbol{\}} \AgdaSymbol{→}\<%
\\
\>[4]\AgdaIndent{8}{}\<[8]%
\>[8]\AgdaDatatype{Term} \AgdaSymbol{\{}\AgdaInductiveConstructor{ε}\AgdaSymbol{\}} \AgdaSymbol{(}\AgdaInductiveConstructor{‘Term’} \AgdaInductiveConstructor{‘’₁} \AgdaInductiveConstructor{⌜} \AgdaInductiveConstructor{ε} \AgdaInductiveConstructor{⌝ᶜ} \AgdaInductiveConstructor{‘’}\<%
\\
\>[8]\AgdaIndent{12}{}\<[12]%
\>[12]\AgdaSymbol{(}\AgdaInductiveConstructor{⌜} \AgdaBound{A} \AgdaInductiveConstructor{⌝ᵀ} \AgdaInductiveConstructor{‘‘’’} \AgdaInductiveConstructor{⌜} \AgdaBound{b} \AgdaInductiveConstructor{⌝ᵗ} \AgdaInductiveConstructor{‘‘→'’’} \AgdaInductiveConstructor{⌜} \AgdaBound{A} \AgdaInductiveConstructor{‘’} \AgdaBound{b} \AgdaInductiveConstructor{⌝ᵀ}\AgdaSymbol{))}\<%
\\
\>[0]\AgdaIndent{4}{}\<[4]%
\>[4]\AgdaInductiveConstructor{‘cast-refl’} \AgdaSymbol{:} \AgdaSymbol{∀} \AgdaSymbol{\{}\AgdaBound{T} \AgdaSymbol{:} \AgdaDatatype{Type} \AgdaSymbol{(}\AgdaInductiveConstructor{ε} \AgdaInductiveConstructor{▻} \AgdaInductiveConstructor{‘Σ’} \AgdaInductiveConstructor{‘Context’} \AgdaInductiveConstructor{‘Type’}\AgdaSymbol{)\}} \AgdaSymbol{→}\<%
\\
\>[4]\AgdaIndent{8}{}\<[8]%
\>[8]\AgdaDatatype{Term} \AgdaSymbol{\{}\AgdaInductiveConstructor{ε}\AgdaSymbol{\}} \AgdaSymbol{(}\AgdaInductiveConstructor{‘Term’} \AgdaInductiveConstructor{‘’₁} \AgdaInductiveConstructor{⌜} \AgdaInductiveConstructor{ε} \AgdaInductiveConstructor{⌝ᶜ} \AgdaInductiveConstructor{‘’}\<%
\\
\>[8]\AgdaIndent{12}{}\<[12]%
\>[12]\AgdaSymbol{((}\AgdaInductiveConstructor{⌜} \AgdaBound{T} \AgdaInductiveConstructor{‘’} \AgdaInductiveConstructor{\_‘,’\_} \AgdaInductiveConstructor{⌜} \AgdaInductiveConstructor{ε} \AgdaInductiveConstructor{▻} \AgdaInductiveConstructor{‘Σ’} \AgdaInductiveConstructor{‘Context’} \AgdaInductiveConstructor{‘Type’} \AgdaInductiveConstructor{⌝ᶜ} \AgdaInductiveConstructor{⌜} \AgdaBound{T} \AgdaInductiveConstructor{⌝ᵀ} \AgdaInductiveConstructor{⌝ᵀ}\AgdaSymbol{)}\<%
\\
\>[12]\AgdaIndent{15}{}\<[15]%
\>[15]\AgdaInductiveConstructor{‘‘→'’’}\<%
\\
\>[12]\AgdaIndent{15}{}\<[15]%
\>[15]\AgdaSymbol{(}\AgdaInductiveConstructor{SW} \AgdaSymbol{(}\AgdaInductiveConstructor{‘cast’} \AgdaInductiveConstructor{‘’ₐ} \AgdaInductiveConstructor{\_‘,’\_} \AgdaInductiveConstructor{⌜} \AgdaInductiveConstructor{ε} \AgdaInductiveConstructor{▻} \AgdaInductiveConstructor{‘Σ’} \AgdaInductiveConstructor{‘Context’} \AgdaInductiveConstructor{‘Type’} \AgdaInductiveConstructor{⌝ᶜ} \AgdaInductiveConstructor{⌜} \AgdaBound{T} \AgdaInductiveConstructor{⌝ᵀ}\AgdaSymbol{)}\<%
\\
\>[15]\AgdaIndent{17}{}\<[17]%
\>[17]\AgdaInductiveConstructor{‘‘’’} \AgdaInductiveConstructor{SW} \AgdaSymbol{(}\AgdaInductiveConstructor{‘quote-Σ’} \AgdaInductiveConstructor{‘’ₐ} \AgdaInductiveConstructor{\_‘,’\_} \AgdaInductiveConstructor{⌜} \AgdaInductiveConstructor{ε} \AgdaInductiveConstructor{▻} \AgdaInductiveConstructor{‘Σ’} \AgdaInductiveConstructor{‘Context’} \AgdaInductiveConstructor{‘Type’} \AgdaInductiveConstructor{⌝ᶜ} \AgdaInductiveConstructor{⌜} \AgdaBound{T} \AgdaInductiveConstructor{⌝ᵀ}\AgdaSymbol{))))}\<%
\\
\>[0]\AgdaIndent{4}{}\<[4]%
\>[4]\AgdaInductiveConstructor{‘cast-refl'’} \AgdaSymbol{:} \AgdaSymbol{∀} \AgdaSymbol{\{}\AgdaBound{T} \AgdaSymbol{:} \AgdaDatatype{Type} \AgdaSymbol{(}\AgdaInductiveConstructor{ε} \AgdaInductiveConstructor{▻} \AgdaInductiveConstructor{‘Σ’} \AgdaInductiveConstructor{‘Context’} \AgdaInductiveConstructor{‘Type’}\AgdaSymbol{)\}} \AgdaSymbol{→}\<%
\\
\>[4]\AgdaIndent{8}{}\<[8]%
\>[8]\AgdaDatatype{Term} \AgdaSymbol{\{}\AgdaInductiveConstructor{ε}\AgdaSymbol{\}} \AgdaSymbol{(}\AgdaInductiveConstructor{‘Term’} \AgdaInductiveConstructor{‘’₁} \AgdaInductiveConstructor{⌜} \AgdaInductiveConstructor{ε} \AgdaInductiveConstructor{⌝ᶜ} \AgdaInductiveConstructor{‘’}\<%
\\
\>[8]\AgdaIndent{12}{}\<[12]%
\>[12]\AgdaSymbol{((}\AgdaInductiveConstructor{SW} \AgdaSymbol{(}\AgdaInductiveConstructor{‘cast’} \AgdaInductiveConstructor{‘’ₐ} \AgdaInductiveConstructor{\_‘,’\_} \AgdaInductiveConstructor{⌜} \AgdaInductiveConstructor{ε} \AgdaInductiveConstructor{▻} \AgdaInductiveConstructor{‘Σ’} \AgdaInductiveConstructor{‘Context’} \AgdaInductiveConstructor{‘Type’} \AgdaInductiveConstructor{⌝ᶜ} \AgdaInductiveConstructor{⌜} \AgdaBound{T} \AgdaInductiveConstructor{⌝ᵀ}\AgdaSymbol{)}\<%
\\
\>[12]\AgdaIndent{17}{}\<[17]%
\>[17]\AgdaInductiveConstructor{‘‘’’} \AgdaInductiveConstructor{SW} \AgdaSymbol{(}\AgdaInductiveConstructor{‘quote-Σ’} \AgdaInductiveConstructor{‘’ₐ} \AgdaInductiveConstructor{\_‘,’\_} \AgdaInductiveConstructor{⌜} \AgdaInductiveConstructor{ε} \AgdaInductiveConstructor{▻} \AgdaInductiveConstructor{‘Σ’} \AgdaInductiveConstructor{‘Context’} \AgdaInductiveConstructor{‘Type’} \AgdaInductiveConstructor{⌝ᶜ} \AgdaInductiveConstructor{⌜} \AgdaBound{T} \AgdaInductiveConstructor{⌝ᵀ}\AgdaSymbol{))}\<%
\\
\>[0]\AgdaIndent{15}{}\<[15]%
\>[15]\AgdaInductiveConstructor{‘‘→'’’}\<%
\\
\>[0]\AgdaIndent{15}{}\<[15]%
\>[15]\AgdaSymbol{(}\AgdaInductiveConstructor{⌜} \AgdaBound{T} \AgdaInductiveConstructor{‘’} \AgdaInductiveConstructor{\_‘,’\_} \AgdaInductiveConstructor{⌜} \AgdaInductiveConstructor{ε} \AgdaInductiveConstructor{▻} \AgdaInductiveConstructor{‘Σ’} \AgdaInductiveConstructor{‘Context’} \AgdaInductiveConstructor{‘Type’} \AgdaInductiveConstructor{⌝ᶜ} \AgdaInductiveConstructor{⌜} \AgdaBound{T} \AgdaInductiveConstructor{⌝ᵀ} \AgdaInductiveConstructor{⌝ᵀ}\AgdaSymbol{)))}\<%
\\
\>[0]\AgdaIndent{4}{}\<[4]%
\>[4]\AgdaInductiveConstructor{‘s→→’} \AgdaSymbol{:} \AgdaSymbol{∀} \AgdaSymbol{\{}\AgdaBound{T} \AgdaBound{B}\AgdaSymbol{\}}\<%
\\
\>[4]\AgdaIndent{11}{}\<[11]%
\>[11]\AgdaSymbol{\{}\AgdaBound{b} \AgdaSymbol{:} \AgdaDatatype{Term} \AgdaSymbol{\{}\AgdaInductiveConstructor{ε}\AgdaSymbol{\}} \AgdaSymbol{(}\AgdaBound{T} \AgdaInductiveConstructor{‘→’} \AgdaInductiveConstructor{W} \AgdaSymbol{(}\AgdaInductiveConstructor{‘Type’} \AgdaInductiveConstructor{‘’} \AgdaInductiveConstructor{⌜} \AgdaInductiveConstructor{ε} \AgdaInductiveConstructor{▻} \AgdaBound{B} \AgdaInductiveConstructor{⌝ᶜ}\AgdaSymbol{))\}}\<%
\\
\>[4]\AgdaIndent{11}{}\<[11]%
\>[11]\AgdaSymbol{\{}\AgdaBound{c} \AgdaSymbol{:} \AgdaDatatype{Term} \AgdaSymbol{\{}\AgdaInductiveConstructor{ε}\AgdaSymbol{\}} \AgdaSymbol{(}\AgdaBound{T} \AgdaInductiveConstructor{‘→’} \AgdaInductiveConstructor{W} \AgdaSymbol{(}\AgdaInductiveConstructor{‘Term’} \AgdaInductiveConstructor{‘’₁} \AgdaInductiveConstructor{⌜} \AgdaInductiveConstructor{ε} \AgdaInductiveConstructor{⌝ᶜ} \AgdaInductiveConstructor{‘’} \AgdaInductiveConstructor{⌜} \AgdaBound{B} \AgdaInductiveConstructor{⌝ᵀ}\AgdaSymbol{))\}}\<%
\\
\>[4]\AgdaIndent{11}{}\<[11]%
\>[11]\AgdaSymbol{\{}\AgdaBound{v} \AgdaSymbol{:} \AgdaDatatype{Term} \AgdaSymbol{\{}\AgdaInductiveConstructor{ε}\AgdaSymbol{\}} \AgdaBound{T}\AgdaSymbol{\}} \AgdaSymbol{→}\<%
\\
\>[0]\AgdaIndent{6}{}\<[6]%
\>[6]\AgdaSymbol{(}\AgdaDatatype{Term} \AgdaSymbol{\{}\AgdaInductiveConstructor{ε}\AgdaSymbol{\}} \AgdaSymbol{(}\AgdaInductiveConstructor{‘Term’} \AgdaInductiveConstructor{‘’₁} \AgdaInductiveConstructor{⌜} \AgdaInductiveConstructor{ε} \AgdaInductiveConstructor{⌝ᶜ}\<%
\\
\>[6]\AgdaIndent{11}{}\<[11]%
\>[11]\AgdaInductiveConstructor{‘’} \AgdaSymbol{((}\AgdaInductiveConstructor{SW} \AgdaSymbol{(((}\AgdaInductiveConstructor{‘λ’} \AgdaSymbol{(}\AgdaInductiveConstructor{SW} \AgdaSymbol{(}\AgdaInductiveConstructor{w→} \AgdaBound{b} \AgdaInductiveConstructor{‘’ₐ} \AgdaInductiveConstructor{‘VAR₀’}\AgdaSymbol{)} \AgdaInductiveConstructor{w‘‘’’} \AgdaInductiveConstructor{SW} \AgdaSymbol{(}\AgdaInductiveConstructor{w→} \AgdaBound{c} \AgdaInductiveConstructor{‘’ₐ} \AgdaInductiveConstructor{‘VAR₀’}\AgdaSymbol{))} \AgdaInductiveConstructor{‘’ₐ} \AgdaBound{v}\AgdaSymbol{))))}\<%
\\
\>[11]\AgdaIndent{17}{}\<[17]%
\>[17]\AgdaInductiveConstructor{‘‘→'’’} \AgdaSymbol{(}\AgdaInductiveConstructor{SW} \AgdaSymbol{(}\AgdaBound{b} \AgdaInductiveConstructor{‘’ₐ} \AgdaBound{v}\AgdaSymbol{)} \AgdaInductiveConstructor{‘‘’’} \AgdaInductiveConstructor{SW} \AgdaSymbol{(}\AgdaBound{c} \AgdaInductiveConstructor{‘’ₐ} \AgdaBound{v}\AgdaSymbol{)))))}\<%
\\
\>[0]\AgdaIndent{4}{}\<[4]%
\>[4]\AgdaInductiveConstructor{‘s←←’} \AgdaSymbol{:} \AgdaSymbol{∀} \AgdaSymbol{\{}\AgdaBound{T} \AgdaBound{B}\AgdaSymbol{\}}\<%
\\
\>[4]\AgdaIndent{11}{}\<[11]%
\>[11]\AgdaSymbol{\{}\AgdaBound{b} \AgdaSymbol{:} \AgdaDatatype{Term} \AgdaSymbol{\{}\AgdaInductiveConstructor{ε}\AgdaSymbol{\}} \AgdaSymbol{(}\AgdaBound{T} \AgdaInductiveConstructor{‘→’} \AgdaInductiveConstructor{W} \AgdaSymbol{(}\AgdaInductiveConstructor{‘Type’} \AgdaInductiveConstructor{‘’} \AgdaInductiveConstructor{⌜} \AgdaInductiveConstructor{ε} \AgdaInductiveConstructor{▻} \AgdaBound{B} \AgdaInductiveConstructor{⌝ᶜ}\AgdaSymbol{))\}}\<%
\\
\>[4]\AgdaIndent{11}{}\<[11]%
\>[11]\AgdaSymbol{\{}\AgdaBound{c} \AgdaSymbol{:} \AgdaDatatype{Term} \AgdaSymbol{\{}\AgdaInductiveConstructor{ε}\AgdaSymbol{\}} \AgdaSymbol{(}\AgdaBound{T} \AgdaInductiveConstructor{‘→’} \AgdaInductiveConstructor{W} \AgdaSymbol{(}\AgdaInductiveConstructor{‘Term’} \AgdaInductiveConstructor{‘’₁} \AgdaInductiveConstructor{⌜} \AgdaInductiveConstructor{ε} \AgdaInductiveConstructor{⌝ᶜ} \AgdaInductiveConstructor{‘’} \AgdaInductiveConstructor{⌜} \AgdaBound{B} \AgdaInductiveConstructor{⌝ᵀ}\AgdaSymbol{))\}}\<%
\\
\>[4]\AgdaIndent{11}{}\<[11]%
\>[11]\AgdaSymbol{\{}\AgdaBound{v} \AgdaSymbol{:} \AgdaDatatype{Term} \AgdaSymbol{\{}\AgdaInductiveConstructor{ε}\AgdaSymbol{\}} \AgdaBound{T}\AgdaSymbol{\}} \AgdaSymbol{→}\<%
\\
\>[0]\AgdaIndent{6}{}\<[6]%
\>[6]\AgdaSymbol{(}\AgdaDatatype{Term} \AgdaSymbol{\{}\AgdaInductiveConstructor{ε}\AgdaSymbol{\}} \AgdaSymbol{(}\AgdaInductiveConstructor{‘Term’} \AgdaInductiveConstructor{‘’₁} \AgdaInductiveConstructor{⌜} \AgdaInductiveConstructor{ε} \AgdaInductiveConstructor{⌝ᶜ}\<%
\\
\>[6]\AgdaIndent{11}{}\<[11]%
\>[11]\AgdaInductiveConstructor{‘’} \AgdaSymbol{((}\AgdaInductiveConstructor{SW} \AgdaSymbol{(}\AgdaBound{b} \AgdaInductiveConstructor{‘’ₐ} \AgdaBound{v}\AgdaSymbol{)} \AgdaInductiveConstructor{‘‘’’} \AgdaInductiveConstructor{SW} \AgdaSymbol{(}\AgdaBound{c} \AgdaInductiveConstructor{‘’ₐ} \AgdaBound{v}\AgdaSymbol{))}\<%
\\
\>[11]\AgdaIndent{17}{}\<[17]%
\>[17]\AgdaInductiveConstructor{‘‘→'’’} \AgdaSymbol{(}\AgdaInductiveConstructor{SW} \AgdaSymbol{(((}\AgdaInductiveConstructor{‘λ’} \AgdaSymbol{(}\AgdaInductiveConstructor{SW} \AgdaSymbol{(}\AgdaInductiveConstructor{w→} \AgdaBound{b} \AgdaInductiveConstructor{‘’ₐ} \AgdaInductiveConstructor{‘VAR₀’}\AgdaSymbol{)} \AgdaInductiveConstructor{w‘‘’’} \AgdaInductiveConstructor{SW} \AgdaSymbol{(}\AgdaInductiveConstructor{w→} \AgdaBound{c} \AgdaInductiveConstructor{‘’ₐ} \AgdaInductiveConstructor{‘VAR₀’}\AgdaSymbol{))} \AgdaInductiveConstructor{‘’ₐ} \AgdaBound{v}\AgdaSymbol{)))))))}\<%
\\
\>\<%
\end{code}

\begin{code}%
\> \AgdaKeyword{module} \AgdaModule{well-typed-syntax-helpers} \AgdaKeyword{where}\<%
\\
\>[0]\AgdaIndent{2}{}\<[2]%
\>[2]\AgdaKeyword{open} \AgdaModule{well-typed-syntax}\<%
\\
%
\\
\>[0]\AgdaIndent{2}{}\<[2]%
\>[2]\AgdaKeyword{infixl} \AgdaNumber{3} \AgdaFixityOp{\_‘'’ₐ\_}\<%
\\
\>[0]\AgdaIndent{2}{}\<[2]%
\>[2]\AgdaKeyword{infixr} \AgdaNumber{1} \AgdaFixityOp{\_‘→'’\_}\<%
\\
\>[0]\AgdaIndent{2}{}\<[2]%
\>[2]\AgdaKeyword{infixl} \AgdaNumber{3} \AgdaFixityOp{\_‘’ᵗ\_}\<%
\\
\>[0]\AgdaIndent{2}{}\<[2]%
\>[2]\AgdaKeyword{infixl} \AgdaNumber{3} \AgdaFixityOp{\_‘’ᵗ₁\_}\<%
\\
\>[0]\AgdaIndent{2}{}\<[2]%
\>[2]\AgdaKeyword{infixl} \AgdaNumber{3} \AgdaFixityOp{\_‘’ᵗ₂\_}\<%
\\
\>[0]\AgdaIndent{2}{}\<[2]%
\>[2]\AgdaKeyword{infixr} \AgdaNumber{2} \AgdaFixityOp{\_‘∘’\_}\<%
\\
%
\\
\>[0]\AgdaIndent{2}{}\<[2]%
\>[2]\AgdaFunction{\_‘→'’\_} \AgdaSymbol{:} \AgdaSymbol{∀} \AgdaSymbol{\{}\AgdaBound{Γ}\AgdaSymbol{\}} \AgdaSymbol{→} \AgdaDatatype{Type} \AgdaBound{Γ} \AgdaSymbol{→} \AgdaDatatype{Type} \AgdaBound{Γ} \AgdaSymbol{→} \AgdaDatatype{Type} \AgdaBound{Γ}\<%
\\
\>[0]\AgdaIndent{2}{}\<[2]%
\>[2]\AgdaFunction{\_‘→'’\_} \AgdaSymbol{\{}\AgdaBound{Γ}\AgdaSymbol{\}} \AgdaBound{A} \AgdaBound{B} \AgdaSymbol{=} \AgdaInductiveConstructor{\_‘→’\_} \AgdaSymbol{\{}\AgdaBound{Γ}\AgdaSymbol{\}} \AgdaBound{A} \AgdaSymbol{(}\AgdaInductiveConstructor{W} \AgdaSymbol{\{}\AgdaBound{Γ}\AgdaSymbol{\}} \AgdaSymbol{\{}\AgdaBound{A}\AgdaSymbol{\}} \AgdaBound{B}\AgdaSymbol{)}\<%
\\
%
\\
\>[0]\AgdaIndent{2}{}\<[2]%
\>[2]\AgdaFunction{\_‘'’ₐ\_} \AgdaSymbol{:} \AgdaSymbol{∀} \AgdaSymbol{\{}\AgdaBound{Γ} \AgdaBound{A} \AgdaBound{B}\AgdaSymbol{\}} \AgdaSymbol{→} \AgdaDatatype{Term} \AgdaSymbol{\{}\AgdaBound{Γ}\AgdaSymbol{\}} \AgdaSymbol{(}\AgdaBound{A} \AgdaFunction{‘→'’} \AgdaBound{B}\AgdaSymbol{)} \AgdaSymbol{→} \AgdaDatatype{Term} \AgdaBound{A} \AgdaSymbol{→} \AgdaDatatype{Term} \AgdaBound{B}\<%
\\
\>[0]\AgdaIndent{2}{}\<[2]%
\>[2]\AgdaFunction{\_‘'’ₐ\_} \AgdaSymbol{\{}\AgdaBound{Γ}\AgdaSymbol{\}} \AgdaSymbol{\{}\AgdaBound{A}\AgdaSymbol{\}} \AgdaSymbol{\{}\AgdaBound{B}\AgdaSymbol{\}} \AgdaBound{f} \AgdaBound{x} \AgdaSymbol{=} \AgdaInductiveConstructor{SW} \AgdaSymbol{(}\AgdaInductiveConstructor{\_‘’ₐ\_} \AgdaSymbol{\{}\AgdaBound{Γ}\AgdaSymbol{\}} \AgdaSymbol{\{}\AgdaBound{A}\AgdaSymbol{\}} \AgdaSymbol{\{}\AgdaInductiveConstructor{W} \AgdaBound{B}\AgdaSymbol{\}} \AgdaBound{f} \AgdaBound{x}\AgdaSymbol{)}\<%
\\
%
\\
\>[0]\AgdaIndent{2}{}\<[2]%
\>[2]\AgdaFunction{\_‘’ᵗ\_} \AgdaSymbol{:} \AgdaSymbol{∀} \AgdaSymbol{\{}\AgdaBound{Γ} \AgdaBound{A}\AgdaSymbol{\}} \AgdaSymbol{\{}\AgdaBound{B} \AgdaSymbol{:} \AgdaDatatype{Type} \AgdaSymbol{(}\AgdaBound{Γ} \AgdaInductiveConstructor{▻} \AgdaBound{A}\AgdaSymbol{)\}} \AgdaSymbol{→} \AgdaSymbol{(}\AgdaBound{b} \AgdaSymbol{:} \AgdaDatatype{Term} \AgdaSymbol{\{}\AgdaBound{Γ} \AgdaInductiveConstructor{▻} \AgdaBound{A}\AgdaSymbol{\}} \AgdaBound{B}\AgdaSymbol{)} \AgdaSymbol{→} \AgdaSymbol{(}\AgdaBound{a} \AgdaSymbol{:} \AgdaDatatype{Term} \AgdaSymbol{\{}\AgdaBound{Γ}\AgdaSymbol{\}} \AgdaBound{A}\AgdaSymbol{)} \AgdaSymbol{→} \AgdaDatatype{Term} \AgdaSymbol{\{}\AgdaBound{Γ}\AgdaSymbol{\}} \AgdaSymbol{(}\AgdaBound{B} \AgdaInductiveConstructor{‘’} \AgdaBound{a}\AgdaSymbol{)}\<%
\\
\>[0]\AgdaIndent{2}{}\<[2]%
\>[2]\AgdaBound{b} \AgdaFunction{‘’ᵗ} \AgdaBound{a} \AgdaSymbol{=} \AgdaInductiveConstructor{‘λ’} \AgdaBound{b} \AgdaInductiveConstructor{‘’ₐ} \AgdaBound{a}\<%
\\
%
\\
\>[0]\AgdaIndent{2}{}\<[2]%
\>[2]\AgdaFunction{S∀} \AgdaSymbol{:} \AgdaSymbol{∀} \AgdaSymbol{\{}\AgdaBound{Γ} \AgdaBound{T} \AgdaBound{A} \AgdaBound{B}\AgdaSymbol{\}} \AgdaSymbol{\{}\AgdaBound{a} \AgdaSymbol{:} \AgdaDatatype{Term} \AgdaSymbol{\{}\AgdaBound{Γ}\AgdaSymbol{\}} \AgdaBound{T}\AgdaSymbol{\}} \AgdaSymbol{→}\<%
\\
\>[2]\AgdaIndent{27}{}\<[27]%
\>[27]\AgdaDatatype{Term} \AgdaSymbol{\{}\AgdaBound{Γ}\AgdaSymbol{\}} \AgdaSymbol{((}\AgdaBound{A} \AgdaInductiveConstructor{‘→’} \AgdaBound{B}\AgdaSymbol{)} \AgdaInductiveConstructor{‘’} \AgdaBound{a}\AgdaSymbol{)}\<%
\\
\>[2]\AgdaIndent{27}{}\<[27]%
\>[27]\AgdaSymbol{→} \AgdaDatatype{Term} \AgdaSymbol{\{}\AgdaBound{Γ}\AgdaSymbol{\}} \AgdaSymbol{(}\AgdaInductiveConstructor{\_‘→’\_} \AgdaSymbol{\{}\AgdaBound{Γ}\AgdaSymbol{\}} \AgdaSymbol{(}\AgdaBound{A} \AgdaInductiveConstructor{‘’} \AgdaBound{a}\AgdaSymbol{)} \AgdaSymbol{(}\AgdaBound{B} \AgdaInductiveConstructor{‘’₁} \AgdaBound{a}\AgdaSymbol{))}\<%
\\
\>[0]\AgdaIndent{2}{}\<[2]%
\>[2]\AgdaFunction{S∀} \AgdaSymbol{\{}\AgdaBound{Γ}\AgdaSymbol{\}} \AgdaSymbol{\{}\AgdaBound{T}\AgdaSymbol{\}} \AgdaSymbol{\{}\AgdaBound{A}\AgdaSymbol{\}} \AgdaSymbol{\{}\AgdaBound{B}\AgdaSymbol{\}} \AgdaSymbol{\{}\AgdaBound{a}\AgdaSymbol{\}} \AgdaBound{x} \AgdaSymbol{=} \AgdaInductiveConstructor{SW} \AgdaSymbol{((}\AgdaInductiveConstructor{WS∀} \AgdaSymbol{(}\AgdaInductiveConstructor{w} \AgdaBound{x}\AgdaSymbol{))} \AgdaFunction{‘’ᵗ} \AgdaBound{a}\AgdaSymbol{)}\<%
\\
%
\\
\>[0]\AgdaIndent{2}{}\<[2]%
\>[2]\AgdaFunction{‘λ'∙’} \AgdaSymbol{:} \AgdaSymbol{∀} \AgdaSymbol{\{}\AgdaBound{Γ} \AgdaBound{A} \AgdaBound{B}\AgdaSymbol{\}} \AgdaSymbol{→} \AgdaDatatype{Term} \AgdaSymbol{\{}\AgdaBound{Γ} \AgdaInductiveConstructor{▻} \AgdaBound{A}\AgdaSymbol{\}} \AgdaSymbol{(}\AgdaInductiveConstructor{W} \AgdaBound{B}\AgdaSymbol{)} \AgdaSymbol{->} \AgdaDatatype{Term} \AgdaSymbol{(}\AgdaBound{A} \AgdaFunction{‘→'’} \AgdaBound{B}\AgdaSymbol{)}\<%
\\
\>[0]\AgdaIndent{2}{}\<[2]%
\>[2]\AgdaFunction{‘λ'∙’} \AgdaBound{f} \AgdaSymbol{=} \AgdaInductiveConstructor{‘λ’} \AgdaBound{f}\<%
\\
%
\\
\>[0]\AgdaIndent{2}{}\<[2]%
\>[2]\AgdaFunction{un‘λ’} \AgdaSymbol{:} \AgdaSymbol{∀} \AgdaSymbol{\{}\AgdaBound{Γ} \AgdaBound{A} \AgdaBound{B}\AgdaSymbol{\}} \AgdaSymbol{→} \AgdaDatatype{Term} \AgdaSymbol{(}\AgdaBound{A} \AgdaInductiveConstructor{‘→’} \AgdaBound{B}\AgdaSymbol{)} \AgdaSymbol{→} \AgdaDatatype{Term} \AgdaSymbol{\{}\AgdaBound{Γ} \AgdaInductiveConstructor{▻} \AgdaBound{A}\AgdaSymbol{\}} \AgdaBound{B}\<%
\\
\>[0]\AgdaIndent{2}{}\<[2]%
\>[2]\AgdaFunction{un‘λ’} \AgdaBound{f} \AgdaSymbol{=} \AgdaInductiveConstructor{SW₁V} \AgdaSymbol{(}\AgdaInductiveConstructor{W∀} \AgdaSymbol{(}\AgdaInductiveConstructor{w} \AgdaBound{f}\AgdaSymbol{)} \AgdaInductiveConstructor{‘’ₐ} \AgdaInductiveConstructor{‘VAR₀’}\AgdaSymbol{)}\<%
\\
%
\\
\>[0]\AgdaIndent{2}{}\<[2]%
\>[2]\AgdaFunction{w∀} \AgdaSymbol{:} \AgdaSymbol{∀} \AgdaSymbol{\{}\AgdaBound{Γ} \AgdaBound{A} \AgdaBound{B} \AgdaBound{C}\AgdaSymbol{\}} \AgdaSymbol{→}\<%
\\
\>[2]\AgdaIndent{28}{}\<[28]%
\>[28]\AgdaDatatype{Term} \AgdaSymbol{\{}\AgdaBound{Γ}\AgdaSymbol{\}} \AgdaSymbol{(}\AgdaBound{A} \AgdaInductiveConstructor{‘→’} \AgdaBound{B}\AgdaSymbol{)}\<%
\\
\>[2]\AgdaIndent{28}{}\<[28]%
\>[28]\AgdaSymbol{→} \AgdaDatatype{Term} \AgdaSymbol{\{}\AgdaBound{Γ} \AgdaInductiveConstructor{▻} \AgdaBound{C}\AgdaSymbol{\}} \AgdaSymbol{(}\AgdaInductiveConstructor{W} \AgdaBound{A} \AgdaInductiveConstructor{‘→’} \AgdaInductiveConstructor{W₁} \AgdaBound{B}\AgdaSymbol{)}\<%
\\
\>[0]\AgdaIndent{2}{}\<[2]%
\>[2]\AgdaFunction{w∀} \AgdaSymbol{\{}\AgdaBound{Γ}\AgdaSymbol{\}} \AgdaSymbol{\{}\AgdaBound{A}\AgdaSymbol{\}} \AgdaSymbol{\{}\AgdaBound{B}\AgdaSymbol{\}} \AgdaSymbol{\{}\AgdaBound{C}\AgdaSymbol{\}} \AgdaBound{x} \AgdaSymbol{=} \AgdaInductiveConstructor{W∀} \AgdaSymbol{(}\AgdaInductiveConstructor{w} \AgdaBound{x}\AgdaSymbol{)}\<%
\\
%
\\
\>[0]\AgdaIndent{2}{}\<[2]%
\>[2]\AgdaFunction{w₁} \AgdaSymbol{:} \AgdaSymbol{∀} \AgdaSymbol{\{}\AgdaBound{Γ} \AgdaBound{A} \AgdaBound{B} \AgdaBound{C}\AgdaSymbol{\}} \AgdaSymbol{→} \AgdaDatatype{Term} \AgdaSymbol{\{}\AgdaBound{Γ} \AgdaInductiveConstructor{▻} \AgdaBound{B}\AgdaSymbol{\}} \AgdaBound{C} \AgdaSymbol{→} \AgdaDatatype{Term} \AgdaSymbol{\{}\AgdaBound{Γ} \AgdaInductiveConstructor{▻} \AgdaBound{A} \AgdaInductiveConstructor{▻} \AgdaInductiveConstructor{W} \AgdaSymbol{\{}\AgdaBound{Γ}\AgdaSymbol{\}} \AgdaSymbol{\{}\AgdaBound{A}\AgdaSymbol{\}} \AgdaBound{B}\AgdaSymbol{\}} \AgdaSymbol{(}\AgdaInductiveConstructor{W₁} \AgdaSymbol{\{}\AgdaBound{Γ}\AgdaSymbol{\}} \AgdaSymbol{\{}\AgdaBound{A}\AgdaSymbol{\}} \AgdaSymbol{\{}\AgdaArgument{B} \AgdaSymbol{=} \AgdaBound{B}\AgdaSymbol{\}} \AgdaBound{C}\AgdaSymbol{)}\<%
\\
\>[0]\AgdaIndent{2}{}\<[2]%
\>[2]\AgdaFunction{w₁} \AgdaBound{x} \AgdaSymbol{=} \AgdaFunction{un‘λ’} \AgdaSymbol{(}\AgdaInductiveConstructor{W∀} \AgdaSymbol{(}\AgdaInductiveConstructor{w} \AgdaSymbol{(}\AgdaInductiveConstructor{‘λ’} \AgdaBound{x}\AgdaSymbol{)))}\<%
\\
%
\\
\>[0]\AgdaIndent{2}{}\<[2]%
\>[2]\AgdaFunction{\_‘’ᵗ₁\_} \AgdaSymbol{:} \AgdaSymbol{∀} \AgdaSymbol{\{}\AgdaBound{Γ} \AgdaBound{A} \AgdaBound{B} \AgdaBound{C}\AgdaSymbol{\}} \AgdaSymbol{→} \AgdaSymbol{(}\AgdaBound{c} \AgdaSymbol{:} \AgdaDatatype{Term} \AgdaSymbol{\{}\AgdaBound{Γ} \AgdaInductiveConstructor{▻} \AgdaBound{A} \AgdaInductiveConstructor{▻} \AgdaBound{B}\AgdaSymbol{\}} \AgdaBound{C}\AgdaSymbol{)} \AgdaSymbol{→} \AgdaSymbol{(}\AgdaBound{a} \AgdaSymbol{:} \AgdaDatatype{Term} \AgdaSymbol{\{}\AgdaBound{Γ}\AgdaSymbol{\}} \AgdaBound{A}\AgdaSymbol{)} \AgdaSymbol{→} \AgdaDatatype{Term} \AgdaSymbol{\{}\AgdaBound{Γ} \AgdaInductiveConstructor{▻} \AgdaBound{B} \AgdaInductiveConstructor{‘’} \AgdaBound{a}\AgdaSymbol{\}} \AgdaSymbol{(}\AgdaBound{C} \AgdaInductiveConstructor{‘’₁} \AgdaBound{a}\AgdaSymbol{)}\<%
\\
\>[0]\AgdaIndent{2}{}\<[2]%
\>[2]\AgdaBound{f} \AgdaFunction{‘’ᵗ₁} \AgdaBound{x} \AgdaSymbol{=} \AgdaFunction{un‘λ’} \AgdaSymbol{(}\AgdaFunction{S∀} \AgdaSymbol{(}\AgdaInductiveConstructor{‘λ’} \AgdaSymbol{(}\AgdaInductiveConstructor{‘λ’} \AgdaBound{f}\AgdaSymbol{)} \AgdaInductiveConstructor{‘’ₐ} \AgdaBound{x}\AgdaSymbol{))}\<%
\\
\>[0]\AgdaIndent{2}{}\<[2]%
\>[2]\AgdaFunction{\_‘’ᵗ₂\_} \AgdaSymbol{:} \AgdaSymbol{∀} \AgdaSymbol{\{}\AgdaBound{Γ} \AgdaBound{A} \AgdaBound{B} \AgdaBound{C} \AgdaBound{D}\AgdaSymbol{\}} \AgdaSymbol{→} \AgdaSymbol{(}\AgdaBound{c} \AgdaSymbol{:} \AgdaDatatype{Term} \AgdaSymbol{\{}\AgdaBound{Γ} \AgdaInductiveConstructor{▻} \AgdaBound{A} \AgdaInductiveConstructor{▻} \AgdaBound{B} \AgdaInductiveConstructor{▻} \AgdaBound{C}\AgdaSymbol{\}} \AgdaBound{D}\AgdaSymbol{)} \AgdaSymbol{→} \AgdaSymbol{(}\AgdaBound{a} \AgdaSymbol{:} \AgdaDatatype{Term} \AgdaSymbol{\{}\AgdaBound{Γ}\AgdaSymbol{\}} \AgdaBound{A}\AgdaSymbol{)} \AgdaSymbol{→} \AgdaDatatype{Term} \AgdaSymbol{\{}\AgdaBound{Γ} \AgdaInductiveConstructor{▻} \AgdaBound{B} \AgdaInductiveConstructor{‘’} \AgdaBound{a} \AgdaInductiveConstructor{▻} \AgdaBound{C} \AgdaInductiveConstructor{‘’₁} \AgdaBound{a}\AgdaSymbol{\}} \AgdaSymbol{(}\AgdaBound{D} \AgdaInductiveConstructor{‘’₂} \AgdaBound{a}\AgdaSymbol{)}\<%
\\
\>[0]\AgdaIndent{2}{}\<[2]%
\>[2]\AgdaBound{f} \AgdaFunction{‘’ᵗ₂} \AgdaBound{x} \AgdaSymbol{=} \AgdaFunction{un‘λ’} \AgdaSymbol{(}\AgdaInductiveConstructor{S₁∀} \AgdaSymbol{(}\AgdaFunction{un‘λ’} \AgdaSymbol{(}\AgdaFunction{S∀} \AgdaSymbol{(}\AgdaInductiveConstructor{‘λ’} \AgdaSymbol{(}\AgdaInductiveConstructor{‘λ’} \AgdaSymbol{(}\AgdaInductiveConstructor{‘λ’} \AgdaBound{f}\AgdaSymbol{))} \AgdaInductiveConstructor{‘’ₐ} \AgdaBound{x}\AgdaSymbol{))))}\<%
\\
%
\\
\>[0]\AgdaIndent{2}{}\<[2]%
\>[2]\AgdaFunction{S₁₀WW} \AgdaSymbol{:} \AgdaSymbol{∀} \AgdaSymbol{\{}\AgdaBound{Γ} \AgdaBound{T} \AgdaBound{A}\AgdaSymbol{\}} \AgdaSymbol{\{}\AgdaBound{B} \AgdaSymbol{:} \AgdaDatatype{Type} \AgdaSymbol{(}\AgdaBound{Γ} \AgdaInductiveConstructor{▻} \AgdaBound{A}\AgdaSymbol{)\}}\<%
\\
\>[2]\AgdaIndent{6}{}\<[6]%
\>[6]\AgdaSymbol{→} \AgdaSymbol{\{}\AgdaBound{a} \AgdaSymbol{:} \AgdaDatatype{Term} \AgdaSymbol{\{}\AgdaBound{Γ}\AgdaSymbol{\}} \AgdaBound{A}\AgdaSymbol{\}}\<%
\\
\>[2]\AgdaIndent{6}{}\<[6]%
\>[6]\AgdaSymbol{→} \AgdaSymbol{\{}\AgdaBound{b} \AgdaSymbol{:} \AgdaDatatype{Term} \AgdaSymbol{\{}\AgdaBound{Γ}\AgdaSymbol{\}} \AgdaSymbol{(}\AgdaBound{B} \AgdaInductiveConstructor{‘’} \AgdaBound{a}\AgdaSymbol{)\}}\<%
\\
\>[2]\AgdaIndent{6}{}\<[6]%
\>[6]\AgdaSymbol{→} \AgdaDatatype{Term} \AgdaSymbol{\{}\AgdaBound{Γ}\AgdaSymbol{\}} \AgdaSymbol{(}\AgdaInductiveConstructor{W} \AgdaSymbol{(}\AgdaInductiveConstructor{W} \AgdaBound{T}\AgdaSymbol{)} \AgdaInductiveConstructor{‘’₁} \AgdaBound{a} \AgdaInductiveConstructor{‘’} \AgdaBound{b}\AgdaSymbol{)}\<%
\\
\>[2]\AgdaIndent{6}{}\<[6]%
\>[6]\AgdaSymbol{→} \AgdaDatatype{Term} \AgdaSymbol{\{}\AgdaBound{Γ}\AgdaSymbol{\}} \AgdaBound{T}\<%
\\
\>[0]\AgdaIndent{2}{}\<[2]%
\>[2]\AgdaFunction{S₁₀WW} \AgdaBound{x} \AgdaSymbol{=} \AgdaInductiveConstructor{SW} \AgdaSymbol{(}\AgdaInductiveConstructor{S₁₀W} \AgdaBound{x}\AgdaSymbol{)}\<%
\\
%
\\
%
\\
\>[0]\AgdaIndent{2}{}\<[2]%
\>[2]\AgdaFunction{S₂₁₀WW} \AgdaSymbol{:} \AgdaSymbol{∀} \AgdaSymbol{\{}\AgdaBound{Γ} \AgdaBound{A} \AgdaBound{B} \AgdaBound{C} \AgdaBound{T}\AgdaSymbol{\}}\<%
\\
\>[2]\AgdaIndent{11}{}\<[11]%
\>[11]\AgdaSymbol{\{}\AgdaBound{a} \AgdaSymbol{:} \AgdaDatatype{Term} \AgdaSymbol{\{}\AgdaBound{Γ}\AgdaSymbol{\}} \AgdaBound{A}\AgdaSymbol{\}}\<%
\\
\>[2]\AgdaIndent{11}{}\<[11]%
\>[11]\AgdaSymbol{\{}\AgdaBound{b} \AgdaSymbol{:} \AgdaDatatype{Term} \AgdaSymbol{\{}\AgdaBound{Γ}\AgdaSymbol{\}} \AgdaSymbol{(}\AgdaBound{B} \AgdaInductiveConstructor{‘’} \AgdaBound{a}\AgdaSymbol{)\}}\<%
\\
\>[2]\AgdaIndent{11}{}\<[11]%
\>[11]\AgdaSymbol{\{}\AgdaBound{c} \AgdaSymbol{:} \AgdaDatatype{Term} \AgdaSymbol{\{}\AgdaBound{Γ}\AgdaSymbol{\}} \AgdaSymbol{(}\AgdaBound{C} \AgdaInductiveConstructor{‘’₁} \AgdaBound{a} \AgdaInductiveConstructor{‘’} \AgdaBound{b}\AgdaSymbol{)\}} \AgdaSymbol{→}\<%
\\
\>[0]\AgdaIndent{6}{}\<[6]%
\>[6]\AgdaDatatype{Term} \AgdaSymbol{\{}\AgdaBound{Γ}\AgdaSymbol{\}} \AgdaSymbol{(}\AgdaInductiveConstructor{W} \AgdaSymbol{(}\AgdaInductiveConstructor{W} \AgdaBound{T}\AgdaSymbol{)} \AgdaInductiveConstructor{‘’₂} \AgdaBound{a} \AgdaInductiveConstructor{‘’₁} \AgdaBound{b} \AgdaInductiveConstructor{‘’} \AgdaBound{c}\AgdaSymbol{)}\<%
\\
\>[0]\AgdaIndent{6}{}\<[6]%
\>[6]\AgdaSymbol{→} \AgdaDatatype{Term} \AgdaSymbol{\{}\AgdaBound{Γ}\AgdaSymbol{\}} \AgdaSymbol{(}\AgdaBound{T} \AgdaInductiveConstructor{‘’} \AgdaBound{a}\AgdaSymbol{)}\<%
\\
\>[0]\AgdaIndent{2}{}\<[2]%
\>[2]\AgdaFunction{S₂₁₀WW} \AgdaBound{x} \AgdaSymbol{=} \AgdaInductiveConstructor{S₁₀W} \AgdaSymbol{(}\AgdaInductiveConstructor{S₂₁₀W} \AgdaBound{x}\AgdaSymbol{)}\<%
\\
%
\\
\>[0]\AgdaIndent{2}{}\<[2]%
\>[2]\AgdaFunction{W₁₀₁W} \AgdaSymbol{:} \AgdaSymbol{∀} \AgdaSymbol{\{}\AgdaBound{Γ} \AgdaBound{A} \AgdaBound{B} \AgdaBound{C} \AgdaBound{T}\AgdaSymbol{\}} \AgdaSymbol{→} \AgdaDatatype{Term} \AgdaSymbol{\{}\AgdaBound{Γ} \AgdaInductiveConstructor{▻} \AgdaBound{A} \AgdaInductiveConstructor{▻} \AgdaBound{B} \AgdaInductiveConstructor{▻} \AgdaInductiveConstructor{W} \AgdaSymbol{(}\AgdaInductiveConstructor{W} \AgdaBound{C}\AgdaSymbol{)\}} \AgdaSymbol{(}\AgdaInductiveConstructor{W₁} \AgdaSymbol{(}\AgdaInductiveConstructor{W₁} \AgdaSymbol{(}\AgdaInductiveConstructor{W} \AgdaBound{T}\AgdaSymbol{)))} \AgdaSymbol{→} \AgdaDatatype{Term} \AgdaSymbol{\{}\AgdaBound{Γ} \AgdaInductiveConstructor{▻} \AgdaBound{A} \AgdaInductiveConstructor{▻} \AgdaBound{B} \AgdaInductiveConstructor{▻} \AgdaInductiveConstructor{W} \AgdaSymbol{(}\AgdaInductiveConstructor{W} \AgdaBound{C}\AgdaSymbol{)\}} \AgdaSymbol{(}\AgdaInductiveConstructor{W₁} \AgdaSymbol{(}\AgdaInductiveConstructor{W} \AgdaSymbol{(}\AgdaInductiveConstructor{W} \AgdaBound{T}\AgdaSymbol{)))}\<%
\\
\>[0]\AgdaIndent{2}{}\<[2]%
\>[2]\AgdaFunction{W₁₀₁W} \AgdaSymbol{=} \AgdaInductiveConstructor{W₁₀₁₀}\<%
\\
%
\\
\>[0]\AgdaIndent{2}{}\<[2]%
\>[2]\AgdaFunction{W∀-nd} \AgdaSymbol{:} \AgdaSymbol{∀} \AgdaSymbol{\{}\AgdaBound{Γ} \AgdaBound{A} \AgdaBound{B} \AgdaBound{C}\AgdaSymbol{\}} \AgdaSymbol{→}\<%
\\
\>[2]\AgdaIndent{28}{}\<[28]%
\>[28]\AgdaDatatype{Term} \AgdaSymbol{\{}\AgdaBound{Γ} \AgdaInductiveConstructor{▻} \AgdaBound{C}\AgdaSymbol{\}} \AgdaSymbol{(}\AgdaInductiveConstructor{W} \AgdaSymbol{(}\AgdaBound{A} \AgdaFunction{‘→'’} \AgdaBound{B}\AgdaSymbol{))}\<%
\\
\>[2]\AgdaIndent{28}{}\<[28]%
\>[28]\AgdaSymbol{→} \AgdaDatatype{Term} \AgdaSymbol{\{}\AgdaBound{Γ} \AgdaInductiveConstructor{▻} \AgdaBound{C}\AgdaSymbol{\}} \AgdaSymbol{(}\AgdaInductiveConstructor{W} \AgdaBound{A} \AgdaFunction{‘→'’} \AgdaInductiveConstructor{W} \AgdaBound{B}\AgdaSymbol{)}\<%
\\
\>[0]\AgdaIndent{2}{}\<[2]%
\>[2]\AgdaFunction{W∀-nd} \AgdaBound{x} \AgdaSymbol{=} \AgdaFunction{‘λ'∙’} \AgdaSymbol{(}\AgdaInductiveConstructor{W₁₀} \AgdaSymbol{(}\AgdaInductiveConstructor{SW₁V} \AgdaSymbol{(}\AgdaInductiveConstructor{W∀} \AgdaSymbol{(}\AgdaInductiveConstructor{w} \AgdaSymbol{(}\AgdaInductiveConstructor{W∀} \AgdaBound{x}\AgdaSymbol{))} \AgdaInductiveConstructor{‘’ₐ} \AgdaInductiveConstructor{‘VAR₀’}\AgdaSymbol{)))}\<%
\\
%
\\
\>[0]\AgdaIndent{2}{}\<[2]%
\>[2]\AgdaFunction{w∀-nd} \AgdaSymbol{:} \AgdaSymbol{∀} \AgdaSymbol{\{}\AgdaBound{Γ} \AgdaBound{A} \AgdaBound{B} \AgdaBound{C}\AgdaSymbol{\}} \AgdaSymbol{→}\<%
\\
\>[2]\AgdaIndent{31}{}\<[31]%
\>[31]\AgdaDatatype{Term} \AgdaSymbol{(}\AgdaBound{A} \AgdaFunction{‘→'’} \AgdaBound{B}\AgdaSymbol{)}\<%
\\
\>[2]\AgdaIndent{31}{}\<[31]%
\>[31]\AgdaSymbol{→} \AgdaDatatype{Term} \AgdaSymbol{\{}\AgdaBound{Γ} \AgdaInductiveConstructor{▻} \AgdaBound{C}\AgdaSymbol{\}} \AgdaSymbol{(}\AgdaInductiveConstructor{W} \AgdaBound{A} \AgdaFunction{‘→'’} \AgdaInductiveConstructor{W} \AgdaBound{B}\AgdaSymbol{)}\<%
\\
\>[0]\AgdaIndent{2}{}\<[2]%
\>[2]\AgdaFunction{w∀-nd} \AgdaSymbol{\{}\AgdaBound{Γ}\AgdaSymbol{\}} \AgdaSymbol{\{}\AgdaBound{A}\AgdaSymbol{\}} \AgdaSymbol{\{}\AgdaBound{B}\AgdaSymbol{\}} \AgdaSymbol{\{}\AgdaBound{C}\AgdaSymbol{\}} \AgdaBound{x} \AgdaSymbol{=} \AgdaFunction{W∀-nd} \AgdaSymbol{(}\AgdaInductiveConstructor{w} \AgdaBound{x}\AgdaSymbol{)}\<%
\\
%
\\
%
\\
%
\\
%
\\
\>[0]\AgdaIndent{2}{}\<[2]%
\>[2]\AgdaFunction{W∀-nd-∀-nd} \AgdaSymbol{:} \AgdaSymbol{∀} \AgdaSymbol{\{}\AgdaBound{Γ} \AgdaBound{A} \AgdaBound{B} \AgdaBound{C} \AgdaBound{D}\AgdaSymbol{\}} \AgdaSymbol{→}\<%
\\
\>[2]\AgdaIndent{6}{}\<[6]%
\>[6]\AgdaDatatype{Term} \AgdaSymbol{\{}\AgdaBound{Γ} \AgdaInductiveConstructor{▻} \AgdaBound{D}\AgdaSymbol{\}} \AgdaSymbol{(}\AgdaInductiveConstructor{W} \AgdaSymbol{(}\AgdaBound{A} \AgdaFunction{‘→'’} \AgdaBound{B} \AgdaFunction{‘→'’} \AgdaBound{C}\AgdaSymbol{))}\<%
\\
\>[2]\AgdaIndent{6}{}\<[6]%
\>[6]\AgdaSymbol{→} \AgdaDatatype{Term} \AgdaSymbol{\{}\AgdaBound{Γ} \AgdaInductiveConstructor{▻} \AgdaBound{D}\AgdaSymbol{\}} \AgdaSymbol{(}\AgdaInductiveConstructor{W} \AgdaBound{A} \AgdaFunction{‘→'’} \AgdaInductiveConstructor{W} \AgdaBound{B} \AgdaFunction{‘→'’} \AgdaInductiveConstructor{W} \AgdaBound{C}\AgdaSymbol{)}\<%
\\
\>[0]\AgdaIndent{2}{}\<[2]%
\>[2]\AgdaFunction{W∀-nd-∀-nd} \AgdaBound{x} \AgdaSymbol{=} \AgdaInductiveConstructor{‘λ’} \AgdaSymbol{(}\AgdaInductiveConstructor{W∀⁻¹} \AgdaSymbol{(}\AgdaInductiveConstructor{‘λ’} \AgdaSymbol{(}\AgdaFunction{W₁₀₁W} \AgdaSymbol{(}\AgdaInductiveConstructor{SW₁V} \AgdaSymbol{(}\AgdaFunction{w∀} \AgdaSymbol{(}\AgdaInductiveConstructor{W∀} \AgdaSymbol{(}\AgdaInductiveConstructor{WW∀} \AgdaSymbol{(}\AgdaInductiveConstructor{w→} \AgdaSymbol{(}\AgdaFunction{W∀-nd} \AgdaBound{x}\AgdaSymbol{)} \AgdaFunction{‘'’ₐ} \AgdaInductiveConstructor{‘VAR₀’}\AgdaSymbol{)))} \AgdaInductiveConstructor{‘’ₐ} \AgdaInductiveConstructor{‘VAR₀’}\AgdaSymbol{)))))}\<%
\\
%
\\
\>[0]\AgdaIndent{2}{}\<[2]%
\>[2]\AgdaFunction{w→→} \AgdaSymbol{:} \AgdaSymbol{∀} \AgdaSymbol{\{}\AgdaBound{Γ} \AgdaBound{A} \AgdaBound{B} \AgdaBound{C} \AgdaBound{D}\AgdaSymbol{\}} \AgdaSymbol{→}\<%
\\
\>[2]\AgdaIndent{6}{}\<[6]%
\>[6]\AgdaDatatype{Term} \AgdaSymbol{(}\AgdaBound{A} \AgdaFunction{‘→'’} \AgdaBound{B} \AgdaFunction{‘→'’} \AgdaBound{C}\AgdaSymbol{)}\<%
\\
\>[2]\AgdaIndent{6}{}\<[6]%
\>[6]\AgdaSymbol{→} \AgdaDatatype{Term} \AgdaSymbol{\{}\AgdaBound{Γ} \AgdaInductiveConstructor{▻} \AgdaBound{D}\AgdaSymbol{\}} \AgdaSymbol{(}\AgdaInductiveConstructor{W} \AgdaBound{A} \AgdaFunction{‘→'’} \AgdaInductiveConstructor{W} \AgdaBound{B} \AgdaFunction{‘→'’} \AgdaInductiveConstructor{W} \AgdaBound{C}\AgdaSymbol{)}\<%
\\
\>[0]\AgdaIndent{2}{}\<[2]%
\>[2]\AgdaFunction{w→→} \AgdaSymbol{\{}\AgdaBound{Γ}\AgdaSymbol{\}} \AgdaSymbol{\{}\AgdaBound{A}\AgdaSymbol{\}} \AgdaSymbol{\{}\AgdaBound{B}\AgdaSymbol{\}} \AgdaSymbol{\{}\AgdaBound{C}\AgdaSymbol{\}} \AgdaSymbol{\{}\AgdaBound{D}\AgdaSymbol{\}} \AgdaBound{x} \AgdaSymbol{=} \AgdaFunction{W∀-nd-∀-nd} \AgdaSymbol{(}\AgdaInductiveConstructor{w} \AgdaBound{x}\AgdaSymbol{)}\<%
\\
%
\\
\>[0]\AgdaIndent{2}{}\<[2]%
\>[2]\AgdaFunction{W₁S₁W⁻¹} \AgdaSymbol{:} \AgdaSymbol{∀} \AgdaSymbol{\{}\AgdaBound{Γ} \AgdaBound{A} \AgdaBound{T''} \AgdaBound{T'} \AgdaBound{T}\AgdaSymbol{\}} \AgdaSymbol{\{}\AgdaBound{a} \AgdaSymbol{:} \AgdaDatatype{Term} \AgdaSymbol{\{}\AgdaBound{Γ}\AgdaSymbol{\}} \AgdaBound{A}\AgdaSymbol{\}}\<%
\\
\>[2]\AgdaIndent{8}{}\<[8]%
\>[8]\AgdaSymbol{→} \AgdaDatatype{Term} \AgdaSymbol{\{}\AgdaBound{Γ} \AgdaInductiveConstructor{▻} \AgdaBound{T''} \AgdaInductiveConstructor{▻} \AgdaInductiveConstructor{W} \AgdaSymbol{(}\AgdaBound{T'} \AgdaInductiveConstructor{‘’} \AgdaBound{a}\AgdaSymbol{)\}} \AgdaSymbol{(}\AgdaInductiveConstructor{W₁} \AgdaSymbol{(}\AgdaInductiveConstructor{W} \AgdaSymbol{(}\AgdaBound{T} \AgdaInductiveConstructor{‘’} \AgdaBound{a}\AgdaSymbol{)))}\<%
\\
\>[2]\AgdaIndent{8}{}\<[8]%
\>[8]\AgdaSymbol{→} \AgdaDatatype{Term} \AgdaSymbol{\{}\AgdaBound{Γ} \AgdaInductiveConstructor{▻} \AgdaBound{T''} \AgdaInductiveConstructor{▻} \AgdaInductiveConstructor{W} \AgdaSymbol{(}\AgdaBound{T'} \AgdaInductiveConstructor{‘’} \AgdaBound{a}\AgdaSymbol{)\}} \AgdaSymbol{(}\AgdaInductiveConstructor{W₁} \AgdaSymbol{(}\AgdaInductiveConstructor{W} \AgdaBound{T} \AgdaInductiveConstructor{‘’₁} \AgdaBound{a}\AgdaSymbol{))}\<%
\\
\>[0]\AgdaIndent{2}{}\<[2]%
\>[2]\AgdaFunction{W₁S₁W⁻¹} \AgdaSymbol{=} \AgdaInductiveConstructor{W₁SW₁⁻¹}\<%
\\
%
\\
%
\\
\>[0]\AgdaIndent{2}{}\<[2]%
\>[2]\AgdaFunction{SW₁⁻¹} \AgdaSymbol{:} \AgdaSymbol{∀} \AgdaSymbol{\{}\AgdaBound{Γ} \AgdaBound{A} \AgdaBound{T'} \AgdaBound{T}\AgdaSymbol{\}}\<%
\\
\>[2]\AgdaIndent{11}{}\<[11]%
\>[11]\AgdaSymbol{\{}\AgdaBound{a} \AgdaSymbol{:} \AgdaDatatype{Term} \AgdaSymbol{\{}\AgdaBound{Γ}\AgdaSymbol{\}} \AgdaBound{A}\AgdaSymbol{\}} \AgdaSymbol{→}\<%
\\
\>[0]\AgdaIndent{6}{}\<[6]%
\>[6]\AgdaDatatype{Term} \AgdaSymbol{\{(}\AgdaBound{Γ} \AgdaInductiveConstructor{▻} \AgdaBound{T'} \AgdaInductiveConstructor{‘’} \AgdaBound{a}\AgdaSymbol{)\}} \AgdaSymbol{(}\AgdaInductiveConstructor{W} \AgdaSymbol{(}\AgdaBound{T} \AgdaInductiveConstructor{‘’} \AgdaBound{a}\AgdaSymbol{))}\<%
\\
\>[0]\AgdaIndent{6}{}\<[6]%
\>[6]\AgdaSymbol{→} \AgdaDatatype{Term} \AgdaSymbol{\{(}\AgdaBound{Γ} \AgdaInductiveConstructor{▻} \AgdaBound{T'} \AgdaInductiveConstructor{‘’} \AgdaBound{a}\AgdaSymbol{)\}} \AgdaSymbol{(}\AgdaInductiveConstructor{W} \AgdaBound{T} \AgdaInductiveConstructor{‘’₁} \AgdaBound{a}\AgdaSymbol{)}\<%
\\
\>[0]\AgdaIndent{2}{}\<[2]%
\>[2]\AgdaFunction{SW₁⁻¹} \AgdaSymbol{\{}\AgdaArgument{a} \AgdaSymbol{=} \AgdaBound{a}\AgdaSymbol{\}} \AgdaBound{x} \AgdaSymbol{=} \AgdaInductiveConstructor{S₁W₁} \AgdaSymbol{(}\AgdaFunction{W₁S₁W⁻¹} \AgdaSymbol{(}\AgdaFunction{w₁} \AgdaBound{x}\AgdaSymbol{)} \AgdaFunction{‘’ᵗ₁} \AgdaBound{a}\AgdaSymbol{)}\<%
\\
%
\\
\>[0]\AgdaIndent{2}{}\<[2]%
\>[2]\AgdaFunction{S₁W⁻¹} \AgdaSymbol{=} \AgdaFunction{SW₁⁻¹}\<%
\\
%
\\
\>[0]\AgdaIndent{2}{}\<[2]%
\>[2]\AgdaFunction{\_‘∘’\_} \AgdaSymbol{:} \AgdaSymbol{∀} \AgdaSymbol{\{}\AgdaBound{Γ} \AgdaBound{A} \AgdaBound{B} \AgdaBound{C}\AgdaSymbol{\}}\<%
\\
\>[2]\AgdaIndent{6}{}\<[6]%
\>[6]\AgdaSymbol{→} \AgdaDatatype{Term} \AgdaSymbol{\{}\AgdaBound{Γ}\AgdaSymbol{\}} \AgdaSymbol{(}\AgdaBound{A} \AgdaFunction{‘→'’} \AgdaBound{B}\AgdaSymbol{)}\<%
\\
\>[2]\AgdaIndent{6}{}\<[6]%
\>[6]\AgdaSymbol{→} \AgdaDatatype{Term} \AgdaSymbol{\{}\AgdaBound{Γ}\AgdaSymbol{\}} \AgdaSymbol{(}\AgdaBound{B} \AgdaFunction{‘→'’} \AgdaBound{C}\AgdaSymbol{)}\<%
\\
\>[2]\AgdaIndent{6}{}\<[6]%
\>[6]\AgdaSymbol{→} \AgdaDatatype{Term} \AgdaSymbol{\{}\AgdaBound{Γ}\AgdaSymbol{\}} \AgdaSymbol{(}\AgdaBound{A} \AgdaFunction{‘→'’} \AgdaBound{C}\AgdaSymbol{)}\<%
\\
\>[0]\AgdaIndent{2}{}\<[2]%
\>[2]\AgdaBound{g} \AgdaFunction{‘∘’} \AgdaBound{f} \AgdaSymbol{=} \AgdaInductiveConstructor{‘λ’} \AgdaSymbol{(}\AgdaInductiveConstructor{w→} \AgdaBound{f} \AgdaFunction{‘'’ₐ} \AgdaSymbol{(}\AgdaInductiveConstructor{w→} \AgdaBound{g} \AgdaFunction{‘'’ₐ} \AgdaInductiveConstructor{‘VAR₀’}\AgdaSymbol{))}\<%
\\
%
\\
%
\\
\>[0]\AgdaIndent{2}{}\<[2]%
\>[2]\AgdaFunction{WSSW₁'} \AgdaSymbol{:} \AgdaSymbol{∀} \AgdaSymbol{\{}\AgdaBound{Γ} \AgdaBound{T'} \AgdaBound{B} \AgdaBound{A}\AgdaSymbol{\}} \AgdaSymbol{\{}\AgdaBound{b} \AgdaSymbol{:} \AgdaDatatype{Term} \AgdaSymbol{\{}\AgdaBound{Γ}\AgdaSymbol{\}} \AgdaBound{B}\AgdaSymbol{\}} \AgdaSymbol{\{}\AgdaBound{a} \AgdaSymbol{:} \AgdaDatatype{Term} \AgdaSymbol{\{}\AgdaBound{Γ} \AgdaInductiveConstructor{▻} \AgdaBound{B}\AgdaSymbol{\}} \AgdaSymbol{(}\AgdaInductiveConstructor{W} \AgdaBound{A}\AgdaSymbol{)\}} \AgdaSymbol{\{}\AgdaBound{T} \AgdaSymbol{:} \AgdaDatatype{Type} \AgdaSymbol{(}\AgdaBound{Γ} \AgdaInductiveConstructor{▻} \AgdaBound{A}\AgdaSymbol{)\}}\<%
\\
\>[2]\AgdaIndent{8}{}\<[8]%
\>[8]\AgdaSymbol{→} \AgdaDatatype{Term} \AgdaSymbol{\{}\AgdaBound{Γ} \AgdaInductiveConstructor{▻} \AgdaBound{T'}\AgdaSymbol{\}} \AgdaSymbol{(}\AgdaInductiveConstructor{W} \AgdaSymbol{(}\AgdaBound{T} \AgdaInductiveConstructor{‘’} \AgdaSymbol{(}\AgdaInductiveConstructor{SW} \AgdaSymbol{(}\AgdaBound{a} \AgdaFunction{‘’ᵗ} \AgdaBound{b}\AgdaSymbol{))))}\<%
\\
\>[2]\AgdaIndent{8}{}\<[8]%
\>[8]\AgdaSymbol{→} \AgdaDatatype{Term} \AgdaSymbol{\{}\AgdaBound{Γ} \AgdaInductiveConstructor{▻} \AgdaBound{T'}\AgdaSymbol{\}} \AgdaSymbol{(}\AgdaInductiveConstructor{W} \AgdaSymbol{(}\AgdaInductiveConstructor{W₁} \AgdaBound{T} \AgdaInductiveConstructor{‘’} \AgdaBound{a} \AgdaInductiveConstructor{‘’} \AgdaBound{b}\AgdaSymbol{))}\<%
\\
\>[0]\AgdaIndent{2}{}\<[2]%
\>[2]\AgdaFunction{WSSW₁'} \AgdaSymbol{=} \AgdaInductiveConstructor{WSSW₁⁻¹}\<%
\\
%
\\
\>[0]\AgdaIndent{2}{}\<[2]%
\>[2]\AgdaFunction{SSW₁⁻¹-arr} \AgdaSymbol{:} \AgdaSymbol{∀} \AgdaSymbol{\{}\AgdaBound{Γ} \AgdaBound{B} \AgdaBound{A}\AgdaSymbol{\}}\<%
\\
\>[2]\AgdaIndent{11}{}\<[11]%
\>[11]\AgdaSymbol{\{}\AgdaBound{b} \AgdaSymbol{:} \AgdaDatatype{Term} \AgdaSymbol{\{}\AgdaBound{Γ}\AgdaSymbol{\}} \AgdaBound{B}\AgdaSymbol{\}}\<%
\\
\>[2]\AgdaIndent{11}{}\<[11]%
\>[11]\AgdaSymbol{\{}\AgdaBound{a} \AgdaSymbol{:} \AgdaDatatype{Term} \AgdaSymbol{\{}\AgdaBound{Γ} \AgdaInductiveConstructor{▻} \AgdaBound{B}\AgdaSymbol{\}} \AgdaSymbol{(}\AgdaInductiveConstructor{W} \AgdaBound{A}\AgdaSymbol{)\}}\<%
\\
\>[2]\AgdaIndent{11}{}\<[11]%
\>[11]\AgdaSymbol{\{}\AgdaBound{T} \AgdaSymbol{:} \AgdaDatatype{Type} \AgdaSymbol{(}\AgdaBound{Γ} \AgdaInductiveConstructor{▻} \AgdaBound{A}\AgdaSymbol{)\}}\<%
\\
\>[2]\AgdaIndent{11}{}\<[11]%
\>[11]\AgdaSymbol{\{}\AgdaBound{X}\AgdaSymbol{\}} \AgdaSymbol{→}\<%
\\
\>[0]\AgdaIndent{6}{}\<[6]%
\>[6]\AgdaDatatype{Term} \AgdaSymbol{\{}\AgdaBound{Γ}\AgdaSymbol{\}} \AgdaSymbol{(}\AgdaBound{T} \AgdaInductiveConstructor{‘’} \AgdaSymbol{(}\AgdaInductiveConstructor{SW} \AgdaSymbol{(}\AgdaBound{a} \AgdaFunction{‘’ᵗ} \AgdaBound{b}\AgdaSymbol{))} \AgdaFunction{‘→'’} \AgdaBound{X}\AgdaSymbol{)}\<%
\\
\>[0]\AgdaIndent{6}{}\<[6]%
\>[6]\AgdaSymbol{→} \AgdaDatatype{Term} \AgdaSymbol{\{}\AgdaBound{Γ}\AgdaSymbol{\}} \AgdaSymbol{(}\AgdaInductiveConstructor{W₁} \AgdaBound{T} \AgdaInductiveConstructor{‘’} \AgdaBound{a} \AgdaInductiveConstructor{‘’} \AgdaBound{b} \AgdaFunction{‘→'’} \AgdaBound{X}\AgdaSymbol{)}\<%
\\
\>[0]\AgdaIndent{2}{}\<[2]%
\>[2]\AgdaFunction{SSW₁⁻¹-arr} \AgdaBound{x} \AgdaSymbol{=} \AgdaInductiveConstructor{‘λ’} \AgdaSymbol{(}\AgdaInductiveConstructor{w→} \AgdaBound{x} \AgdaFunction{‘'’ₐ} \AgdaInductiveConstructor{WSSW₁} \AgdaInductiveConstructor{‘VAR₀’}\AgdaSymbol{)}\<%
\\
%
\\
\>[0]\AgdaIndent{2}{}\<[2]%
\>[2]\AgdaFunction{SSW₁'→} \AgdaSymbol{=} \AgdaFunction{SSW₁⁻¹-arr}\<%
\\
%
\\
\>[0]\AgdaIndent{2}{}\<[2]%
\>[2]\AgdaFunction{SSW₁-arr⁻¹} \AgdaSymbol{:} \AgdaSymbol{∀} \AgdaSymbol{\{}\AgdaBound{Γ} \AgdaBound{B} \AgdaBound{A}\AgdaSymbol{\}}\<%
\\
\>[2]\AgdaIndent{11}{}\<[11]%
\>[11]\AgdaSymbol{\{}\AgdaBound{b} \AgdaSymbol{:} \AgdaDatatype{Term} \AgdaSymbol{\{}\AgdaBound{Γ}\AgdaSymbol{\}} \AgdaBound{B}\AgdaSymbol{\}}\<%
\\
\>[2]\AgdaIndent{11}{}\<[11]%
\>[11]\AgdaSymbol{\{}\AgdaBound{a} \AgdaSymbol{:} \AgdaDatatype{Term} \AgdaSymbol{\{}\AgdaBound{Γ} \AgdaInductiveConstructor{▻} \AgdaBound{B}\AgdaSymbol{\}} \AgdaSymbol{(}\AgdaInductiveConstructor{W} \AgdaBound{A}\AgdaSymbol{)\}}\<%
\\
\>[2]\AgdaIndent{11}{}\<[11]%
\>[11]\AgdaSymbol{\{}\AgdaBound{T} \AgdaSymbol{:} \AgdaDatatype{Type} \AgdaSymbol{(}\AgdaBound{Γ} \AgdaInductiveConstructor{▻} \AgdaBound{A}\AgdaSymbol{)\}}\<%
\\
\>[2]\AgdaIndent{11}{}\<[11]%
\>[11]\AgdaSymbol{\{}\AgdaBound{X}\AgdaSymbol{\}} \AgdaSymbol{→}\<%
\\
\>[0]\AgdaIndent{6}{}\<[6]%
\>[6]\AgdaDatatype{Term} \AgdaSymbol{\{}\AgdaBound{Γ}\AgdaSymbol{\}} \AgdaSymbol{(}\AgdaBound{X} \AgdaFunction{‘→'’} \AgdaBound{T} \AgdaInductiveConstructor{‘’} \AgdaSymbol{(}\AgdaInductiveConstructor{SW} \AgdaSymbol{(}\AgdaBound{a} \AgdaFunction{‘’ᵗ} \AgdaBound{b}\AgdaSymbol{)))}\<%
\\
\>[0]\AgdaIndent{6}{}\<[6]%
\>[6]\AgdaSymbol{→} \AgdaDatatype{Term} \AgdaSymbol{\{}\AgdaBound{Γ}\AgdaSymbol{\}} \AgdaSymbol{(}\AgdaBound{X} \AgdaFunction{‘→'’} \AgdaInductiveConstructor{W₁} \AgdaBound{T} \AgdaInductiveConstructor{‘’} \AgdaBound{a} \AgdaInductiveConstructor{‘’} \AgdaBound{b}\AgdaSymbol{)}\<%
\\
\>[0]\AgdaIndent{2}{}\<[2]%
\>[2]\AgdaFunction{SSW₁-arr⁻¹} \AgdaBound{x} \AgdaSymbol{=} \AgdaInductiveConstructor{‘λ’} \AgdaSymbol{(}\AgdaFunction{WSSW₁'} \AgdaSymbol{(}\AgdaFunction{un‘λ’} \AgdaBound{x}\AgdaSymbol{))}\<%
\\
%
\\
\>[0]\AgdaIndent{2}{}\<[2]%
\>[2]\AgdaFunction{SSW₁'←} \AgdaSymbol{=} \AgdaFunction{SSW₁-arr⁻¹}\<%
\\
%
\\
%
\\
\>[0]\AgdaIndent{2}{}\<[2]%
\>[2]\AgdaFunction{SSW₁} \AgdaSymbol{:} \AgdaSymbol{∀} \AgdaSymbol{\{}\AgdaBound{Γ} \AgdaBound{B} \AgdaBound{A}\AgdaSymbol{\}}\<%
\\
\>[2]\AgdaIndent{11}{}\<[11]%
\>[11]\AgdaSymbol{\{}\AgdaBound{b} \AgdaSymbol{:} \AgdaDatatype{Term} \AgdaSymbol{\{}\AgdaBound{Γ}\AgdaSymbol{\}} \AgdaBound{B}\AgdaSymbol{\}}\<%
\\
\>[2]\AgdaIndent{11}{}\<[11]%
\>[11]\AgdaSymbol{\{}\AgdaBound{a} \AgdaSymbol{:} \AgdaDatatype{Term} \AgdaSymbol{\{}\AgdaBound{Γ} \AgdaInductiveConstructor{▻} \AgdaBound{B}\AgdaSymbol{\}} \AgdaSymbol{(}\AgdaInductiveConstructor{W} \AgdaBound{A}\AgdaSymbol{)\}}\<%
\\
\>[2]\AgdaIndent{11}{}\<[11]%
\>[11]\AgdaSymbol{\{}\AgdaBound{T} \AgdaSymbol{:} \AgdaDatatype{Type} \AgdaSymbol{(}\AgdaBound{Γ} \AgdaInductiveConstructor{▻} \AgdaBound{A}\AgdaSymbol{)\}} \AgdaSymbol{→}\<%
\\
\>[0]\AgdaIndent{6}{}\<[6]%
\>[6]\AgdaDatatype{Term} \AgdaSymbol{\{}\AgdaBound{Γ}\AgdaSymbol{\}} \AgdaSymbol{(}\AgdaInductiveConstructor{W₁} \AgdaBound{T} \AgdaInductiveConstructor{‘’} \AgdaBound{a} \AgdaInductiveConstructor{‘’} \AgdaBound{b}\AgdaSymbol{)}\<%
\\
\>[0]\AgdaIndent{6}{}\<[6]%
\>[6]\AgdaSymbol{→} \AgdaDatatype{Term} \AgdaSymbol{\{}\AgdaBound{Γ}\AgdaSymbol{\}} \AgdaSymbol{(}\AgdaBound{T} \AgdaInductiveConstructor{‘’} \AgdaSymbol{(}\AgdaInductiveConstructor{SW} \AgdaSymbol{(}\AgdaBound{a} \AgdaFunction{‘’ᵗ} \AgdaBound{b}\AgdaSymbol{)))}\<%
\\
\>[0]\AgdaIndent{2}{}\<[2]%
\>[2]\AgdaFunction{SSW₁} \AgdaBound{x} \AgdaSymbol{=} \AgdaSymbol{(}\AgdaInductiveConstructor{SW} \AgdaSymbol{(}\AgdaInductiveConstructor{WSSW₁} \AgdaSymbol{(}\AgdaInductiveConstructor{w} \AgdaBound{x}\AgdaSymbol{)} \AgdaFunction{‘’ᵗ} \AgdaBound{x}\AgdaSymbol{))}\<%
\\
%
\\
\>[0]\AgdaIndent{2}{}\<[2]%
\>[2]\AgdaFunction{S₁₀W₂W} \AgdaSymbol{:} \AgdaSymbol{∀} \AgdaSymbol{\{}\AgdaBound{Γ} \AgdaBound{T'} \AgdaBound{A} \AgdaBound{B} \AgdaBound{T}\AgdaSymbol{\}} \AgdaSymbol{\{}\AgdaBound{a} \AgdaSymbol{:} \AgdaDatatype{Term} \AgdaSymbol{\{}\AgdaBound{Γ} \AgdaInductiveConstructor{▻} \AgdaBound{T'}\AgdaSymbol{\}} \AgdaSymbol{(}\AgdaInductiveConstructor{W} \AgdaBound{A}\AgdaSymbol{)\}} \AgdaSymbol{\{}\AgdaBound{b} \AgdaSymbol{:} \AgdaDatatype{Term} \AgdaSymbol{\{}\AgdaBound{Γ} \AgdaInductiveConstructor{▻} \AgdaBound{T'}\AgdaSymbol{\}} \AgdaSymbol{(}\AgdaInductiveConstructor{W₁} \AgdaBound{B} \AgdaInductiveConstructor{‘’} \AgdaBound{a}\AgdaSymbol{)\}}\<%
\\
\>[2]\AgdaIndent{8}{}\<[8]%
\>[8]\AgdaSymbol{→} \AgdaDatatype{Term} \AgdaSymbol{\{}\AgdaBound{Γ} \AgdaInductiveConstructor{▻} \AgdaBound{T'}\AgdaSymbol{\}} \AgdaSymbol{(}\AgdaInductiveConstructor{W₂} \AgdaSymbol{(}\AgdaInductiveConstructor{W} \AgdaBound{T}\AgdaSymbol{)} \AgdaInductiveConstructor{‘’₁} \AgdaBound{a} \AgdaInductiveConstructor{‘’} \AgdaBound{b}\AgdaSymbol{)}\<%
\\
\>[2]\AgdaIndent{8}{}\<[8]%
\>[8]\AgdaSymbol{→} \AgdaDatatype{Term} \AgdaSymbol{\{}\AgdaBound{Γ} \AgdaInductiveConstructor{▻} \AgdaBound{T'}\AgdaSymbol{\}} \AgdaSymbol{(}\AgdaInductiveConstructor{W₁} \AgdaBound{T} \AgdaInductiveConstructor{‘’} \AgdaBound{a}\AgdaSymbol{)}\<%
\\
\>[0]\AgdaIndent{2}{}\<[2]%
\>[2]\AgdaFunction{S₁₀W₂W} \AgdaSymbol{=} \AgdaInductiveConstructor{S₁₀W₂₀}\<%
\end{code}

\begin{code}%
\> \AgdaKeyword{module} \AgdaModule{well-typed-syntax-context-helpers} \AgdaKeyword{where}\<%
\\
\>[0]\AgdaIndent{2}{}\<[2]%
\>[2]\AgdaKeyword{open} \AgdaModule{well-typed-syntax}\<%
\\
\>[0]\AgdaIndent{2}{}\<[2]%
\>[2]\AgdaKeyword{open} \AgdaModule{well-typed-syntax-helpers}\<%
\\
%
\\
\>[0]\AgdaIndent{2}{}\<[2]%
\>[2]\AgdaFunction{□\_} \AgdaSymbol{:} \AgdaDatatype{Type} \AgdaInductiveConstructor{ε} \AgdaSymbol{→} \AgdaPrimitiveType{Set}\<%
\\
\>[0]\AgdaIndent{2}{}\<[2]%
\>[2]\AgdaFunction{□\_} \AgdaBound{T} \AgdaSymbol{=} \AgdaDatatype{Term} \AgdaSymbol{\{}\AgdaInductiveConstructor{ε}\AgdaSymbol{\}} \AgdaBound{T}\<%
\end{code}

\begin{code}%
\> \AgdaKeyword{module} \AgdaModule{well-typed-quoted-syntax-defs} \AgdaKeyword{where}\<%
\\
\>[0]\AgdaIndent{2}{}\<[2]%
\>[2]\AgdaKeyword{open} \AgdaModule{well-typed-syntax}\<%
\\
\>[0]\AgdaIndent{2}{}\<[2]%
\>[2]\AgdaKeyword{open} \AgdaModule{well-typed-syntax-helpers}\<%
\\
\>[0]\AgdaIndent{2}{}\<[2]%
\>[2]\AgdaKeyword{open} \AgdaModule{well-typed-syntax-context-helpers}\<%
\\
%
\\
\>[0]\AgdaIndent{2}{}\<[2]%
\>[2]\AgdaFunction{‘ε’} \AgdaSymbol{:} \AgdaDatatype{Term} \AgdaSymbol{\{}\AgdaInductiveConstructor{ε}\AgdaSymbol{\}} \AgdaInductiveConstructor{‘Context’}\<%
\\
\>[0]\AgdaIndent{2}{}\<[2]%
\>[2]\AgdaFunction{‘ε’} \AgdaSymbol{=} \AgdaInductiveConstructor{⌜} \AgdaInductiveConstructor{ε} \AgdaInductiveConstructor{⌝ᶜ}\<%
\\
%
\\
\>[0]\AgdaIndent{2}{}\<[2]%
\>[2]\AgdaFunction{‘□’} \AgdaSymbol{:} \AgdaDatatype{Type} \AgdaSymbol{(}\AgdaInductiveConstructor{ε} \AgdaInductiveConstructor{▻} \AgdaInductiveConstructor{‘Type’} \AgdaInductiveConstructor{‘’} \AgdaFunction{‘ε’}\AgdaSymbol{)}\<%
\\
\>[0]\AgdaIndent{2}{}\<[2]%
\>[2]\AgdaFunction{‘□’} \AgdaSymbol{=} \AgdaInductiveConstructor{‘Term’} \AgdaInductiveConstructor{‘’₁} \AgdaFunction{‘ε’}\<%
\\
\>\<%
\end{code}

\begin{code}%
\> \AgdaKeyword{module} \AgdaModule{well-typed-syntax-eq-dec} \AgdaKeyword{where}\<%
\\
\>[0]\AgdaIndent{2}{}\<[2]%
\>[2]\AgdaKeyword{open} \AgdaModule{well-typed-syntax}\<%
\\
%
\\
\>[0]\AgdaIndent{2}{}\<[2]%
\>[2]\AgdaFunction{context-pick-if} \AgdaSymbol{:} \AgdaSymbol{∀} \AgdaSymbol{\{}\AgdaBound{ℓ}\AgdaSymbol{\}} \AgdaSymbol{\{}\AgdaBound{P} \AgdaSymbol{:} \AgdaDatatype{Context} \AgdaSymbol{→} \AgdaPrimitiveType{Set} \AgdaBound{ℓ}\AgdaSymbol{\}}\<%
\\
\>[2]\AgdaIndent{26}{}\<[26]%
\>[26]\AgdaSymbol{\{}\AgdaBound{Γ} \AgdaSymbol{:} \AgdaDatatype{Context}\AgdaSymbol{\}}\<%
\\
\>[2]\AgdaIndent{26}{}\<[26]%
\>[26]\AgdaSymbol{(}\AgdaBound{dummy} \AgdaSymbol{:} \AgdaBound{P} \AgdaSymbol{(}\AgdaInductiveConstructor{ε} \AgdaInductiveConstructor{▻} \AgdaInductiveConstructor{‘Σ’} \AgdaInductiveConstructor{‘Context’} \AgdaInductiveConstructor{‘Type’}\AgdaSymbol{))}\<%
\\
\>[2]\AgdaIndent{26}{}\<[26]%
\>[26]\AgdaSymbol{(}\AgdaBound{val} \AgdaSymbol{:} \AgdaBound{P} \AgdaBound{Γ}\AgdaSymbol{)} \AgdaSymbol{→}\<%
\\
\>[0]\AgdaIndent{24}{}\<[24]%
\>[24]\AgdaBound{P} \AgdaSymbol{(}\AgdaInductiveConstructor{ε} \AgdaInductiveConstructor{▻} \AgdaInductiveConstructor{‘Σ’} \AgdaInductiveConstructor{‘Context’} \AgdaInductiveConstructor{‘Type’}\AgdaSymbol{)}\<%
\\
\>[0]\AgdaIndent{2}{}\<[2]%
\>[2]\AgdaFunction{context-pick-if} \AgdaSymbol{\{}\AgdaArgument{P} \AgdaSymbol{=} \AgdaBound{P}\AgdaSymbol{\}} \AgdaSymbol{\{}\AgdaInductiveConstructor{ε} \AgdaInductiveConstructor{▻} \AgdaInductiveConstructor{‘Σ’} \AgdaInductiveConstructor{‘Context’} \AgdaInductiveConstructor{‘Type’}\AgdaSymbol{\}} \AgdaBound{dummy} \AgdaBound{val} \AgdaSymbol{=} \AgdaBound{val}\<%
\\
\>[0]\AgdaIndent{2}{}\<[2]%
\>[2]\AgdaFunction{context-pick-if} \AgdaSymbol{\{}\AgdaArgument{P} \AgdaSymbol{=} \AgdaBound{P}\AgdaSymbol{\}} \AgdaSymbol{\{}\AgdaBound{Γ}\AgdaSymbol{\}} \AgdaBound{dummy} \AgdaBound{val} \AgdaSymbol{=} \AgdaBound{dummy}\<%
\\
%
\\
\>[0]\AgdaIndent{2}{}\<[2]%
\>[2]\AgdaFunction{context-pick-if-refl} \AgdaSymbol{:} \AgdaSymbol{∀} \AgdaSymbol{\{}\AgdaBound{ℓ} \AgdaBound{P} \AgdaBound{dummy} \AgdaBound{val}\AgdaSymbol{\}} \AgdaSymbol{→}\<%
\\
\>[2]\AgdaIndent{29}{}\<[29]%
\>[29]\AgdaFunction{context-pick-if} \AgdaSymbol{\{}\AgdaBound{ℓ}\AgdaSymbol{\}} \AgdaSymbol{\{}\AgdaBound{P}\AgdaSymbol{\}} \AgdaSymbol{\{}\AgdaInductiveConstructor{ε} \AgdaInductiveConstructor{▻} \AgdaInductiveConstructor{‘Σ’} \AgdaInductiveConstructor{‘Context’} \AgdaInductiveConstructor{‘Type’}\AgdaSymbol{\}} \AgdaBound{dummy} \AgdaBound{val} \AgdaDatatype{≡} \AgdaBound{val}\<%
\\
\>[0]\AgdaIndent{2}{}\<[2]%
\>[2]\AgdaFunction{context-pick-if-refl} \AgdaSymbol{\{}\AgdaArgument{P} \AgdaSymbol{=} \AgdaBound{P}\AgdaSymbol{\}} \AgdaSymbol{=} \AgdaInductiveConstructor{refl}\<%
\\
\>\<%
\end{code}

\begin{code}%
\> \AgdaKeyword{module} \AgdaModule{well-typed-quoted-syntax} \AgdaKeyword{where}\<%
\\
\>[0]\AgdaIndent{2}{}\<[2]%
\>[2]\AgdaKeyword{open} \AgdaModule{well-typed-syntax}\<%
\\
\>[0]\AgdaIndent{2}{}\<[2]%
\>[2]\AgdaKeyword{open} \AgdaModule{well-typed-syntax-helpers} \AgdaKeyword{public}\<%
\\
\>[0]\AgdaIndent{2}{}\<[2]%
\>[2]\AgdaKeyword{open} \AgdaModule{well-typed-quoted-syntax-defs} \AgdaKeyword{public}\<%
\\
\>[0]\AgdaIndent{2}{}\<[2]%
\>[2]\AgdaKeyword{open} \AgdaModule{well-typed-syntax-context-helpers} \AgdaKeyword{public}\<%
\\
\>[0]\AgdaIndent{2}{}\<[2]%
\>[2]\AgdaKeyword{open} \AgdaModule{well-typed-syntax-eq-dec} \AgdaKeyword{public}\<%
\\
%
\\
\>[0]\AgdaIndent{2}{}\<[2]%
\>[2]\AgdaKeyword{infixr} \AgdaNumber{2} \AgdaFixityOp{\_‘‘∘’’\_}\<%
\\
%
\\
\>[0]\AgdaIndent{2}{}\<[2]%
\>[2]\AgdaFunction{quote-Σ} \AgdaSymbol{:} \AgdaSymbol{(}\AgdaBound{Γv} \AgdaSymbol{:} \AgdaRecord{Σ} \AgdaDatatype{Context} \AgdaDatatype{Type}\AgdaSymbol{)} \AgdaSymbol{→} \AgdaDatatype{Term} \AgdaSymbol{\{}\AgdaInductiveConstructor{ε}\AgdaSymbol{\}} \AgdaSymbol{(}\AgdaInductiveConstructor{‘Σ’} \AgdaInductiveConstructor{‘Context’} \AgdaInductiveConstructor{‘Type’}\AgdaSymbol{)}\<%
\\
\>[0]\AgdaIndent{2}{}\<[2]%
\>[2]\AgdaFunction{quote-Σ} \AgdaSymbol{(}\AgdaBound{Γ} \AgdaInductiveConstructor{,} \AgdaBound{v}\AgdaSymbol{)} \AgdaSymbol{=} \AgdaInductiveConstructor{\_‘,’\_} \AgdaInductiveConstructor{⌜} \AgdaBound{Γ} \AgdaInductiveConstructor{⌝ᶜ} \AgdaInductiveConstructor{⌜} \AgdaBound{v} \AgdaInductiveConstructor{⌝ᵀ}\<%
\\
%
\\
\>[0]\AgdaIndent{2}{}\<[2]%
\>[2]\AgdaFunction{\_‘‘∘’’\_} \AgdaSymbol{:} \AgdaSymbol{∀} \AgdaSymbol{\{}\AgdaBound{A} \AgdaBound{B} \AgdaBound{C}\AgdaSymbol{\}}\<%
\\
\>[2]\AgdaIndent{6}{}\<[6]%
\>[6]\AgdaSymbol{→} \AgdaFunction{□} \AgdaSymbol{(}\AgdaFunction{‘□’} \AgdaInductiveConstructor{‘’} \AgdaSymbol{(}\AgdaBound{C} \AgdaInductiveConstructor{‘‘→'’’} \AgdaBound{B}\AgdaSymbol{))}\<%
\\
\>[2]\AgdaIndent{6}{}\<[6]%
\>[6]\AgdaSymbol{→} \AgdaFunction{□} \AgdaSymbol{(}\AgdaFunction{‘□’} \AgdaInductiveConstructor{‘’} \AgdaSymbol{(}\AgdaBound{A} \AgdaInductiveConstructor{‘‘→'’’} \AgdaBound{C}\AgdaSymbol{))}\<%
\\
\>[2]\AgdaIndent{6}{}\<[6]%
\>[6]\AgdaSymbol{→} \AgdaFunction{□} \AgdaSymbol{(}\AgdaFunction{‘□’} \AgdaInductiveConstructor{‘’} \AgdaSymbol{(}\AgdaBound{A} \AgdaInductiveConstructor{‘‘→'’’} \AgdaBound{B}\AgdaSymbol{))}\<%
\\
\>[0]\AgdaIndent{2}{}\<[2]%
\>[2]\AgdaBound{g} \AgdaFunction{‘‘∘’’} \AgdaBound{f} \AgdaSymbol{=} \AgdaSymbol{(}\AgdaInductiveConstructor{‘‘∘-nd’’} \AgdaFunction{‘'’ₐ} \AgdaBound{f} \AgdaFunction{‘'’ₐ} \AgdaBound{g}\AgdaSymbol{)}\<%
\\
%
\\
\>[0]\AgdaIndent{2}{}\<[2]%
\>[2]\AgdaFunction{Conv0} \AgdaSymbol{:} \AgdaSymbol{∀} \AgdaSymbol{\{}\AgdaBound{qH0} \AgdaBound{qX}\AgdaSymbol{\}} \AgdaSymbol{→}\<%
\\
\>[2]\AgdaIndent{6}{}\<[6]%
\>[6]\AgdaDatatype{Term} \AgdaSymbol{\{(}\AgdaInductiveConstructor{ε} \AgdaInductiveConstructor{▻} \AgdaFunction{‘□’} \AgdaInductiveConstructor{‘’} \AgdaBound{qH0}\AgdaSymbol{)\}}\<%
\\
\>[6]\AgdaIndent{12}{}\<[12]%
\>[12]\AgdaSymbol{(}\AgdaInductiveConstructor{W} \AgdaSymbol{(}\AgdaFunction{‘□’} \AgdaInductiveConstructor{‘’} \AgdaInductiveConstructor{⌜} \AgdaFunction{‘□’} \AgdaInductiveConstructor{‘’} \AgdaBound{qH0} \AgdaFunction{‘→'’} \AgdaBound{qX} \AgdaInductiveConstructor{⌝ᵀ}\AgdaSymbol{))}\<%
\\
\>[0]\AgdaIndent{6}{}\<[6]%
\>[6]\AgdaSymbol{→} \AgdaDatatype{Term} \AgdaSymbol{\{(}\AgdaInductiveConstructor{ε} \AgdaInductiveConstructor{▻} \AgdaFunction{‘□’} \AgdaInductiveConstructor{‘’} \AgdaBound{qH0}\AgdaSymbol{)\}}\<%
\\
\>[6]\AgdaIndent{15}{}\<[15]%
\>[15]\AgdaSymbol{(}\AgdaInductiveConstructor{W}\<%
\\
\>[15]\AgdaIndent{18}{}\<[18]%
\>[18]\AgdaSymbol{(}\AgdaFunction{‘□’} \AgdaInductiveConstructor{‘’} \AgdaSymbol{(}\AgdaInductiveConstructor{⌜} \AgdaFunction{‘□’} \AgdaInductiveConstructor{‘’} \AgdaBound{qH0} \AgdaInductiveConstructor{⌝ᵀ} \AgdaInductiveConstructor{‘‘→'’’} \AgdaInductiveConstructor{⌜} \AgdaBound{qX} \AgdaInductiveConstructor{⌝ᵀ}\AgdaSymbol{)))}\<%
\\
\>[0]\AgdaIndent{2}{}\<[2]%
\>[2]\AgdaFunction{Conv0} \AgdaSymbol{\{}\AgdaBound{qH0}\AgdaSymbol{\}} \AgdaSymbol{\{}\AgdaBound{qX}\AgdaSymbol{\}} \AgdaBound{x} \AgdaSymbol{=} \AgdaInductiveConstructor{w→} \AgdaInductiveConstructor{⌜→'⌝} \AgdaFunction{‘'’ₐ} \AgdaBound{x}\<%
\\
\>\<%
\end{code}

\begin{code}%
\> \AgdaKeyword{module} \AgdaModule{well-typed-syntax-pre-interpreter} \AgdaKeyword{where}\<%
\\
\>[0]\AgdaIndent{2}{}\<[2]%
\>[2]\AgdaKeyword{open} \AgdaModule{well-typed-syntax}\<%
\\
\>[0]\AgdaIndent{2}{}\<[2]%
\>[2]\AgdaKeyword{open} \AgdaModule{well-typed-syntax-helpers}\<%
\\
%
\\
\>[0]\AgdaIndent{2}{}\<[2]%
\>[2]\AgdaFunction{max-level} \AgdaSymbol{:} \AgdaPostulate{Level}\<%
\\
\>[0]\AgdaIndent{2}{}\<[2]%
\>[2]\AgdaFunction{max-level} \AgdaSymbol{=} \AgdaPrimitive{lsuc} \AgdaPrimitive{lzero}\<%
\\
%
\\
\>[0]\AgdaIndent{2}{}\<[2]%
\>[2]\AgdaKeyword{module} \AgdaModule{inner}\<%
\\
\>[2]\AgdaIndent{4}{}\<[4]%
\>[4]\AgdaSymbol{(}\AgdaBound{context-pick-if'} \AgdaSymbol{:} \AgdaSymbol{∀} \AgdaBound{ℓ} \AgdaSymbol{(}\AgdaBound{P} \AgdaSymbol{:} \AgdaDatatype{Context} \AgdaSymbol{→} \AgdaPrimitiveType{Set} \AgdaBound{ℓ}\AgdaSymbol{)}\<%
\\
\>[4]\AgdaIndent{28}{}\<[28]%
\>[28]\AgdaSymbol{(}\AgdaBound{Γ} \AgdaSymbol{:} \AgdaDatatype{Context}\AgdaSymbol{)}\<%
\\
\>[4]\AgdaIndent{28}{}\<[28]%
\>[28]\AgdaSymbol{(}\AgdaBound{dummy} \AgdaSymbol{:} \AgdaBound{P} \AgdaSymbol{(}\AgdaInductiveConstructor{ε} \AgdaInductiveConstructor{▻} \AgdaInductiveConstructor{‘Σ’} \AgdaInductiveConstructor{‘Context’} \AgdaInductiveConstructor{‘Type’}\AgdaSymbol{))}\<%
\\
\>[4]\AgdaIndent{28}{}\<[28]%
\>[28]\AgdaSymbol{(}\AgdaBound{val} \AgdaSymbol{:} \AgdaBound{P} \AgdaBound{Γ}\AgdaSymbol{)} \AgdaSymbol{→}\<%
\\
\>[0]\AgdaIndent{26}{}\<[26]%
\>[26]\AgdaBound{P} \AgdaSymbol{(}\AgdaInductiveConstructor{ε} \AgdaInductiveConstructor{▻} \AgdaInductiveConstructor{‘Σ’} \AgdaInductiveConstructor{‘Context’} \AgdaInductiveConstructor{‘Type’}\AgdaSymbol{))}\<%
\\
\>[0]\AgdaIndent{4}{}\<[4]%
\>[4]\AgdaSymbol{(}\AgdaBound{context-pick-if-refl'} \AgdaSymbol{:} \AgdaSymbol{∀} \AgdaBound{ℓ} \AgdaBound{P} \AgdaBound{dummy} \AgdaBound{val} \AgdaSymbol{→}\<%
\\
\>[4]\AgdaIndent{30}{}\<[30]%
\>[30]\AgdaBound{context-pick-if'} \AgdaBound{ℓ} \AgdaBound{P} \AgdaSymbol{(}\AgdaInductiveConstructor{ε} \AgdaInductiveConstructor{▻} \AgdaInductiveConstructor{‘Σ’} \AgdaInductiveConstructor{‘Context’} \AgdaInductiveConstructor{‘Type’}\AgdaSymbol{)} \AgdaBound{dummy} \AgdaBound{val} \AgdaDatatype{≡} \AgdaBound{val}\AgdaSymbol{)}\<%
\\
\>[0]\AgdaIndent{4}{}\<[4]%
\>[4]\AgdaKeyword{where}\<%
\\
%
\\
\>[0]\AgdaIndent{4}{}\<[4]%
\>[4]\AgdaFunction{context-pick-if} \AgdaSymbol{:} \AgdaSymbol{∀} \AgdaSymbol{\{}\AgdaBound{ℓ}\AgdaSymbol{\}} \AgdaSymbol{\{}\AgdaBound{P} \AgdaSymbol{:} \AgdaDatatype{Context} \AgdaSymbol{→} \AgdaPrimitiveType{Set} \AgdaBound{ℓ}\AgdaSymbol{\}}\<%
\\
\>[4]\AgdaIndent{28}{}\<[28]%
\>[28]\AgdaSymbol{\{}\AgdaBound{Γ} \AgdaSymbol{:} \AgdaDatatype{Context}\AgdaSymbol{\}}\<%
\\
\>[4]\AgdaIndent{28}{}\<[28]%
\>[28]\AgdaSymbol{(}\AgdaBound{dummy} \AgdaSymbol{:} \AgdaBound{P} \AgdaSymbol{(}\AgdaInductiveConstructor{ε} \AgdaInductiveConstructor{▻} \AgdaInductiveConstructor{‘Σ’} \AgdaInductiveConstructor{‘Context’} \AgdaInductiveConstructor{‘Type’}\AgdaSymbol{))}\<%
\\
\>[4]\AgdaIndent{28}{}\<[28]%
\>[28]\AgdaSymbol{(}\AgdaBound{val} \AgdaSymbol{:} \AgdaBound{P} \AgdaBound{Γ}\AgdaSymbol{)} \AgdaSymbol{→}\<%
\\
\>[0]\AgdaIndent{26}{}\<[26]%
\>[26]\AgdaBound{P} \AgdaSymbol{(}\AgdaInductiveConstructor{ε} \AgdaInductiveConstructor{▻} \AgdaInductiveConstructor{‘Σ’} \AgdaInductiveConstructor{‘Context’} \AgdaInductiveConstructor{‘Type’}\AgdaSymbol{)}\<%
\\
\>[0]\AgdaIndent{4}{}\<[4]%
\>[4]\AgdaFunction{context-pick-if} \AgdaSymbol{\{}\AgdaArgument{P} \AgdaSymbol{=} \AgdaBound{P}\AgdaSymbol{\}} \AgdaBound{dummy} \AgdaBound{val} \AgdaSymbol{=} \AgdaBound{context-pick-if'} \AgdaSymbol{\_} \AgdaBound{P} \AgdaSymbol{\_} \AgdaBound{dummy} \AgdaBound{val}\<%
\\
\>[0]\AgdaIndent{4}{}\<[4]%
\>[4]\AgdaFunction{context-pick-if-refl} \AgdaSymbol{:} \AgdaSymbol{∀} \AgdaSymbol{\{}\AgdaBound{ℓ} \AgdaBound{P} \AgdaBound{dummy} \AgdaBound{val}\AgdaSymbol{\}} \AgdaSymbol{→}\<%
\\
\>[4]\AgdaIndent{30}{}\<[30]%
\>[30]\AgdaFunction{context-pick-if} \AgdaSymbol{\{}\AgdaBound{ℓ}\AgdaSymbol{\}} \AgdaSymbol{\{}\AgdaBound{P}\AgdaSymbol{\}} \AgdaSymbol{\{}\AgdaInductiveConstructor{ε} \AgdaInductiveConstructor{▻} \AgdaInductiveConstructor{‘Σ’} \AgdaInductiveConstructor{‘Context’} \AgdaInductiveConstructor{‘Type’}\AgdaSymbol{\}} \AgdaBound{dummy} \AgdaBound{val} \AgdaDatatype{≡} \AgdaBound{val}\<%
\\
\>[0]\AgdaIndent{4}{}\<[4]%
\>[4]\AgdaFunction{context-pick-if-refl} \AgdaSymbol{\{}\AgdaArgument{P} \AgdaSymbol{=} \AgdaBound{P}\AgdaSymbol{\}} \AgdaSymbol{=} \AgdaBound{context-pick-if-refl'} \AgdaSymbol{\_} \AgdaBound{P} \AgdaSymbol{\_} \AgdaSymbol{\_}\<%
\\
%
\\
\>[0]\AgdaIndent{4}{}\<[4]%
\>[4]\AgdaKeyword{private}\<%
\\
\>[4]\AgdaIndent{6}{}\<[6]%
\>[6]\AgdaFunction{dummy} \AgdaSymbol{:} \AgdaDatatype{Type} \AgdaInductiveConstructor{ε}\<%
\\
\>[4]\AgdaIndent{6}{}\<[6]%
\>[6]\AgdaFunction{dummy} \AgdaSymbol{=} \AgdaInductiveConstructor{‘Context’}\<%
\\
%
\\
\>[0]\AgdaIndent{4}{}\<[4]%
\>[4]\AgdaFunction{cast-helper} \AgdaSymbol{:} \AgdaSymbol{∀} \AgdaSymbol{\{}\AgdaBound{X} \AgdaBound{T} \AgdaBound{A}\AgdaSymbol{\}} \AgdaSymbol{\{}\AgdaBound{x} \AgdaSymbol{:} \AgdaDatatype{Term} \AgdaBound{X}\AgdaSymbol{\}} \AgdaSymbol{→} \AgdaBound{A} \AgdaDatatype{≡} \AgdaBound{T} \AgdaSymbol{→} \AgdaDatatype{Term} \AgdaSymbol{\{}\AgdaInductiveConstructor{ε}\AgdaSymbol{\}} \AgdaSymbol{(}\AgdaBound{T} \AgdaInductiveConstructor{‘’} \AgdaBound{x} \AgdaFunction{‘→'’} \AgdaBound{A} \AgdaInductiveConstructor{‘’} \AgdaBound{x}\AgdaSymbol{)}\<%
\\
\>[0]\AgdaIndent{4}{}\<[4]%
\>[4]\AgdaFunction{cast-helper} \AgdaInductiveConstructor{refl} \AgdaSymbol{=} \AgdaInductiveConstructor{‘λ’} \AgdaInductiveConstructor{‘VAR₀’}\<%
\\
%
\\
\>[0]\AgdaIndent{4}{}\<[4]%
\>[4]\AgdaFunction{cast'-proof} \AgdaSymbol{:} \AgdaSymbol{∀} \AgdaSymbol{\{}\AgdaBound{T}\AgdaSymbol{\}} \AgdaSymbol{→} \AgdaDatatype{Term} \AgdaSymbol{\{}\AgdaInductiveConstructor{ε}\AgdaSymbol{\}} \AgdaSymbol{(}\AgdaFunction{context-pick-if} \AgdaSymbol{\{}\AgdaArgument{P} \AgdaSymbol{=} \AgdaDatatype{Type}\AgdaSymbol{\}} \AgdaSymbol{(}\AgdaInductiveConstructor{W} \AgdaFunction{dummy}\AgdaSymbol{)} \AgdaBound{T} \AgdaInductiveConstructor{‘’} \AgdaInductiveConstructor{\_‘,’\_} \AgdaInductiveConstructor{⌜} \AgdaInductiveConstructor{ε} \AgdaInductiveConstructor{▻} \AgdaInductiveConstructor{‘Σ’} \AgdaInductiveConstructor{‘Context’} \AgdaInductiveConstructor{‘Type’} \AgdaInductiveConstructor{⌝ᶜ} \AgdaInductiveConstructor{⌜} \AgdaBound{T} \AgdaInductiveConstructor{⌝ᵀ}\<%
\\
\>[4]\AgdaIndent{18}{}\<[18]%
\>[18]\AgdaFunction{‘→'’} \AgdaBound{T} \AgdaInductiveConstructor{‘’} \AgdaInductiveConstructor{\_‘,’\_} \AgdaInductiveConstructor{⌜} \AgdaInductiveConstructor{ε} \AgdaInductiveConstructor{▻} \AgdaInductiveConstructor{‘Σ’} \AgdaInductiveConstructor{‘Context’} \AgdaInductiveConstructor{‘Type’} \AgdaInductiveConstructor{⌝ᶜ} \AgdaInductiveConstructor{⌜} \AgdaBound{T} \AgdaInductiveConstructor{⌝ᵀ}\AgdaSymbol{)}\<%
\\
\>[0]\AgdaIndent{4}{}\<[4]%
\>[4]\AgdaFunction{cast'-proof} \AgdaSymbol{\{}\AgdaBound{T}\AgdaSymbol{\}} \AgdaSymbol{=} \AgdaFunction{cast-helper} \AgdaSymbol{\{}\AgdaInductiveConstructor{‘Σ’} \AgdaInductiveConstructor{‘Context’} \AgdaInductiveConstructor{‘Type’}\AgdaSymbol{\}}\<%
\\
\>[4]\AgdaIndent{24}{}\<[24]%
\>[24]\AgdaSymbol{\{}\AgdaFunction{context-pick-if} \AgdaSymbol{\{}\AgdaArgument{P} \AgdaSymbol{=} \AgdaDatatype{Type}\AgdaSymbol{\}} \AgdaSymbol{\{}\AgdaInductiveConstructor{ε} \AgdaInductiveConstructor{▻} \AgdaInductiveConstructor{‘Σ’} \AgdaInductiveConstructor{‘Context’} \AgdaInductiveConstructor{‘Type’}\AgdaSymbol{\}} \AgdaSymbol{(}\AgdaInductiveConstructor{W} \AgdaFunction{dummy}\AgdaSymbol{)} \AgdaBound{T}\AgdaSymbol{\}}\<%
\\
\>[4]\AgdaIndent{24}{}\<[24]%
\>[24]\AgdaSymbol{\{}\AgdaBound{T}\AgdaSymbol{\}} \AgdaSymbol{(}\AgdaFunction{sym} \AgdaSymbol{(}\AgdaFunction{context-pick-if-refl} \AgdaSymbol{\{}\AgdaArgument{P} \AgdaSymbol{=} \AgdaDatatype{Type}\AgdaSymbol{\}} \AgdaSymbol{\{}\AgdaArgument{dummy} \AgdaSymbol{=} \AgdaInductiveConstructor{W} \AgdaFunction{dummy}\AgdaSymbol{\}))}\<%
\\
%
\\
\>[0]\AgdaIndent{4}{}\<[4]%
\>[4]\AgdaFunction{cast-proof} \AgdaSymbol{:} \AgdaSymbol{∀} \AgdaSymbol{\{}\AgdaBound{T}\AgdaSymbol{\}} \AgdaSymbol{→} \AgdaDatatype{Term} \AgdaSymbol{\{}\AgdaInductiveConstructor{ε}\AgdaSymbol{\}} \AgdaSymbol{(}\AgdaBound{T} \AgdaInductiveConstructor{‘’} \AgdaInductiveConstructor{\_‘,’\_} \AgdaInductiveConstructor{⌜} \AgdaInductiveConstructor{ε} \AgdaInductiveConstructor{▻} \AgdaInductiveConstructor{‘Σ’} \AgdaInductiveConstructor{‘Context’} \AgdaInductiveConstructor{‘Type’} \AgdaInductiveConstructor{⌝ᶜ} \AgdaInductiveConstructor{⌜} \AgdaBound{T} \AgdaInductiveConstructor{⌝ᵀ}\<%
\\
\>[4]\AgdaIndent{18}{}\<[18]%
\>[18]\AgdaFunction{‘→'’} \AgdaFunction{context-pick-if} \AgdaSymbol{\{}\AgdaArgument{P} \AgdaSymbol{=} \AgdaDatatype{Type}\AgdaSymbol{\}} \AgdaSymbol{(}\AgdaInductiveConstructor{W} \AgdaFunction{dummy}\AgdaSymbol{)} \AgdaBound{T} \AgdaInductiveConstructor{‘’} \AgdaInductiveConstructor{\_‘,’\_} \AgdaInductiveConstructor{⌜} \AgdaInductiveConstructor{ε} \AgdaInductiveConstructor{▻} \AgdaInductiveConstructor{‘Σ’} \AgdaInductiveConstructor{‘Context’} \AgdaInductiveConstructor{‘Type’} \AgdaInductiveConstructor{⌝ᶜ} \AgdaInductiveConstructor{⌜} \AgdaBound{T} \AgdaInductiveConstructor{⌝ᵀ}\AgdaSymbol{)}\<%
\\
\>[0]\AgdaIndent{4}{}\<[4]%
\>[4]\AgdaFunction{cast-proof} \AgdaSymbol{\{}\AgdaBound{T}\AgdaSymbol{\}} \AgdaSymbol{=} \AgdaFunction{cast-helper} \AgdaSymbol{\{}\AgdaInductiveConstructor{‘Σ’} \AgdaInductiveConstructor{‘Context’} \AgdaInductiveConstructor{‘Type’}\AgdaSymbol{\}} \AgdaSymbol{\{}\AgdaBound{T}\AgdaSymbol{\}}\<%
\\
\>[4]\AgdaIndent{24}{}\<[24]%
\>[24]\AgdaSymbol{\{}\AgdaFunction{context-pick-if} \AgdaSymbol{\{}\AgdaArgument{P} \AgdaSymbol{=} \AgdaDatatype{Type}\AgdaSymbol{\}} \AgdaSymbol{\{}\AgdaInductiveConstructor{ε} \AgdaInductiveConstructor{▻} \AgdaInductiveConstructor{‘Σ’} \AgdaInductiveConstructor{‘Context’} \AgdaInductiveConstructor{‘Type’}\AgdaSymbol{\}} \AgdaSymbol{(}\AgdaInductiveConstructor{W} \AgdaFunction{dummy}\AgdaSymbol{)} \AgdaBound{T}\AgdaSymbol{\}}\<%
\\
\>[4]\AgdaIndent{24}{}\<[24]%
\>[24]\AgdaSymbol{(}\AgdaFunction{context-pick-if-refl} \AgdaSymbol{\{}\AgdaArgument{P} \AgdaSymbol{=} \AgdaDatatype{Type}\AgdaSymbol{\}} \AgdaSymbol{\{}\AgdaArgument{dummy} \AgdaSymbol{=} \AgdaInductiveConstructor{W} \AgdaFunction{dummy}\AgdaSymbol{\})}\<%
\\
%
\\
\>[0]\AgdaIndent{4}{}\<[4]%
\>[4]\AgdaFunction{‘idfun’} \AgdaSymbol{:} \AgdaSymbol{∀} \AgdaSymbol{\{}\AgdaBound{T}\AgdaSymbol{\}} \AgdaSymbol{→} \AgdaDatatype{Term} \AgdaSymbol{\{}\AgdaInductiveConstructor{ε}\AgdaSymbol{\}} \AgdaSymbol{(}\AgdaBound{T} \AgdaFunction{‘→'’} \AgdaBound{T}\AgdaSymbol{)}\<%
\\
\>[0]\AgdaIndent{4}{}\<[4]%
\>[4]\AgdaFunction{‘idfun’} \AgdaSymbol{=} \AgdaInductiveConstructor{‘λ’} \AgdaInductiveConstructor{‘VAR₀’}\<%
\\
%
\\
\>[0]\AgdaIndent{4}{}\<[4]%
\>[4]\AgdaKeyword{mutual}\<%
\\
\>[4]\AgdaIndent{6}{}\<[6]%
\>[6]\AgdaFunction{⟦\_⟧ᶜ} \AgdaSymbol{:} \AgdaSymbol{(}\AgdaBound{Γ} \AgdaSymbol{:} \AgdaDatatype{Context}\AgdaSymbol{)} \AgdaSymbol{→} \AgdaPrimitiveType{Set} \AgdaSymbol{(}\AgdaPrimitive{lsuc} \AgdaFunction{max-level}\AgdaSymbol{)}\<%
\\
\>[4]\AgdaIndent{6}{}\<[6]%
\>[6]\AgdaFunction{⟦\_⟧ᵀ} \AgdaSymbol{:} \AgdaSymbol{\{}\AgdaBound{Γ} \AgdaSymbol{:} \AgdaDatatype{Context}\AgdaSymbol{\}} \AgdaSymbol{→} \AgdaDatatype{Type} \AgdaBound{Γ} \AgdaSymbol{→} \AgdaFunction{⟦\_⟧ᶜ} \AgdaBound{Γ} \AgdaSymbol{→} \AgdaPrimitiveType{Set} \AgdaFunction{max-level}\<%
\\
%
\\
\>[4]\AgdaIndent{6}{}\<[6]%
\>[6]\AgdaFunction{⟦\_⟧ᶜ} \AgdaInductiveConstructor{ε} \AgdaSymbol{=} \AgdaRecord{⊤}\<%
\\
\>[4]\AgdaIndent{6}{}\<[6]%
\>[6]\AgdaFunction{⟦\_⟧ᶜ} \AgdaSymbol{(}\AgdaBound{Γ} \AgdaInductiveConstructor{▻} \AgdaBound{T}\AgdaSymbol{)} \AgdaSymbol{=} \AgdaRecord{Σ} \AgdaSymbol{(}\AgdaFunction{⟦\_⟧ᶜ} \AgdaBound{Γ}\AgdaSymbol{)} \AgdaSymbol{(λ} \AgdaBound{Γ'} \AgdaSymbol{→} \AgdaFunction{⟦\_⟧ᵀ} \AgdaBound{T} \AgdaBound{Γ'}\AgdaSymbol{)}\<%
\\
%
\\
\>[4]\AgdaIndent{6}{}\<[6]%
\>[6]\AgdaFunction{⟦\_⟧ᵀ} \AgdaSymbol{(}\AgdaBound{T₁} \AgdaInductiveConstructor{‘’} \AgdaBound{x}\AgdaSymbol{)} \AgdaBound{⟦Γ⟧} \AgdaSymbol{=} \AgdaFunction{⟦\_⟧ᵀ} \AgdaBound{T₁} \AgdaSymbol{(}\AgdaBound{⟦Γ⟧} \AgdaInductiveConstructor{,} \AgdaFunction{⟦\_⟧ᵗ} \AgdaBound{x} \AgdaBound{⟦Γ⟧}\AgdaSymbol{)}\<%
\\
\>[4]\AgdaIndent{6}{}\<[6]%
\>[6]\AgdaFunction{⟦\_⟧ᵀ} \AgdaSymbol{(}\AgdaBound{T₂} \AgdaInductiveConstructor{‘’₁} \AgdaBound{a}\AgdaSymbol{)} \AgdaSymbol{(}\AgdaBound{⟦Γ⟧} \AgdaInductiveConstructor{,} \AgdaBound{A⇓}\AgdaSymbol{)} \AgdaSymbol{=} \AgdaFunction{⟦\_⟧ᵀ} \AgdaBound{T₂} \AgdaSymbol{((}\AgdaBound{⟦Γ⟧} \AgdaInductiveConstructor{,} \AgdaFunction{⟦\_⟧ᵗ} \AgdaBound{a} \AgdaBound{⟦Γ⟧}\AgdaSymbol{)} \AgdaInductiveConstructor{,} \AgdaBound{A⇓}\AgdaSymbol{)}\<%
\\
\>[4]\AgdaIndent{6}{}\<[6]%
\>[6]\AgdaFunction{⟦\_⟧ᵀ} \AgdaSymbol{(}\AgdaBound{T₃} \AgdaInductiveConstructor{‘’₂} \AgdaBound{a}\AgdaSymbol{)} \AgdaSymbol{((}\AgdaBound{⟦Γ⟧} \AgdaInductiveConstructor{,} \AgdaBound{A⇓}\AgdaSymbol{)} \AgdaInductiveConstructor{,} \AgdaBound{B⇓}\AgdaSymbol{)} \AgdaSymbol{=} \AgdaFunction{⟦\_⟧ᵀ} \AgdaBound{T₃} \AgdaSymbol{(((}\AgdaBound{⟦Γ⟧} \AgdaInductiveConstructor{,} \AgdaFunction{⟦\_⟧ᵗ} \AgdaBound{a} \AgdaBound{⟦Γ⟧}\AgdaSymbol{)} \AgdaInductiveConstructor{,} \AgdaBound{A⇓}\AgdaSymbol{)} \AgdaInductiveConstructor{,} \AgdaBound{B⇓}\AgdaSymbol{)}\<%
\\
\>[4]\AgdaIndent{6}{}\<[6]%
\>[6]\AgdaFunction{⟦\_⟧ᵀ} \AgdaSymbol{(}\AgdaBound{T₃} \AgdaInductiveConstructor{‘’₃} \AgdaBound{a}\AgdaSymbol{)} \AgdaSymbol{(((}\AgdaBound{⟦Γ⟧} \AgdaInductiveConstructor{,} \AgdaBound{A⇓}\AgdaSymbol{)} \AgdaInductiveConstructor{,} \AgdaBound{B⇓}\AgdaSymbol{)} \AgdaInductiveConstructor{,} \AgdaBound{C⇓}\AgdaSymbol{)} \AgdaSymbol{=} \AgdaFunction{⟦\_⟧ᵀ} \AgdaBound{T₃} \AgdaSymbol{((((}\AgdaBound{⟦Γ⟧} \AgdaInductiveConstructor{,} \AgdaFunction{⟦\_⟧ᵗ} \AgdaBound{a} \AgdaBound{⟦Γ⟧}\AgdaSymbol{)} \AgdaInductiveConstructor{,} \AgdaBound{A⇓}\AgdaSymbol{)} \AgdaInductiveConstructor{,} \AgdaBound{B⇓}\AgdaSymbol{)} \AgdaInductiveConstructor{,} \AgdaBound{C⇓}\AgdaSymbol{)}\<%
\\
\>[4]\AgdaIndent{6}{}\<[6]%
\>[6]\AgdaFunction{⟦\_⟧ᵀ} \AgdaSymbol{(}\AgdaInductiveConstructor{W} \AgdaBound{T₁}\AgdaSymbol{)} \AgdaSymbol{(}\AgdaBound{⟦Γ⟧} \AgdaInductiveConstructor{,} \AgdaSymbol{\_)} \AgdaSymbol{=} \AgdaFunction{⟦\_⟧ᵀ} \AgdaBound{T₁} \AgdaBound{⟦Γ⟧}\<%
\\
\>[4]\AgdaIndent{6}{}\<[6]%
\>[6]\AgdaFunction{⟦\_⟧ᵀ} \AgdaSymbol{(}\AgdaInductiveConstructor{W₁} \AgdaBound{T₂}\AgdaSymbol{)} \AgdaSymbol{((}\AgdaBound{⟦Γ⟧} \AgdaInductiveConstructor{,} \AgdaBound{A⇓}\AgdaSymbol{)} \AgdaInductiveConstructor{,} \AgdaBound{B⇓}\AgdaSymbol{)} \AgdaSymbol{=} \AgdaFunction{⟦\_⟧ᵀ} \AgdaBound{T₂} \AgdaSymbol{(}\AgdaBound{⟦Γ⟧} \AgdaInductiveConstructor{,} \AgdaBound{B⇓}\AgdaSymbol{)}\<%
\\
\>[4]\AgdaIndent{6}{}\<[6]%
\>[6]\AgdaFunction{⟦\_⟧ᵀ} \AgdaSymbol{(}\AgdaInductiveConstructor{W₂} \AgdaBound{T₃}\AgdaSymbol{)} \AgdaSymbol{(((}\AgdaBound{⟦Γ⟧} \AgdaInductiveConstructor{,} \AgdaBound{A⇓}\AgdaSymbol{)} \AgdaInductiveConstructor{,} \AgdaBound{B⇓}\AgdaSymbol{)} \AgdaInductiveConstructor{,} \AgdaBound{C⇓}\AgdaSymbol{)} \AgdaSymbol{=} \AgdaFunction{⟦\_⟧ᵀ} \AgdaBound{T₃} \AgdaSymbol{((}\AgdaBound{⟦Γ⟧} \AgdaInductiveConstructor{,} \AgdaBound{B⇓}\AgdaSymbol{)} \AgdaInductiveConstructor{,} \AgdaBound{C⇓}\AgdaSymbol{)}\<%
\\
\>[4]\AgdaIndent{6}{}\<[6]%
\>[6]\AgdaFunction{⟦\_⟧ᵀ} \AgdaSymbol{(}\AgdaBound{T} \AgdaInductiveConstructor{‘→’} \AgdaBound{T₁}\AgdaSymbol{)} \AgdaBound{⟦Γ⟧} \AgdaSymbol{=} \AgdaSymbol{(}\AgdaBound{T⇓} \AgdaSymbol{:} \AgdaFunction{⟦\_⟧ᵀ} \AgdaBound{T} \AgdaBound{⟦Γ⟧}\AgdaSymbol{)} \AgdaSymbol{→} \AgdaFunction{⟦\_⟧ᵀ} \AgdaBound{T₁} \AgdaSymbol{(}\AgdaBound{⟦Γ⟧} \AgdaInductiveConstructor{,} \AgdaBound{T⇓}\AgdaSymbol{)}\<%
\\
\>[4]\AgdaIndent{6}{}\<[6]%
\>[6]\AgdaFunction{⟦\_⟧ᵀ} \AgdaInductiveConstructor{‘Context’} \AgdaBound{⟦Γ⟧} \AgdaSymbol{=} \AgdaDatatype{Lifted} \AgdaDatatype{Context}\<%
\\
\>[4]\AgdaIndent{6}{}\<[6]%
\>[6]\AgdaFunction{⟦\_⟧ᵀ} \AgdaInductiveConstructor{‘Type’} \AgdaSymbol{(}\AgdaBound{⟦Γ⟧} \AgdaInductiveConstructor{,} \AgdaBound{T⇓}\AgdaSymbol{)} \AgdaSymbol{=} \AgdaDatatype{Lifted} \AgdaSymbol{(}\AgdaDatatype{Type} \AgdaSymbol{(}\AgdaFunction{lower} \AgdaBound{T⇓}\AgdaSymbol{))}\<%
\\
\>[4]\AgdaIndent{6}{}\<[6]%
\>[6]\AgdaFunction{⟦\_⟧ᵀ} \AgdaInductiveConstructor{‘Term’} \AgdaSymbol{(}\AgdaBound{⟦Γ⟧} \AgdaInductiveConstructor{,} \AgdaBound{T⇓} \AgdaInductiveConstructor{,} \AgdaBound{t⇓}\AgdaSymbol{)} \AgdaSymbol{=} \AgdaDatatype{Lifted} \AgdaSymbol{(}\AgdaDatatype{Term} \AgdaSymbol{(}\AgdaFunction{lower} \AgdaBound{t⇓}\AgdaSymbol{))}\<%
\\
\>[4]\AgdaIndent{6}{}\<[6]%
\>[6]\AgdaFunction{⟦\_⟧ᵀ} \AgdaSymbol{(}\AgdaInductiveConstructor{‘Σ’} \AgdaBound{T} \AgdaBound{T₁}\AgdaSymbol{)} \AgdaBound{⟦Γ⟧} \AgdaSymbol{=} \AgdaRecord{Σ} \AgdaSymbol{(}\AgdaFunction{⟦\_⟧ᵀ} \AgdaBound{T} \AgdaBound{⟦Γ⟧}\AgdaSymbol{)} \AgdaSymbol{(λ} \AgdaBound{T⇓} \AgdaSymbol{→} \AgdaFunction{⟦\_⟧ᵀ} \AgdaBound{T₁} \AgdaSymbol{(}\AgdaBound{⟦Γ⟧} \AgdaInductiveConstructor{,} \AgdaBound{T⇓}\AgdaSymbol{))}\<%
\\
%
\\
\>[4]\AgdaIndent{6}{}\<[6]%
\>[6]\AgdaFunction{⟦\_⟧ᵗ} \AgdaSymbol{:} \AgdaSymbol{∀} \AgdaSymbol{\{}\AgdaBound{Γ} \AgdaSymbol{:} \AgdaDatatype{Context}\AgdaSymbol{\}} \AgdaSymbol{\{}\AgdaBound{T} \AgdaSymbol{:} \AgdaDatatype{Type} \AgdaBound{Γ}\AgdaSymbol{\}} \AgdaSymbol{→} \AgdaDatatype{Term} \AgdaBound{T} \AgdaSymbol{→} \AgdaSymbol{(}\AgdaBound{⟦Γ⟧} \AgdaSymbol{:} \AgdaFunction{⟦\_⟧ᶜ} \AgdaBound{Γ}\AgdaSymbol{)} \AgdaSymbol{→} \AgdaFunction{⟦\_⟧ᵀ} \AgdaBound{T} \AgdaBound{⟦Γ⟧}\<%
\\
\>[4]\AgdaIndent{6}{}\<[6]%
\>[6]\AgdaFunction{⟦\_⟧ᵗ} \AgdaSymbol{(}\AgdaInductiveConstructor{w} \AgdaBound{t}\AgdaSymbol{)} \AgdaSymbol{(}\AgdaBound{⟦Γ⟧} \AgdaInductiveConstructor{,} \AgdaBound{A⇓}\AgdaSymbol{)} \AgdaSymbol{=} \AgdaFunction{⟦\_⟧ᵗ} \AgdaBound{t} \AgdaBound{⟦Γ⟧}\<%
\\
\>[4]\AgdaIndent{6}{}\<[6]%
\>[6]\AgdaFunction{⟦\_⟧ᵗ} \AgdaSymbol{(}\AgdaInductiveConstructor{‘λ’} \AgdaBound{t}\AgdaSymbol{)} \AgdaBound{⟦Γ⟧} \AgdaBound{T⇓} \AgdaSymbol{=} \AgdaFunction{⟦\_⟧ᵗ} \AgdaBound{t} \AgdaSymbol{(}\AgdaBound{⟦Γ⟧} \AgdaInductiveConstructor{,} \AgdaBound{T⇓}\AgdaSymbol{)}\<%
\\
\>[4]\AgdaIndent{6}{}\<[6]%
\>[6]\AgdaFunction{⟦\_⟧ᵗ} \AgdaSymbol{(}\AgdaBound{t} \AgdaInductiveConstructor{‘’ₐ} \AgdaBound{t₁}\AgdaSymbol{)} \AgdaBound{⟦Γ⟧} \AgdaSymbol{=} \AgdaFunction{⟦\_⟧ᵗ} \AgdaBound{t} \AgdaBound{⟦Γ⟧} \AgdaSymbol{(}\AgdaFunction{⟦\_⟧ᵗ} \AgdaBound{t₁} \AgdaBound{⟦Γ⟧}\AgdaSymbol{)}\<%
\\
\>[4]\AgdaIndent{6}{}\<[6]%
\>[6]\AgdaFunction{⟦\_⟧ᵗ} \AgdaInductiveConstructor{‘VAR₀’} \AgdaSymbol{(}\AgdaBound{⟦Γ⟧} \AgdaInductiveConstructor{,} \AgdaBound{A⇓}\AgdaSymbol{)} \AgdaSymbol{=} \AgdaBound{A⇓}\<%
\\
\>[4]\AgdaIndent{6}{}\<[6]%
\>[6]\AgdaFunction{⟦\_⟧ᵗ} \AgdaSymbol{(}\AgdaInductiveConstructor{⌜} \AgdaBound{Γ} \AgdaInductiveConstructor{⌝ᶜ}\AgdaSymbol{)} \AgdaBound{⟦Γ⟧} \AgdaSymbol{=} \AgdaInductiveConstructor{lift} \AgdaBound{Γ}\<%
\\
\>[4]\AgdaIndent{6}{}\<[6]%
\>[6]\AgdaFunction{⟦\_⟧ᵗ} \AgdaSymbol{(}\AgdaInductiveConstructor{⌜} \AgdaBound{T} \AgdaInductiveConstructor{⌝ᵀ}\AgdaSymbol{)} \AgdaBound{⟦Γ⟧} \AgdaSymbol{=} \AgdaInductiveConstructor{lift} \AgdaBound{T}\<%
\\
\>[4]\AgdaIndent{6}{}\<[6]%
\>[6]\AgdaFunction{⟦\_⟧ᵗ} \AgdaSymbol{(}\AgdaInductiveConstructor{⌜} \AgdaBound{t} \AgdaInductiveConstructor{⌝ᵗ}\AgdaSymbol{)} \AgdaBound{⟦Γ⟧} \AgdaSymbol{=} \AgdaInductiveConstructor{lift} \AgdaBound{t}\<%
\\
\>[4]\AgdaIndent{6}{}\<[6]%
\>[6]\AgdaFunction{⟦\_⟧ᵗ} \AgdaInductiveConstructor{‘⌜\_⌝ᵗ’} \AgdaBound{⟦Γ⟧} \AgdaSymbol{(}\AgdaInductiveConstructor{lift} \AgdaBound{T⇓}\AgdaSymbol{)} \AgdaSymbol{=} \AgdaInductiveConstructor{lift} \AgdaInductiveConstructor{⌜} \AgdaBound{T⇓} \AgdaInductiveConstructor{⌝ᵗ}\<%
\\
\>[4]\AgdaIndent{6}{}\<[6]%
\>[6]\AgdaFunction{⟦\_⟧ᵗ} \AgdaSymbol{(}\AgdaInductiveConstructor{‘quote-Σ’} \AgdaSymbol{\{}\AgdaBound{Γ₀}\AgdaSymbol{\}} \AgdaSymbol{\{}\AgdaBound{Γ₁}\AgdaSymbol{\})} \AgdaBound{⟦Γ⟧} \AgdaSymbol{(}\AgdaInductiveConstructor{lift} \AgdaBound{Γ} \AgdaInductiveConstructor{,} \AgdaInductiveConstructor{lift} \AgdaBound{T}\AgdaSymbol{)} \AgdaSymbol{=} \AgdaInductiveConstructor{lift} \AgdaSymbol{(}\AgdaInductiveConstructor{\_‘,’\_} \AgdaSymbol{\{}\AgdaBound{Γ₁}\AgdaSymbol{\}} \AgdaInductiveConstructor{⌜} \AgdaBound{Γ} \AgdaInductiveConstructor{⌝ᶜ} \AgdaInductiveConstructor{⌜} \AgdaBound{T} \AgdaInductiveConstructor{⌝ᵀ}\AgdaSymbol{)}\<%
\\
\>[4]\AgdaIndent{6}{}\<[6]%
\>[6]\AgdaFunction{⟦\_⟧ᵗ} \AgdaInductiveConstructor{‘cast’} \AgdaBound{⟦Γ⟧} \AgdaBound{T⇓} \AgdaSymbol{=} \AgdaInductiveConstructor{lift} \AgdaSymbol{(}\AgdaFunction{context-pick-if}\<%
\\
\>[6]\AgdaIndent{32}{}\<[32]%
\>[32]\AgdaSymbol{\{}\AgdaArgument{P} \AgdaSymbol{=} \AgdaDatatype{Type}\AgdaSymbol{\}}\<%
\\
\>[6]\AgdaIndent{32}{}\<[32]%
\>[32]\AgdaSymbol{\{}\AgdaFunction{lower} \AgdaSymbol{(}\AgdaField{Σ.fst} \AgdaBound{T⇓}\AgdaSymbol{)\}}\<%
\\
\>[6]\AgdaIndent{32}{}\<[32]%
\>[32]\AgdaSymbol{(}\AgdaInductiveConstructor{W} \AgdaFunction{dummy}\AgdaSymbol{)}\<%
\\
\>[6]\AgdaIndent{32}{}\<[32]%
\>[32]\AgdaSymbol{(}\AgdaFunction{lower} \AgdaSymbol{(}\AgdaField{Σ.snd} \AgdaBound{T⇓}\AgdaSymbol{)))}\<%
\\
\>[0]\AgdaIndent{6}{}\<[6]%
\>[6]\AgdaFunction{⟦\_⟧ᵗ} \AgdaSymbol{(}\AgdaInductiveConstructor{SW} \AgdaBound{t}\AgdaSymbol{)} \AgdaBound{⟦Γ⟧} \AgdaSymbol{=} \AgdaFunction{⟦\_⟧ᵗ} \AgdaBound{t} \AgdaBound{⟦Γ⟧}\<%
\\
\>[0]\AgdaIndent{6}{}\<[6]%
\>[6]\AgdaFunction{⟦\_⟧ᵗ} \AgdaSymbol{(}\AgdaInductiveConstructor{WS∀} \AgdaBound{t}\AgdaSymbol{)} \AgdaBound{⟦Γ⟧} \AgdaBound{T⇓} \AgdaSymbol{=} \AgdaFunction{⟦\_⟧ᵗ} \AgdaBound{t} \AgdaBound{⟦Γ⟧} \AgdaBound{T⇓}\<%
\\
\>[0]\AgdaIndent{6}{}\<[6]%
\>[6]\AgdaFunction{⟦\_⟧ᵗ} \AgdaSymbol{(}\AgdaInductiveConstructor{SW₁V} \AgdaBound{t}\AgdaSymbol{)} \AgdaBound{⟦Γ⟧} \AgdaSymbol{=} \AgdaFunction{⟦\_⟧ᵗ} \AgdaBound{t} \AgdaBound{⟦Γ⟧}\<%
\\
\>[0]\AgdaIndent{6}{}\<[6]%
\>[6]\AgdaFunction{⟦\_⟧ᵗ} \AgdaSymbol{(}\AgdaInductiveConstructor{W∀} \AgdaBound{t}\AgdaSymbol{)} \AgdaBound{⟦Γ⟧} \AgdaBound{T⇓} \AgdaSymbol{=} \AgdaFunction{⟦\_⟧ᵗ} \AgdaBound{t} \AgdaBound{⟦Γ⟧} \AgdaBound{T⇓}\<%
\\
\>[0]\AgdaIndent{6}{}\<[6]%
\>[6]\AgdaFunction{⟦\_⟧ᵗ} \AgdaSymbol{(}\AgdaInductiveConstructor{W∀⁻¹} \AgdaBound{t}\AgdaSymbol{)} \AgdaBound{⟦Γ⟧} \AgdaBound{T⇓} \AgdaSymbol{=} \AgdaFunction{⟦\_⟧ᵗ} \AgdaBound{t} \AgdaBound{⟦Γ⟧} \AgdaBound{T⇓}\<%
\\
\>[0]\AgdaIndent{6}{}\<[6]%
\>[6]\AgdaFunction{⟦\_⟧ᵗ} \AgdaSymbol{(}\AgdaInductiveConstructor{WW∀} \AgdaBound{t}\AgdaSymbol{)} \AgdaBound{⟦Γ⟧} \AgdaBound{T⇓} \AgdaSymbol{=} \AgdaFunction{⟦\_⟧ᵗ} \AgdaBound{t} \AgdaBound{⟦Γ⟧} \AgdaBound{T⇓}\<%
\\
\>[0]\AgdaIndent{6}{}\<[6]%
\>[6]\AgdaFunction{⟦\_⟧ᵗ} \AgdaSymbol{(}\AgdaInductiveConstructor{S₁∀} \AgdaBound{t}\AgdaSymbol{)} \AgdaBound{⟦Γ⟧} \AgdaBound{T⇓} \AgdaSymbol{=} \AgdaFunction{⟦\_⟧ᵗ} \AgdaBound{t} \AgdaBound{⟦Γ⟧} \AgdaBound{T⇓}\<%
\\
\>[0]\AgdaIndent{6}{}\<[6]%
\>[6]\AgdaFunction{⟦\_⟧ᵗ} \AgdaSymbol{(}\AgdaInductiveConstructor{S₁₀W⁻¹} \AgdaBound{t}\AgdaSymbol{)} \AgdaBound{⟦Γ⟧} \AgdaSymbol{=} \AgdaFunction{⟦\_⟧ᵗ} \AgdaBound{t} \AgdaBound{⟦Γ⟧}\<%
\\
\>[0]\AgdaIndent{6}{}\<[6]%
\>[6]\AgdaFunction{⟦\_⟧ᵗ} \AgdaSymbol{(}\AgdaInductiveConstructor{S₁₀W} \AgdaBound{t}\AgdaSymbol{)} \AgdaBound{⟦Γ⟧} \AgdaSymbol{=} \AgdaFunction{⟦\_⟧ᵗ} \AgdaBound{t} \AgdaBound{⟦Γ⟧}\<%
\\
\>[0]\AgdaIndent{6}{}\<[6]%
\>[6]\AgdaFunction{⟦\_⟧ᵗ} \AgdaSymbol{(}\AgdaInductiveConstructor{WWS₁₀W} \AgdaBound{t}\AgdaSymbol{)} \AgdaBound{⟦Γ⟧} \AgdaSymbol{=} \AgdaFunction{⟦\_⟧ᵗ} \AgdaBound{t} \AgdaBound{⟦Γ⟧}\<%
\\
\>[0]\AgdaIndent{6}{}\<[6]%
\>[6]\AgdaFunction{⟦\_⟧ᵗ} \AgdaSymbol{(}\AgdaInductiveConstructor{WS₂₁₀W⁻¹} \AgdaBound{t}\AgdaSymbol{)} \AgdaBound{⟦Γ⟧} \AgdaSymbol{=} \AgdaFunction{⟦\_⟧ᵗ} \AgdaBound{t} \AgdaBound{⟦Γ⟧}\<%
\\
\>[0]\AgdaIndent{6}{}\<[6]%
\>[6]\AgdaFunction{⟦\_⟧ᵗ} \AgdaSymbol{(}\AgdaInductiveConstructor{S₂₁₀W} \AgdaBound{t}\AgdaSymbol{)} \AgdaBound{⟦Γ⟧} \AgdaSymbol{=} \AgdaFunction{⟦\_⟧ᵗ} \AgdaBound{t} \AgdaBound{⟦Γ⟧}\<%
\\
\>[0]\AgdaIndent{6}{}\<[6]%
\>[6]\AgdaFunction{⟦\_⟧ᵗ} \AgdaSymbol{(}\AgdaInductiveConstructor{W₁₀} \AgdaBound{t}\AgdaSymbol{)} \AgdaBound{⟦Γ⟧} \AgdaSymbol{=} \AgdaFunction{⟦\_⟧ᵗ} \AgdaBound{t} \AgdaBound{⟦Γ⟧}\<%
\\
\>[0]\AgdaIndent{6}{}\<[6]%
\>[6]\AgdaFunction{⟦\_⟧ᵗ} \AgdaSymbol{(}\AgdaInductiveConstructor{W₁₀⁻¹} \AgdaBound{t}\AgdaSymbol{)} \AgdaBound{⟦Γ⟧} \AgdaSymbol{=} \AgdaFunction{⟦\_⟧ᵗ} \AgdaBound{t} \AgdaBound{⟦Γ⟧}\<%
\\
\>[0]\AgdaIndent{6}{}\<[6]%
\>[6]\AgdaFunction{⟦\_⟧ᵗ} \AgdaSymbol{(}\AgdaInductiveConstructor{W₁₀₁₀} \AgdaBound{t}\AgdaSymbol{)} \AgdaBound{⟦Γ⟧} \AgdaSymbol{=} \AgdaFunction{⟦\_⟧ᵗ} \AgdaBound{t} \AgdaBound{⟦Γ⟧}\<%
\\
\>[0]\AgdaIndent{6}{}\<[6]%
\>[6]\AgdaFunction{⟦\_⟧ᵗ} \AgdaSymbol{(}\AgdaInductiveConstructor{S₁W₁} \AgdaBound{t}\AgdaSymbol{)} \AgdaBound{⟦Γ⟧} \AgdaSymbol{=} \AgdaFunction{⟦\_⟧ᵗ} \AgdaBound{t} \AgdaBound{⟦Γ⟧}\<%
\\
\>[0]\AgdaIndent{6}{}\<[6]%
\>[6]\AgdaFunction{⟦\_⟧ᵗ} \AgdaSymbol{(}\AgdaInductiveConstructor{W₁SW₁⁻¹} \AgdaBound{t}\AgdaSymbol{)} \AgdaBound{⟦Γ⟧} \AgdaSymbol{=} \AgdaFunction{⟦\_⟧ᵗ} \AgdaBound{t} \AgdaBound{⟦Γ⟧}\<%
\\
\>[0]\AgdaIndent{6}{}\<[6]%
\>[6]\AgdaFunction{⟦\_⟧ᵗ} \AgdaSymbol{(}\AgdaInductiveConstructor{W₁SW₁} \AgdaBound{t}\AgdaSymbol{)} \AgdaBound{⟦Γ⟧} \AgdaSymbol{=} \AgdaFunction{⟦\_⟧ᵗ} \AgdaBound{t} \AgdaBound{⟦Γ⟧}\<%
\\
\>[0]\AgdaIndent{6}{}\<[6]%
\>[6]\AgdaFunction{⟦\_⟧ᵗ} \AgdaSymbol{(}\AgdaInductiveConstructor{WSSW₁} \AgdaBound{t}\AgdaSymbol{)} \AgdaBound{⟦Γ⟧} \AgdaSymbol{=} \AgdaFunction{⟦\_⟧ᵗ} \AgdaBound{t} \AgdaBound{⟦Γ⟧}\<%
\\
\>[0]\AgdaIndent{6}{}\<[6]%
\>[6]\AgdaFunction{⟦\_⟧ᵗ} \AgdaSymbol{(}\AgdaInductiveConstructor{WSSW₁⁻¹} \AgdaBound{t}\AgdaSymbol{)} \AgdaBound{⟦Γ⟧} \AgdaSymbol{=} \AgdaFunction{⟦\_⟧ᵗ} \AgdaBound{t} \AgdaBound{⟦Γ⟧}\<%
\\
\>[0]\AgdaIndent{6}{}\<[6]%
\>[6]\AgdaFunction{⟦\_⟧ᵗ} \AgdaSymbol{(}\AgdaInductiveConstructor{SW₁₀} \AgdaBound{t}\AgdaSymbol{)} \AgdaBound{⟦Γ⟧} \AgdaSymbol{=} \AgdaFunction{⟦\_⟧ᵗ} \AgdaBound{t} \AgdaBound{⟦Γ⟧}\<%
\\
\>[0]\AgdaIndent{6}{}\<[6]%
\>[6]\AgdaFunction{⟦\_⟧ᵗ} \AgdaSymbol{(}\AgdaInductiveConstructor{WS₂₁₀W₁} \AgdaBound{t}\AgdaSymbol{)} \AgdaBound{⟦Γ⟧} \AgdaSymbol{=} \AgdaFunction{⟦\_⟧ᵗ} \AgdaBound{t} \AgdaBound{⟦Γ⟧}\<%
\\
\>[0]\AgdaIndent{6}{}\<[6]%
\>[6]\AgdaFunction{⟦\_⟧ᵗ} \AgdaSymbol{(}\AgdaInductiveConstructor{S₁₀∀} \AgdaBound{t}\AgdaSymbol{)} \AgdaBound{⟦Γ⟧} \AgdaBound{T⇓} \AgdaSymbol{=} \AgdaFunction{⟦\_⟧ᵗ} \AgdaBound{t} \AgdaBound{⟦Γ⟧} \AgdaBound{T⇓}\<%
\\
\>[0]\AgdaIndent{6}{}\<[6]%
\>[6]\AgdaFunction{⟦\_⟧ᵗ} \AgdaSymbol{(}\AgdaInductiveConstructor{S₂₀₀W₁₀₀} \AgdaBound{t}\AgdaSymbol{)} \AgdaBound{⟦Γ⟧} \AgdaSymbol{=} \AgdaFunction{⟦\_⟧ᵗ} \AgdaBound{t} \AgdaBound{⟦Γ⟧}\<%
\\
\>[0]\AgdaIndent{6}{}\<[6]%
\>[6]\AgdaFunction{⟦\_⟧ᵗ} \AgdaSymbol{(}\AgdaInductiveConstructor{S₁₀W₂₀} \AgdaBound{t}\AgdaSymbol{)} \AgdaBound{⟦Γ⟧} \AgdaSymbol{=} \AgdaFunction{⟦\_⟧ᵗ} \AgdaBound{t} \AgdaBound{⟦Γ⟧}\<%
\\
\>[0]\AgdaIndent{6}{}\<[6]%
\>[6]\AgdaFunction{⟦\_⟧ᵗ} \AgdaSymbol{(}\AgdaInductiveConstructor{W₀₁₀} \AgdaBound{t}\AgdaSymbol{)} \AgdaBound{⟦Γ⟧} \AgdaSymbol{=} \AgdaFunction{⟦\_⟧ᵗ} \AgdaBound{t} \AgdaBound{⟦Γ⟧}\<%
\\
\>[0]\AgdaIndent{6}{}\<[6]%
\>[6]\AgdaFunction{⟦\_⟧ᵗ} \AgdaSymbol{(}\AgdaInductiveConstructor{β-under-S} \AgdaBound{t}\AgdaSymbol{)} \AgdaBound{⟦Γ⟧} \AgdaSymbol{=} \AgdaFunction{⟦\_⟧ᵗ} \AgdaBound{t} \AgdaBound{⟦Γ⟧}\<%
\\
\>[0]\AgdaIndent{6}{}\<[6]%
\>[6]\AgdaFunction{⟦\_⟧ᵗ} \AgdaInductiveConstructor{‘fst'’} \AgdaBound{⟦Γ⟧} \AgdaSymbol{(}\AgdaBound{x} \AgdaInductiveConstructor{,} \AgdaBound{p}\AgdaSymbol{)} \AgdaSymbol{=} \AgdaBound{x}\<%
\\
\>[0]\AgdaIndent{6}{}\<[6]%
\>[6]\AgdaFunction{⟦\_⟧ᵗ} \AgdaInductiveConstructor{‘snd'’} \AgdaSymbol{(}\AgdaBound{⟦Γ⟧} \AgdaInductiveConstructor{,} \AgdaSymbol{(}\AgdaBound{x} \AgdaInductiveConstructor{,} \AgdaBound{p}\AgdaSymbol{))} \AgdaSymbol{=} \AgdaBound{p}\<%
\\
\>[0]\AgdaIndent{6}{}\<[6]%
\>[6]\AgdaFunction{⟦\_⟧ᵗ} \AgdaSymbol{(}\AgdaInductiveConstructor{\_‘,’\_} \AgdaBound{x} \AgdaBound{p}\AgdaSymbol{)} \AgdaBound{⟦Γ⟧} \AgdaSymbol{=} \AgdaFunction{⟦\_⟧ᵗ} \AgdaBound{x} \AgdaBound{⟦Γ⟧} \AgdaInductiveConstructor{,} \AgdaFunction{⟦\_⟧ᵗ} \AgdaBound{p} \AgdaBound{⟦Γ⟧}\<%
\\
\>[0]\AgdaIndent{6}{}\<[6]%
\>[6]\AgdaFunction{⟦\_⟧ᵗ} \AgdaSymbol{(}\AgdaBound{f} \AgdaInductiveConstructor{‘‘’’} \AgdaBound{x}\AgdaSymbol{)} \AgdaBound{⟦Γ⟧} \AgdaSymbol{=} \AgdaInductiveConstructor{lift} \AgdaSymbol{(}\AgdaFunction{lower} \AgdaSymbol{(}\AgdaFunction{⟦\_⟧ᵗ} \AgdaBound{f} \AgdaBound{⟦Γ⟧}\AgdaSymbol{)} \AgdaInductiveConstructor{‘’} \AgdaFunction{lower} \AgdaSymbol{(}\AgdaFunction{⟦\_⟧ᵗ} \AgdaBound{x} \AgdaBound{⟦Γ⟧}\AgdaSymbol{))}\<%
\\
\>[0]\AgdaIndent{6}{}\<[6]%
\>[6]\AgdaFunction{⟦\_⟧ᵗ} \AgdaSymbol{(}\AgdaBound{f} \AgdaInductiveConstructor{w‘‘’’} \AgdaBound{x}\AgdaSymbol{)} \AgdaBound{⟦Γ⟧} \AgdaSymbol{=} \AgdaInductiveConstructor{lift} \AgdaSymbol{(}\AgdaFunction{lower} \AgdaSymbol{(}\AgdaFunction{⟦\_⟧ᵗ} \AgdaBound{f} \AgdaBound{⟦Γ⟧}\AgdaSymbol{)} \AgdaInductiveConstructor{‘’} \AgdaFunction{lower} \AgdaSymbol{(}\AgdaFunction{⟦\_⟧ᵗ} \AgdaBound{x} \AgdaBound{⟦Γ⟧}\AgdaSymbol{))}\<%
\\
\>[0]\AgdaIndent{6}{}\<[6]%
\>[6]\AgdaFunction{⟦\_⟧ᵗ} \AgdaSymbol{(}\AgdaBound{f} \AgdaInductiveConstructor{‘‘→'’’} \AgdaBound{x}\AgdaSymbol{)} \AgdaBound{⟦Γ⟧} \AgdaSymbol{=} \AgdaInductiveConstructor{lift} \AgdaSymbol{(}\AgdaFunction{lower} \AgdaSymbol{(}\AgdaFunction{⟦\_⟧ᵗ} \AgdaBound{f} \AgdaBound{⟦Γ⟧}\AgdaSymbol{)} \AgdaFunction{‘→'’} \AgdaFunction{lower} \AgdaSymbol{(}\AgdaFunction{⟦\_⟧ᵗ} \AgdaBound{x} \AgdaBound{⟦Γ⟧}\AgdaSymbol{))}\<%
\\
\>[0]\AgdaIndent{6}{}\<[6]%
\>[6]\AgdaFunction{⟦\_⟧ᵗ} \AgdaSymbol{(}\AgdaBound{f} \AgdaInductiveConstructor{w‘‘→'’’} \AgdaBound{x}\AgdaSymbol{)} \AgdaBound{⟦Γ⟧} \AgdaSymbol{=} \AgdaInductiveConstructor{lift} \AgdaSymbol{(}\AgdaFunction{lower} \AgdaSymbol{(}\AgdaFunction{⟦\_⟧ᵗ} \AgdaBound{f} \AgdaBound{⟦Γ⟧}\AgdaSymbol{)} \AgdaFunction{‘→'’} \AgdaFunction{lower} \AgdaSymbol{(}\AgdaFunction{⟦\_⟧ᵗ} \AgdaBound{x} \AgdaBound{⟦Γ⟧}\AgdaSymbol{))}\<%
\\
\>[0]\AgdaIndent{6}{}\<[6]%
\>[6]\AgdaFunction{⟦\_⟧ᵗ} \AgdaSymbol{(}\AgdaInductiveConstructor{w→} \AgdaBound{x}\AgdaSymbol{)} \AgdaBound{⟦Γ⟧} \AgdaBound{A⇓} \AgdaSymbol{=} \AgdaFunction{⟦\_⟧ᵗ} \AgdaBound{x} \AgdaSymbol{(}\AgdaField{Σ.fst} \AgdaBound{⟦Γ⟧}\AgdaSymbol{)} \AgdaBound{A⇓}\<%
\\
\>[0]\AgdaIndent{6}{}\<[6]%
\>[6]\AgdaFunction{⟦\_⟧ᵗ} \AgdaInductiveConstructor{w‘‘→'’’→‘‘→'’’} \AgdaBound{⟦Γ⟧} \AgdaBound{T⇓} \AgdaSymbol{=} \AgdaBound{T⇓}\<%
\\
\>[0]\AgdaIndent{6}{}\<[6]%
\>[6]\AgdaFunction{⟦\_⟧ᵗ} \AgdaInductiveConstructor{‘‘→'’’→w‘‘→'’’} \AgdaBound{⟦Γ⟧} \AgdaBound{T⇓} \AgdaSymbol{=} \AgdaBound{T⇓}\<%
\\
\>[0]\AgdaIndent{6}{}\<[6]%
\>[6]\AgdaFunction{⟦\_⟧ᵗ} \AgdaInductiveConstructor{‘tApp-nd’} \AgdaBound{⟦Γ⟧} \AgdaBound{f⇓} \AgdaBound{x⇓} \AgdaSymbol{=} \AgdaInductiveConstructor{lift} \AgdaSymbol{(}\AgdaInductiveConstructor{SW} \AgdaSymbol{(}\AgdaFunction{lower} \AgdaBound{f⇓} \AgdaInductiveConstructor{‘’ₐ} \AgdaFunction{lower} \AgdaBound{x⇓}\AgdaSymbol{))}\<%
\\
\>[0]\AgdaIndent{6}{}\<[6]%
\>[6]\AgdaFunction{⟦\_⟧ᵗ} \AgdaInductiveConstructor{⌜←'⌝} \AgdaBound{⟦Γ⟧} \AgdaBound{T⇓} \AgdaSymbol{=} \AgdaBound{T⇓}\<%
\\
\>[0]\AgdaIndent{6}{}\<[6]%
\>[6]\AgdaFunction{⟦\_⟧ᵗ} \AgdaInductiveConstructor{⌜→'⌝} \AgdaBound{⟦Γ⟧} \AgdaBound{T⇓} \AgdaSymbol{=} \AgdaBound{T⇓}\<%
\\
\>[0]\AgdaIndent{6}{}\<[6]%
\>[6]\AgdaFunction{⟦\_⟧ᵗ} \AgdaSymbol{(}\AgdaInductiveConstructor{‘‘∘-nd’’} \AgdaSymbol{\{}\AgdaBound{A}\AgdaSymbol{\}} \AgdaSymbol{\{}\AgdaBound{B}\AgdaSymbol{\}} \AgdaSymbol{\{}\AgdaBound{C}\AgdaSymbol{\})} \AgdaBound{⟦Γ⟧} \AgdaBound{g⇓} \AgdaBound{f⇓} \AgdaSymbol{=} \AgdaInductiveConstructor{lift} \AgdaSymbol{(}\AgdaFunction{\_‘∘’\_} \AgdaSymbol{\{}\AgdaInductiveConstructor{ε}\AgdaSymbol{\}} \AgdaSymbol{(}\AgdaFunction{lower} \AgdaBound{g⇓}\AgdaSymbol{)} \AgdaSymbol{(}\AgdaFunction{lower} \AgdaBound{f⇓}\AgdaSymbol{))}\<%
\\
\>[0]\AgdaIndent{6}{}\<[6]%
\>[6]\AgdaFunction{⟦\_⟧ᵗ} \AgdaSymbol{(}\AgdaInductiveConstructor{⌜‘’⌝} \AgdaSymbol{\{}\AgdaBound{B}\AgdaSymbol{\}} \AgdaSymbol{\{}\AgdaBound{A}\AgdaSymbol{\}} \AgdaSymbol{\{}\AgdaBound{b}\AgdaSymbol{\})} \AgdaBound{⟦Γ⟧} \AgdaSymbol{=} \AgdaInductiveConstructor{lift} \AgdaSymbol{(}\AgdaInductiveConstructor{‘λ’} \AgdaSymbol{\{}\AgdaInductiveConstructor{ε}\AgdaSymbol{\}} \AgdaSymbol{(}\AgdaInductiveConstructor{‘VAR₀’} \AgdaSymbol{\{}\AgdaInductiveConstructor{ε}\AgdaSymbol{\}} \AgdaSymbol{\{}\AgdaInductiveConstructor{\_‘’\_} \AgdaSymbol{\{}\AgdaInductiveConstructor{ε}\AgdaSymbol{\}} \AgdaBound{A} \AgdaBound{b}\AgdaSymbol{\}))}\<%
\\
\>[0]\AgdaIndent{6}{}\<[6]%
\>[6]\AgdaFunction{⟦\_⟧ᵗ} \AgdaSymbol{(}\AgdaInductiveConstructor{⌜‘’⌝'} \AgdaSymbol{\{}\AgdaBound{B}\AgdaSymbol{\}} \AgdaSymbol{\{}\AgdaBound{A}\AgdaSymbol{\}} \AgdaSymbol{\{}\AgdaBound{b}\AgdaSymbol{\})} \AgdaBound{⟦Γ⟧} \AgdaSymbol{=} \AgdaInductiveConstructor{lift} \AgdaSymbol{(}\AgdaInductiveConstructor{‘λ’} \AgdaSymbol{\{}\AgdaInductiveConstructor{ε}\AgdaSymbol{\}} \AgdaSymbol{(}\AgdaInductiveConstructor{‘VAR₀’} \AgdaSymbol{\{}\AgdaInductiveConstructor{ε}\AgdaSymbol{\}} \AgdaSymbol{\{}\AgdaInductiveConstructor{\_‘’\_} \AgdaSymbol{\{}\AgdaInductiveConstructor{ε}\AgdaSymbol{\}} \AgdaBound{A} \AgdaBound{b}\AgdaSymbol{\}))}\<%
\\
\>[0]\AgdaIndent{6}{}\<[6]%
\>[6]\AgdaFunction{⟦\_⟧ᵗ} \AgdaSymbol{(}\AgdaInductiveConstructor{‘cast-refl’} \AgdaSymbol{\{}\AgdaBound{T}\AgdaSymbol{\})} \AgdaBound{⟦Γ⟧} \AgdaSymbol{=} \AgdaInductiveConstructor{lift} \AgdaSymbol{(}\AgdaFunction{cast-proof} \AgdaSymbol{\{}\AgdaBound{T}\AgdaSymbol{\})}\<%
\\
\>[0]\AgdaIndent{6}{}\<[6]%
\>[6]\AgdaFunction{⟦\_⟧ᵗ} \AgdaSymbol{(}\AgdaInductiveConstructor{‘cast-refl'’} \AgdaSymbol{\{}\AgdaBound{T}\AgdaSymbol{\})} \AgdaBound{⟦Γ⟧} \AgdaSymbol{=} \AgdaInductiveConstructor{lift} \AgdaSymbol{(}\AgdaFunction{cast'-proof} \AgdaSymbol{\{}\AgdaBound{T}\AgdaSymbol{\})}\<%
\\
\>[0]\AgdaIndent{6}{}\<[6]%
\>[6]\AgdaFunction{⟦\_⟧ᵗ} \AgdaSymbol{(}\AgdaInductiveConstructor{‘s→→’} \AgdaSymbol{\{}\AgdaBound{T}\AgdaSymbol{\}} \AgdaSymbol{\{}\AgdaBound{B}\AgdaSymbol{\}} \AgdaSymbol{\{}\AgdaBound{b}\AgdaSymbol{\}} \AgdaSymbol{\{}\AgdaBound{c}\AgdaSymbol{\}} \AgdaSymbol{\{}\AgdaBound{v}\AgdaSymbol{\})} \AgdaBound{⟦Γ⟧} \AgdaSymbol{=} \AgdaInductiveConstructor{lift} \AgdaSymbol{(}\AgdaFunction{‘idfun’} \AgdaSymbol{\{}\AgdaInductiveConstructor{\_‘’\_} \AgdaSymbol{\{}\AgdaInductiveConstructor{ε}\AgdaSymbol{\}} \AgdaSymbol{(}\AgdaFunction{lower} \AgdaSymbol{(}\AgdaFunction{⟦\_⟧ᵗ} \AgdaBound{b} \AgdaInductiveConstructor{tt} \AgdaSymbol{(}\AgdaFunction{⟦\_⟧ᵗ} \AgdaBound{v} \AgdaBound{⟦Γ⟧}\AgdaSymbol{)))} \AgdaSymbol{(}\AgdaFunction{lower} \AgdaSymbol{(}\AgdaFunction{⟦\_⟧ᵗ} \AgdaBound{c} \AgdaInductiveConstructor{tt} \AgdaSymbol{(}\AgdaFunction{⟦\_⟧ᵗ} \AgdaBound{v} \AgdaBound{⟦Γ⟧}\AgdaSymbol{)))\})}\<%
\\
\>[0]\AgdaIndent{6}{}\<[6]%
\>[6]\AgdaFunction{⟦\_⟧ᵗ} \AgdaSymbol{(}\AgdaInductiveConstructor{‘s←←’} \AgdaSymbol{\{}\AgdaBound{T}\AgdaSymbol{\}} \AgdaSymbol{\{}\AgdaBound{B}\AgdaSymbol{\}} \AgdaSymbol{\{}\AgdaBound{b}\AgdaSymbol{\}} \AgdaSymbol{\{}\AgdaBound{c}\AgdaSymbol{\}} \AgdaSymbol{\{}\AgdaBound{v}\AgdaSymbol{\})} \AgdaBound{⟦Γ⟧} \AgdaSymbol{=} \AgdaInductiveConstructor{lift} \AgdaSymbol{(}\AgdaFunction{‘idfun’} \AgdaSymbol{\{}\AgdaInductiveConstructor{\_‘’\_} \AgdaSymbol{\{}\AgdaInductiveConstructor{ε}\AgdaSymbol{\}} \AgdaSymbol{(}\AgdaFunction{lower} \AgdaSymbol{(}\AgdaFunction{⟦\_⟧ᵗ} \AgdaBound{b} \AgdaInductiveConstructor{tt} \AgdaSymbol{(}\AgdaFunction{⟦\_⟧ᵗ} \AgdaBound{v} \AgdaBound{⟦Γ⟧}\AgdaSymbol{)))} \AgdaSymbol{(}\AgdaFunction{lower} \AgdaSymbol{(}\AgdaFunction{⟦\_⟧ᵗ} \AgdaBound{c} \AgdaInductiveConstructor{tt} \AgdaSymbol{(}\AgdaFunction{⟦\_⟧ᵗ} \AgdaBound{v} \AgdaBound{⟦Γ⟧}\AgdaSymbol{)))\})}\<%
\end{code}

\begin{code}%
\> \AgdaKeyword{module} \AgdaModule{well-typed-syntax-interpreter} \AgdaKeyword{where}\<%
\\
\>[0]\AgdaIndent{2}{}\<[2]%
\>[2]\AgdaKeyword{open} \AgdaModule{well-typed-syntax}\<%
\\
\>[0]\AgdaIndent{2}{}\<[2]%
\>[2]\AgdaKeyword{open} \AgdaModule{well-typed-syntax-eq-dec}\<%
\\
%
\\
\>[0]\AgdaIndent{2}{}\<[2]%
\>[2]\AgdaFunction{max-level} \AgdaSymbol{:} \AgdaPostulate{Level}\<%
\\
\>[0]\AgdaIndent{2}{}\<[2]%
\>[2]\AgdaFunction{max-level} \AgdaSymbol{=} \AgdaFunction{well-typed-syntax-pre-interpreter.max-level}\<%
\\
%
\\
\>[0]\AgdaIndent{2}{}\<[2]%
\>[2]\AgdaFunction{⟦\_⟧ᶜ} \AgdaSymbol{:} \AgdaSymbol{(}\AgdaBound{Γ} \AgdaSymbol{:} \AgdaDatatype{Context}\AgdaSymbol{)} \AgdaSymbol{→} \AgdaPrimitiveType{Set} \AgdaSymbol{(}\AgdaPrimitive{lsuc} \AgdaFunction{max-level}\AgdaSymbol{)}\<%
\\
\>[0]\AgdaIndent{2}{}\<[2]%
\>[2]\AgdaFunction{⟦\_⟧ᶜ} \AgdaSymbol{=} \AgdaFunction{well-typed-syntax-pre-interpreter.inner.⟦\_⟧ᶜ}\<%
\\
\>[2]\AgdaIndent{13}{}\<[13]%
\>[13]\AgdaSymbol{(λ} \AgdaBound{ℓ} \AgdaBound{P} \AgdaBound{Γ'} \AgdaBound{dummy} \AgdaBound{val} \AgdaSymbol{→} \AgdaFunction{context-pick-if} \AgdaSymbol{\{}\AgdaArgument{P} \AgdaSymbol{=} \AgdaBound{P}\AgdaSymbol{\}} \AgdaBound{dummy} \AgdaBound{val}\AgdaSymbol{)}\<%
\\
\>[2]\AgdaIndent{13}{}\<[13]%
\>[13]\AgdaSymbol{(λ} \AgdaBound{ℓ} \AgdaBound{P} \AgdaBound{dummy} \AgdaBound{val} \AgdaSymbol{→} \AgdaFunction{context-pick-if-refl} \AgdaSymbol{\{}\AgdaArgument{P} \AgdaSymbol{=} \AgdaBound{P}\AgdaSymbol{\}} \AgdaSymbol{\{}\AgdaBound{dummy}\AgdaSymbol{\})}\<%
\\
%
\\
\>[0]\AgdaIndent{2}{}\<[2]%
\>[2]\AgdaFunction{⟦\_⟧ᵀ} \AgdaSymbol{:} \AgdaSymbol{\{}\AgdaBound{Γ} \AgdaSymbol{:} \AgdaDatatype{Context}\AgdaSymbol{\}} \AgdaSymbol{→} \AgdaDatatype{Type} \AgdaBound{Γ} \AgdaSymbol{→} \AgdaFunction{⟦\_⟧ᶜ} \AgdaBound{Γ} \AgdaSymbol{→} \AgdaPrimitiveType{Set} \AgdaFunction{max-level}\<%
\\
\>[0]\AgdaIndent{2}{}\<[2]%
\>[2]\AgdaFunction{⟦\_⟧ᵀ} \AgdaSymbol{=} \AgdaFunction{well-typed-syntax-pre-interpreter.inner.⟦\_⟧ᵀ}\<%
\\
\>[2]\AgdaIndent{9}{}\<[9]%
\>[9]\AgdaSymbol{(λ} \AgdaBound{ℓ} \AgdaBound{P} \AgdaBound{Γ'} \AgdaBound{dummy} \AgdaBound{val} \AgdaSymbol{→} \AgdaFunction{context-pick-if} \AgdaSymbol{\{}\AgdaArgument{P} \AgdaSymbol{=} \AgdaBound{P}\AgdaSymbol{\}} \AgdaBound{dummy} \AgdaBound{val}\AgdaSymbol{)}\<%
\\
\>[2]\AgdaIndent{9}{}\<[9]%
\>[9]\AgdaSymbol{(λ} \AgdaBound{ℓ} \AgdaBound{P} \AgdaBound{dummy} \AgdaBound{val} \AgdaSymbol{→} \AgdaFunction{context-pick-if-refl} \AgdaSymbol{\{}\AgdaArgument{P} \AgdaSymbol{=} \AgdaBound{P}\AgdaSymbol{\}} \AgdaSymbol{\{}\AgdaBound{dummy}\AgdaSymbol{\})}\<%
\\
%
\\
\>[0]\AgdaIndent{2}{}\<[2]%
\>[2]\AgdaFunction{⟦\_⟧ᵗ} \AgdaSymbol{:} \AgdaSymbol{∀} \AgdaSymbol{\{}\AgdaBound{Γ} \AgdaSymbol{:} \AgdaDatatype{Context}\AgdaSymbol{\}} \AgdaSymbol{\{}\AgdaBound{T} \AgdaSymbol{:} \AgdaDatatype{Type} \AgdaBound{Γ}\AgdaSymbol{\}} \AgdaSymbol{→} \AgdaDatatype{Term} \AgdaBound{T} \AgdaSymbol{→} \AgdaSymbol{(}\AgdaBound{⟦Γ⟧} \AgdaSymbol{:} \AgdaFunction{⟦\_⟧ᶜ} \AgdaBound{Γ}\AgdaSymbol{)} \AgdaSymbol{→} \AgdaFunction{⟦\_⟧ᵀ} \AgdaBound{T} \AgdaBound{⟦Γ⟧}\<%
\\
\>[0]\AgdaIndent{2}{}\<[2]%
\>[2]\AgdaFunction{⟦\_⟧ᵗ} \AgdaSymbol{=} \AgdaFunction{well-typed-syntax-pre-interpreter.inner.⟦\_⟧ᵗ}\<%
\\
\>[2]\AgdaIndent{10}{}\<[10]%
\>[10]\AgdaSymbol{(λ} \AgdaBound{ℓ} \AgdaBound{P} \AgdaBound{Γ'} \AgdaBound{dummy} \AgdaBound{val} \AgdaSymbol{→} \AgdaFunction{context-pick-if} \AgdaSymbol{\{}\AgdaArgument{P} \AgdaSymbol{=} \AgdaBound{P}\AgdaSymbol{\}} \AgdaBound{dummy} \AgdaBound{val}\AgdaSymbol{)}\<%
\\
\>[2]\AgdaIndent{10}{}\<[10]%
\>[10]\AgdaSymbol{(λ} \AgdaBound{ℓ} \AgdaBound{P} \AgdaBound{dummy} \AgdaBound{val} \AgdaSymbol{→} \AgdaFunction{context-pick-if-refl} \AgdaSymbol{\{}\AgdaArgument{P} \AgdaSymbol{=} \AgdaBound{P}\AgdaSymbol{\}} \AgdaSymbol{\{}\AgdaBound{dummy}\AgdaSymbol{\})}\<%
\\
\>\<%
\end{code}

\begin{code}%
\> \AgdaKeyword{module} \AgdaModule{well-typed-syntax-interpreter-full} \AgdaKeyword{where}\<%
\\
\>[0]\AgdaIndent{2}{}\<[2]%
\>[2]\AgdaKeyword{open} \AgdaModule{well-typed-syntax}\<%
\\
\>[0]\AgdaIndent{2}{}\<[2]%
\>[2]\AgdaKeyword{open} \AgdaModule{well-typed-syntax-interpreter}\<%
\\
%
\\
\>[0]\AgdaIndent{2}{}\<[2]%
\>[2]\AgdaFunction{⟦ε⟧ᶜ} \AgdaSymbol{:} \AgdaFunction{⟦} \AgdaInductiveConstructor{ε} \AgdaFunction{⟧ᶜ}\<%
\\
\>[0]\AgdaIndent{2}{}\<[2]%
\>[2]\AgdaFunction{⟦ε⟧ᶜ} \AgdaSymbol{=} \AgdaInductiveConstructor{tt}\<%
\\
%
\\
\>[0]\AgdaIndent{2}{}\<[2]%
\>[2]\AgdaFunction{⟦\_⟧ᵀε} \AgdaSymbol{:} \AgdaDatatype{Type} \AgdaInductiveConstructor{ε} \AgdaSymbol{→} \AgdaPrimitiveType{Set} \AgdaFunction{max-level}\<%
\\
\>[0]\AgdaIndent{2}{}\<[2]%
\>[2]\AgdaFunction{⟦} \AgdaBound{T} \AgdaFunction{⟧ᵀε} \AgdaSymbol{=} \AgdaFunction{⟦} \AgdaBound{T} \AgdaFunction{⟧ᵀ} \AgdaFunction{⟦ε⟧ᶜ}\<%
\\
%
\\
\>[0]\AgdaIndent{2}{}\<[2]%
\>[2]\AgdaFunction{⟦\_⟧ᵗε} \AgdaSymbol{:} \AgdaSymbol{\{}\AgdaBound{T} \AgdaSymbol{:} \AgdaDatatype{Type} \AgdaInductiveConstructor{ε}\AgdaSymbol{\}} \AgdaSymbol{→} \AgdaDatatype{Term} \AgdaBound{T} \AgdaSymbol{→} \AgdaFunction{⟦\_⟧ᵀε} \AgdaBound{T}\<%
\\
\>[0]\AgdaIndent{2}{}\<[2]%
\>[2]\AgdaFunction{⟦} \AgdaBound{t} \AgdaFunction{⟧ᵗε} \AgdaSymbol{=} \AgdaFunction{⟦} \AgdaBound{t} \AgdaFunction{⟧ᵗ} \AgdaFunction{⟦ε⟧ᶜ}\<%
\\
%
\\
\>[0]\AgdaIndent{2}{}\<[2]%
\>[2]\AgdaFunction{⟦\_⟧ᵀε▻} \AgdaSymbol{:} \AgdaSymbol{∀} \AgdaSymbol{\{}\AgdaBound{A}\AgdaSymbol{\}} \AgdaSymbol{→} \AgdaDatatype{Type} \AgdaSymbol{(}\AgdaInductiveConstructor{ε} \AgdaInductiveConstructor{▻} \AgdaBound{A}\AgdaSymbol{)} \AgdaSymbol{→} \AgdaFunction{⟦} \AgdaBound{A} \AgdaFunction{⟧ᵀε} \AgdaSymbol{→} \AgdaPrimitiveType{Set} \AgdaFunction{max-level}\<%
\\
\>[0]\AgdaIndent{2}{}\<[2]%
\>[2]\AgdaFunction{⟦} \AgdaBound{T} \AgdaFunction{⟧ᵀε▻} \AgdaBound{⟦A⟧} \AgdaSymbol{=} \AgdaFunction{⟦} \AgdaBound{T} \AgdaFunction{⟧ᵀ} \AgdaSymbol{(}\AgdaFunction{⟦ε⟧ᶜ} \AgdaInductiveConstructor{,} \AgdaBound{⟦A⟧}\AgdaSymbol{)}\<%
\\
%
\\
\>[0]\AgdaIndent{2}{}\<[2]%
\>[2]\AgdaFunction{⟦\_⟧ᵗε▻} \AgdaSymbol{:} \AgdaSymbol{∀} \AgdaSymbol{\{}\AgdaBound{A}\AgdaSymbol{\}} \AgdaSymbol{→} \AgdaSymbol{\{}\AgdaBound{T} \AgdaSymbol{:} \AgdaDatatype{Type} \AgdaSymbol{(}\AgdaInductiveConstructor{ε} \AgdaInductiveConstructor{▻} \AgdaBound{A}\AgdaSymbol{)\}} \AgdaSymbol{→} \AgdaDatatype{Term} \AgdaBound{T} \AgdaSymbol{→} \AgdaSymbol{(}\AgdaBound{x} \AgdaSymbol{:} \AgdaFunction{⟦} \AgdaBound{A} \AgdaFunction{⟧ᵀε}\AgdaSymbol{)} \AgdaSymbol{→} \AgdaFunction{⟦} \AgdaBound{T} \AgdaFunction{⟧ᵀε▻} \AgdaBound{x}\<%
\\
\>[0]\AgdaIndent{2}{}\<[2]%
\>[2]\AgdaFunction{⟦} \AgdaBound{t} \AgdaFunction{⟧ᵗε▻} \AgdaBound{x} \AgdaSymbol{=} \AgdaFunction{⟦} \AgdaBound{t} \AgdaFunction{⟧ᵗ} \AgdaSymbol{(}\AgdaFunction{⟦ε⟧ᶜ} \AgdaInductiveConstructor{,} \AgdaBound{x}\AgdaSymbol{)}\<%
\\
\>\<%
\end{code}

\begin{code}%
\> \AgdaKeyword{module} \AgdaModule{lӧb} \AgdaKeyword{where}\<%
\\
\>[0]\AgdaIndent{2}{}\<[2]%
\>[2]\AgdaKeyword{open} \AgdaModule{well-typed-syntax}\<%
\\
\>[0]\AgdaIndent{2}{}\<[2]%
\>[2]\AgdaKeyword{open} \AgdaModule{well-typed-quoted-syntax}\<%
\\
\>[0]\AgdaIndent{2}{}\<[2]%
\>[2]\AgdaKeyword{open} \AgdaModule{well-typed-syntax-interpreter-full}\<%
\\
%
\\
\>[0]\AgdaIndent{2}{}\<[2]%
\>[2]\AgdaKeyword{module} \AgdaModule{inner} \AgdaSymbol{(}\AgdaBound{‘X’} \AgdaSymbol{:} \AgdaDatatype{Type} \AgdaInductiveConstructor{ε}\AgdaSymbol{)} \AgdaSymbol{(}\AgdaBound{‘f’} \AgdaSymbol{:} \AgdaDatatype{Term} \AgdaSymbol{\{}\AgdaInductiveConstructor{ε} \AgdaInductiveConstructor{▻} \AgdaSymbol{(}\AgdaFunction{‘□’} \AgdaInductiveConstructor{‘’} \AgdaInductiveConstructor{⌜} \AgdaBound{‘X’} \AgdaInductiveConstructor{⌝ᵀ}\AgdaSymbol{)\}} \AgdaSymbol{(}\AgdaInductiveConstructor{W} \AgdaBound{‘X’}\AgdaSymbol{))} \AgdaKeyword{where}\<%
\\
\>[2]\AgdaIndent{4}{}\<[4]%
\>[4]\AgdaFunction{X} \AgdaSymbol{:} \AgdaPrimitiveType{Set} \AgdaSymbol{\_}\<%
\\
\>[2]\AgdaIndent{4}{}\<[4]%
\>[4]\AgdaFunction{X} \AgdaSymbol{=} \AgdaFunction{⟦} \AgdaBound{‘X’} \AgdaFunction{⟧ᵀε}\<%
\\
%
\\
\>[2]\AgdaIndent{4}{}\<[4]%
\>[4]\AgdaFunction{f''} \AgdaSymbol{:} \AgdaSymbol{(}\AgdaBound{x} \AgdaSymbol{:} \AgdaFunction{⟦\_⟧ᵀε} \AgdaSymbol{(}\AgdaFunction{‘□’} \AgdaInductiveConstructor{‘’} \AgdaInductiveConstructor{⌜} \AgdaBound{‘X’} \AgdaInductiveConstructor{⌝ᵀ}\AgdaSymbol{))} \AgdaSymbol{→} \AgdaFunction{⟦\_⟧ᵀε▻} \AgdaSymbol{\{}\AgdaFunction{‘□’} \AgdaInductiveConstructor{‘’} \AgdaInductiveConstructor{⌜} \AgdaBound{‘X’} \AgdaInductiveConstructor{⌝ᵀ}\AgdaSymbol{\}} \AgdaSymbol{(}\AgdaInductiveConstructor{W} \AgdaBound{‘X’}\AgdaSymbol{)} \AgdaBound{x}\<%
\\
\>[2]\AgdaIndent{4}{}\<[4]%
\>[4]\AgdaFunction{f''} \AgdaSymbol{=} \AgdaFunction{⟦} \AgdaBound{‘f’} \AgdaFunction{⟧ᵗε▻}\<%
\\
%
\\
\>[2]\AgdaIndent{4}{}\<[4]%
\>[4]\AgdaFunction{dummy} \AgdaSymbol{:} \AgdaDatatype{Type} \AgdaInductiveConstructor{ε}\<%
\\
\>[2]\AgdaIndent{4}{}\<[4]%
\>[4]\AgdaFunction{dummy} \AgdaSymbol{=} \AgdaInductiveConstructor{‘Context’}\<%
\\
%
\\
\>[2]\AgdaIndent{4}{}\<[4]%
\>[4]\AgdaFunction{cast} \AgdaSymbol{:} \AgdaSymbol{(}\AgdaBound{Γv} \AgdaSymbol{:} \AgdaRecord{Σ} \AgdaDatatype{Context} \AgdaDatatype{Type}\AgdaSymbol{)} \AgdaSymbol{→} \AgdaDatatype{Type} \AgdaSymbol{(}\AgdaInductiveConstructor{ε} \AgdaInductiveConstructor{▻} \AgdaInductiveConstructor{‘Σ’} \AgdaInductiveConstructor{‘Context’} \AgdaInductiveConstructor{‘Type’}\AgdaSymbol{)}\<%
\\
\>[2]\AgdaIndent{4}{}\<[4]%
\>[4]\AgdaFunction{cast} \AgdaSymbol{(}\AgdaBound{Γ} \AgdaInductiveConstructor{,} \AgdaBound{v}\AgdaSymbol{)} \AgdaSymbol{=} \AgdaFunction{context-pick-if} \AgdaSymbol{\{}\AgdaArgument{P} \AgdaSymbol{=} \AgdaDatatype{Type}\AgdaSymbol{\}} \AgdaSymbol{\{}\AgdaBound{Γ}\AgdaSymbol{\}} \AgdaSymbol{(}\AgdaInductiveConstructor{W} \AgdaFunction{dummy}\AgdaSymbol{)} \AgdaBound{v}\<%
\\
%
\\
\>[2]\AgdaIndent{4}{}\<[4]%
\>[4]\AgdaFunction{Hf} \AgdaSymbol{:} \AgdaSymbol{(}\AgdaBound{h} \AgdaSymbol{:} \AgdaRecord{Σ} \AgdaDatatype{Context} \AgdaDatatype{Type}\AgdaSymbol{)} \AgdaSymbol{→} \AgdaDatatype{Type} \AgdaInductiveConstructor{ε}\<%
\\
\>[2]\AgdaIndent{4}{}\<[4]%
\>[4]\AgdaFunction{Hf} \AgdaBound{h} \AgdaSymbol{=} \AgdaSymbol{(}\AgdaFunction{cast} \AgdaBound{h} \AgdaInductiveConstructor{‘’} \AgdaFunction{quote-Σ} \AgdaBound{h} \AgdaFunction{‘→'’} \AgdaBound{‘X’}\AgdaSymbol{)}\<%
\\
%
\\
\>[2]\AgdaIndent{4}{}\<[4]%
\>[4]\AgdaFunction{qh} \AgdaSymbol{:} \AgdaDatatype{Term} \AgdaSymbol{\{(}\AgdaInductiveConstructor{ε} \AgdaInductiveConstructor{▻} \AgdaInductiveConstructor{‘Σ’} \AgdaInductiveConstructor{‘Context’} \AgdaInductiveConstructor{‘Type’}\AgdaSymbol{)\}} \AgdaSymbol{(}\AgdaInductiveConstructor{W} \AgdaSymbol{(}\AgdaInductiveConstructor{‘Type’} \AgdaInductiveConstructor{‘’} \AgdaFunction{‘ε’}\AgdaSymbol{))}\<%
\\
\>[2]\AgdaIndent{4}{}\<[4]%
\>[4]\AgdaFunction{qh} \AgdaSymbol{=} \AgdaFunction{f'} \AgdaInductiveConstructor{w‘‘’’} \AgdaFunction{x}\<%
\\
\>[4]\AgdaIndent{6}{}\<[6]%
\>[6]\AgdaKeyword{where}\<%
\\
\>[6]\AgdaIndent{8}{}\<[8]%
\>[8]\AgdaFunction{f'} \AgdaSymbol{:} \AgdaDatatype{Term} \AgdaSymbol{(}\AgdaInductiveConstructor{W} \AgdaSymbol{(}\AgdaInductiveConstructor{‘Type’} \AgdaInductiveConstructor{‘’} \AgdaInductiveConstructor{⌜} \AgdaInductiveConstructor{ε} \AgdaInductiveConstructor{▻} \AgdaInductiveConstructor{‘Σ’} \AgdaInductiveConstructor{‘Context’} \AgdaInductiveConstructor{‘Type’} \AgdaInductiveConstructor{⌝ᶜ}\AgdaSymbol{))}\<%
\\
\>[6]\AgdaIndent{8}{}\<[8]%
\>[8]\AgdaFunction{f'} \AgdaSymbol{=} \AgdaInductiveConstructor{w→} \AgdaInductiveConstructor{‘cast’} \AgdaFunction{‘'’ₐ} \AgdaInductiveConstructor{‘VAR₀’}\<%
\\
%
\\
\>[6]\AgdaIndent{8}{}\<[8]%
\>[8]\AgdaFunction{x} \AgdaSymbol{:} \AgdaDatatype{Term} \AgdaSymbol{(}\AgdaInductiveConstructor{W} \AgdaSymbol{(}\AgdaInductiveConstructor{‘Term’} \AgdaInductiveConstructor{‘’₁} \AgdaInductiveConstructor{⌜} \AgdaInductiveConstructor{ε} \AgdaInductiveConstructor{⌝ᶜ} \AgdaInductiveConstructor{‘’} \AgdaInductiveConstructor{⌜} \AgdaInductiveConstructor{‘Σ’} \AgdaInductiveConstructor{‘Context’} \AgdaInductiveConstructor{‘Type’} \AgdaInductiveConstructor{⌝ᵀ}\AgdaSymbol{))}\<%
\\
\>[6]\AgdaIndent{8}{}\<[8]%
\>[8]\AgdaFunction{x} \AgdaSymbol{=} \AgdaSymbol{(}\AgdaInductiveConstructor{w→} \AgdaInductiveConstructor{‘quote-Σ’} \AgdaFunction{‘'’ₐ} \AgdaInductiveConstructor{‘VAR₀’}\AgdaSymbol{)}\<%
\\
%
\\
\>[0]\AgdaIndent{4}{}\<[4]%
\>[4]\AgdaFunction{h₂} \AgdaSymbol{:} \AgdaDatatype{Type} \AgdaSymbol{(}\AgdaInductiveConstructor{ε} \AgdaInductiveConstructor{▻} \AgdaInductiveConstructor{‘Σ’} \AgdaInductiveConstructor{‘Context’} \AgdaInductiveConstructor{‘Type’}\AgdaSymbol{)}\<%
\\
\>[0]\AgdaIndent{4}{}\<[4]%
\>[4]\AgdaFunction{h₂} \AgdaSymbol{=} \AgdaSymbol{(}\AgdaInductiveConstructor{W₁} \AgdaFunction{‘□’} \AgdaInductiveConstructor{‘’} \AgdaSymbol{(}\AgdaFunction{qh} \AgdaInductiveConstructor{w‘‘→'’’} \AgdaInductiveConstructor{w} \AgdaInductiveConstructor{⌜} \AgdaBound{‘X’} \AgdaInductiveConstructor{⌝ᵀ}\AgdaSymbol{))}\<%
\\
%
\\
\>[0]\AgdaIndent{4}{}\<[4]%
\>[4]\AgdaFunction{h} \AgdaSymbol{:} \AgdaRecord{Σ} \AgdaDatatype{Context} \AgdaDatatype{Type}\<%
\\
\>[0]\AgdaIndent{4}{}\<[4]%
\>[4]\AgdaFunction{h} \AgdaSymbol{=} \AgdaSymbol{((}\AgdaInductiveConstructor{ε} \AgdaInductiveConstructor{▻} \AgdaInductiveConstructor{‘Σ’} \AgdaInductiveConstructor{‘Context’} \AgdaInductiveConstructor{‘Type’}\AgdaSymbol{)} \AgdaInductiveConstructor{,} \AgdaFunction{h₂}\AgdaSymbol{)}\<%
\\
%
\\
\>[0]\AgdaIndent{4}{}\<[4]%
\>[4]\AgdaFunction{H0} \AgdaSymbol{:} \AgdaDatatype{Type} \AgdaInductiveConstructor{ε}\<%
\\
\>[0]\AgdaIndent{4}{}\<[4]%
\>[4]\AgdaFunction{H0} \AgdaSymbol{=} \AgdaFunction{Hf} \AgdaFunction{h}\<%
\\
%
\\
\>[0]\AgdaIndent{4}{}\<[4]%
\>[4]\AgdaFunction{H} \AgdaSymbol{:} \AgdaPrimitiveType{Set}\<%
\\
\>[0]\AgdaIndent{4}{}\<[4]%
\>[4]\AgdaFunction{H} \AgdaSymbol{=} \AgdaDatatype{Term} \AgdaSymbol{\{}\AgdaInductiveConstructor{ε}\AgdaSymbol{\}} \AgdaFunction{H0}\<%
\\
%
\\
\>[0]\AgdaIndent{4}{}\<[4]%
\>[4]\AgdaFunction{‘H0’} \AgdaSymbol{:} \AgdaFunction{□} \AgdaSymbol{(}\AgdaInductiveConstructor{‘Type’} \AgdaInductiveConstructor{‘’} \AgdaInductiveConstructor{⌜} \AgdaInductiveConstructor{ε} \AgdaInductiveConstructor{⌝ᶜ}\AgdaSymbol{)}\<%
\\
\>[0]\AgdaIndent{4}{}\<[4]%
\>[4]\AgdaFunction{‘H0’} \AgdaSymbol{=} \AgdaInductiveConstructor{⌜} \AgdaFunction{H0} \AgdaInductiveConstructor{⌝ᵀ}\<%
\\
%
\\
\>[0]\AgdaIndent{4}{}\<[4]%
\>[4]\AgdaFunction{‘H’} \AgdaSymbol{:} \AgdaDatatype{Type} \AgdaInductiveConstructor{ε}\<%
\\
\>[0]\AgdaIndent{4}{}\<[4]%
\>[4]\AgdaFunction{‘H’} \AgdaSymbol{=} \AgdaFunction{‘□’} \AgdaInductiveConstructor{‘’} \AgdaFunction{‘H0’}\<%
\\
%
\\
\>[0]\AgdaIndent{4}{}\<[4]%
\>[4]\AgdaFunction{H0'} \AgdaSymbol{:} \AgdaDatatype{Type} \AgdaInductiveConstructor{ε}\<%
\\
\>[0]\AgdaIndent{4}{}\<[4]%
\>[4]\AgdaFunction{H0'} \AgdaSymbol{=} \AgdaFunction{‘H’} \AgdaFunction{‘→'’} \AgdaBound{‘X’}\<%
\\
%
\\
\>[0]\AgdaIndent{4}{}\<[4]%
\>[4]\AgdaFunction{H'} \AgdaSymbol{:} \AgdaPrimitiveType{Set}\<%
\\
\>[0]\AgdaIndent{4}{}\<[4]%
\>[4]\AgdaFunction{H'} \AgdaSymbol{=} \AgdaDatatype{Term} \AgdaSymbol{\{}\AgdaInductiveConstructor{ε}\AgdaSymbol{\}} \AgdaFunction{H0'}\<%
\\
%
\\
\>[0]\AgdaIndent{4}{}\<[4]%
\>[4]\AgdaFunction{‘H0'’} \AgdaSymbol{:} \AgdaFunction{□} \AgdaSymbol{(}\AgdaInductiveConstructor{‘Type’} \AgdaInductiveConstructor{‘’} \AgdaInductiveConstructor{⌜} \AgdaInductiveConstructor{ε} \AgdaInductiveConstructor{⌝ᶜ}\AgdaSymbol{)}\<%
\\
\>[0]\AgdaIndent{4}{}\<[4]%
\>[4]\AgdaFunction{‘H0'’} \AgdaSymbol{=} \AgdaInductiveConstructor{⌜} \AgdaFunction{H0'} \AgdaInductiveConstructor{⌝ᵀ}\<%
\\
%
\\
\>[0]\AgdaIndent{4}{}\<[4]%
\>[4]\AgdaFunction{‘H'’} \AgdaSymbol{:} \AgdaDatatype{Type} \AgdaInductiveConstructor{ε}\<%
\\
\>[0]\AgdaIndent{4}{}\<[4]%
\>[4]\AgdaFunction{‘H'’} \AgdaSymbol{=} \AgdaFunction{‘□’} \AgdaInductiveConstructor{‘’} \AgdaFunction{‘H0'’}\<%
\\
%
\\
\>[0]\AgdaIndent{4}{}\<[4]%
\>[4]\AgdaFunction{toH-helper-helper} \AgdaSymbol{:} \AgdaSymbol{∀} \AgdaSymbol{\{}\AgdaBound{k}\AgdaSymbol{\}} \AgdaSymbol{→} \AgdaFunction{h₂} \AgdaDatatype{≡} \AgdaBound{k}\<%
\\
\>[4]\AgdaIndent{6}{}\<[6]%
\>[6]\AgdaSymbol{→} \AgdaFunction{□} \AgdaSymbol{(}\AgdaFunction{h₂} \AgdaInductiveConstructor{‘’} \AgdaFunction{quote-Σ} \AgdaFunction{h} \AgdaFunction{‘→'’} \AgdaFunction{‘□’} \AgdaInductiveConstructor{‘’} \AgdaInductiveConstructor{⌜} \AgdaFunction{h₂} \AgdaInductiveConstructor{‘’} \AgdaFunction{quote-Σ} \AgdaFunction{h} \AgdaFunction{‘→'’} \AgdaBound{‘X’} \AgdaInductiveConstructor{⌝ᵀ}\AgdaSymbol{)}\<%
\\
\>[4]\AgdaIndent{6}{}\<[6]%
\>[6]\AgdaSymbol{→} \AgdaFunction{□} \AgdaSymbol{(}\AgdaBound{k} \AgdaInductiveConstructor{‘’} \AgdaFunction{quote-Σ} \AgdaFunction{h} \AgdaFunction{‘→'’} \AgdaFunction{‘□’} \AgdaInductiveConstructor{‘’} \AgdaInductiveConstructor{⌜} \AgdaBound{k} \AgdaInductiveConstructor{‘’} \AgdaFunction{quote-Σ} \AgdaFunction{h} \AgdaFunction{‘→'’} \AgdaBound{‘X’} \AgdaInductiveConstructor{⌝ᵀ}\AgdaSymbol{)}\<%
\\
\>[0]\AgdaIndent{4}{}\<[4]%
\>[4]\AgdaFunction{toH-helper-helper} \AgdaBound{p} \AgdaBound{x} \AgdaSymbol{=} \AgdaFunction{transport} \AgdaSymbol{(λ} \AgdaBound{k} \AgdaSymbol{→} \AgdaFunction{□} \AgdaSymbol{(}\AgdaBound{k} \AgdaInductiveConstructor{‘’} \AgdaFunction{quote-Σ} \AgdaFunction{h} \AgdaFunction{‘→'’} \AgdaFunction{‘□’} \AgdaInductiveConstructor{‘’} \AgdaInductiveConstructor{⌜} \AgdaBound{k} \AgdaInductiveConstructor{‘’} \AgdaFunction{quote-Σ} \AgdaFunction{h} \AgdaFunction{‘→'’} \AgdaBound{‘X’} \AgdaInductiveConstructor{⌝ᵀ}\AgdaSymbol{))} \AgdaBound{p} \AgdaBound{x}\<%
\\
%
\\
\>[0]\AgdaIndent{4}{}\<[4]%
\>[4]\AgdaFunction{toH-helper} \AgdaSymbol{:} \AgdaFunction{□} \AgdaSymbol{(}\AgdaFunction{cast} \AgdaFunction{h} \AgdaInductiveConstructor{‘’} \AgdaFunction{quote-Σ} \AgdaFunction{h} \AgdaFunction{‘→'’} \AgdaFunction{‘H’}\AgdaSymbol{)}\<%
\\
\>[0]\AgdaIndent{4}{}\<[4]%
\>[4]\AgdaFunction{toH-helper} \AgdaSymbol{=} \AgdaFunction{toH-helper-helper}\<%
\\
\>[4]\AgdaIndent{6}{}\<[6]%
\>[6]\AgdaSymbol{\{}\AgdaArgument{k} \AgdaSymbol{=} \AgdaFunction{context-pick-if} \AgdaSymbol{\{}\AgdaArgument{P} \AgdaSymbol{=} \AgdaDatatype{Type}\AgdaSymbol{\}} \AgdaSymbol{\{}\AgdaInductiveConstructor{ε} \AgdaInductiveConstructor{▻} \AgdaInductiveConstructor{‘Σ’} \AgdaInductiveConstructor{‘Context’} \AgdaInductiveConstructor{‘Type’}\AgdaSymbol{\}} \AgdaSymbol{(}\AgdaInductiveConstructor{W} \AgdaFunction{dummy}\AgdaSymbol{)} \AgdaFunction{h₂}\AgdaSymbol{\}}\<%
\\
\>[4]\AgdaIndent{6}{}\<[6]%
\>[6]\AgdaSymbol{(}\AgdaFunction{sym} \AgdaSymbol{(}\AgdaFunction{context-pick-if-refl} \AgdaSymbol{\{}\AgdaArgument{P} \AgdaSymbol{=} \AgdaDatatype{Type}\AgdaSymbol{\}} \AgdaSymbol{\{}\AgdaInductiveConstructor{W} \AgdaFunction{dummy}\AgdaSymbol{\}} \AgdaSymbol{\{}\AgdaFunction{h₂}\AgdaSymbol{\}))}\<%
\\
\>[4]\AgdaIndent{6}{}\<[6]%
\>[6]\AgdaSymbol{(}\AgdaFunction{SSW₁'→} \AgdaSymbol{((}\AgdaInductiveConstructor{‘‘→'’’→w‘‘→'’’} \AgdaFunction{‘∘’} \AgdaInductiveConstructor{‘‘∘-nd’’} \AgdaFunction{‘'’ₐ} \AgdaSymbol{(}\AgdaInductiveConstructor{‘s←←’} \AgdaFunction{‘‘∘’’} \AgdaInductiveConstructor{‘cast-refl’} \AgdaFunction{‘‘∘’’} \AgdaInductiveConstructor{⌜→'⌝} \AgdaFunction{‘'’ₐ} \AgdaInductiveConstructor{⌜} \AgdaInductiveConstructor{‘λ’} \AgdaInductiveConstructor{‘VAR₀’} \AgdaInductiveConstructor{⌝ᵗ}\AgdaSymbol{))} \AgdaFunction{‘∘’} \AgdaInductiveConstructor{⌜←'⌝}\AgdaSymbol{))}\<%
\\
%
\\
\>[0]\AgdaIndent{4}{}\<[4]%
\>[4]\AgdaFunction{‘toH’} \AgdaSymbol{:} \AgdaFunction{□} \AgdaSymbol{(}\AgdaFunction{‘H'’} \AgdaFunction{‘→'’} \AgdaFunction{‘H’}\AgdaSymbol{)}\<%
\\
\>[0]\AgdaIndent{4}{}\<[4]%
\>[4]\AgdaFunction{‘toH’} \AgdaSymbol{=} \AgdaInductiveConstructor{⌜→'⌝} \AgdaFunction{‘∘’} \AgdaInductiveConstructor{‘‘∘-nd’’} \AgdaFunction{‘'’ₐ} \AgdaSymbol{(}\AgdaInductiveConstructor{⌜→'⌝} \AgdaFunction{‘'’ₐ} \AgdaInductiveConstructor{⌜} \AgdaFunction{toH-helper} \AgdaInductiveConstructor{⌝ᵗ}\AgdaSymbol{)} \AgdaFunction{‘∘’} \AgdaInductiveConstructor{⌜←'⌝}\<%
\\
%
\\
\>[0]\AgdaIndent{4}{}\<[4]%
\>[4]\AgdaFunction{toH} \AgdaSymbol{:} \AgdaFunction{H'} \AgdaSymbol{→} \AgdaFunction{H}\<%
\\
\>[0]\AgdaIndent{4}{}\<[4]%
\>[4]\AgdaFunction{toH} \AgdaBound{h'} \AgdaSymbol{=} \AgdaFunction{toH-helper} \AgdaFunction{‘∘’} \AgdaBound{h'}\<%
\\
%
\\
\>[0]\AgdaIndent{4}{}\<[4]%
\>[4]\AgdaFunction{fromH-helper-helper} \AgdaSymbol{:} \AgdaSymbol{∀} \AgdaSymbol{\{}\AgdaBound{k}\AgdaSymbol{\}} \AgdaSymbol{→} \AgdaFunction{h₂} \AgdaDatatype{≡} \AgdaBound{k}\<%
\\
\>[4]\AgdaIndent{6}{}\<[6]%
\>[6]\AgdaSymbol{→} \AgdaFunction{□} \AgdaSymbol{(}\AgdaFunction{‘□’} \AgdaInductiveConstructor{‘’} \AgdaInductiveConstructor{⌜} \AgdaFunction{h₂} \AgdaInductiveConstructor{‘’} \AgdaFunction{quote-Σ} \AgdaFunction{h} \AgdaFunction{‘→'’} \AgdaBound{‘X’} \AgdaInductiveConstructor{⌝ᵀ} \AgdaFunction{‘→'’} \AgdaFunction{h₂} \AgdaInductiveConstructor{‘’} \AgdaFunction{quote-Σ} \AgdaFunction{h}\AgdaSymbol{)}\<%
\\
\>[4]\AgdaIndent{6}{}\<[6]%
\>[6]\AgdaSymbol{→} \AgdaFunction{□} \AgdaSymbol{(}\AgdaFunction{‘□’} \AgdaInductiveConstructor{‘’} \AgdaInductiveConstructor{⌜} \AgdaBound{k} \AgdaInductiveConstructor{‘’} \AgdaFunction{quote-Σ} \AgdaFunction{h} \AgdaFunction{‘→'’} \AgdaBound{‘X’} \AgdaInductiveConstructor{⌝ᵀ} \AgdaFunction{‘→'’} \AgdaBound{k} \AgdaInductiveConstructor{‘’} \AgdaFunction{quote-Σ} \AgdaFunction{h}\AgdaSymbol{)}\<%
\\
\>[0]\AgdaIndent{4}{}\<[4]%
\>[4]\AgdaFunction{fromH-helper-helper} \AgdaBound{p} \AgdaBound{x} \AgdaSymbol{=} \AgdaFunction{transport} \AgdaSymbol{(λ} \AgdaBound{k} \AgdaSymbol{→} \AgdaFunction{□} \AgdaSymbol{(}\AgdaFunction{‘□’} \AgdaInductiveConstructor{‘’} \AgdaInductiveConstructor{⌜} \AgdaBound{k} \AgdaInductiveConstructor{‘’} \AgdaFunction{quote-Σ} \AgdaFunction{h} \AgdaFunction{‘→'’} \AgdaBound{‘X’} \AgdaInductiveConstructor{⌝ᵀ} \AgdaFunction{‘→'’} \AgdaBound{k} \AgdaInductiveConstructor{‘’} \AgdaFunction{quote-Σ} \AgdaFunction{h}\AgdaSymbol{))} \AgdaBound{p} \AgdaBound{x}\<%
\\
%
\\
\>[0]\AgdaIndent{4}{}\<[4]%
\>[4]\AgdaFunction{fromH-helper} \AgdaSymbol{:} \AgdaFunction{□} \AgdaSymbol{(}\AgdaFunction{‘H’} \AgdaFunction{‘→'’} \AgdaFunction{cast} \AgdaFunction{h} \AgdaInductiveConstructor{‘’} \AgdaFunction{quote-Σ} \AgdaFunction{h}\AgdaSymbol{)}\<%
\\
\>[0]\AgdaIndent{4}{}\<[4]%
\>[4]\AgdaFunction{fromH-helper} \AgdaSymbol{=} \AgdaFunction{fromH-helper-helper}\<%
\\
\>[4]\AgdaIndent{6}{}\<[6]%
\>[6]\AgdaSymbol{\{}\AgdaArgument{k} \AgdaSymbol{=} \AgdaFunction{context-pick-if} \AgdaSymbol{\{}\AgdaArgument{P} \AgdaSymbol{=} \AgdaDatatype{Type}\AgdaSymbol{\}} \AgdaSymbol{\{}\AgdaInductiveConstructor{ε} \AgdaInductiveConstructor{▻} \AgdaInductiveConstructor{‘Σ’} \AgdaInductiveConstructor{‘Context’} \AgdaInductiveConstructor{‘Type’}\AgdaSymbol{\}} \AgdaSymbol{(}\AgdaInductiveConstructor{W} \AgdaFunction{dummy}\AgdaSymbol{)} \AgdaFunction{h₂}\AgdaSymbol{\}}\<%
\\
\>[4]\AgdaIndent{6}{}\<[6]%
\>[6]\AgdaSymbol{(}\AgdaFunction{sym} \AgdaSymbol{(}\AgdaFunction{context-pick-if-refl} \AgdaSymbol{\{}\AgdaArgument{P} \AgdaSymbol{=} \AgdaDatatype{Type}\AgdaSymbol{\}} \AgdaSymbol{\{}\AgdaInductiveConstructor{W} \AgdaFunction{dummy}\AgdaSymbol{\}} \AgdaSymbol{\{}\AgdaFunction{h₂}\AgdaSymbol{\}))}\<%
\\
\>[4]\AgdaIndent{6}{}\<[6]%
\>[6]\AgdaSymbol{(}\AgdaFunction{SSW₁'←} \AgdaSymbol{(}\AgdaInductiveConstructor{⌜→'⌝} \AgdaFunction{‘∘’} \AgdaInductiveConstructor{‘‘∘-nd’’} \AgdaFunction{‘'’ₐ} \AgdaSymbol{(}\AgdaInductiveConstructor{⌜→'⌝} \AgdaFunction{‘'’ₐ} \AgdaInductiveConstructor{⌜} \AgdaInductiveConstructor{‘λ’} \AgdaInductiveConstructor{‘VAR₀’} \AgdaInductiveConstructor{⌝ᵗ} \AgdaFunction{‘‘∘’’} \AgdaInductiveConstructor{‘cast-refl'’} \AgdaFunction{‘‘∘’’} \AgdaInductiveConstructor{‘s→→’}\AgdaSymbol{)} \AgdaFunction{‘∘’} \AgdaInductiveConstructor{w‘‘→'’’→‘‘→'’’}\AgdaSymbol{))}\<%
\\
%
\\
\>[0]\AgdaIndent{4}{}\<[4]%
\>[4]\AgdaFunction{‘fromH’} \AgdaSymbol{:} \AgdaFunction{□} \AgdaSymbol{(}\AgdaFunction{‘H’} \AgdaFunction{‘→'’} \AgdaFunction{‘H'’}\AgdaSymbol{)}\<%
\\
\>[0]\AgdaIndent{4}{}\<[4]%
\>[4]\AgdaFunction{‘fromH’} \AgdaSymbol{=} \AgdaInductiveConstructor{⌜→'⌝} \AgdaFunction{‘∘’} \AgdaInductiveConstructor{‘‘∘-nd’’} \AgdaFunction{‘'’ₐ} \AgdaSymbol{(}\AgdaInductiveConstructor{⌜→'⌝} \AgdaFunction{‘'’ₐ} \AgdaInductiveConstructor{⌜} \AgdaFunction{fromH-helper} \AgdaInductiveConstructor{⌝ᵗ}\AgdaSymbol{)} \AgdaFunction{‘∘’} \AgdaInductiveConstructor{⌜←'⌝}\<%
\\
%
\\
\>[0]\AgdaIndent{4}{}\<[4]%
\>[4]\AgdaFunction{fromH} \AgdaSymbol{:} \AgdaFunction{H} \AgdaSymbol{→} \AgdaFunction{H'}\<%
\\
\>[0]\AgdaIndent{4}{}\<[4]%
\>[4]\AgdaFunction{fromH} \AgdaBound{h'} \AgdaSymbol{=} \AgdaFunction{fromH-helper} \AgdaFunction{‘∘’} \AgdaBound{h'}\<%
\\
%
\\
\>[0]\AgdaIndent{4}{}\<[4]%
\>[4]\AgdaFunction{lob} \AgdaSymbol{:} \AgdaFunction{□} \AgdaBound{‘X’}\<%
\\
\>[0]\AgdaIndent{4}{}\<[4]%
\>[4]\AgdaFunction{lob} \AgdaSymbol{=} \AgdaFunction{fromH} \AgdaFunction{h'} \AgdaFunction{‘'’ₐ} \AgdaInductiveConstructor{⌜} \AgdaFunction{h'} \AgdaInductiveConstructor{⌝ᵗ}\<%
\\
\>[4]\AgdaIndent{6}{}\<[6]%
\>[6]\AgdaKeyword{where}\<%
\\
\>[6]\AgdaIndent{8}{}\<[8]%
\>[8]\AgdaFunction{f'} \AgdaSymbol{:} \AgdaDatatype{Term} \AgdaSymbol{\{}\AgdaInductiveConstructor{ε} \AgdaInductiveConstructor{▻} \AgdaFunction{‘□’} \AgdaInductiveConstructor{‘’} \AgdaFunction{‘H0’}\AgdaSymbol{\}} \AgdaSymbol{(}\AgdaInductiveConstructor{W} \AgdaSymbol{(}\AgdaFunction{‘□’} \AgdaInductiveConstructor{‘’} \AgdaSymbol{(}\AgdaInductiveConstructor{⌜} \AgdaFunction{‘□’} \AgdaInductiveConstructor{‘’} \AgdaFunction{‘H0’} \AgdaInductiveConstructor{⌝ᵀ} \AgdaInductiveConstructor{‘‘→'’’} \AgdaInductiveConstructor{⌜} \AgdaBound{‘X’} \AgdaInductiveConstructor{⌝ᵀ}\AgdaSymbol{)))}\<%
\\
\>[6]\AgdaIndent{8}{}\<[8]%
\>[8]\AgdaFunction{f'} \AgdaSymbol{=} \AgdaFunction{Conv0} \AgdaSymbol{\{}\AgdaFunction{‘H0’}\AgdaSymbol{\}} \AgdaSymbol{\{}\AgdaBound{‘X’}\AgdaSymbol{\}} \AgdaSymbol{(}\AgdaInductiveConstructor{SW₁₀} \AgdaSymbol{(}\AgdaFunction{w∀} \AgdaFunction{‘fromH’} \AgdaInductiveConstructor{‘’ₐ} \AgdaInductiveConstructor{‘VAR₀’}\AgdaSymbol{))}\<%
\\
%
\\
\>[6]\AgdaIndent{8}{}\<[8]%
\>[8]\AgdaFunction{x} \AgdaSymbol{:} \AgdaDatatype{Term} \AgdaSymbol{\{}\AgdaInductiveConstructor{ε} \AgdaInductiveConstructor{▻} \AgdaFunction{‘□’} \AgdaInductiveConstructor{‘’} \AgdaFunction{‘H0’}\AgdaSymbol{\}} \AgdaSymbol{(}\AgdaInductiveConstructor{W} \AgdaSymbol{(}\AgdaFunction{‘□’} \AgdaInductiveConstructor{‘’} \AgdaInductiveConstructor{⌜} \AgdaFunction{‘H’} \AgdaInductiveConstructor{⌝ᵀ}\AgdaSymbol{))}\<%
\\
\>[6]\AgdaIndent{8}{}\<[8]%
\>[8]\AgdaFunction{x} \AgdaSymbol{=} \AgdaInductiveConstructor{w→} \AgdaInductiveConstructor{‘⌜\_⌝ᵗ’} \AgdaFunction{‘'’ₐ} \AgdaInductiveConstructor{‘VAR₀’}\<%
\\
%
\\
\>[6]\AgdaIndent{8}{}\<[8]%
\>[8]\AgdaFunction{h'} \AgdaSymbol{:} \AgdaFunction{H}\<%
\\
\>[6]\AgdaIndent{8}{}\<[8]%
\>[8]\AgdaFunction{h'} \AgdaSymbol{=} \AgdaFunction{toH} \AgdaSymbol{(}\AgdaInductiveConstructor{‘λ’} \AgdaSymbol{(}\AgdaInductiveConstructor{w→} \AgdaSymbol{(}\AgdaInductiveConstructor{‘λ’} \AgdaBound{‘f’}\AgdaSymbol{)} \AgdaFunction{‘'’ₐ} \AgdaSymbol{(}\AgdaFunction{w→→} \AgdaInductiveConstructor{‘tApp-nd’} \AgdaFunction{‘'’ₐ} \AgdaFunction{f'} \AgdaFunction{‘'’ₐ} \AgdaFunction{x}\AgdaSymbol{)))}\<%
\\
%
\\
\>[0]\AgdaIndent{2}{}\<[2]%
\>[2]\AgdaFunction{lob} \AgdaSymbol{:} \AgdaSymbol{\{}\AgdaBound{‘X’} \AgdaSymbol{:} \AgdaDatatype{Type} \AgdaInductiveConstructor{ε}\AgdaSymbol{\}} \AgdaSymbol{→} \AgdaFunction{□} \AgdaSymbol{((}\AgdaFunction{‘□’} \AgdaInductiveConstructor{‘’} \AgdaInductiveConstructor{⌜} \AgdaBound{‘X’} \AgdaInductiveConstructor{⌝ᵀ}\AgdaSymbol{)} \AgdaFunction{‘→'’} \AgdaBound{‘X’}\AgdaSymbol{)} \AgdaSymbol{→} \AgdaFunction{□} \AgdaBound{‘X’}\<%
\\
\>[0]\AgdaIndent{2}{}\<[2]%
\>[2]\AgdaFunction{lob} \AgdaSymbol{\{}\AgdaBound{‘X’}\AgdaSymbol{\}} \AgdaBound{‘f’} \AgdaSymbol{=} \AgdaFunction{inner.lob} \AgdaBound{‘X’} \AgdaSymbol{(}\AgdaFunction{un‘λ’} \AgdaBound{‘f’}\AgdaSymbol{)}\<%
\\
\>\<%
\end{code}

\inputacknowledgements

%\printbibliography
\bibliographystyle{abbrvnat}
\bibliography{lob}

\end{document}
